\section{Obit\-Ion\-Cal.h File Reference}
\label{ObitIonCal_8h}\index{ObitIonCal.h@{ObitIonCal.h}}
{\bf Obit\-Ion\-Cal}{\rm (p.\,\pageref{structObitIonCal})} Ionospheric model calibration class. 

{\tt \#include \char`\"{}Obit.h\char`\"{}}\par
{\tt \#include \char`\"{}Obit\-Err.h\char`\"{}}\par
{\tt \#include \char`\"{}Obit\-Image.h\char`\"{}}\par
{\tt \#include \char`\"{}Obit\-Image\-Mosaic.h\char`\"{}}\par
\subsection*{Classes}
\begin{CompactItemize}
\item 
struct {\bf Obit\-Ion\-Cal}
\begin{CompactList}\small\item\em Obit\-Ion\-Cal Class structure. \item\end{CompactList}\item 
struct {\bf Obit\-Ion\-Cal\-Class\-Info}
\begin{CompactList}\small\item\em Class\-Info Structure. \item\end{CompactList}\end{CompactItemize}
\subsection*{Defines}
\begin{CompactItemize}
\item 
\#define {\bf Obit\-Ion\-Cal\-Unref}(in)\ Obit\-Unref (in)
\begin{CompactList}\small\item\em Macro to unreference (and possibly destroy) an {\bf Obit\-Ion\-Cal}{\rm (p.\,\pageref{structObitIonCal})} returns a Obit\-Ion\-Cal$\ast$. \item\end{CompactList}\item 
\#define {\bf Obit\-Ion\-Cal\-Ref}(in)\ Obit\-Ref (in)
\begin{CompactList}\small\item\em Macro to reference (update reference count) an {\bf Obit\-Ion\-Cal}{\rm (p.\,\pageref{structObitIonCal})}. \item\end{CompactList}\item 
\#define {\bf Obit\-Ion\-Cal\-Is\-A}(in)\ Obit\-Is\-A (in, Obit\-Ion\-Cal\-Get\-Class())
\begin{CompactList}\small\item\em Macro to determine if an object is the member of this or a derived class. \item\end{CompactList}\end{CompactItemize}
\subsection*{Typedefs}
\begin{CompactItemize}
\item 
typedef {\bf Obit\-Ion\-Cal} $\ast$($\ast$ {\bf Obit\-Ion\-Cal\-Create\-FP} )(gchar $\ast$name)
\begin{CompactList}\small\item\em Typedef for definition of class pointer structure. \item\end{CompactList}\end{CompactItemize}
\subsection*{Functions}
\begin{CompactItemize}
\item 
void {\bf Obit\-Ion\-Cal\-Class\-Init} (void)
\begin{CompactList}\small\item\em Public: Class initializer. \item\end{CompactList}\item 
{\bf Obit\-Ion\-Cal} $\ast$ {\bf new\-Obit\-Ion\-Cal} (gchar $\ast$name)
\begin{CompactList}\small\item\em Public: Default Constructor. \item\end{CompactList}\item 
{\bf Obit\-Ion\-Cal} $\ast$ {\bf Obit\-Ion\-Cal\-Create} (gchar $\ast$name)
\begin{CompactList}\small\item\em Public: Create/initialize {\bf Obit\-Ion\-Cal}{\rm (p.\,\pageref{structObitIonCal})} structures. \item\end{CompactList}\item 
gconstpointer {\bf Obit\-Ion\-Cal\-Get\-Class} (void)
\begin{CompactList}\small\item\em Public: Class\-Info pointer. \item\end{CompactList}\item 
{\bf Obit\-Ion\-Cal} $\ast$ {\bf Obit\-Ion\-Cal\-Copy} ({\bf Obit\-Ion\-Cal} $\ast$in, {\bf Obit\-Ion\-Cal} $\ast$out, {\bf Obit\-Err} $\ast$err)
\begin{CompactList}\small\item\em Public: Copy (deep) constructor. \item\end{CompactList}\item 
void {\bf Obit\-Ion\-Cal\-Clone} ({\bf Obit\-Ion\-Cal} $\ast$in, {\bf Obit\-Ion\-Cal} $\ast$out, {\bf Obit\-Err} $\ast$err)
\begin{CompactList}\small\item\em Public: Copy structure. \item\end{CompactList}\item 
void {\bf Obit\-Ion\-Cal\-Set\-Data} ({\bf Obit\-Ion\-Cal} $\ast$in, {\bf Obit\-UV} $\ast$in\-UV)
\begin{CompactList}\small\item\em Public: Attach uv data. \item\end{CompactList}\item 
void {\bf Obit\-Ion\-Cal\-Find\-Image} ({\bf Obit\-Ion\-Cal} $\ast$in, {\bf Obit\-Image} $\ast$image, {\bf Obit\-Err} $\ast$err)
\begin{CompactList}\small\item\em Public: Lookup calibrators for image. \item\end{CompactList}\item 
void {\bf Obit\-Ion\-Cal\-Pos\-Mul} ({\bf Obit\-Ion\-Cal} $\ast$in, {\bf Obit\-Image} $\ast$image, {\bf Obit\-Err} $\ast$err)
\begin{CompactList}\small\item\em Public: Fit multiple calibrators in same image. \item\end{CompactList}\item 
{\bf ofloat} {\bf Obit\-Ion\-Cal\-Fit1} ({\bf Obit\-Ion\-Cal} $\ast$in, {\bf olong} epoch, {\bf ofloat} $\ast$coef, {\bf Obit\-Err} $\ast$err)
\begin{CompactList}\small\item\em Public: Fit single epoch Zernike model. \item\end{CompactList}\item 
void {\bf Obit\-Ion\-Caldo\-Cal} ({\bf Obit\-Ion\-Cal} $\ast$in, {\bf Obit\-Err} $\ast$err)
\begin{CompactList}\small\item\em Public: Determine Ionospheric calibration for a UV data. \item\end{CompactList}\item 
void {\bf Obit\-Ion\-Cal\-Pos\-Mosaic} ({\bf Obit\-Ion\-Cal} $\ast$in, {\bf Obit\-Image\-Mosaic} $\ast$mosaic, {\bf olong} epoch, {\bf Obit\-Err} $\ast$err)
\begin{CompactList}\small\item\em Public: Fit position offsets to sources expected at centers of a mosaic. \item\end{CompactList}\end{CompactItemize}


\subsection{Detailed Description}
{\bf Obit\-Ion\-Cal}{\rm (p.\,\pageref{structObitIonCal})} Ionospheric model calibration class. 

This class is derived from the {\bf Obit}{\rm (p.\,\pageref{structObit})} class.\subsection{Creators and Destructors}\label{ObitIonCal_8h_ObitIonCalaccess}
An {\bf Obit\-Ion\-Cal}{\rm (p.\,\pageref{structObitIonCal})} will usually be created using Obit\-Ion\-Cal\-Create which allows specifying a name for the object as well as other information.

A copy of a pointer to an {\bf Obit\-Ion\-Cal}{\rm (p.\,\pageref{structObitIonCal})} should always be made using the {\bf Obit\-Ion\-Cal\-Ref}{\rm (p.\,\pageref{ObitIonCal_8h_a1})} function which updates the reference count in the object. Then whenever freeing an {\bf Obit\-Ion\-Cal}{\rm (p.\,\pageref{structObitIonCal})} or changing a pointer, the function {\bf Obit\-Ion\-Cal\-Unref}{\rm (p.\,\pageref{ObitIonCal_8h_a0})} will decrement the reference count and destroy the object when the reference count hits 0. There is no explicit destructor.

\subsection{Define Documentation}
\index{ObitIonCal.h@{Obit\-Ion\-Cal.h}!ObitIonCalIsA@{ObitIonCalIsA}}
\index{ObitIonCalIsA@{ObitIonCalIsA}!ObitIonCal.h@{Obit\-Ion\-Cal.h}}
\subsubsection{\setlength{\rightskip}{0pt plus 5cm}\#define Obit\-Ion\-Cal\-Is\-A(in)\ Obit\-Is\-A (in, Obit\-Ion\-Cal\-Get\-Class())}\label{ObitIonCal_8h_a2}


Macro to determine if an object is the member of this or a derived class. 

Returns TRUE if a member, else FALSE in = object to reference \index{ObitIonCal.h@{Obit\-Ion\-Cal.h}!ObitIonCalRef@{ObitIonCalRef}}
\index{ObitIonCalRef@{ObitIonCalRef}!ObitIonCal.h@{Obit\-Ion\-Cal.h}}
\subsubsection{\setlength{\rightskip}{0pt plus 5cm}\#define Obit\-Ion\-Cal\-Ref(in)\ Obit\-Ref (in)}\label{ObitIonCal_8h_a1}


Macro to reference (update reference count) an {\bf Obit\-Ion\-Cal}{\rm (p.\,\pageref{structObitIonCal})}. 

returns a Obit\-Ion\-Cal$\ast$. in = object to reference \index{ObitIonCal.h@{Obit\-Ion\-Cal.h}!ObitIonCalUnref@{ObitIonCalUnref}}
\index{ObitIonCalUnref@{ObitIonCalUnref}!ObitIonCal.h@{Obit\-Ion\-Cal.h}}
\subsubsection{\setlength{\rightskip}{0pt plus 5cm}\#define Obit\-Ion\-Cal\-Unref(in)\ Obit\-Unref (in)}\label{ObitIonCal_8h_a0}


Macro to unreference (and possibly destroy) an {\bf Obit\-Ion\-Cal}{\rm (p.\,\pageref{structObitIonCal})} returns a Obit\-Ion\-Cal$\ast$. 

in = object to unreference 

\subsection{Typedef Documentation}
\index{ObitIonCal.h@{Obit\-Ion\-Cal.h}!ObitIonCalCreateFP@{ObitIonCalCreateFP}}
\index{ObitIonCalCreateFP@{ObitIonCalCreateFP}!ObitIonCal.h@{Obit\-Ion\-Cal.h}}
\subsubsection{\setlength{\rightskip}{0pt plus 5cm}typedef {\bf Obit\-Ion\-Cal}$\ast$($\ast$ {\bf Obit\-Ion\-Cal\-Create\-FP})(gchar $\ast$name)}\label{ObitIonCal_8h_a3}


Typedef for definition of class pointer structure. 



\subsection{Function Documentation}
\index{ObitIonCal.h@{Obit\-Ion\-Cal.h}!newObitIonCal@{newObitIonCal}}
\index{newObitIonCal@{newObitIonCal}!ObitIonCal.h@{Obit\-Ion\-Cal.h}}
\subsubsection{\setlength{\rightskip}{0pt plus 5cm}{\bf Obit\-Ion\-Cal}$\ast$ new\-Obit\-Ion\-Cal (gchar $\ast$ {\em name})}\label{ObitIonCal_8h_a5}


Public: Default Constructor. 

Initializes class if needed on first call. \begin{Desc}
\item[Parameters:]
\begin{description}
\item[{\em name}]An optional name for the object. \end{description}
\end{Desc}
\begin{Desc}
\item[Returns:]the new object. \end{Desc}
\index{ObitIonCal.h@{Obit\-Ion\-Cal.h}!ObitIonCalClassInit@{ObitIonCalClassInit}}
\index{ObitIonCalClassInit@{ObitIonCalClassInit}!ObitIonCal.h@{Obit\-Ion\-Cal.h}}
\subsubsection{\setlength{\rightskip}{0pt plus 5cm}void Obit\-Ion\-Cal\-Class\-Init (void)}\label{ObitIonCal_8h_a4}


Public: Class initializer. 

\index{ObitIonCal.h@{Obit\-Ion\-Cal.h}!ObitIonCalClone@{ObitIonCalClone}}
\index{ObitIonCalClone@{ObitIonCalClone}!ObitIonCal.h@{Obit\-Ion\-Cal.h}}
\subsubsection{\setlength{\rightskip}{0pt plus 5cm}void Obit\-Ion\-Cal\-Clone ({\bf Obit\-Ion\-Cal} $\ast$ {\em in}, {\bf Obit\-Ion\-Cal} $\ast$ {\em out}, {\bf Obit\-Err} $\ast$ {\em err})}\label{ObitIonCal_8h_a9}


Public: Copy structure. 

\begin{Desc}
\item[Parameters:]
\begin{description}
\item[{\em in}]The object to copy \item[{\em out}]An existing object pointer for output, must be defined. \item[{\em err}]{\bf Obit}{\rm (p.\,\pageref{structObit})} error stack object. \end{description}
\end{Desc}
\index{ObitIonCal.h@{Obit\-Ion\-Cal.h}!ObitIonCalCopy@{ObitIonCalCopy}}
\index{ObitIonCalCopy@{ObitIonCalCopy}!ObitIonCal.h@{Obit\-Ion\-Cal.h}}
\subsubsection{\setlength{\rightskip}{0pt plus 5cm}{\bf Obit\-Ion\-Cal}$\ast$ Obit\-Ion\-Cal\-Copy ({\bf Obit\-Ion\-Cal} $\ast$ {\em in}, {\bf Obit\-Ion\-Cal} $\ast$ {\em out}, {\bf Obit\-Err} $\ast$ {\em err})}\label{ObitIonCal_8h_a8}


Public: Copy (deep) constructor. 

\begin{Desc}
\item[Parameters:]
\begin{description}
\item[{\em in}]The object to copy \item[{\em out}]An existing object pointer for output or NULL if none exists. \item[{\em err}]{\bf Obit}{\rm (p.\,\pageref{structObit})} error stack object. \end{description}
\end{Desc}
\begin{Desc}
\item[Returns:]pointer to the new object. \end{Desc}
\index{ObitIonCal.h@{Obit\-Ion\-Cal.h}!ObitIonCalCreate@{ObitIonCalCreate}}
\index{ObitIonCalCreate@{ObitIonCalCreate}!ObitIonCal.h@{Obit\-Ion\-Cal.h}}
\subsubsection{\setlength{\rightskip}{0pt plus 5cm}{\bf Obit\-Ion\-Cal}$\ast$ Obit\-Ion\-Cal\-Create (gchar $\ast$ {\em name})}\label{ObitIonCal_8h_a6}


Public: Create/initialize {\bf Obit\-Ion\-Cal}{\rm (p.\,\pageref{structObitIonCal})} structures. 

\begin{Desc}
\item[Parameters:]
\begin{description}
\item[{\em name}]An optional name for the object. \end{description}
\end{Desc}
\begin{Desc}
\item[Returns:]the new object. \end{Desc}
\index{ObitIonCal.h@{Obit\-Ion\-Cal.h}!ObitIonCaldoCal@{ObitIonCaldoCal}}
\index{ObitIonCaldoCal@{ObitIonCaldoCal}!ObitIonCal.h@{Obit\-Ion\-Cal.h}}
\subsubsection{\setlength{\rightskip}{0pt plus 5cm}void Obit\-Ion\-Caldo\-Cal ({\bf Obit\-Ion\-Cal} $\ast$ {\em in}, {\bf Obit\-Err} $\ast$ {\em err})}\label{ObitIonCal_8h_a14}


Public: Determine Ionospheric calibration for a UV data. 

Loops over time slices, imaging and deconvolving selected fields. Then determines position offsets and fits an ionospheric model. Results are stored in an 'NI' table attached to in\-UV. Current maximum 1024 epochs. Routine translated from the AIPSish IONCAL.FOR/IONCAL \begin{Desc}
\item[Parameters:]
\begin{description}
\item[{\em in}]Ion\-Cal object, must have my\-Data UV data attached. If my\-Data is a multisource file, then selection (1 source) , calibration and editing controls must be set on the info member. Control parameters on the my\-Data info member. \begin{itemize}
\item Catalog OBIT\_\-char (?,1,1) AIPSVZ format FITS catalog for defining outliers, 'Default' or blank = use default catalog. \item cat\-Disk OBIT\_\-int (1,1,1) FITS disk for catalog [def 1] \item Outlier\-Dist OBIT\_\-float (1,1,1) How far from pointing to add calibrators \item Outlier\-Flux OBIT\_\-float (1,1,1) Minimum estimated flux density include outlier fields from Catalog. [default 0.1 Jy ] \item Outlier\-SI OBIT\_\-float (1,1,1) Spectral index to use to convert catalog flux density to observed frequency. [default = -0.75] \item Niter OBIT\_\-int (1,1,1) Max. number of components to clean \item min\-Flux OBIT\_\-float (1,1,1) Minimum flux density to CLEAN \item auto\-Window OBIT\_\-boolean (1,1,1)True if auto\-Window feature wanted. \item disp\-URL\char`\"{} OBIT\_\-string (1,1,1) URL of display server\end{itemize}
Control parameters on the in-$>$info member. \begin{itemize}
\item n\-Zern OBIT\_\-int (1,1,1) Zernike polynomial order requested [def 5] \item Max\-Qual OBIT\_\-int (1,1,1) Max. cal. quality code [def 1] \item prt\-Lv OBIT\_\-int (1,1,1) Print level $>$=2 =$>$ give list [def 0] \item Max\-RMS OBIT\_\-float (1,1,1) Maximum allowable RMS in arcsec. [def 20] \item Min\-Rat OBIT\_\-float (1,1,1) Minimum acceptable ratio to average flux [def 0.1] \item Fit\-Dist OBIT\_\-int (1,1,1) Dist, from expected location to search asec [10 pixels] \item Min\-Peak OBIT\_\-float (1,1,1) Min. acceptable image peak (Jy) [1.0] If not given Outlier\-Flux is used \item Max\-Dist OBIT\_\-float (1,1,1) Max. distance (deg/10) to accept calibrator [1.] \item Max\-Wt OBIT\_\-float (1,1,1) Max. weight [10.0] \item do\-INEdit OBIT\_\-boolean (1,1,1) If true flag solutions for which the seeing residual could not be determined or exceeds Max\-RMS [def TRUE] \item do\-SN OBIT\_\-boolean (1,1,1) If true, convert IN table to an SN table attached to in\-UV. [def False] \item sol\-Int OBIT\_\-float (1,1,1) Solution interval (min). [def 1] \end{itemize}
\item[{\em err}]Error code: 0 =$>$ ok, -1 =$>$ all data flagged \end{description}
\end{Desc}
\index{ObitIonCal.h@{Obit\-Ion\-Cal.h}!ObitIonCalFindImage@{ObitIonCalFindImage}}
\index{ObitIonCalFindImage@{ObitIonCalFindImage}!ObitIonCal.h@{Obit\-Ion\-Cal.h}}
\subsubsection{\setlength{\rightskip}{0pt plus 5cm}void Obit\-Ion\-Cal\-Find\-Image ({\bf Obit\-Ion\-Cal} $\ast$ {\em in}, {\bf Obit\-Image} $\ast$ {\em image}, {\bf Obit\-Err} $\ast$ {\em err})}\label{ObitIonCal_8h_a11}


Public: Lookup calibrators for image. 

Previous contents of the {\bf Cal\-List}{\rm (p.\,\pageref{structCalList})} are cleared \begin{Desc}
\item[Parameters:]
\begin{description}
\item[{\em in}]Ion\-Cal object Control parameters are on the info member. \begin{itemize}
\item Catalog OBIT\_\-char (?,1,1) AIPSVZ format FITS catalog for defining outliers, 'Default' or blank = use default catalog. \item cat\-Disk OBIT\_\-int (1,1,1) FITS disk for catalog [def 1] \item Outlier\-Flux OBIT\_\-float (1,1,1) Minimum estimated flux density include cal. fields from Catalog. [default 0.1 Jy ] \item Outlier\-SI OBIT\_\-float (1,1,1) Spectral index to use to convert catalog flux density to observed frequency. [default = -0.75] \item Max\-Qual OBIT\_\-int (1,1,1) Max. cal. quality code [def 1] \item prt\-Lv OBIT\_\-int (1,1,1) Print level $>$=2 =$>$ give list [def 0]\end{itemize}
cal\-List Calibrator list each element of which has: \begin{itemize}
\item ra position RA (deg) of calibrators \item dec position Dec (deg) of calibrators \item shift Offset in field [x,y] on unit Zernike circle of position \item pixel Expected pixel in reference image \item flux Estimated catalog flux density \item offset Measured offset [x,y] from expected position (deg) \item peak Measured peak flux density (image units) \item fint Measured integrated flux density (image units) \item wt Determined weight \item qual Catalog quality code \end{itemize}
\item[{\em image}]Image object \item[{\em err}]Error stack \end{description}
\end{Desc}
\index{ObitIonCal.h@{Obit\-Ion\-Cal.h}!ObitIonCalFit1@{ObitIonCalFit1}}
\index{ObitIonCalFit1@{ObitIonCalFit1}!ObitIonCal.h@{Obit\-Ion\-Cal.h}}
\subsubsection{\setlength{\rightskip}{0pt plus 5cm}{\bf ofloat} Obit\-Ion\-Cal\-Fit1 ({\bf Obit\-Ion\-Cal} $\ast$ {\em in}, {\bf olong} {\em epoch}, {\bf ofloat} $\ast$ {\em coef}, {\bf Obit\-Err} $\ast$ {\em err})}\label{ObitIonCal_8h_a13}


Public: Fit single epoch Zernike model. 

Iteratively edits most discrepant point if needed to get RMS residual down to Max\-RMS. \begin{Desc}
\item[Parameters:]
\begin{description}
\item[{\em in}]Ion\-Cal object Control parameters are on the info member. \begin{itemize}
\item \char`\"{}n\-Zern\char`\"{} OBIT\_\-int (1,1,1) Zernike polynomial order requested [def 5] \item \char`\"{}Max\-RMS\char`\"{} OBIT\_\-float (1,1,1) Target RMS residual (asec), default 10 asec \item prt\-Lv OBIT\_\-int (1,1,1) Print level $>$=3 =$>$ give fitting diagnostics [def. 0] \end{itemize}
\item[{\em epoch}]1-rel time index i measurements \item[{\em coef}][out] Fitted coefficients \item[{\em err}]Error stack \end{description}
\end{Desc}
\begin{Desc}
\item[Returns:]RMS residual in deg, -1 on error \end{Desc}
\index{ObitIonCal.h@{Obit\-Ion\-Cal.h}!ObitIonCalGetClass@{ObitIonCalGetClass}}
\index{ObitIonCalGetClass@{ObitIonCalGetClass}!ObitIonCal.h@{Obit\-Ion\-Cal.h}}
\subsubsection{\setlength{\rightskip}{0pt plus 5cm}gconstpointer Obit\-Ion\-Cal\-Get\-Class (void)}\label{ObitIonCal_8h_a7}


Public: Class\-Info pointer. 

\begin{Desc}
\item[Returns:]pointer to the class structure. \end{Desc}
\index{ObitIonCal.h@{Obit\-Ion\-Cal.h}!ObitIonCalPosMosaic@{ObitIonCalPosMosaic}}
\index{ObitIonCalPosMosaic@{ObitIonCalPosMosaic}!ObitIonCal.h@{Obit\-Ion\-Cal.h}}
\subsubsection{\setlength{\rightskip}{0pt plus 5cm}void Obit\-Ion\-Cal\-Pos\-Mosaic ({\bf Obit\-Ion\-Cal} $\ast$ {\em in}, {\bf Obit\-Image\-Mosaic} $\ast$ {\em mosaic}, {\bf olong} {\em epoch}, {\bf Obit\-Err} $\ast$ {\em err})}\label{ObitIonCal_8h_a15}


Public: Fit position offsets to sources expected at centers of a mosaic. 

Asumes given a cal\-List with entries corresponding to entries in an Image mosaic, fit the actual positioins in the mosaic and write new entries in the {\bf Cal\-List}{\rm (p.\,\pageref{structCalList})} with offsets for acceptable fits. Resultant positions are all referred to a tangent plane at the pointing center, this is referred to as the Zernike plane as this plane will be used to fit the phase screen. The \char`\"{}Zernike Unit Circle\char`\"{} defines the phase screen. Source position offsets are in the X and Y (RA, Dec) as defined by this plane. Routine adapted from the AIPSish CALPOS.FOR/CALPOS \begin{Desc}
\item[Parameters:]
\begin{description}
\item[{\em in}]Ion\-Cal object Control parameters are on the info member. \begin{itemize}
\item \char`\"{}n\-Zern\char`\"{} OBIT\_\-int (1,1,1) Zernike polynomial order requested [def 5] \item Fit\-Dist OBIT\_\-int (1,1,1) dist, from expected location to search asec [10 pixels] \item Min\-Peak OBIT\_\-float (1,1,1) Min. acceptable image peak (Jy) [1.0] \item Max\-Dist OBIT\_\-float (1,1,1) Max. distance (deg/10) to accept calibrator [1.] \item Max\-Wt OBIT\_\-float (1,1,1) Max. weight [10.0] \item Max\-Qual OBIT\_\-int (1,1,1) Max. cal. quality code [def 1] \item prt\-Lv OBIT\_\-int (1,1,1) Print level $>$=1 =$>$ give fits \end{itemize}
\item[{\em mosaic}]Image mosaic object \item[{\em cal\-List}]Calibrator list each element of which has: \begin{itemize}
\item ra position RA (deg) of calibrators \item dec position Dec (deg) of calibrators \item shift Offset in field [x,y] on unit Zernike circle of position \item pixel Expected pixel in reference image \item flux Estimated catalog flux density \item offset Measured offset [x,y] from expected position (deg) \item peak Measured peak flux density (image units) \item fint Measured integrated flux density (image units) \item wt Determined weight \item qual Catalog quality code \end{itemize}
\item[{\em err}]Error stack \item[{\em epoch}]Epoch number \end{description}
\end{Desc}
\index{ObitIonCal.h@{Obit\-Ion\-Cal.h}!ObitIonCalPosMul@{ObitIonCalPosMul}}
\index{ObitIonCalPosMul@{ObitIonCalPosMul}!ObitIonCal.h@{Obit\-Ion\-Cal.h}}
\subsubsection{\setlength{\rightskip}{0pt plus 5cm}void Obit\-Ion\-Cal\-Pos\-Mul ({\bf Obit\-Ion\-Cal} $\ast$ {\em in}, {\bf Obit\-Image} $\ast$ {\em image}, {\bf Obit\-Err} $\ast$ {\em err})}\label{ObitIonCal_8h_a12}


Public: Fit multiple calibrators in same image. 

Given an image containing calibrator sources, return fitted fluxes and position offsets. Resultant positions are all referred to a tangent plane at the pointing center, this is referred to as the Zernike plane as this plane will be used to fit the phase screen. The \char`\"{}Zernike Unit Circle\char`\"{} defines the phase screen. Source position offsets are in the X and Y (RA, Dec) as defined by this plane. Routine translated from the AIPSish CALPOS.FOR/CALPOS \begin{Desc}
\item[Parameters:]
\begin{description}
\item[{\em in}]Ion\-Cal object Control parameters are on the info member. \begin{itemize}
\item \char`\"{}Fit\-Dist\char`\"{} OBIT\_\-int (1,1,1) dist, from expected location to search asec [10 pixels] \item \char`\"{}Min\-Peak\char`\"{} OBIT\_\-float (1,1,1) Min. acceptable image peak (Jy) [1.0] \item \char`\"{}Max\-Dist\char`\"{} OBIT\_\-float (1,1,1) Max. distance (deg/10) to accept calibrator [1.] \item \char`\"{}Max\-Wt\char`\"{} OBIT\_\-float (1,1,1) Max. weight [10.0] \item prt\-Lv OBIT\_\-int (1,1,1) Print level $>$=1 =$>$ give fits \end{itemize}
\item[{\em image}]Image object \item[{\em cal\-List}]Calibrator list each element of which has: \begin{itemize}
\item ra position RA (deg) of calibrators \item dec position Dec (deg) of calibrators \item shift Offset in field [x,y] on unit Zernike circle of position \item pixel Expected pixel in reference image \item flux Estimated catalog flux density \item offset Measured offset [x,y] from expected position (deg) \item peak Measured peak flux density (image units) \item fint Measured integrated flux density (image units) \item wt Determined weight \item qual Catalog quality code \end{itemize}
\item[{\em err}]Error stack \end{description}
\end{Desc}
\index{ObitIonCal.h@{Obit\-Ion\-Cal.h}!ObitIonCalSetData@{ObitIonCalSetData}}
\index{ObitIonCalSetData@{ObitIonCalSetData}!ObitIonCal.h@{Obit\-Ion\-Cal.h}}
\subsubsection{\setlength{\rightskip}{0pt plus 5cm}void Obit\-Ion\-Cal\-Set\-Data ({\bf Obit\-Ion\-Cal} $\ast$ {\em in}, {\bf Obit\-UV} $\ast$ {\em in\-UV})}\label{ObitIonCal_8h_a10}


Public: Attach uv data. 

\begin{Desc}
\item[Parameters:]
\begin{description}
\item[{\em in}]Object to attach UV data to \item[{\em in\-UV}]UV data to attach \end{description}
\end{Desc}
