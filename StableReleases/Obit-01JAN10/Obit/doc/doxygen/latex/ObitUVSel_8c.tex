\section{Obit\-UVSel.c File Reference}
\label{ObitUVSel_8c}\index{ObitUVSel.c@{ObitUVSel.c}}
{\bf Obit\-UVSel}{\rm (p.\,\pageref{structObitUVSel})} {\bf Obit}{\rm (p.\,\pageref{structObit})} uv data selector class definition. 

{\tt \#include \char`\"{}Obit.h\char`\"{}}\par
{\tt \#include \char`\"{}Obit\-UVSel.h\char`\"{}}\par
{\tt \#include \char`\"{}Obit\-Table\-NX.h\char`\"{}}\par
{\tt \#include \char`\"{}Obit\-Table\-SU.h\char`\"{}}\par
{\tt \#include \char`\"{}Obit\-Table\-SUUtil.h\char`\"{}}\par
\subsection*{Functions}
\begin{CompactItemize}
\item 
void {\bf Obit\-UVSel\-Init} (gpointer in)
\begin{CompactList}\small\item\em Private: Initialize newly instantiated object. \item\end{CompactList}\item 
void {\bf Obit\-UVSel\-Clear} (gpointer in)
\begin{CompactList}\small\item\em Private: Deallocate members. \item\end{CompactList}\item 
{\bf Obit\-UVSel} $\ast$ {\bf new\-Obit\-UVSel} (gchar $\ast$name)
\begin{CompactList}\small\item\em Public: Constructor. \item\end{CompactList}\item 
gconstpointer {\bf Obit\-UVSel\-Get\-Class} (void)
\begin{CompactList}\small\item\em Public: Return class pointer. \item\end{CompactList}\item 
{\bf Obit\-UVSel} $\ast$ {\bf Obit\-UVSel\-Copy} ({\bf Obit\-UVSel} $\ast$in, {\bf Obit\-UVSel} $\ast$out, {\bf Obit\-Err} $\ast$err)
\begin{CompactList}\small\item\em Public: Copy UVSel. \item\end{CompactList}\item 
{\bf olong} {\bf Obit\-UVSel\-Buffer\-Size} ({\bf Obit\-UVDesc} $\ast$desc, {\bf Obit\-UVSel} $\ast$sel)
\begin{CompactList}\small\item\em Public: How big a buffer is needed for a data transfer? \item\end{CompactList}\item 
void {\bf Obit\-UVSel\-Default} ({\bf Obit\-UVDesc} $\ast$in, {\bf Obit\-UVSel} $\ast$sel)
\begin{CompactList}\small\item\em Public: Enforces defaults in inaxes, blc, trc. \item\end{CompactList}\item 
void {\bf Obit\-UVSel\-Get\-Desc} ({\bf Obit\-UVDesc} $\ast$in, {\bf Obit\-UVSel} $\ast$sel, {\bf Obit\-UVDesc} $\ast$out, {\bf Obit\-Err} $\ast$err)
\begin{CompactList}\small\item\em Public: Applies selection to a Descriptor for writing. \item\end{CompactList}\item 
void {\bf Obit\-UVSel\-Set\-Desc} ({\bf Obit\-UVDesc} $\ast$in, {\bf Obit\-UVSel} $\ast$sel, {\bf Obit\-UVDesc} $\ast$out, {\bf Obit\-Err} $\ast$err)
\begin{CompactList}\small\item\em Public: Applies selection to a Descriptor for reading. \item\end{CompactList}\item 
void {\bf Obit\-UVSel\-Next\-Init} ({\bf Obit\-UVSel} $\ast$in, {\bf Obit\-UVDesc} $\ast$desc, {\bf Obit\-Err} $\ast$err)
\begin{CompactList}\small\item\em See if an NX table exists and if so initialize it to use in deciding which visibilities to read. \item\end{CompactList}\item 
gboolean {\bf Obit\-UVSel\-Next} ({\bf Obit\-UVSel} $\ast$in, {\bf Obit\-UVDesc} $\ast$desc, {\bf Obit\-Err} $\ast$err)
\begin{CompactList}\small\item\em Uses selector member to decide which visibilities to read next. \item\end{CompactList}\item 
void {\bf Obit\-UVSel\-Shutdown} ({\bf Obit\-UVSel} $\ast$in, {\bf Obit\-Err} $\ast$err)
\begin{CompactList}\small\item\em Close NX table if open . \item\end{CompactList}\item 
void {\bf Obit\-UVSel\-Set\-Sour} ({\bf Obit\-UVSel} $\ast$sel, gpointer in\-Data, {\bf olong} Qual, gchar $\ast$sou\-Code, gchar $\ast$Sources, {\bf olong} lsou, {\bf olong} nsou, {\bf Obit\-Err} $\ast$err)
\begin{CompactList}\small\item\em Set selector for source selection. \item\end{CompactList}\item 
void {\bf Obit\-UVSel\-Set\-Ant} ({\bf Obit\-UVSel} $\ast$sel, {\bf olong} $\ast$Antennas, {\bf olong} nant)
\begin{CompactList}\small\item\em Set selector for antenna selection. \item\end{CompactList}\item 
gboolean {\bf Obit\-UVSel\-Want\-Sour} ({\bf Obit\-UVSel} $\ast$sel, {\bf olong} Sour\-ID)
\begin{CompactList}\small\item\em Determine if a given source is selected. \item\end{CompactList}\item 
gboolean {\bf Obit\-UVSel\-Want\-Ant} ({\bf Obit\-UVSel} $\ast$sel, {\bf olong} ant)
\begin{CompactList}\small\item\em Determine if a given antenna is selected. \item\end{CompactList}\item 
{\bf ofloat} {\bf Obit\-UVSel\-Sub\-Scan} ({\bf Obit\-UVSel} $\ast$sel)
\begin{CompactList}\small\item\em Suggest a length for a sub interval of the current scan such that the scan is evenly divided. \item\end{CompactList}\item 
void {\bf Obit\-UVSel\-Class\-Init} (void)
\begin{CompactList}\small\item\em Public: Class initializer. \item\end{CompactList}\end{CompactItemize}


\subsection{Detailed Description}
{\bf Obit\-UVSel}{\rm (p.\,\pageref{structObitUVSel})} {\bf Obit}{\rm (p.\,\pageref{structObit})} uv data selector class definition. 

This contains information about data selection and calibration.

\subsection{Function Documentation}
\index{ObitUVSel.c@{Obit\-UVSel.c}!newObitUVSel@{newObitUVSel}}
\index{newObitUVSel@{newObitUVSel}!ObitUVSel.c@{Obit\-UVSel.c}}
\subsubsection{\setlength{\rightskip}{0pt plus 5cm}{\bf Obit\-UVSel}$\ast$ new\-Obit\-UVSel (gchar $\ast$ {\em name})}\label{ObitUVSel_8c_a7}


Public: Constructor. 

\begin{Desc}
\item[Returns:]pointer to object created. \end{Desc}
\index{ObitUVSel.c@{Obit\-UVSel.c}!ObitUVSelBufferSize@{ObitUVSelBufferSize}}
\index{ObitUVSelBufferSize@{ObitUVSelBufferSize}!ObitUVSel.c@{Obit\-UVSel.c}}
\subsubsection{\setlength{\rightskip}{0pt plus 5cm}{\bf olong} Obit\-UVSel\-Buffer\-Size ({\bf Obit\-UVDesc} $\ast$ {\em desc}, {\bf Obit\-UVSel} $\ast$ {\em sel})}\label{ObitUVSel_8c_a10}


Public: How big a buffer is needed for a data transfer? 

The buffer is intended for the uncompressed versions of uv data records. \begin{Desc}
\item[Parameters:]
\begin{description}
\item[{\em desc}]Pointer input descriptor. \item[{\em sel}]UV selector. \end{description}
\end{Desc}
\begin{Desc}
\item[Returns:]size in floats needed for I/O. \end{Desc}
\index{ObitUVSel.c@{Obit\-UVSel.c}!ObitUVSelClassInit@{ObitUVSelClassInit}}
\index{ObitUVSelClassInit@{ObitUVSelClassInit}!ObitUVSel.c@{Obit\-UVSel.c}}
\subsubsection{\setlength{\rightskip}{0pt plus 5cm}void Obit\-UVSel\-Class\-Init (void)}\label{ObitUVSel_8c_a22}


Public: Class initializer. 

\index{ObitUVSel.c@{Obit\-UVSel.c}!ObitUVSelClear@{ObitUVSelClear}}
\index{ObitUVSelClear@{ObitUVSelClear}!ObitUVSel.c@{Obit\-UVSel.c}}
\subsubsection{\setlength{\rightskip}{0pt plus 5cm}void Obit\-UVSel\-Clear (gpointer {\em inn})}\label{ObitUVSel_8c_a4}


Private: Deallocate members. 

Does (recursive) deallocation of parent class members. \begin{Desc}
\item[Parameters:]
\begin{description}
\item[{\em inn}]Pointer to the object to deallocate. \end{description}
\end{Desc}
\index{ObitUVSel.c@{Obit\-UVSel.c}!ObitUVSelCopy@{ObitUVSelCopy}}
\index{ObitUVSelCopy@{ObitUVSelCopy}!ObitUVSel.c@{Obit\-UVSel.c}}
\subsubsection{\setlength{\rightskip}{0pt plus 5cm}{\bf Obit\-UVSel}$\ast$ Obit\-UVSel\-Copy ({\bf Obit\-UVSel} $\ast$ {\em in}, {\bf Obit\-UVSel} $\ast$ {\em out}, {\bf Obit\-Err} $\ast$ {\em err})}\label{ObitUVSel_8c_a9}


Public: Copy UVSel. 

\begin{Desc}
\item[Parameters:]
\begin{description}
\item[{\em in}]Pointer to object to be copied. \item[{\em out}]Pointer to object to be written. If NULL then a new structure is created. \item[{\em err}]{\bf Obit\-Err}{\rm (p.\,\pageref{structObitErr})} error stack \end{description}
\end{Desc}
\begin{Desc}
\item[Returns:]Pointer to new object. \end{Desc}
\index{ObitUVSel.c@{Obit\-UVSel.c}!ObitUVSelDefault@{ObitUVSelDefault}}
\index{ObitUVSelDefault@{ObitUVSelDefault}!ObitUVSel.c@{Obit\-UVSel.c}}
\subsubsection{\setlength{\rightskip}{0pt plus 5cm}void Obit\-UVSel\-Default ({\bf Obit\-UVDesc} $\ast$ {\em in}, {\bf Obit\-UVSel} $\ast$ {\em sel})}\label{ObitUVSel_8c_a11}


Public: Enforces defaults in inaxes, blc, trc. 

\begin{Desc}
\item[Parameters:]
\begin{description}
\item[{\em in}]Pointer to descriptor. \item[{\em sel}]UV selector, output vis descriptor changed if needed. \end{description}
\end{Desc}
\index{ObitUVSel.c@{Obit\-UVSel.c}!ObitUVSelGetClass@{ObitUVSelGetClass}}
\index{ObitUVSelGetClass@{ObitUVSelGetClass}!ObitUVSel.c@{Obit\-UVSel.c}}
\subsubsection{\setlength{\rightskip}{0pt plus 5cm}gconstpointer Obit\-UVSel\-Get\-Class (void)}\label{ObitUVSel_8c_a8}


Public: Return class pointer. 

Initializes class if needed on first call. \begin{Desc}
\item[Returns:]pointer to the class structure. \end{Desc}
\index{ObitUVSel.c@{Obit\-UVSel.c}!ObitUVSelGetDesc@{ObitUVSelGetDesc}}
\index{ObitUVSelGetDesc@{ObitUVSelGetDesc}!ObitUVSel.c@{Obit\-UVSel.c}}
\subsubsection{\setlength{\rightskip}{0pt plus 5cm}void Obit\-UVSel\-Get\-Desc ({\bf Obit\-UVDesc} $\ast$ {\em in}, {\bf Obit\-UVSel} $\ast$ {\em sel}, {\bf Obit\-UVDesc} $\ast$ {\em out}, {\bf Obit\-Err} $\ast$ {\em err})}\label{ObitUVSel_8c_a12}


Public: Applies selection to a Descriptor for writing. 

\begin{Desc}
\item[Parameters:]
\begin{description}
\item[{\em in}]Pointer to input descriptor, this describes the data as they appear in memory. \item[{\em sel}]UV selector, blc, trc members changed if needed. \item[{\em out}]Pointer to output descriptor, describing form on disk. \item[{\em err}]{\bf Obit}{\rm (p.\,\pageref{structObit})} error stack \end{description}
\end{Desc}
\index{ObitUVSel.c@{Obit\-UVSel.c}!ObitUVSelInit@{ObitUVSelInit}}
\index{ObitUVSelInit@{ObitUVSelInit}!ObitUVSel.c@{Obit\-UVSel.c}}
\subsubsection{\setlength{\rightskip}{0pt plus 5cm}void Obit\-UVSel\-Init (gpointer {\em inn})}\label{ObitUVSel_8c_a3}


Private: Initialize newly instantiated object. 

Does (recursive) initialization of base class members before this class. \begin{Desc}
\item[Parameters:]
\begin{description}
\item[{\em inn}]Pointer to the object to initialize. \end{description}
\end{Desc}
\index{ObitUVSel.c@{Obit\-UVSel.c}!ObitUVSelNext@{ObitUVSelNext}}
\index{ObitUVSelNext@{ObitUVSelNext}!ObitUVSel.c@{Obit\-UVSel.c}}
\subsubsection{\setlength{\rightskip}{0pt plus 5cm}gboolean Obit\-UVSel\-Next ({\bf Obit\-UVSel} $\ast$ {\em in}, {\bf Obit\-UVDesc} $\ast$ {\em desc}, {\bf Obit\-Err} $\ast$ {\em err})}\label{ObitUVSel_8c_a15}


Uses selector member to decide which visibilities to read next. 

If do\-Index is TRUE, then visibilities are selected from the NX table. \begin{Desc}
\item[Parameters:]
\begin{description}
\item[{\em in}]Pointer to the object. \item[{\em desc}]UV descriptor from IO where the next visibility to read and the number will be stored. 0 causes an initialization.\&nleft) \item[{\em err}]Error stack \end{description}
\end{Desc}
\begin{Desc}
\item[Returns:]TRUE is finished, else FALSE \end{Desc}
\index{ObitUVSel.c@{Obit\-UVSel.c}!ObitUVSelNextInit@{ObitUVSelNextInit}}
\index{ObitUVSelNextInit@{ObitUVSelNextInit}!ObitUVSel.c@{Obit\-UVSel.c}}
\subsubsection{\setlength{\rightskip}{0pt plus 5cm}void Obit\-UVSel\-Next\-Init ({\bf Obit\-UVSel} $\ast$ {\em in}, {\bf Obit\-UVDesc} $\ast$ {\em desc}, {\bf Obit\-Err} $\ast$ {\em err})}\label{ObitUVSel_8c_a14}


See if an NX table exists and if so initialize it to use in deciding which visibilities to read. 

\begin{Desc}
\item[Parameters:]
\begin{description}
\item[{\em in}]Pointer to the object. \item[{\em desc}]UV descriptor from IO where the next visibility to read and the number will be stored. \item[{\em err}]Error stack \end{description}
\end{Desc}
\begin{Desc}
\item[Returns:]TRUE is finished, else FALSE \end{Desc}
\index{ObitUVSel.c@{Obit\-UVSel.c}!ObitUVSelSetAnt@{ObitUVSelSetAnt}}
\index{ObitUVSelSetAnt@{ObitUVSelSetAnt}!ObitUVSel.c@{Obit\-UVSel.c}}
\subsubsection{\setlength{\rightskip}{0pt plus 5cm}void Obit\-UVSel\-Set\-Ant ({\bf Obit\-UVSel} $\ast$ {\em sel}, {\bf olong} $\ast$ {\em Antennas}, {\bf olong} {\em nant})}\label{ObitUVSel_8c_a18}


Set selector for antenna selection. 

\begin{Desc}
\item[Parameters:]
\begin{description}
\item[{\em sel}]UV selector. \item[{\em Antennas}]List of selected Antennas, NULL or all 0 =$>$ all, zero entries after first non zero are ignored. Any negative values means all named are deselected \item[{\em nant}]Number of entries in Antennas \end{description}
\end{Desc}
\index{ObitUVSel.c@{Obit\-UVSel.c}!ObitUVSelSetDesc@{ObitUVSelSetDesc}}
\index{ObitUVSelSetDesc@{ObitUVSelSetDesc}!ObitUVSel.c@{Obit\-UVSel.c}}
\subsubsection{\setlength{\rightskip}{0pt plus 5cm}void Obit\-UVSel\-Set\-Desc ({\bf Obit\-UVDesc} $\ast$ {\em in}, {\bf Obit\-UVSel} $\ast$ {\em sel}, {\bf Obit\-UVDesc} $\ast$ {\em out}, {\bf Obit\-Err} $\ast$ {\em err})}\label{ObitUVSel_8c_a13}


Public: Applies selection to a Descriptor for reading. 

Note: many operations associated with data selection are done in Obit\-UVCal\-Select\-Init. Also sets previously undefined values on sel. \begin{Desc}
\item[Parameters:]
\begin{description}
\item[{\em in}]Pointer to input descriptor, this describes the data as they appear on disk (possibly compressed). \item[{\em sel}]UV selector, members changed if needed. \item[{\em out}]Pointer to output descriptor, this describes the data after any processing when read, or before any compression on output. \item[{\em err}]{\bf Obit}{\rm (p.\,\pageref{structObit})} error stack \end{description}
\end{Desc}
\index{ObitUVSel.c@{Obit\-UVSel.c}!ObitUVSelSetSour@{ObitUVSelSetSour}}
\index{ObitUVSelSetSour@{ObitUVSelSetSour}!ObitUVSel.c@{Obit\-UVSel.c}}
\subsubsection{\setlength{\rightskip}{0pt plus 5cm}void Obit\-UVSel\-Set\-Sour ({\bf Obit\-UVSel} $\ast$ {\em sel}, gpointer {\em in\-Data}, {\bf olong} {\em Qual}, gchar $\ast$ {\em sou\-Code}, gchar $\ast$ {\em Sources}, {\bf olong} {\em lsou}, {\bf olong} {\em nsou}, {\bf Obit\-Err} $\ast$ {\em err})}\label{ObitUVSel_8c_a17}


Set selector for source selection. 

\begin{Desc}
\item[Parameters:]
\begin{description}
\item[{\em sel}]UV selector. \item[{\em in\-Data}]Associated UV data (as gpointer to avoid recursive definition) \item[{\em Qual}]Source qualifier, -1 =$>$ any \item[{\em sou\-Code}]selection of Source by Calcode,if not specified in Source ' ' =$>$ any calibrator code selected '$\ast$ ' =$>$ any non blank code (cal. only) '-CAL' =$>$ blank codes only (no calibrators) anything else = calibrator code to select. NB: The sou\-Code test is applied in addition to the other tests, i.e. Sources and Qual, in the selection of sources to process \item[{\em Sources}]Selected source names, [0] blank=$>$ any this is passed as a lsou x nsou array of characters \item[{\em lsou}]length of source name in Sources \item[{\em nsou}]maximum number of entries in Sources \item[{\em err}]{\bf Obit}{\rm (p.\,\pageref{structObit})} error/message stack \end{description}
\end{Desc}
\index{ObitUVSel.c@{Obit\-UVSel.c}!ObitUVSelShutdown@{ObitUVSelShutdown}}
\index{ObitUVSelShutdown@{ObitUVSelShutdown}!ObitUVSel.c@{Obit\-UVSel.c}}
\subsubsection{\setlength{\rightskip}{0pt plus 5cm}void Obit\-UVSel\-Shutdown ({\bf Obit\-UVSel} $\ast$ {\em in}, {\bf Obit\-Err} $\ast$ {\em err})}\label{ObitUVSel_8c_a16}


Close NX table if open . 

If do\-Index is TRUE, then visibilities are selected from the NX table. \begin{Desc}
\item[Parameters:]
\begin{description}
\item[{\em in}]Pointer to the Selector. \item[{\em err}]Error stack \end{description}
\end{Desc}
\begin{Desc}
\item[Returns:]TRUE is finished, else FALSE \end{Desc}
\index{ObitUVSel.c@{Obit\-UVSel.c}!ObitUVSelSubScan@{ObitUVSelSubScan}}
\index{ObitUVSelSubScan@{ObitUVSelSubScan}!ObitUVSel.c@{Obit\-UVSel.c}}
\subsubsection{\setlength{\rightskip}{0pt plus 5cm}{\bf ofloat} Obit\-UVSel\-Sub\-Scan ({\bf Obit\-UVSel} $\ast$ {\em sel})}\label{ObitUVSel_8c_a21}


Suggest a length for a sub interval of the current scan such that the scan is evenly divided. 

This is based on the target value Sub\-Scan\-Time. This is only useful for Read\-Select operations on an indexed {\bf Obit\-UV}{\rm (p.\,\pageref{structObitUV})}. \begin{Desc}
\item[Parameters:]
\begin{description}
\item[{\em sel}]UV selector. \end{description}
\end{Desc}
\begin{Desc}
\item[Returns:]suggested subscan length in days; \end{Desc}
\index{ObitUVSel.c@{Obit\-UVSel.c}!ObitUVSelWantAnt@{ObitUVSelWantAnt}}
\index{ObitUVSelWantAnt@{ObitUVSelWantAnt}!ObitUVSel.c@{Obit\-UVSel.c}}
\subsubsection{\setlength{\rightskip}{0pt plus 5cm}gboolean Obit\-UVSel\-Want\-Ant ({\bf Obit\-UVSel} $\ast$ {\em sel}, {\bf olong} {\em ant})}\label{ObitUVSel_8c_a20}


Determine if a given antenna is selected. 

\begin{Desc}
\item[Parameters:]
\begin{description}
\item[{\em sel}]UV selector. \item[{\em ant}]antenna id to test \end{description}
\end{Desc}
\begin{Desc}
\item[Returns:]TRUE if antenna selected. \end{Desc}
\index{ObitUVSel.c@{Obit\-UVSel.c}!ObitUVSelWantSour@{ObitUVSelWantSour}}
\index{ObitUVSelWantSour@{ObitUVSelWantSour}!ObitUVSel.c@{Obit\-UVSel.c}}
\subsubsection{\setlength{\rightskip}{0pt plus 5cm}gboolean Obit\-UVSel\-Want\-Sour ({\bf Obit\-UVSel} $\ast$ {\em sel}, {\bf olong} {\em Sour\-ID})}\label{ObitUVSel_8c_a19}


Determine if a given source is selected. 

\begin{Desc}
\item[Parameters:]
\begin{description}
\item[{\em sel}]UV selector. \item[{\em Sour\-ID}]Source ID to be tested \end{description}
\end{Desc}
\begin{Desc}
\item[Returns:]TRUE if source selected. \end{Desc}
