\section{Obit\-FArray.c File Reference}
\label{ObitFArray_8c}\index{ObitFArray.c@{ObitFArray.c}}
{\bf Obit\-FArray}{\rm (p.\,\pageref{structObitFArray})} class function definitions. 

{\tt \#include \char`\"{}Obit\-Thread.h\char`\"{}}\par
{\tt \#include \char`\"{}Obit\-FArray.h\char`\"{}}\par
{\tt \#include \char`\"{}Obit\-Mem.h\char`\"{}}\par
\subsection*{Classes}
\begin{CompactItemize}
\item 
struct {\bf FAFunc\-Arg}
\end{CompactItemize}
\subsection*{Defines}
\begin{CompactItemize}
\item 
\#define {\bf MAXSYMINVSIZE}\ 50
\end{CompactItemize}
\subsection*{Functions}
\begin{CompactItemize}
\item 
void {\bf Obit\-FArray\-Init} (gpointer in)
\begin{CompactList}\small\item\em Private: Initialize newly instantiated object. \item\end{CompactList}\item 
void {\bf Obit\-FArray\-Clear} (gpointer in)
\begin{CompactList}\small\item\em Private: Deallocate members. \item\end{CompactList}\item 
{\bf Obit\-FArray} $\ast$ {\bf new\-Obit\-FArray} (gchar $\ast$name)
\begin{CompactList}\small\item\em Public: Default Constructor. \item\end{CompactList}\item 
gconstpointer {\bf Obit\-FArray\-Get\-Class} (void)
\begin{CompactList}\small\item\em Public: Class\-Info pointer. \item\end{CompactList}\item 
{\bf Obit\-FArray} $\ast$ {\bf Obit\-FArray\-Copy} ({\bf Obit\-FArray} $\ast$in, {\bf Obit\-FArray} $\ast$out, {\bf Obit\-Err} $\ast$err)
\begin{CompactList}\small\item\em Public: Copy (deep) constructor. \item\end{CompactList}\item 
void {\bf Obit\-FArray\-Clone} ({\bf Obit\-FArray} $\ast$in, {\bf Obit\-FArray} $\ast$out, {\bf Obit\-Err} $\ast$err)
\begin{CompactList}\small\item\em Public: Copy structure. \item\end{CompactList}\item 
gboolean {\bf Obit\-FArray\-Is\-Compatable} ({\bf Obit\-FArray} $\ast$in1, {\bf Obit\-FArray} $\ast$in2)
\begin{CompactList}\small\item\em Public: Are two FArrays of compatable geometry. \item\end{CompactList}\item 
{\bf Obit\-FArray} $\ast$ {\bf Obit\-FArray\-Create} (gchar $\ast$name, {\bf olong} ndim, {\bf olong} $\ast$naxis)
\begin{CompactList}\small\item\em Public: Create/initialize {\bf Obit\-FArray}{\rm (p.\,\pageref{structObitFArray})} structures. \item\end{CompactList}\item 
{\bf Obit\-FArray} $\ast$ {\bf Obit\-FArray\-Sub\-Arr} ({\bf Obit\-FArray} $\ast$in, {\bf olong} $\ast$blc, {\bf olong} $\ast$trc, {\bf Obit\-Err} $\ast$err)
\begin{CompactList}\small\item\em Public: Copy Subarray constructor. \item\end{CompactList}\item 
{\bf Obit\-FArray} $\ast$ {\bf Obit\-FArray\-Transpose} ({\bf Obit\-FArray} $\ast$in, {\bf olong} $\ast$order, {\bf Obit\-Err} $\ast$err)
\begin{CompactList}\small\item\em Public: Transpose constructor. \item\end{CompactList}\item 
{\bf Obit\-FArray} $\ast$ {\bf Obit\-FArray\-Realloc} ({\bf Obit\-FArray} $\ast$in, {\bf olong} ndim, {\bf olong} $\ast$naxis)
\begin{CompactList}\small\item\em Public: Reallocate/initialize {\bf Obit\-FArray}{\rm (p.\,\pageref{structObitFArray})} structures. \item\end{CompactList}\item 
{\bf ofloat} $\ast$ {\bf Obit\-FArray\-Index} ({\bf Obit\-FArray} $\ast$in, {\bf olong} $\ast$pos)
\begin{CompactList}\small\item\em Public: return pointer to a specified element. \item\end{CompactList}\item 
{\bf ofloat} {\bf Obit\-FArray\-Max} ({\bf Obit\-FArray} $\ast$in, {\bf olong} $\ast$pos)
\begin{CompactList}\small\item\em Public: Find Maximum value in an {\bf Obit\-FArray}{\rm (p.\,\pageref{structObitFArray})}. \item\end{CompactList}\item 
{\bf ofloat} {\bf Obit\-FArray\-Max\-Abs} ({\bf Obit\-FArray} $\ast$in, {\bf olong} $\ast$pos)
\begin{CompactList}\small\item\em Public: Find Maximum abs value in an {\bf Obit\-FArray}{\rm (p.\,\pageref{structObitFArray})}. \item\end{CompactList}\item 
{\bf ofloat} {\bf Obit\-FArray\-Min} ({\bf Obit\-FArray} $\ast$in, {\bf olong} $\ast$pos)
\begin{CompactList}\small\item\em Public: Find Minimum value in an {\bf Obit\-FArray}{\rm (p.\,\pageref{structObitFArray})}. \item\end{CompactList}\item 
void {\bf Obit\-FArray\-Deblank} ({\bf Obit\-FArray} $\ast$in, {\bf ofloat} scalar)
\begin{CompactList}\small\item\em Public: Replace blanks in an {\bf Obit\-FArray}{\rm (p.\,\pageref{structObitFArray})}. \item\end{CompactList}\item 
{\bf ofloat} {\bf Obit\-FArray\-RMS} ({\bf Obit\-FArray} $\ast$in)
\begin{CompactList}\small\item\em Public: RMS of pixel distribution from histogram. \item\end{CompactList}\item 
{\bf ofloat} {\bf Obit\-FArray\-Raw\-RMS} ({\bf Obit\-FArray} $\ast$in)
\begin{CompactList}\small\item\em Public: RMS of pixel distribution. \item\end{CompactList}\item 
{\bf ofloat} {\bf Obit\-FArray\-RMS0} ({\bf Obit\-FArray} $\ast$in)
\begin{CompactList}\small\item\em Public: RMS of pixel about zero. \item\end{CompactList}\item 
{\bf ofloat} {\bf Obit\-FArray\-RMSQuant} ({\bf Obit\-FArray} $\ast$in)
\begin{CompactList}\small\item\em Public: RMS of pixel in potentially quantized image. \item\end{CompactList}\item 
void {\bf Obit\-FArray\-Quant} ({\bf Obit\-FArray} $\ast$in, {\bf ofloat} $\ast$quant, {\bf ofloat} $\ast$zero)
\begin{CompactList}\small\item\em Public: Determine quantization and offset in an image. \item\end{CompactList}\item 
{\bf ofloat} {\bf Obit\-FArray\-Mode} ({\bf Obit\-FArray} $\ast$in)
\begin{CompactList}\small\item\em Public: Mode of pixel distribution. \item\end{CompactList}\item 
{\bf ofloat} {\bf Obit\-FArray\-Mean} ({\bf Obit\-FArray} $\ast$in)
\begin{CompactList}\small\item\em Public: Mean of pixel distribution. \item\end{CompactList}\item 
void {\bf Obit\-FArray\-Fill} ({\bf Obit\-FArray} $\ast$in, {\bf ofloat} scalar)
\begin{CompactList}\small\item\em Public: fill elements of an FArray. \item\end{CompactList}\item 
void {\bf Obit\-FArray\-Neg} ({\bf Obit\-FArray} $\ast$in)
\begin{CompactList}\small\item\em Public: negate elements of an FArray. \item\end{CompactList}\item 
void {\bf Obit\-FArray\-Sin} ({\bf Obit\-FArray} $\ast$in)
\begin{CompactList}\small\item\em Public: sine of elements of an FArray. \item\end{CompactList}\item 
void {\bf Obit\-FArray\-Cos} ({\bf Obit\-FArray} $\ast$in)
\begin{CompactList}\small\item\em Public: cosine of elements of an FArray. \item\end{CompactList}\item 
void {\bf Obit\-FArray\-Sqrt} ({\bf Obit\-FArray} $\ast$in)
\begin{CompactList}\small\item\em Public: square root of elements of an FArray. \item\end{CompactList}\item 
{\bf ofloat} {\bf Obit\-FArray\-Sum} ({\bf Obit\-FArray} $\ast$in)
\begin{CompactList}\small\item\em Public: sum elements of an FArray. \item\end{CompactList}\item 
{\bf olong} {\bf Obit\-FArray\-Count} ({\bf Obit\-FArray} $\ast$in)
\begin{CompactList}\small\item\em Public: number of valid elements in an FArray. \item\end{CompactList}\item 
void {\bf Obit\-FArray\-SAdd} ({\bf Obit\-FArray} $\ast$in, {\bf ofloat} scalar)
\begin{CompactList}\small\item\em Public: Add a scalar to elements of an FArray. \item\end{CompactList}\item 
void {\bf Obit\-FArray\-SMul} ({\bf Obit\-FArray} $\ast$in, {\bf ofloat} scalar)
\begin{CompactList}\small\item\em Public: Multiply elements of an FArray by a scalar. \item\end{CompactList}\item 
void {\bf Obit\-FArray\-SDiv} ({\bf Obit\-FArray} $\ast$in, {\bf ofloat} scalar)
\begin{CompactList}\small\item\em Public: Divide elements of an FArray into a scalar. \item\end{CompactList}\item 
void {\bf Obit\-FArray\-Clip} ({\bf Obit\-FArray} $\ast$in, {\bf ofloat} min\-Val, {\bf ofloat} max\-Val, {\bf ofloat} new\-Val)
\begin{CompactList}\small\item\em Public: Clip elements of an FArray outside of a given range. \item\end{CompactList}\item 
void {\bf Obit\-FArray\-In\-Clip} ({\bf Obit\-FArray} $\ast$in, {\bf ofloat} min\-Val, {\bf ofloat} max\-Val, {\bf ofloat} new\-Val)
\begin{CompactList}\small\item\em Public: Clip elements of an FArray inside of a given range. \item\end{CompactList}\item 
void {\bf Obit\-FArray\-Blank} ({\bf Obit\-FArray} $\ast$in1, {\bf Obit\-FArray} $\ast$in2, {\bf Obit\-FArray} $\ast$out)
\begin{CompactList}\small\item\em Public: Blank elements of an array where another is blanked. \item\end{CompactList}\item 
void {\bf Obit\-FArray\-Max\-Arr} ({\bf Obit\-FArray} $\ast$in1, {\bf Obit\-FArray} $\ast$in2, {\bf Obit\-FArray} $\ast$out)
\begin{CompactList}\small\item\em Public: Get larger elements of two FArrays. \item\end{CompactList}\item 
void {\bf Obit\-FArray\-Min\-Arr} ({\bf Obit\-FArray} $\ast$in1, {\bf Obit\-FArray} $\ast$in2, {\bf Obit\-FArray} $\ast$out)
\begin{CompactList}\small\item\em Public: Get lesser elements of two FArrays. \item\end{CompactList}\item 
void {\bf Obit\-FArray\-Sum\-Arr} ({\bf Obit\-FArray} $\ast$in1, {\bf Obit\-FArray} $\ast$in2, {\bf Obit\-FArray} $\ast$out)
\begin{CompactList}\small\item\em Public: Sum nonblanked elements of two FArrays. \item\end{CompactList}\item 
void {\bf Obit\-FArray\-Avg\-Arr} ({\bf Obit\-FArray} $\ast$in1, {\bf Obit\-FArray} $\ast$in2, {\bf Obit\-FArray} $\ast$out)
\begin{CompactList}\small\item\em Public: Average nonblanked elements of two FArrays. \item\end{CompactList}\item 
void {\bf Obit\-FArray\-Add} ({\bf Obit\-FArray} $\ast$in1, {\bf Obit\-FArray} $\ast$in2, {\bf Obit\-FArray} $\ast$out)
\begin{CompactList}\small\item\em Public: Add elements of two FArrays. \item\end{CompactList}\item 
void {\bf Obit\-FArray\-Sub} ({\bf Obit\-FArray} $\ast$in1, {\bf Obit\-FArray} $\ast$in2, {\bf Obit\-FArray} $\ast$out)
\begin{CompactList}\small\item\em Public: Subtract elements of two FArrays. \item\end{CompactList}\item 
void {\bf Obit\-FArray\-Mul} ({\bf Obit\-FArray} $\ast$in1, {\bf Obit\-FArray} $\ast$in2, {\bf Obit\-FArray} $\ast$out)
\begin{CompactList}\small\item\em Public: Multiply elements of two FArrays. \item\end{CompactList}\item 
void {\bf Obit\-FArray\-Div} ({\bf Obit\-FArray} $\ast$in1, {\bf Obit\-FArray} $\ast$in2, {\bf Obit\-FArray} $\ast$out)
\begin{CompactList}\small\item\em Public: Divide elements of two FArrays. \item\end{CompactList}\item 
void {\bf Obit\-FArray\-Div\-Clip} ({\bf Obit\-FArray} $\ast$in1, {\bf Obit\-FArray} $\ast$in2, {\bf ofloat} min\-Val, {\bf Obit\-FArray} $\ast$out)
\begin{CompactList}\small\item\em Public: Divide elements of two FArrays with clipping. \item\end{CompactList}\item 
{\bf ofloat} {\bf Obit\-FArray\-Dot} ({\bf Obit\-FArray} $\ast$in1, {\bf Obit\-FArray} $\ast$in2)
\begin{CompactList}\small\item\em Public: \char`\"{}Dot\char`\"{} product to two arrays. \item\end{CompactList}\item 
void {\bf Obit\-FArray\-Mul\-Col\-Row} ({\bf Obit\-FArray} $\ast$in, {\bf Obit\-FArray} $\ast$row, {\bf Obit\-FArray} $\ast$col, {\bf Obit\-FArray} $\ast$out)
\begin{CompactList}\small\item\em Public: Multiply a 2D array by a Col vector $\ast$ Row vector. \item\end{CompactList}\item 
void {\bf Obit\-FArray1DCenter} ({\bf Obit\-FArray} $\ast$in)
\begin{CompactList}\small\item\em Public: Convert a 1D \char`\"{}center at edges\char`\"{} array to proper order. \item\end{CompactList}\item 
void {\bf Obit\-FArray2DCenter} ({\bf Obit\-FArray} $\ast$in)
\begin{CompactList}\small\item\em Public: Convert a 2D \char`\"{}center at edges\char`\"{} array to proper order. \item\end{CompactList}\item 
void {\bf Obit\-FArray2DSym\-Inv} ({\bf Obit\-FArray} $\ast$in, {\bf olong} $\ast$ierr)
\begin{CompactList}\small\item\em Public: inplace invert a symmetric 2D array. \item\end{CompactList}\item 
void {\bf Obit\-FArray2DCGauss} ({\bf Obit\-FArray} $\ast$array, {\bf olong} Cen[2], {\bf ofloat} FWHM)
\begin{CompactList}\small\item\em Public: Make 2-D Circular Gaussian in FArray. \item\end{CompactList}\item 
void {\bf Obit\-FArray2DEGauss} ({\bf Obit\-FArray} $\ast$array, {\bf ofloat} amp, {\bf ofloat} Cen[2], {\bf ofloat} Gau\-Mod[3])
\begin{CompactList}\small\item\em Public: Make 2-D Eliptical Gaussian in FArray. \item\end{CompactList}\item 
void {\bf Obit\-FArray\-Shift\-Add} ({\bf Obit\-FArray} $\ast$in1, {\bf olong} $\ast$pos1, {\bf Obit\-FArray} $\ast$in2, {\bf olong} $\ast$pos2, {\bf ofloat} scalar, {\bf Obit\-FArray} $\ast$out)
\begin{CompactList}\small\item\em Public: Shift and Add scaled array. \item\end{CompactList}\item 
void {\bf Obit\-FArray\-Pad} ({\bf Obit\-FArray} $\ast$in, {\bf Obit\-FArray} $\ast$out, {\bf ofloat} factor)
\begin{CompactList}\small\item\em Public: Zero pad an array. \item\end{CompactList}\item 
void {\bf Obit\-FArray\-Conv\-Gaus} ({\bf Obit\-FArray} $\ast$in, {\bf Obit\-FArray} $\ast$list, {\bf olong} ncomp, {\bf ofloat} gauss[3])
\begin{CompactList}\small\item\em Public: Convolve a list of Gaussians onto an FArray. \item\end{CompactList}\item 
void {\bf Obit\-FArray\-Sel\-Inc} ({\bf Obit\-FArray} $\ast$in, {\bf Obit\-FArray} $\ast$out, {\bf olong} $\ast$blc, {\bf olong} $\ast$trc, {\bf olong} $\ast$inc, {\bf Obit\-Err} $\ast$err)
\begin{CompactList}\small\item\em Public: Select elements in an FArray by increment. \item\end{CompactList}\item 
void {\bf Obit\-FArray\-Class\-Init} (void)
\begin{CompactList}\small\item\em Public: Class initializer. \item\end{CompactList}\end{CompactItemize}


\subsection{Detailed Description}
{\bf Obit\-FArray}{\rm (p.\,\pageref{structObitFArray})} class function definitions. 

This class is derived from the {\bf Obit}{\rm (p.\,\pageref{structObit})} base class.

\subsection{Define Documentation}
\index{ObitFArray.c@{Obit\-FArray.c}!MAXSYMINVSIZE@{MAXSYMINVSIZE}}
\index{MAXSYMINVSIZE@{MAXSYMINVSIZE}!ObitFArray.c@{Obit\-FArray.c}}
\subsubsection{\setlength{\rightskip}{0pt plus 5cm}\#define MAXSYMINVSIZE\ 50}\label{ObitFArray_8c_a0}




\subsection{Function Documentation}
\index{ObitFArray.c@{Obit\-FArray.c}!newObitFArray@{newObitFArray}}
\index{newObitFArray@{newObitFArray}!ObitFArray.c@{Obit\-FArray.c}}
\subsubsection{\setlength{\rightskip}{0pt plus 5cm}{\bf Obit\-FArray}$\ast$ new\-Obit\-FArray (gchar $\ast$ {\em name})}\label{ObitFArray_8c_a14}


Public: Default Constructor. 

Initializes class if needed on first call. \begin{Desc}
\item[Parameters:]
\begin{description}
\item[{\em name}]An optional name for the object. \end{description}
\end{Desc}
\begin{Desc}
\item[Returns:]the new object. \end{Desc}
\index{ObitFArray.c@{Obit\-FArray.c}!ObitFArray1DCenter@{ObitFArray1DCenter}}
\index{ObitFArray1DCenter@{ObitFArray1DCenter}!ObitFArray.c@{Obit\-FArray.c}}
\subsubsection{\setlength{\rightskip}{0pt plus 5cm}void Obit\-FArray1DCenter ({\bf Obit\-FArray} $\ast$ {\em in})}\label{ObitFArray_8c_a59}


Public: Convert a 1D \char`\"{}center at edges\char`\"{} array to proper order. 

This is needed for the peculiar order of FFTs. FFTs don't like blanked values. \begin{Desc}
\item[Parameters:]
\begin{description}
\item[{\em in}]1D array to reorder \end{description}
\end{Desc}
\index{ObitFArray.c@{Obit\-FArray.c}!ObitFArray2DCenter@{ObitFArray2DCenter}}
\index{ObitFArray2DCenter@{ObitFArray2DCenter}!ObitFArray.c@{Obit\-FArray.c}}
\subsubsection{\setlength{\rightskip}{0pt plus 5cm}void Obit\-FArray2DCenter ({\bf Obit\-FArray} $\ast$ {\em in})}\label{ObitFArray_8c_a60}


Public: Convert a 2D \char`\"{}center at edges\char`\"{} array to proper order. 

This is needed for the peculiar order of FFTs. FFTs don't like blanked values. \begin{Desc}
\item[Parameters:]
\begin{description}
\item[{\em in}]2D array to reorder \end{description}
\end{Desc}
\index{ObitFArray.c@{Obit\-FArray.c}!ObitFArray2DCGauss@{ObitFArray2DCGauss}}
\index{ObitFArray2DCGauss@{ObitFArray2DCGauss}!ObitFArray.c@{Obit\-FArray.c}}
\subsubsection{\setlength{\rightskip}{0pt plus 5cm}void Obit\-FArray2DCGauss ({\bf Obit\-FArray} $\ast$ {\em array}, {\bf olong} {\em Cen}[2], {\bf ofloat} {\em FWHM})}\label{ObitFArray_8c_a62}


Public: Make 2-D Circular Gaussian in FArray. 

\begin{Desc}
\item[Parameters:]
\begin{description}
\item[{\em array}]Array to fill in \item[{\em Cen}]0-rel center pixel \item[{\em Size}]FWHM of Gaussian in pixels. \end{description}
\end{Desc}
\index{ObitFArray.c@{Obit\-FArray.c}!ObitFArray2DEGauss@{ObitFArray2DEGauss}}
\index{ObitFArray2DEGauss@{ObitFArray2DEGauss}!ObitFArray.c@{Obit\-FArray.c}}
\subsubsection{\setlength{\rightskip}{0pt plus 5cm}void Obit\-FArray2DEGauss ({\bf Obit\-FArray} $\ast$ {\em array}, {\bf ofloat} {\em amp}, {\bf ofloat} {\em Cen}[2], {\bf ofloat} {\em Gau\-Mod}[3])}\label{ObitFArray_8c_a63}


Public: Make 2-D Eliptical Gaussian in FArray. 

\begin{Desc}
\item[Parameters:]
\begin{description}
\item[{\em array}]Array to fill in \item[{\em amp}]Peak value of Gaussian \item[{\em Cen}]0-rel center pixel \item[{\em Gau\-Mod}]Gaussian parameters, Major axis, FWHM, minor axis FWHM (both in pixels) and rotation angle wrt \char`\"{}X\char`\"{} axis (deg). \end{description}
\end{Desc}
\index{ObitFArray.c@{Obit\-FArray.c}!ObitFArray2DSymInv@{ObitFArray2DSymInv}}
\index{ObitFArray2DSymInv@{ObitFArray2DSymInv}!ObitFArray.c@{Obit\-FArray.c}}
\subsubsection{\setlength{\rightskip}{0pt plus 5cm}void Obit\-FArray2DSym\-Inv ({\bf Obit\-FArray} $\ast$ {\em in}, {\bf olong} $\ast$ {\em ierr})}\label{ObitFArray_8c_a61}


Public: inplace invert a symmetric 2D array. 

\begin{Desc}
\item[Parameters:]
\begin{description}
\item[{\em in}]2D array to invers (max dim 50x50) \item[{\em ierr}]return code, 0=$>$OK, else could not invert. \end{description}
\end{Desc}
\index{ObitFArray.c@{Obit\-FArray.c}!ObitFArrayAdd@{ObitFArrayAdd}}
\index{ObitFArrayAdd@{ObitFArrayAdd}!ObitFArray.c@{Obit\-FArray.c}}
\subsubsection{\setlength{\rightskip}{0pt plus 5cm}void Obit\-FArray\-Add ({\bf Obit\-FArray} $\ast$ {\em in1}, {\bf Obit\-FArray} $\ast$ {\em in2}, {\bf Obit\-FArray} $\ast$ {\em out})}\label{ObitFArray_8c_a52}


Public: Add elements of two FArrays. 

out = in1 + in2, if either is blanked the result is blanked \begin{Desc}
\item[Parameters:]
\begin{description}
\item[{\em in1}]Input object with data \item[{\em in2}]Input object with data \item[{\em out}]Output array (may be an input array). \end{description}
\end{Desc}
\index{ObitFArray.c@{Obit\-FArray.c}!ObitFArrayAvgArr@{ObitFArrayAvgArr}}
\index{ObitFArrayAvgArr@{ObitFArrayAvgArr}!ObitFArray.c@{Obit\-FArray.c}}
\subsubsection{\setlength{\rightskip}{0pt plus 5cm}void Obit\-FArray\-Avg\-Arr ({\bf Obit\-FArray} $\ast$ {\em in1}, {\bf Obit\-FArray} $\ast$ {\em in2}, {\bf Obit\-FArray} $\ast$ {\em out})}\label{ObitFArray_8c_a51}


Public: Average nonblanked elements of two FArrays. 

out = (in1 + in2)/2 or whichever is not blanked \begin{Desc}
\item[Parameters:]
\begin{description}
\item[{\em in1}]Input object with data \item[{\em in2}]Input object with data \item[{\em out}]Output array (may be an input array). \end{description}
\end{Desc}
\index{ObitFArray.c@{Obit\-FArray.c}!ObitFArrayBlank@{ObitFArrayBlank}}
\index{ObitFArrayBlank@{ObitFArrayBlank}!ObitFArray.c@{Obit\-FArray.c}}
\subsubsection{\setlength{\rightskip}{0pt plus 5cm}void Obit\-FArray\-Blank ({\bf Obit\-FArray} $\ast$ {\em in1}, {\bf Obit\-FArray} $\ast$ {\em in2}, {\bf Obit\-FArray} $\ast$ {\em out})}\label{ObitFArray_8c_a47}


Public: Blank elements of an array where another is blanked. 

\begin{Desc}
\item[Parameters:]
\begin{description}
\item[{\em in1}]Input object with data \item[{\em in2}]Input object with blanking \item[{\em out}]Output array (may be an input array). \end{description}
\end{Desc}
\index{ObitFArray.c@{Obit\-FArray.c}!ObitFArrayClassInit@{ObitFArrayClassInit}}
\index{ObitFArrayClassInit@{ObitFArrayClassInit}!ObitFArray.c@{Obit\-FArray.c}}
\subsubsection{\setlength{\rightskip}{0pt plus 5cm}void Obit\-FArray\-Class\-Init (void)}\label{ObitFArray_8c_a68}


Public: Class initializer. 

\index{ObitFArray.c@{Obit\-FArray.c}!ObitFArrayClear@{ObitFArrayClear}}
\index{ObitFArrayClear@{ObitFArrayClear}!ObitFArray.c@{Obit\-FArray.c}}
\subsubsection{\setlength{\rightskip}{0pt plus 5cm}void Obit\-FArray\-Clear (gpointer {\em inn})}\label{ObitFArray_8c_a5}


Private: Deallocate members. 

Does (recursive) deallocation of parent class members. For some reason this wasn't build into the GType class. \begin{Desc}
\item[Parameters:]
\begin{description}
\item[{\em inn}]Pointer to the object to deallocate. Actually it should be an Obit\-FArray$\ast$ cast to an Obit$\ast$. \end{description}
\end{Desc}
\index{ObitFArray.c@{Obit\-FArray.c}!ObitFArrayClip@{ObitFArrayClip}}
\index{ObitFArrayClip@{ObitFArrayClip}!ObitFArray.c@{Obit\-FArray.c}}
\subsubsection{\setlength{\rightskip}{0pt plus 5cm}void Obit\-FArray\-Clip ({\bf Obit\-FArray} $\ast$ {\em in}, {\bf ofloat} {\em min\-Val}, {\bf ofloat} {\em max\-Val}, {\bf ofloat} {\em new\-Val})}\label{ObitFArray_8c_a45}


Public: Clip elements of an FArray outside of a given range. 

\begin{Desc}
\item[Parameters:]
\begin{description}
\item[{\em in}]Input object with data \item[{\em min\-Val}]Minimum allowed value \item[{\em max\-Val}]Maximum allowed value \item[{\em new\-Val}]Value to use if out of range. \end{description}
\end{Desc}
\index{ObitFArray.c@{Obit\-FArray.c}!ObitFArrayClone@{ObitFArrayClone}}
\index{ObitFArrayClone@{ObitFArrayClone}!ObitFArray.c@{Obit\-FArray.c}}
\subsubsection{\setlength{\rightskip}{0pt plus 5cm}void Obit\-FArray\-Clone ({\bf Obit\-FArray} $\ast$ {\em in}, {\bf Obit\-FArray} $\ast$ {\em out}, {\bf Obit\-Err} $\ast$ {\em err})}\label{ObitFArray_8c_a17}


Public: Copy structure. 

\begin{Desc}
\item[Parameters:]
\begin{description}
\item[{\em in}]The object to copy \item[{\em out}]An existing object pointer for output, must be defined. \item[{\em err}]{\bf Obit}{\rm (p.\,\pageref{structObit})} error stack object. \end{description}
\end{Desc}
\index{ObitFArray.c@{Obit\-FArray.c}!ObitFArrayConvGaus@{ObitFArrayConvGaus}}
\index{ObitFArrayConvGaus@{ObitFArrayConvGaus}!ObitFArray.c@{Obit\-FArray.c}}
\subsubsection{\setlength{\rightskip}{0pt plus 5cm}void Obit\-FArray\-Conv\-Gaus ({\bf Obit\-FArray} $\ast$ {\em in}, {\bf Obit\-FArray} $\ast$ {\em list}, {\bf olong} {\em ncomp}, {\bf ofloat} {\em gauss}[3])}\label{ObitFArray_8c_a66}


Public: Convolve a list of Gaussians onto an FArray. 

\begin{Desc}
\item[Parameters:]
\begin{description}
\item[{\em in}]2-D array to add Gaussians to. \item[{\em list}]List of positions and fluxes of the Gaussians (x pixel, y pixel, flux) \item[{\em ncomp}]Number of components in list (generally less than size of FArray). \item[{\em gauss}]Gaussian coefficients for (d\_\-x$\ast$d\_\-x, d\_\-y$\ast$d\_\-y, d\_\-x$\ast$d\_\-y) Gaussian maj = major axis FWHM, min=minor, pa = posn. angle cr=cos(pa+rotation), sr=sin(pa+rotation), cell\_\-x, cell\_\-y x, y cell spacing is same units as maj, min [0] = \{(cr/min)$^\wedge$2 + ((sr/maj)$^\wedge$2)\}$\ast$(cell\_\-x$^\wedge$2)$\ast$4$\ast$log(2) [1] = \{(sr/min)$^\wedge$2 + ((cr/maj)$^\wedge$2)\}$\ast$(cell\_\-y$^\wedge$2)$\ast$4$\ast$log(2) [2] = \{(1/min)$^\wedge$2 - ((1/maj)$^\wedge$2)\}$\ast$sr$\ast$cr$\ast$abs(cell\_\-x$\ast$cell\_\-y)$\ast$8$\ast$log(2) \end{description}
\end{Desc}
\index{ObitFArray.c@{Obit\-FArray.c}!ObitFArrayCopy@{ObitFArrayCopy}}
\index{ObitFArrayCopy@{ObitFArrayCopy}!ObitFArray.c@{Obit\-FArray.c}}
\subsubsection{\setlength{\rightskip}{0pt plus 5cm}{\bf Obit\-FArray}$\ast$ Obit\-FArray\-Copy ({\bf Obit\-FArray} $\ast$ {\em in}, {\bf Obit\-FArray} $\ast$ {\em out}, {\bf Obit\-Err} $\ast$ {\em err})}\label{ObitFArray_8c_a16}


Public: Copy (deep) constructor. 

\begin{Desc}
\item[Parameters:]
\begin{description}
\item[{\em in}]The object to copy \item[{\em out}]An existing object pointer for output or NULL if none exists. \item[{\em err}]{\bf Obit}{\rm (p.\,\pageref{structObit})} error stack object. \end{description}
\end{Desc}
\begin{Desc}
\item[Returns:]pointer to the new object. \end{Desc}
\index{ObitFArray.c@{Obit\-FArray.c}!ObitFArrayCos@{ObitFArrayCos}}
\index{ObitFArrayCos@{ObitFArrayCos}!ObitFArray.c@{Obit\-FArray.c}}
\subsubsection{\setlength{\rightskip}{0pt plus 5cm}void Obit\-FArray\-Cos ({\bf Obit\-FArray} $\ast$ {\em in})}\label{ObitFArray_8c_a38}


Public: cosine of elements of an FArray. 

in = cos(in). \begin{Desc}
\item[Parameters:]
\begin{description}
\item[{\em in}]Input object with data \end{description}
\end{Desc}
\index{ObitFArray.c@{Obit\-FArray.c}!ObitFArrayCount@{ObitFArrayCount}}
\index{ObitFArrayCount@{ObitFArrayCount}!ObitFArray.c@{Obit\-FArray.c}}
\subsubsection{\setlength{\rightskip}{0pt plus 5cm}{\bf olong} Obit\-FArray\-Count ({\bf Obit\-FArray} $\ast$ {\em in})}\label{ObitFArray_8c_a41}


Public: number of valid elements in an FArray. 

\begin{Desc}
\item[Parameters:]
\begin{description}
\item[{\em in}]Input object with data \end{description}
\end{Desc}
\begin{Desc}
\item[Returns:]count of valid elements \end{Desc}
\index{ObitFArray.c@{Obit\-FArray.c}!ObitFArrayCreate@{ObitFArrayCreate}}
\index{ObitFArrayCreate@{ObitFArrayCreate}!ObitFArray.c@{Obit\-FArray.c}}
\subsubsection{\setlength{\rightskip}{0pt plus 5cm}{\bf Obit\-FArray}$\ast$ Obit\-FArray\-Create (gchar $\ast$ {\em name}, {\bf olong} {\em ndim}, {\bf olong} $\ast$ {\em naxis})}\label{ObitFArray_8c_a19}


Public: Create/initialize {\bf Obit\-FArray}{\rm (p.\,\pageref{structObitFArray})} structures. 

\begin{Desc}
\item[Parameters:]
\begin{description}
\item[{\em name}]An optional name for the object. \item[{\em ndim}]Number of dimensions desired, if $<$=0 data array not created. maximum value = MAXFARRAYDIM. \item[{\em naxis}]Dimensionality along each axis. NULL =$>$ don't create array. \end{description}
\end{Desc}
\begin{Desc}
\item[Returns:]the new object. \end{Desc}
\index{ObitFArray.c@{Obit\-FArray.c}!ObitFArrayDeblank@{ObitFArrayDeblank}}
\index{ObitFArrayDeblank@{ObitFArrayDeblank}!ObitFArray.c@{Obit\-FArray.c}}
\subsubsection{\setlength{\rightskip}{0pt plus 5cm}void Obit\-FArray\-Deblank ({\bf Obit\-FArray} $\ast$ {\em in}, {\bf ofloat} {\em scalar})}\label{ObitFArray_8c_a27}


Public: Replace blanks in an {\bf Obit\-FArray}{\rm (p.\,\pageref{structObitFArray})}. 

\begin{Desc}
\item[Parameters:]
\begin{description}
\item[{\em in}]Object with data to deblank \item[{\em scalar}]Value to replace blanks. \end{description}
\end{Desc}
\index{ObitFArray.c@{Obit\-FArray.c}!ObitFArrayDiv@{ObitFArrayDiv}}
\index{ObitFArrayDiv@{ObitFArrayDiv}!ObitFArray.c@{Obit\-FArray.c}}
\subsubsection{\setlength{\rightskip}{0pt plus 5cm}void Obit\-FArray\-Div ({\bf Obit\-FArray} $\ast$ {\em in1}, {\bf Obit\-FArray} $\ast$ {\em in2}, {\bf Obit\-FArray} $\ast$ {\em out})}\label{ObitFArray_8c_a55}


Public: Divide elements of two FArrays. 

out = in1 / in2 \begin{Desc}
\item[Parameters:]
\begin{description}
\item[{\em in1}]Input object with data \item[{\em in2}]Input object with data \item[{\em out}]Output array (may be an input array). \end{description}
\end{Desc}
\index{ObitFArray.c@{Obit\-FArray.c}!ObitFArrayDivClip@{ObitFArrayDivClip}}
\index{ObitFArrayDivClip@{ObitFArrayDivClip}!ObitFArray.c@{Obit\-FArray.c}}
\subsubsection{\setlength{\rightskip}{0pt plus 5cm}void Obit\-FArray\-Div\-Clip ({\bf Obit\-FArray} $\ast$ {\em in1}, {\bf Obit\-FArray} $\ast$ {\em in2}, {\bf ofloat} {\em min\-Val}, {\bf Obit\-FArray} $\ast$ {\em out})}\label{ObitFArray_8c_a56}


Public: Divide elements of two FArrays with clipping. 

out = in1 / in2 where in2$>$min\-Val, else blanked \begin{Desc}
\item[Parameters:]
\begin{description}
\item[{\em in1}]Input object with data \item[{\em in2}]Input object with data \item[{\em min\-Val}]minimum allowed value for in2 \item[{\em out}]Output array (may be an input array). \end{description}
\end{Desc}
\index{ObitFArray.c@{Obit\-FArray.c}!ObitFArrayDot@{ObitFArrayDot}}
\index{ObitFArrayDot@{ObitFArrayDot}!ObitFArray.c@{Obit\-FArray.c}}
\subsubsection{\setlength{\rightskip}{0pt plus 5cm}{\bf ofloat} Obit\-FArray\-Dot ({\bf Obit\-FArray} $\ast$ {\em in1}, {\bf Obit\-FArray} $\ast$ {\em in2})}\label{ObitFArray_8c_a57}


Public: \char`\"{}Dot\char`\"{} product to two arrays. 

\begin{Desc}
\item[Parameters:]
\begin{description}
\item[{\em in1}]Input object with data \item[{\em in2}]Input object with data \end{description}
\end{Desc}
\begin{Desc}
\item[Returns:]sum of product of elements \end{Desc}
\index{ObitFArray.c@{Obit\-FArray.c}!ObitFArrayFill@{ObitFArrayFill}}
\index{ObitFArrayFill@{ObitFArrayFill}!ObitFArray.c@{Obit\-FArray.c}}
\subsubsection{\setlength{\rightskip}{0pt plus 5cm}void Obit\-FArray\-Fill ({\bf Obit\-FArray} $\ast$ {\em in}, {\bf ofloat} {\em scalar})}\label{ObitFArray_8c_a35}


Public: fill elements of an FArray. 

in = scalar. \begin{Desc}
\item[Parameters:]
\begin{description}
\item[{\em in}]Input object with data \item[{\em scalar}]Scalar value \end{description}
\end{Desc}
\index{ObitFArray.c@{Obit\-FArray.c}!ObitFArrayGetClass@{ObitFArrayGetClass}}
\index{ObitFArrayGetClass@{ObitFArrayGetClass}!ObitFArray.c@{Obit\-FArray.c}}
\subsubsection{\setlength{\rightskip}{0pt plus 5cm}gconstpointer Obit\-FArray\-Get\-Class (void)}\label{ObitFArray_8c_a15}


Public: Class\-Info pointer. 

\begin{Desc}
\item[Returns:]pointer to the class structure. \end{Desc}
\index{ObitFArray.c@{Obit\-FArray.c}!ObitFArrayInClip@{ObitFArrayInClip}}
\index{ObitFArrayInClip@{ObitFArrayInClip}!ObitFArray.c@{Obit\-FArray.c}}
\subsubsection{\setlength{\rightskip}{0pt plus 5cm}void Obit\-FArray\-In\-Clip ({\bf Obit\-FArray} $\ast$ {\em in}, {\bf ofloat} {\em min\-Val}, {\bf ofloat} {\em max\-Val}, {\bf ofloat} {\em new\-Val})}\label{ObitFArray_8c_a46}


Public: Clip elements of an FArray inside of a given range. 

\begin{Desc}
\item[Parameters:]
\begin{description}
\item[{\em in}]Input object with data \item[{\em min\-Val}]Minimum allowed value \item[{\em max\-Val}]Maximum allowed value \item[{\em new\-Val}]Value to use if out of range. \end{description}
\end{Desc}
\index{ObitFArray.c@{Obit\-FArray.c}!ObitFArrayIndex@{ObitFArrayIndex}}
\index{ObitFArrayIndex@{ObitFArrayIndex}!ObitFArray.c@{Obit\-FArray.c}}
\subsubsection{\setlength{\rightskip}{0pt plus 5cm}{\bf ofloat}$\ast$ Obit\-FArray\-Index ({\bf Obit\-FArray} $\ast$ {\em in}, {\bf olong} $\ast$ {\em pos})}\label{ObitFArray_8c_a23}


Public: return pointer to a specified element. 

Subsequent data are stored in order of increasing dimension (rows, then columns...). \begin{Desc}
\item[Parameters:]
\begin{description}
\item[{\em in}]Object with data \item[{\em pos}]array of 0-rel pixel numbers on each axis \end{description}
\end{Desc}
\begin{Desc}
\item[Returns:]pointer to specified cell; NULL if illegal pixel. \end{Desc}
\index{ObitFArray.c@{Obit\-FArray.c}!ObitFArrayInit@{ObitFArrayInit}}
\index{ObitFArrayInit@{ObitFArrayInit}!ObitFArray.c@{Obit\-FArray.c}}
\subsubsection{\setlength{\rightskip}{0pt plus 5cm}void Obit\-FArray\-Init (gpointer {\em inn})}\label{ObitFArray_8c_a4}


Private: Initialize newly instantiated object. 

Parent classes portions are (recursively) initialized first \begin{Desc}
\item[Parameters:]
\begin{description}
\item[{\em inn}]Pointer to the object to initialize. \end{description}
\end{Desc}
\index{ObitFArray.c@{Obit\-FArray.c}!ObitFArrayIsCompatable@{ObitFArrayIsCompatable}}
\index{ObitFArrayIsCompatable@{ObitFArrayIsCompatable}!ObitFArray.c@{Obit\-FArray.c}}
\subsubsection{\setlength{\rightskip}{0pt plus 5cm}gboolean Obit\-FArray\-Is\-Compatable ({\bf Obit\-FArray} $\ast$ {\em in1}, {\bf Obit\-FArray} $\ast$ {\em in2})}\label{ObitFArray_8c_a18}


Public: Are two FArrays of compatable geometry. 

Must have same number of non degenerate dimensions and each dimension must be the same size. \begin{Desc}
\item[Parameters:]
\begin{description}
\item[{\em in1}]First object to test. \item[{\em in2}]Second object to test. \end{description}
\end{Desc}
\begin{Desc}
\item[Returns:]TRUE if compatable, else FALSE. \end{Desc}
\index{ObitFArray.c@{Obit\-FArray.c}!ObitFArrayMax@{ObitFArrayMax}}
\index{ObitFArrayMax@{ObitFArrayMax}!ObitFArray.c@{Obit\-FArray.c}}
\subsubsection{\setlength{\rightskip}{0pt plus 5cm}{\bf ofloat} Obit\-FArray\-Max ({\bf Obit\-FArray} $\ast$ {\em in}, {\bf olong} $\ast$ {\em pos})}\label{ObitFArray_8c_a24}


Public: Find Maximum value in an {\bf Obit\-FArray}{\rm (p.\,\pageref{structObitFArray})}. 

Return value and location in pos. \begin{Desc}
\item[Parameters:]
\begin{description}
\item[{\em in}]Object with data \item[{\em pos}](out) array of 0-rel pixel numbers on each axis \end{description}
\end{Desc}
\begin{Desc}
\item[Returns:]maximum value. \end{Desc}
\index{ObitFArray.c@{Obit\-FArray.c}!ObitFArrayMaxAbs@{ObitFArrayMaxAbs}}
\index{ObitFArrayMaxAbs@{ObitFArrayMaxAbs}!ObitFArray.c@{Obit\-FArray.c}}
\subsubsection{\setlength{\rightskip}{0pt plus 5cm}{\bf ofloat} Obit\-FArray\-Max\-Abs ({\bf Obit\-FArray} $\ast$ {\em in}, {\bf olong} $\ast$ {\em pos})}\label{ObitFArray_8c_a25}


Public: Find Maximum abs value in an {\bf Obit\-FArray}{\rm (p.\,\pageref{structObitFArray})}. 

Return value and location in pos. \begin{Desc}
\item[Parameters:]
\begin{description}
\item[{\em in}]Object with data \item[{\em pos}](out) array of 0-rel pixel numbers on each axis \end{description}
\end{Desc}
\begin{Desc}
\item[Returns:]maximum absolute (signed) value. \end{Desc}
\index{ObitFArray.c@{Obit\-FArray.c}!ObitFArrayMaxArr@{ObitFArrayMaxArr}}
\index{ObitFArrayMaxArr@{ObitFArrayMaxArr}!ObitFArray.c@{Obit\-FArray.c}}
\subsubsection{\setlength{\rightskip}{0pt plus 5cm}void Obit\-FArray\-Max\-Arr ({\bf Obit\-FArray} $\ast$ {\em in1}, {\bf Obit\-FArray} $\ast$ {\em in2}, {\bf Obit\-FArray} $\ast$ {\em out})}\label{ObitFArray_8c_a48}


Public: Get larger elements of two FArrays. 

out = MAX (in1, in2) or whichever is not blanked \begin{Desc}
\item[Parameters:]
\begin{description}
\item[{\em in1}]Input object with data \item[{\em in2}]Input object with data \item[{\em out}]Output array (may be an input array). \end{description}
\end{Desc}
\index{ObitFArray.c@{Obit\-FArray.c}!ObitFArrayMean@{ObitFArrayMean}}
\index{ObitFArrayMean@{ObitFArrayMean}!ObitFArray.c@{Obit\-FArray.c}}
\subsubsection{\setlength{\rightskip}{0pt plus 5cm}{\bf ofloat} Obit\-FArray\-Mean ({\bf Obit\-FArray} $\ast$ {\em in})}\label{ObitFArray_8c_a34}


Public: Mean of pixel distribution. 

\begin{Desc}
\item[Parameters:]
\begin{description}
\item[{\em in}]Input object with data \end{description}
\end{Desc}
\begin{Desc}
\item[Returns:]mean of distribution \end{Desc}
\index{ObitFArray.c@{Obit\-FArray.c}!ObitFArrayMin@{ObitFArrayMin}}
\index{ObitFArrayMin@{ObitFArrayMin}!ObitFArray.c@{Obit\-FArray.c}}
\subsubsection{\setlength{\rightskip}{0pt plus 5cm}{\bf ofloat} Obit\-FArray\-Min ({\bf Obit\-FArray} $\ast$ {\em in}, {\bf olong} $\ast$ {\em pos})}\label{ObitFArray_8c_a26}


Public: Find Minimum value in an {\bf Obit\-FArray}{\rm (p.\,\pageref{structObitFArray})}. 

Return value and location in pos. \begin{Desc}
\item[Parameters:]
\begin{description}
\item[{\em in}]Object with data \item[{\em pos}](out) array of 0-rel pixel numbers on each axis \end{description}
\end{Desc}
\begin{Desc}
\item[Returns:]minimum value. \end{Desc}
\index{ObitFArray.c@{Obit\-FArray.c}!ObitFArrayMinArr@{ObitFArrayMinArr}}
\index{ObitFArrayMinArr@{ObitFArrayMinArr}!ObitFArray.c@{Obit\-FArray.c}}
\subsubsection{\setlength{\rightskip}{0pt plus 5cm}void Obit\-FArray\-Min\-Arr ({\bf Obit\-FArray} $\ast$ {\em in1}, {\bf Obit\-FArray} $\ast$ {\em in2}, {\bf Obit\-FArray} $\ast$ {\em out})}\label{ObitFArray_8c_a49}


Public: Get lesser elements of two FArrays. 

out = MIN (in1, in2) or whichever is not blanked \begin{Desc}
\item[Parameters:]
\begin{description}
\item[{\em in1}]Input object with data \item[{\em in2}]Input object with data \item[{\em out}]Output array (may be an input array). \end{description}
\end{Desc}
\index{ObitFArray.c@{Obit\-FArray.c}!ObitFArrayMode@{ObitFArrayMode}}
\index{ObitFArrayMode@{ObitFArrayMode}!ObitFArray.c@{Obit\-FArray.c}}
\subsubsection{\setlength{\rightskip}{0pt plus 5cm}{\bf ofloat} Obit\-FArray\-Mode ({\bf Obit\-FArray} $\ast$ {\em in})}\label{ObitFArray_8c_a33}


Public: Mode of pixel distribution. 

Value is based on a histogram analysis and is determined from the peak in the distribution.. out = Mode (in.) \begin{Desc}
\item[Parameters:]
\begin{description}
\item[{\em in}]Input object with data \end{description}
\end{Desc}
\begin{Desc}
\item[Returns:]mode of distribution \end{Desc}
\index{ObitFArray.c@{Obit\-FArray.c}!ObitFArrayMul@{ObitFArrayMul}}
\index{ObitFArrayMul@{ObitFArrayMul}!ObitFArray.c@{Obit\-FArray.c}}
\subsubsection{\setlength{\rightskip}{0pt plus 5cm}void Obit\-FArray\-Mul ({\bf Obit\-FArray} $\ast$ {\em in1}, {\bf Obit\-FArray} $\ast$ {\em in2}, {\bf Obit\-FArray} $\ast$ {\em out})}\label{ObitFArray_8c_a54}


Public: Multiply elements of two FArrays. 

out = in1 $\ast$ in2 \begin{Desc}
\item[Parameters:]
\begin{description}
\item[{\em in1}]Input object with data \item[{\em in2}]Input object with data \item[{\em out}]Output array (may be an input array). \end{description}
\end{Desc}
\index{ObitFArray.c@{Obit\-FArray.c}!ObitFArrayMulColRow@{ObitFArrayMulColRow}}
\index{ObitFArrayMulColRow@{ObitFArrayMulColRow}!ObitFArray.c@{Obit\-FArray.c}}
\subsubsection{\setlength{\rightskip}{0pt plus 5cm}void Obit\-FArray\-Mul\-Col\-Row ({\bf Obit\-FArray} $\ast$ {\em in}, {\bf Obit\-FArray} $\ast$ {\em row}, {\bf Obit\-FArray} $\ast$ {\em col}, {\bf Obit\-FArray} $\ast$ {\em out})}\label{ObitFArray_8c_a58}


Public: Multiply a 2D array by a Col vector $\ast$ Row vector. 

NOTE: this does not check for magic value blanking, this was causing trouble in its major application - image formation - which should not produce blanks. out[i,j] = in[i,j] $\ast$ row[j] $\ast$ col[i]. \begin{Desc}
\item[Parameters:]
\begin{description}
\item[{\em in}]Input 2D array \item[{\em row}]Input row vector \item[{\em col}]Input column \item[{\em out}]Output array (may be an input array). \end{description}
\end{Desc}
\index{ObitFArray.c@{Obit\-FArray.c}!ObitFArrayNeg@{ObitFArrayNeg}}
\index{ObitFArrayNeg@{ObitFArrayNeg}!ObitFArray.c@{Obit\-FArray.c}}
\subsubsection{\setlength{\rightskip}{0pt plus 5cm}void Obit\-FArray\-Neg ({\bf Obit\-FArray} $\ast$ {\em in})}\label{ObitFArray_8c_a36}


Public: negate elements of an FArray. 

in = -in. \begin{Desc}
\item[Parameters:]
\begin{description}
\item[{\em in}]Input object with data \end{description}
\end{Desc}
\index{ObitFArray.c@{Obit\-FArray.c}!ObitFArrayPad@{ObitFArrayPad}}
\index{ObitFArrayPad@{ObitFArrayPad}!ObitFArray.c@{Obit\-FArray.c}}
\subsubsection{\setlength{\rightskip}{0pt plus 5cm}void Obit\-FArray\-Pad ({\bf Obit\-FArray} $\ast$ {\em in}, {\bf Obit\-FArray} $\ast$ {\em out}, {\bf ofloat} {\em factor})}\label{ObitFArray_8c_a65}


Public: Zero pad an array. 

Any blanks in in are replaced with zero. This routine is intended for zero padding images before an FFT to increase the resolution in the uv plane. \begin{Desc}
\item[Parameters:]
\begin{description}
\item[{\em in}]Object with structures to zero pad \item[{\em out}]Output object \item[{\em factor}]scaling factor for in \end{description}
\end{Desc}
\index{ObitFArray.c@{Obit\-FArray.c}!ObitFArrayQuant@{ObitFArrayQuant}}
\index{ObitFArrayQuant@{ObitFArrayQuant}!ObitFArray.c@{Obit\-FArray.c}}
\subsubsection{\setlength{\rightskip}{0pt plus 5cm}void Obit\-FArray\-Quant ({\bf Obit\-FArray} $\ast$ {\em in}, {\bf ofloat} $\ast$ {\em quant}, {\bf ofloat} $\ast$ {\em zero})}\label{ObitFArray_8c_a32}


Public: Determine quantization and offset in an image. 

\begin{Desc}
\item[Parameters:]
\begin{description}
\item[{\em in}]Input object with data \item[{\em quant}][out] quantization level \item[{\em zero}][out] closest level to zero \end{description}
\end{Desc}
\index{ObitFArray.c@{Obit\-FArray.c}!ObitFArrayRawRMS@{ObitFArrayRawRMS}}
\index{ObitFArrayRawRMS@{ObitFArrayRawRMS}!ObitFArray.c@{Obit\-FArray.c}}
\subsubsection{\setlength{\rightskip}{0pt plus 5cm}{\bf ofloat} Obit\-FArray\-Raw\-RMS ({\bf Obit\-FArray} $\ast$ {\em in})}\label{ObitFArray_8c_a29}


Public: RMS of pixel distribution. 

\begin{Desc}
\item[Parameters:]
\begin{description}
\item[{\em in}]Input object with data \end{description}
\end{Desc}
\begin{Desc}
\item[Returns:]rms of element distribution (-1 on error) \end{Desc}
\index{ObitFArray.c@{Obit\-FArray.c}!ObitFArrayRealloc@{ObitFArrayRealloc}}
\index{ObitFArrayRealloc@{ObitFArrayRealloc}!ObitFArray.c@{Obit\-FArray.c}}
\subsubsection{\setlength{\rightskip}{0pt plus 5cm}{\bf Obit\-FArray}$\ast$ Obit\-FArray\-Realloc ({\bf Obit\-FArray} $\ast$ {\em in}, {\bf olong} {\em ndim}, {\bf olong} $\ast$ {\em naxis})}\label{ObitFArray_8c_a22}


Public: Reallocate/initialize {\bf Obit\-FArray}{\rm (p.\,\pageref{structObitFArray})} structures. 

\begin{Desc}
\item[Parameters:]
\begin{description}
\item[{\em in}]Object with structures to reallocate. \item[{\em ndim}]Number of dimensions desired, if $<$=0 data array not created. maximum value = {\bf MAXFARRAYDIM}{\rm (p.\,\pageref{ObitFArray_8h_a3})}. \item[{\em naxis}]Dimensionality along each axis. NULL =$>$ don't create array. \end{description}
\end{Desc}
\begin{Desc}
\item[Returns:]the resized object. \end{Desc}
\index{ObitFArray.c@{Obit\-FArray.c}!ObitFArrayRMS@{ObitFArrayRMS}}
\index{ObitFArrayRMS@{ObitFArrayRMS}!ObitFArray.c@{Obit\-FArray.c}}
\subsubsection{\setlength{\rightskip}{0pt plus 5cm}{\bf ofloat} Obit\-FArray\-RMS ({\bf Obit\-FArray} $\ast$ {\em in})}\label{ObitFArray_8c_a28}


Public: RMS of pixel distribution from histogram. 

Value is based on a histogram analysis and is determined from the width of the peak around the mode. out = RMS (in.) \begin{Desc}
\item[Parameters:]
\begin{description}
\item[{\em in}]Input object with data \end{description}
\end{Desc}
\begin{Desc}
\item[Returns:]rms of element distribution (-1 on error) \end{Desc}
\index{ObitFArray.c@{Obit\-FArray.c}!ObitFArrayRMS0@{ObitFArrayRMS0}}
\index{ObitFArrayRMS0@{ObitFArrayRMS0}!ObitFArray.c@{Obit\-FArray.c}}
\subsubsection{\setlength{\rightskip}{0pt plus 5cm}{\bf ofloat} Obit\-FArray\-RMS0 ({\bf Obit\-FArray} $\ast$ {\em in})}\label{ObitFArray_8c_a30}


Public: RMS of pixel about zero. 

\begin{Desc}
\item[Parameters:]
\begin{description}
\item[{\em in}]Input object with data \end{description}
\end{Desc}
\begin{Desc}
\item[Returns:]rms of element distribution (-1 on error) \end{Desc}
\index{ObitFArray.c@{Obit\-FArray.c}!ObitFArrayRMSQuant@{ObitFArrayRMSQuant}}
\index{ObitFArrayRMSQuant@{ObitFArrayRMSQuant}!ObitFArray.c@{Obit\-FArray.c}}
\subsubsection{\setlength{\rightskip}{0pt plus 5cm}{\bf ofloat} Obit\-FArray\-RMSQuant ({\bf Obit\-FArray} $\ast$ {\em in})}\label{ObitFArray_8c_a31}


Public: RMS of pixel in potentially quantized image. 

\begin{Desc}
\item[Parameters:]
\begin{description}
\item[{\em in}]Input object with data \end{description}
\end{Desc}
\begin{Desc}
\item[Returns:]rms of element distribution (-1 on error) \end{Desc}
\index{ObitFArray.c@{Obit\-FArray.c}!ObitFArraySAdd@{ObitFArraySAdd}}
\index{ObitFArraySAdd@{ObitFArraySAdd}!ObitFArray.c@{Obit\-FArray.c}}
\subsubsection{\setlength{\rightskip}{0pt plus 5cm}void Obit\-FArray\-SAdd ({\bf Obit\-FArray} $\ast$ {\em in}, {\bf ofloat} {\em scalar})}\label{ObitFArray_8c_a42}


Public: Add a scalar to elements of an FArray. 

in = in + scalar \begin{Desc}
\item[Parameters:]
\begin{description}
\item[{\em in}]Input object with data \item[{\em scalar}]Scalar value \end{description}
\end{Desc}
\index{ObitFArray.c@{Obit\-FArray.c}!ObitFArraySDiv@{ObitFArraySDiv}}
\index{ObitFArraySDiv@{ObitFArraySDiv}!ObitFArray.c@{Obit\-FArray.c}}
\subsubsection{\setlength{\rightskip}{0pt plus 5cm}void Obit\-FArray\-SDiv ({\bf Obit\-FArray} $\ast$ {\em in}, {\bf ofloat} {\em scalar})}\label{ObitFArray_8c_a44}


Public: Divide elements of an FArray into a scalar. 

No check for zeroes is made . in = scalar / in \begin{Desc}
\item[Parameters:]
\begin{description}
\item[{\em in}]Input object with data \item[{\em scalar}]Scalar value \end{description}
\end{Desc}
\index{ObitFArray.c@{Obit\-FArray.c}!ObitFArraySelInc@{ObitFArraySelInc}}
\index{ObitFArraySelInc@{ObitFArraySelInc}!ObitFArray.c@{Obit\-FArray.c}}
\subsubsection{\setlength{\rightskip}{0pt plus 5cm}void Obit\-FArray\-Sel\-Inc ({\bf Obit\-FArray} $\ast$ {\em in}, {\bf Obit\-FArray} $\ast$ {\em out}, {\bf olong} $\ast$ {\em blc}, {\bf olong} $\ast$ {\em trc}, {\bf olong} $\ast$ {\em inc}, {\bf Obit\-Err} $\ast$ {\em err})}\label{ObitFArray_8c_a67}


Public: Select elements in an FArray by increment. 

\begin{Desc}
\item[Parameters:]
\begin{description}
\item[{\em in}]Input Object \item[{\em out}]Output Object \item[{\em blc}](0-rel) lower index of first pixel to copy \item[{\em trc}](0-rel) lower index of highest pixel to copy \item[{\em inc}]increment on each axis \item[{\em err}]{\bf Obit}{\rm (p.\,\pageref{structObit})} error stack object. \end{description}
\end{Desc}
\index{ObitFArray.c@{Obit\-FArray.c}!ObitFArrayShiftAdd@{ObitFArrayShiftAdd}}
\index{ObitFArrayShiftAdd@{ObitFArrayShiftAdd}!ObitFArray.c@{Obit\-FArray.c}}
\subsubsection{\setlength{\rightskip}{0pt plus 5cm}void Obit\-FArray\-Shift\-Add ({\bf Obit\-FArray} $\ast$ {\em in1}, {\bf olong} $\ast$ {\em pos1}, {\bf Obit\-FArray} $\ast$ {\em in2}, {\bf olong} $\ast$ {\em pos2}, {\bf ofloat} {\em scalar}, {\bf Obit\-FArray} $\ast$ {\em out})}\label{ObitFArray_8c_a64}


Public: Shift and Add scaled array. 

Only handles to 3 dimensions. If in1/out are 3D and in2 is 2D then the same plane in in2 is used for all planes in in1/out. NB: this works better if the alignment point is near the center of in2 out = in1 + scalar x in2 in overlap, else in1 \begin{Desc}
\item[Parameters:]
\begin{description}
\item[{\em in1}]First input object with data, may be blanked \item[{\em pos1}]Alignment pixel in in1 (0-rel) \item[{\em in2}]Second input object with data, blanked pixels ignored \item[{\em pos2}]Alignment pixel in in2 (0-rel) \item[{\em scalar}]factor to be multiplied times in2 \item[{\em out}]Output array, may be an input array and MUST have the same the same geometry. \end{description}
\end{Desc}
\index{ObitFArray.c@{Obit\-FArray.c}!ObitFArraySin@{ObitFArraySin}}
\index{ObitFArraySin@{ObitFArraySin}!ObitFArray.c@{Obit\-FArray.c}}
\subsubsection{\setlength{\rightskip}{0pt plus 5cm}void Obit\-FArray\-Sin ({\bf Obit\-FArray} $\ast$ {\em in})}\label{ObitFArray_8c_a37}


Public: sine of elements of an FArray. 

in = sin(in). \begin{Desc}
\item[Parameters:]
\begin{description}
\item[{\em in}]Input object with data \end{description}
\end{Desc}
\index{ObitFArray.c@{Obit\-FArray.c}!ObitFArraySMul@{ObitFArraySMul}}
\index{ObitFArraySMul@{ObitFArraySMul}!ObitFArray.c@{Obit\-FArray.c}}
\subsubsection{\setlength{\rightskip}{0pt plus 5cm}void Obit\-FArray\-SMul ({\bf Obit\-FArray} $\ast$ {\em in}, {\bf ofloat} {\em scalar})}\label{ObitFArray_8c_a43}


Public: Multiply elements of an FArray by a scalar. 

in = in $\ast$ scalar \begin{Desc}
\item[Parameters:]
\begin{description}
\item[{\em in}]Input object with data \item[{\em scalar}]Scalar value \end{description}
\end{Desc}
\index{ObitFArray.c@{Obit\-FArray.c}!ObitFArraySqrt@{ObitFArraySqrt}}
\index{ObitFArraySqrt@{ObitFArraySqrt}!ObitFArray.c@{Obit\-FArray.c}}
\subsubsection{\setlength{\rightskip}{0pt plus 5cm}void Obit\-FArray\-Sqrt ({\bf Obit\-FArray} $\ast$ {\em in})}\label{ObitFArray_8c_a39}


Public: square root of elements of an FArray. 

in = sqrt(MAX(1.0e-20, in)). \begin{Desc}
\item[Parameters:]
\begin{description}
\item[{\em in}]Input object with data \end{description}
\end{Desc}
\index{ObitFArray.c@{Obit\-FArray.c}!ObitFArraySub@{ObitFArraySub}}
\index{ObitFArraySub@{ObitFArraySub}!ObitFArray.c@{Obit\-FArray.c}}
\subsubsection{\setlength{\rightskip}{0pt plus 5cm}void Obit\-FArray\-Sub ({\bf Obit\-FArray} $\ast$ {\em in1}, {\bf Obit\-FArray} $\ast$ {\em in2}, {\bf Obit\-FArray} $\ast$ {\em out})}\label{ObitFArray_8c_a53}


Public: Subtract elements of two FArrays. 

out = in1 - in2, if either is blanked the result is blanked \begin{Desc}
\item[Parameters:]
\begin{description}
\item[{\em in1}]Input object with data \item[{\em in2}]Input object with data \item[{\em out}]Output array (may be an input array). \end{description}
\end{Desc}
\index{ObitFArray.c@{Obit\-FArray.c}!ObitFArraySubArr@{ObitFArraySubArr}}
\index{ObitFArraySubArr@{ObitFArraySubArr}!ObitFArray.c@{Obit\-FArray.c}}
\subsubsection{\setlength{\rightskip}{0pt plus 5cm}{\bf Obit\-FArray}$\ast$ Obit\-FArray\-Sub\-Arr ({\bf Obit\-FArray} $\ast$ {\em in}, {\bf olong} $\ast$ {\em blc}, {\bf olong} $\ast$ {\em trc}, {\bf Obit\-Err} $\ast$ {\em err})}\label{ObitFArray_8c_a20}


Public: Copy Subarray constructor. 

\begin{Desc}
\item[Parameters:]
\begin{description}
\item[{\em in}]Object with structures to subarray. \item[{\em blc}](0-rel) lower index of first pixel to copy \item[{\em trc}](0-rel) lower index of highest pixel to copy \item[{\em err}]{\bf Obit}{\rm (p.\,\pageref{structObit})} error stack object. \end{description}
\end{Desc}
\begin{Desc}
\item[Returns:]the new object. \end{Desc}
\index{ObitFArray.c@{Obit\-FArray.c}!ObitFArraySum@{ObitFArraySum}}
\index{ObitFArraySum@{ObitFArraySum}!ObitFArray.c@{Obit\-FArray.c}}
\subsubsection{\setlength{\rightskip}{0pt plus 5cm}{\bf ofloat} Obit\-FArray\-Sum ({\bf Obit\-FArray} $\ast$ {\em in})}\label{ObitFArray_8c_a40}


Public: sum elements of an FArray. 

out = Sum (in.) \begin{Desc}
\item[Parameters:]
\begin{description}
\item[{\em in}]Input object with data \end{description}
\end{Desc}
\begin{Desc}
\item[Returns:]sum of elements \end{Desc}
\index{ObitFArray.c@{Obit\-FArray.c}!ObitFArraySumArr@{ObitFArraySumArr}}
\index{ObitFArraySumArr@{ObitFArraySumArr}!ObitFArray.c@{Obit\-FArray.c}}
\subsubsection{\setlength{\rightskip}{0pt plus 5cm}void Obit\-FArray\-Sum\-Arr ({\bf Obit\-FArray} $\ast$ {\em in1}, {\bf Obit\-FArray} $\ast$ {\em in2}, {\bf Obit\-FArray} $\ast$ {\em out})}\label{ObitFArray_8c_a50}


Public: Sum nonblanked elements of two FArrays. 

out = (in1 + in2) or whichever is not blanked \begin{Desc}
\item[Parameters:]
\begin{description}
\item[{\em in1}]Input object with data \item[{\em in2}]Input object with data \item[{\em out}]Output array (may be an input array). \end{description}
\end{Desc}
\index{ObitFArray.c@{Obit\-FArray.c}!ObitFArrayTranspose@{ObitFArrayTranspose}}
\index{ObitFArrayTranspose@{ObitFArrayTranspose}!ObitFArray.c@{Obit\-FArray.c}}
\subsubsection{\setlength{\rightskip}{0pt plus 5cm}{\bf Obit\-FArray}$\ast$ Obit\-FArray\-Transpose ({\bf Obit\-FArray} $\ast$ {\em in}, {\bf olong} $\ast$ {\em order}, {\bf Obit\-Err} $\ast$ {\em err})}\label{ObitFArray_8c_a21}


Public: Transpose constructor. 

\begin{Desc}
\item[Parameters:]
\begin{description}
\item[{\em in}]Object with structures to transpose \item[{\em order}]output 1-rel order of the transposed axes, in storage order negative value = reverse order, e,g, [2,1] = transpose 2D array \item[{\em err}]{\bf Obit}{\rm (p.\,\pageref{structObit})} error stack object. \end{description}
\end{Desc}
\begin{Desc}
\item[Returns:]the new object. \end{Desc}
