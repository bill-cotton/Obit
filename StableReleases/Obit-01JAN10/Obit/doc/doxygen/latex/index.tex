{\bf Obit}{\rm (p.\,\pageref{structObit})} uses a class derivation scheme that doxygen does not understand so some care is needed in interpreting this documentation. Class hierarchies are generally noted in the names of modules, i.e. {\bf Obit}{\rm (p.\,\pageref{structObit})} is the base class from which (almost) all others are derived. {\bf Obit}{\rm (p.\,\pageref{structObit})} class derivation is by means of nested include files; each class has an include file for the data members and for the class function pointers. These include files include the corresponding includes of their parent class.

The most effective use of this documentation page is to use the \char`\"{}File List\char`\"{} function and select the header file (.h file) for the given class, this gives the functional interface to the class. Class data members can be viewed using the links to the class structure on this page or use the \char`\"{}Class List\char`\"{} function on the main page and select the desired class.

Each class has a \char`\"{}Class\-Info\char`\"{} structure with class specific information, mostly, function pointers. Each instance of a class has a number of data members, including a pointer to the Class\-Info structure of its class. 