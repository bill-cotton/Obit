\section{Obit\-IOUVAIPS.h File Reference}
\label{ObitIOUVAIPS_8h}\index{ObitIOUVAIPS.h@{ObitIOUVAIPS.h}}
{\bf Obit\-IOUVAIPS}{\rm (p.\,\pageref{structObitIOUVAIPS})} class definition. 

{\tt \#include \char`\"{}fitsio.h\char`\"{}}\par
{\tt \#include \char`\"{}Obit.h\char`\"{}}\par
{\tt \#include \char`\"{}Obit\-IO.h\char`\"{}}\par
{\tt \#include \char`\"{}Obit\-File.h\char`\"{}}\par
{\tt \#include \char`\"{}Obit\-UVDesc.h\char`\"{}}\par
{\tt \#include \char`\"{}Obit\-UVSel.h\char`\"{}}\par
\subsection*{Classes}
\begin{CompactItemize}
\item 
struct {\bf Obit\-IOUVAIPS}
\begin{CompactList}\small\item\em Obit\-IOUVAIPS Class structure. \item\end{CompactList}\item 
struct {\bf Obit\-IOUVAIPSClass\-Info}
\begin{CompactList}\small\item\em Class\-Info Structure. \item\end{CompactList}\end{CompactItemize}
\subsection*{Defines}
\begin{CompactItemize}
\item 
\#define {\bf Obit\-IOUVAIPSUnref}(in)\ Obit\-Unref (in)
\begin{CompactList}\small\item\em Macro to unreference (and possibly destroy) an {\bf Obit\-IOUVAIPS}{\rm (p.\,\pageref{structObitIOUVAIPS})} returns a Obit\-IOUVUVAIPS$\ast$ (NULL). \item\end{CompactList}\item 
\#define {\bf Obit\-IOUVAIPSRef}(in)\ Obit\-Ref (in)
\begin{CompactList}\small\item\em Macro to reference (update reference count) an {\bf Obit\-IOUVAIPS}{\rm (p.\,\pageref{structObitIOUVAIPS})}. \item\end{CompactList}\item 
\#define {\bf Obit\-IOUVAIPSIs\-A}(in)\ Obit\-Is\-A (in, Obit\-IOUVAIPSGet\-Class())
\begin{CompactList}\small\item\em Macro to determine if an object is the member of this or a derived class. \item\end{CompactList}\end{CompactItemize}
\subsection*{Functions}
\begin{CompactItemize}
\item 
void {\bf Obit\-IOUVAIPSClass\-Init} (void)
\begin{CompactList}\small\item\em Public: Class initializer. \item\end{CompactList}\item 
{\bf Obit\-IOUVAIPS} $\ast$ {\bf new\-Obit\-IOUVAIPS} (gchar $\ast$name, {\bf Obit\-Info\-List} $\ast$info, {\bf Obit\-Err} $\ast$err)
\begin{CompactList}\small\item\em Public: Constructor. \item\end{CompactList}\item 
gconstpointer {\bf Obit\-IOUVAIPSGet\-Class} (void)
\begin{CompactList}\small\item\em Public: Class\-Info pointer. \item\end{CompactList}\item 
gboolean {\bf Obit\-IOUVAIPSSame} ({\bf Obit\-IO} $\ast$in, {\bf Obit\-Info\-List} $\ast$in1, {\bf Obit\-Info\-List} $\ast$in2, {\bf Obit\-Err} $\ast$err)
\begin{CompactList}\small\item\em Public: Are underlying structures the same. \item\end{CompactList}\item 
void {\bf Obit\-IOUVAIPSRename} ({\bf Obit\-IO} $\ast$in, {\bf Obit\-Info\-List} $\ast$info, {\bf Obit\-Err} $\ast$err)
\begin{CompactList}\small\item\em Public: Rename underlying structures. \item\end{CompactList}\item 
void {\bf Obit\-IOUVAIPSZap} ({\bf Obit\-IOUVAIPS} $\ast$in, {\bf Obit\-Err} $\ast$err)
\begin{CompactList}\small\item\em Public: Delete underlying structures. \item\end{CompactList}\item 
{\bf Obit\-IOUVAIPS} $\ast$ {\bf Obit\-IOUVAIPSCopy} ({\bf Obit\-IOUVAIPS} $\ast$in, {\bf Obit\-IOUVAIPS} $\ast$out, {\bf Obit\-Err} $\ast$err)
\begin{CompactList}\small\item\em Public: Copy constructor. \item\end{CompactList}\item 
Obit\-IOCode {\bf Obit\-IOUVAIPSOpen} ({\bf Obit\-IOUVAIPS} $\ast$in, Obit\-IOAccess access, {\bf Obit\-Info\-List} $\ast$info, {\bf Obit\-Err} $\ast$err)
\begin{CompactList}\small\item\em Public: Open. \item\end{CompactList}\item 
Obit\-IOCode {\bf Obit\-IOUVAIPSClose} ({\bf Obit\-IOUVAIPS} $\ast$in, {\bf Obit\-Err} $\ast$err)
\begin{CompactList}\small\item\em Public: Close. \item\end{CompactList}\item 
Obit\-IOCode {\bf Obit\-IOUVAIPSSet} ({\bf Obit\-IOUVAIPS} $\ast$in, {\bf Obit\-Info\-List} $\ast$info, {\bf Obit\-Err} $\ast$err)
\begin{CompactList}\small\item\em Public: Init I/O. \item\end{CompactList}\item 
Obit\-IOCode {\bf Obit\-IOUVAIPSRead} ({\bf Obit\-IOUVAIPS} $\ast$in, {\bf ofloat} $\ast$data, {\bf Obit\-Err} $\ast$err)
\begin{CompactList}\small\item\em Public: Read. \item\end{CompactList}\item 
Obit\-IOCode {\bf Obit\-IOUVAIPSRead\-Select} ({\bf Obit\-IOUVAIPS} $\ast$in, {\bf ofloat} $\ast$data, {\bf Obit\-Err} $\ast$err)
\begin{CompactList}\small\item\em Public: Read/cal/select. \item\end{CompactList}\item 
Obit\-IOCode {\bf Obit\-IOUVAIPSRead\-Multi} ({\bf olong} n\-Buff, {\bf Obit\-IOUVAIPS} $\ast$$\ast$in, {\bf ofloat} $\ast$$\ast$data, {\bf Obit\-Err} $\ast$err)
\begin{CompactList}\small\item\em Public: Read to multiple buffers. \item\end{CompactList}\item 
Obit\-IOCode {\bf Obit\-IOUVAIPSRe\-Read\-Multi} ({\bf olong} n\-Buff, {\bf Obit\-IOUVAIPS} $\ast$$\ast$in, {\bf ofloat} $\ast$$\ast$data, {\bf Obit\-Err} $\ast$err)
\begin{CompactList}\small\item\em Public: Reread to multiple buffers. \item\end{CompactList}\item 
Obit\-IOCode {\bf Obit\-IOUVAIPSRead\-Multi\-Select} ({\bf olong} n\-Buff, {\bf Obit\-IOUVAIPS} $\ast$$\ast$in, {\bf ofloat} $\ast$$\ast$data, {\bf Obit\-Err} $\ast$err)
\begin{CompactList}\small\item\em Public: Read/cal/select multiple buffers. \item\end{CompactList}\item 
Obit\-IOCode {\bf Obit\-IOUVAIPSRe\-Read\-Multi\-Select} ({\bf olong} n\-Buff, {\bf Obit\-IOUVAIPS} $\ast$$\ast$in, {\bf ofloat} $\ast$$\ast$data, {\bf Obit\-Err} $\ast$err)
\begin{CompactList}\small\item\em Public: Reread/cal/select multiple buffers. \item\end{CompactList}\item 
Obit\-IOCode {\bf Obit\-IOUVAIPSWrite} ({\bf Obit\-IOUVAIPS} $\ast$in, {\bf ofloat} $\ast$data, {\bf Obit\-Err} $\ast$err)
\begin{CompactList}\small\item\em Public: Write. \item\end{CompactList}\item 
Obit\-IOCode {\bf Obit\-IOUVAIPSFlush} ({\bf Obit\-IOUVAIPS} $\ast$in, {\bf Obit\-Err} $\ast$err)
\begin{CompactList}\small\item\em Public: Flush. \item\end{CompactList}\item 
Obit\-IOCode {\bf Obit\-IOUVAIPSRead\-Descriptor} ({\bf Obit\-IOUVAIPS} $\ast$in, {\bf Obit\-Err} $\ast$err)
\begin{CompactList}\small\item\em Public: Read Descriptor. \item\end{CompactList}\item 
Obit\-IOCode {\bf Obit\-IOUVAIPSWrite\-Descriptor} ({\bf Obit\-IOUVAIPS} $\ast$in, {\bf Obit\-Err} $\ast$err)
\begin{CompactList}\small\item\em Public: Write Descriptor. \item\end{CompactList}\item 
void {\bf Obit\-IOUVAIPSCreate\-Buffer} ({\bf ofloat} $\ast$$\ast$data, {\bf olong} $\ast$size, {\bf Obit\-IOUVAIPS} $\ast$in, {\bf Obit\-Info\-List} $\ast$info, {\bf Obit\-Err} $\ast$err)
\begin{CompactList}\small\item\em Public: Create buffer. \item\end{CompactList}\item 
{\bf Obit} $\ast$ {\bf new\-Obit\-IOUVAIPSTable} ({\bf Obit\-IOUVAIPS} $\ast$in, Obit\-IOAccess access, gchar $\ast$tab\-Type, {\bf olong} $\ast$tabver, {\bf Obit\-Err} $\ast$err)
\begin{CompactList}\small\item\em Public: Create an associated Table Typed as base class to avoid problems. \item\end{CompactList}\item 
Obit\-IOCode {\bf Obit\-IOUVAIPSUpdate\-Tables} ({\bf Obit\-IOUVAIPS} $\ast$in, {\bf Obit\-Info\-List} $\ast$info, {\bf Obit\-Err} $\ast$err)
\begin{CompactList}\small\item\em Public: Update disk resident tables information. \item\end{CompactList}\item 
void {\bf Obit\-IOUVAIPSGet\-File\-Info} ({\bf Obit\-IO} $\ast$in, {\bf Obit\-Info\-List} $\ast$my\-Info, gchar $\ast$prefix, {\bf Obit\-Info\-List} $\ast$out\-List, {\bf Obit\-Err} $\ast$err)
\begin{CompactList}\small\item\em Public: Extract information about underlying file. \item\end{CompactList}\end{CompactItemize}


\subsection{Detailed Description}
{\bf Obit\-IOUVAIPS}{\rm (p.\,\pageref{structObitIOUVAIPS})} class definition. 

This class is derived from the {\bf Obit\-IO}{\rm (p.\,\pageref{structObitIO})} class.\subsection{Usage}\label{ObitIOUVAIPS_8h_ObitIOUVAIPSUsage}
Instances of this class are for access to AIPS uv data files. The {\bf Obit\-AIPS}{\rm (p.\,\pageref{structObitAIPS})} class must be initialized before accessing AIPS files; this uses {\bf Obit\-AIPSClass\-Init}{\rm (p.\,\pageref{ObitAIPS_8c_a5})}. Instances can be made using the {\bf new\-Obit\-IOUVAIPS}{\rm (p.\,\pageref{ObitIOUVAIPS_8c_a10})} constructor, or the {\bf Obit\-IOUVAIPSCopy}{\rm (p.\,\pageref{ObitIOUVAIPS_8c_a15})} copy constructor and pointers copied (with reference pointer update) using {\bf Obit\-IORef}{\rm (p.\,\pageref{ObitIO_8h_a1})}. The destructor (when reference count goes to zero) is {\bf Obit\-IOUnref}{\rm (p.\,\pageref{ObitIO_8h_a0})}. This class should seldom need be accessed directly outside of the {\bf Obit\-IO}{\rm (p.\,\pageref{structObitIO})} class. Parameters needed (passed via {\bf Obit\-Info\-List}{\rm (p.\,\pageref{structObitInfoList})}) are: \begin{itemize}
\item \char`\"{}BLC\char`\"{} OBIT\_\-int (?,1,1) the bottom-left corner desired as expressed in 1-rel pixel indices. If absent, the value (1,1,1...) will be assumed. dimension of this array is [IM\_\-MAXDIM]. \item \char`\"{}TRC\char`\"{} OBIT\_\-int (?,1,1) the top-right corner desired as expressed in 1-rel pixel indices. If absent, all pixels are included. dimension of this array is [IM\_\-MAXDIM]. \item \char`\"{}IOBy\char`\"{} OBIT\_\-int (1,1,1) an Obit\-IOSize enum defined in {\bf Obit\-IO.h}{\rm (p.\,\pageref{ObitIO_8h})} giving values OBIT\_\-IO\_\-by\-Row or OBIT\_\-IO\_\-by\-Plane to specify if the data transfers are to be by row or plane at a time. \item \char`\"{}Disk\char`\"{} OBIT\_\-int (1,1,1) AIPS \char`\"{}disk\char`\"{} number. \item \char`\"{}User\char`\"{} OBIT\_\-int (1,1,1) user number. \item \char`\"{}CNO\char`\"{} OBIT\_\-int (1,1,1) AIPS catalog slot number.\end{itemize}


\subsection{Define Documentation}
\index{ObitIOUVAIPS.h@{Obit\-IOUVAIPS.h}!ObitIOUVAIPSIsA@{ObitIOUVAIPSIsA}}
\index{ObitIOUVAIPSIsA@{ObitIOUVAIPSIsA}!ObitIOUVAIPS.h@{Obit\-IOUVAIPS.h}}
\subsubsection{\setlength{\rightskip}{0pt plus 5cm}\#define Obit\-IOUVAIPSIs\-A(in)\ Obit\-Is\-A (in, Obit\-IOUVAIPSGet\-Class())}\label{ObitIOUVAIPS_8h_a2}


Macro to determine if an object is the member of this or a derived class. 

Returns TRUE if a member, else FALSE in = object to reference \index{ObitIOUVAIPS.h@{Obit\-IOUVAIPS.h}!ObitIOUVAIPSRef@{ObitIOUVAIPSRef}}
\index{ObitIOUVAIPSRef@{ObitIOUVAIPSRef}!ObitIOUVAIPS.h@{Obit\-IOUVAIPS.h}}
\subsubsection{\setlength{\rightskip}{0pt plus 5cm}\#define Obit\-IOUVAIPSRef(in)\ Obit\-Ref (in)}\label{ObitIOUVAIPS_8h_a1}


Macro to reference (update reference count) an {\bf Obit\-IOUVAIPS}{\rm (p.\,\pageref{structObitIOUVAIPS})}. 

returns a Obit\-IOUVAIPS$\ast$. in = object to reference \index{ObitIOUVAIPS.h@{Obit\-IOUVAIPS.h}!ObitIOUVAIPSUnref@{ObitIOUVAIPSUnref}}
\index{ObitIOUVAIPSUnref@{ObitIOUVAIPSUnref}!ObitIOUVAIPS.h@{Obit\-IOUVAIPS.h}}
\subsubsection{\setlength{\rightskip}{0pt plus 5cm}\#define Obit\-IOUVAIPSUnref(in)\ Obit\-Unref (in)}\label{ObitIOUVAIPS_8h_a0}


Macro to unreference (and possibly destroy) an {\bf Obit\-IOUVAIPS}{\rm (p.\,\pageref{structObitIOUVAIPS})} returns a Obit\-IOUVUVAIPS$\ast$ (NULL). 

\begin{itemize}
\item in = object to unreference. \end{itemize}


\subsection{Function Documentation}
\index{ObitIOUVAIPS.h@{Obit\-IOUVAIPS.h}!newObitIOUVAIPS@{newObitIOUVAIPS}}
\index{newObitIOUVAIPS@{newObitIOUVAIPS}!ObitIOUVAIPS.h@{Obit\-IOUVAIPS.h}}
\subsubsection{\setlength{\rightskip}{0pt plus 5cm}{\bf Obit\-IOUVAIPS}$\ast$ new\-Obit\-IOUVAIPS (gchar $\ast$ {\em name}, {\bf Obit\-Info\-List} $\ast$ {\em info}, {\bf Obit\-Err} $\ast$ {\em err})}\label{ObitIOUVAIPS_8h_a4}


Public: Constructor. 

Initializes class on the first call. \begin{Desc}
\item[Parameters:]
\begin{description}
\item[{\em name}]An optional name for the object. \item[{\em info}]if non-NULL it is used to initialize the new object. \item[{\em err}]{\bf Obit\-Err}{\rm (p.\,\pageref{structObitErr})} for error messages. \end{description}
\end{Desc}
\begin{Desc}
\item[Returns:]the new object. \end{Desc}
\index{ObitIOUVAIPS.h@{Obit\-IOUVAIPS.h}!newObitIOUVAIPSTable@{newObitIOUVAIPSTable}}
\index{newObitIOUVAIPSTable@{newObitIOUVAIPSTable}!ObitIOUVAIPS.h@{Obit\-IOUVAIPS.h}}
\subsubsection{\setlength{\rightskip}{0pt plus 5cm}{\bf Obit}$\ast$ new\-Obit\-IOUVAIPSTable ({\bf Obit\-IOUVAIPS} $\ast$ {\em in}, Obit\-IOAccess {\em access}, gchar $\ast$ {\em tab\-Type}, {\bf olong} $\ast$ {\em tab\-Ver}, {\bf Obit\-Err} $\ast$ {\em err})}\label{ObitIOUVAIPS_8h_a24}


Public: Create an associated Table Typed as base class to avoid problems. 

If such an object exists, a reference to it is returned, else a new object is created and entered in the {\bf Obit\-Table\-List}{\rm (p.\,\pageref{structObitTableList})}. Returned object is typed an {\bf Obit}{\rm (p.\,\pageref{structObit})} to prevent circular definitions between the {\bf Obit\-Table}{\rm (p.\,\pageref{structObitTable})} and the {\bf Obit\-IO}{\rm (p.\,\pageref{structObitIO})} classes. \begin{Desc}
\item[Parameters:]
\begin{description}
\item[{\em in}]Pointer to object with associated tables. This MUST have been opened before this call. \item[{\em access}]access (OBIT\_\-IO\_\-Read\-Only,OBIT\_\-IO\_\-Read\-Write, or OBIT\_\-IO\_\-Write\-Only). This is used to determine defaulted version number and a different value may be used for the actual Open. \item[{\em tab\-Type}]The table type (e.g. \char`\"{}AIPS CC\char`\"{}). \item[{\em tab\-Ver}]Desired version number, may be zero in which case the highest extant version is returned for read and the highest+1 for OBIT\_\-IO\_\-Write\-Only. \item[{\em err}]{\bf Obit\-Err}{\rm (p.\,\pageref{structObitErr})} for reporting errors. \end{description}
\end{Desc}
\begin{Desc}
\item[Returns:]pointer to created {\bf Obit\-Table}{\rm (p.\,\pageref{structObitTable})}, NULL on failure. \end{Desc}
\index{ObitIOUVAIPS.h@{Obit\-IOUVAIPS.h}!ObitIOUVAIPSClassInit@{ObitIOUVAIPSClassInit}}
\index{ObitIOUVAIPSClassInit@{ObitIOUVAIPSClassInit}!ObitIOUVAIPS.h@{Obit\-IOUVAIPS.h}}
\subsubsection{\setlength{\rightskip}{0pt plus 5cm}void Obit\-IOUVAIPSClass\-Init (void)}\label{ObitIOUVAIPS_8h_a3}


Public: Class initializer. 

\index{ObitIOUVAIPS.h@{Obit\-IOUVAIPS.h}!ObitIOUVAIPSClose@{ObitIOUVAIPSClose}}
\index{ObitIOUVAIPSClose@{ObitIOUVAIPSClose}!ObitIOUVAIPS.h@{Obit\-IOUVAIPS.h}}
\subsubsection{\setlength{\rightskip}{0pt plus 5cm}Obit\-IOCode Obit\-IOUVAIPSClose ({\bf Obit\-IOUVAIPS} $\ast$ {\em in}, {\bf Obit\-Err} $\ast$ {\em err})}\label{ObitIOUVAIPS_8h_a11}


Public: Close. 

\begin{Desc}
\item[Parameters:]
\begin{description}
\item[{\em in}]Pointer to object to be closed. \item[{\em err}]{\bf Obit\-Err}{\rm (p.\,\pageref{structObitErr})} for reporting errors. \end{description}
\end{Desc}
\begin{Desc}
\item[Returns:]error code, 0=$>$ OK \end{Desc}
\index{ObitIOUVAIPS.h@{Obit\-IOUVAIPS.h}!ObitIOUVAIPSCopy@{ObitIOUVAIPSCopy}}
\index{ObitIOUVAIPSCopy@{ObitIOUVAIPSCopy}!ObitIOUVAIPS.h@{Obit\-IOUVAIPS.h}}
\subsubsection{\setlength{\rightskip}{0pt plus 5cm}{\bf Obit\-IOUVAIPS}$\ast$ Obit\-IOUVAIPSCopy ({\bf Obit\-IOUVAIPS} $\ast$ {\em in}, {\bf Obit\-IOUVAIPS} $\ast$ {\em out}, {\bf Obit\-Err} $\ast$ {\em err})}\label{ObitIOUVAIPS_8h_a9}


Public: Copy constructor. 

The result will have pointers to the more complex members. Parent class members are included but any derived class info is ignored. \begin{Desc}
\item[Parameters:]
\begin{description}
\item[{\em in}]The object to copy \item[{\em out}]An existing object pointer for output or NULL if none exists. \item[{\em err}]{\bf Obit}{\rm (p.\,\pageref{structObit})} error stack object. \end{description}
\end{Desc}
\begin{Desc}
\item[Returns:]pointer to the new object. \end{Desc}
\index{ObitIOUVAIPS.h@{Obit\-IOUVAIPS.h}!ObitIOUVAIPSCreateBuffer@{ObitIOUVAIPSCreateBuffer}}
\index{ObitIOUVAIPSCreateBuffer@{ObitIOUVAIPSCreateBuffer}!ObitIOUVAIPS.h@{Obit\-IOUVAIPS.h}}
\subsubsection{\setlength{\rightskip}{0pt plus 5cm}void Obit\-IOUVAIPSCreate\-Buffer ({\bf ofloat} $\ast$$\ast$ {\em data}, {\bf olong} $\ast$ {\em size}, {\bf Obit\-IOUVAIPS} $\ast$ {\em in}, {\bf Obit\-Info\-List} $\ast$ {\em info}, {\bf Obit\-Err} $\ast$ {\em err})}\label{ObitIOUVAIPS_8h_a23}


Public: Create buffer. 

\begin{Desc}
\item[Parameters:]
\begin{description}
\item[{\em data}](output) pointer to data array \item[{\em size}](output) size of data array in floats. \item[{\em in}]Pointer to object to be accessed. \item[{\em info}]{\bf Obit\-Info\-List}{\rm (p.\,\pageref{structObitInfoList})} with instructions \item[{\em err}]{\bf Obit\-Err}{\rm (p.\,\pageref{structObitErr})} for reporting errors. \end{description}
\end{Desc}
\index{ObitIOUVAIPS.h@{Obit\-IOUVAIPS.h}!ObitIOUVAIPSFlush@{ObitIOUVAIPSFlush}}
\index{ObitIOUVAIPSFlush@{ObitIOUVAIPSFlush}!ObitIOUVAIPS.h@{Obit\-IOUVAIPS.h}}
\subsubsection{\setlength{\rightskip}{0pt plus 5cm}Obit\-IOCode Obit\-IOUVAIPSFlush ({\bf Obit\-IOUVAIPS} $\ast$ {\em in}, {\bf Obit\-Err} $\ast$ {\em err})}\label{ObitIOUVAIPS_8h_a20}


Public: Flush. 

\begin{Desc}
\item[Parameters:]
\begin{description}
\item[{\em in}]Pointer to object to be accessed. \item[{\em err}]{\bf Obit\-Err}{\rm (p.\,\pageref{structObitErr})} for reporting errors. \end{description}
\end{Desc}
\begin{Desc}
\item[Returns:]return code, 0=$>$ OK \end{Desc}
\index{ObitIOUVAIPS.h@{Obit\-IOUVAIPS.h}!ObitIOUVAIPSGetClass@{ObitIOUVAIPSGetClass}}
\index{ObitIOUVAIPSGetClass@{ObitIOUVAIPSGetClass}!ObitIOUVAIPS.h@{Obit\-IOUVAIPS.h}}
\subsubsection{\setlength{\rightskip}{0pt plus 5cm}gconstpointer Obit\-IOUVAIPSGet\-Class (void)}\label{ObitIOUVAIPS_8h_a5}


Public: Class\-Info pointer. 

Initializes class if needed on first call. \begin{Desc}
\item[Returns:]pointer to the class structure. \end{Desc}
\index{ObitIOUVAIPS.h@{Obit\-IOUVAIPS.h}!ObitIOUVAIPSGetFileInfo@{ObitIOUVAIPSGetFileInfo}}
\index{ObitIOUVAIPSGetFileInfo@{ObitIOUVAIPSGetFileInfo}!ObitIOUVAIPS.h@{Obit\-IOUVAIPS.h}}
\subsubsection{\setlength{\rightskip}{0pt plus 5cm}void Obit\-IOUVAIPSGet\-File\-Info ({\bf Obit\-IO} $\ast$ {\em in}, {\bf Obit\-Info\-List} $\ast$ {\em my\-Info}, gchar $\ast$ {\em prefix}, {\bf Obit\-Info\-List} $\ast$ {\em out\-List}, {\bf Obit\-Err} $\ast$ {\em err})}\label{ObitIOUVAIPS_8h_a26}


Public: Extract information about underlying file. 

\begin{Desc}
\item[Parameters:]
\begin{description}
\item[{\em in}]Object of interest. \item[{\em my\-Info}]Info\-List on basic object with selection \item[{\em prefix}]If Non\-Null, string to be added to beginning of out\-List entry name \item[{\em out\-List}]Info\-List to write entries into Following entries for AIPS files (\char`\"{}xxx\char`\"{} = prefix): \begin{itemize}
\item xxx\-Name OBIT\_\-string AIPS file name \item xxx\-Class OBIT\_\-string AIPS file class \item xxx\-Disk OBIT\_\-oint AIPS file disk number \item xxx\-Seq OBIT\_\-oint AIPS file Sequence number \item xxx\-User OBIT\_\-oint AIPS User number \item xxx\-CNO OBIT\_\-oint AIPS Catalog slot number \item xxx\-Dir OBIT\_\-string Directory name for xxx\-Disk\end{itemize}
For all File types types: \begin{itemize}
\item xxx\-Data\-Type OBIT\_\-string \char`\"{}UV\char`\"{} = UV data, \char`\"{}MA\char`\"{}=$>$image, \char`\"{}Table\char`\"{}=Table, \char`\"{}OTF\char`\"{}=OTF, etc \item xxx\-File\-Type OBIT\_\-oint File type as Obit\-IOType, OBIT\_\-IO\_\-FITS, OBIT\_\-IO\_\-AIPS\end{itemize}
For xxx\-Data\-Type = \char`\"{}UV\char`\"{} \begin{itemize}
\item xxxn\-Vis\-PIO OBIT\_\-int (1,1,1) Number of vis. records per IO call \item xxxdo\-Cal\-Select OBIT\_\-bool (1,1,1) Select/calibrate/edit data? \item xxx\-Stokes OBIT\_\-string (4,1,1) Selected output Stokes parameters: \char`\"{}    \char`\"{}=$>$ no translation,\char`\"{}I   \char`\"{},\char`\"{}V   \char`\"{},\char`\"{}Q   \char`\"{}, \char`\"{}U   \char`\"{}, \char`\"{}IQU \char`\"{}, \char`\"{}IQUV\char`\"{}, \char`\"{}IV  \char`\"{}, \char`\"{}RR  \char`\"{}, \char`\"{}LL  \char`\"{}, \char`\"{}RL  \char`\"{}, \char`\"{}LR  \char`\"{}, \char`\"{}HALF\char`\"{} = RR,LL, \char`\"{}FULL\char`\"{}=RR,LL,RL,LR. [default \char`\"{}    \char`\"{}] In the above 'F' can substitute for \char`\"{}formal\char`\"{} 'I' (both RR+LL). \item xxx\-BChan OBIT\_\-int (1,1,1) First spectral channel selected. [def all] \item xxx\-EChan OBIT\_\-int (1,1,1) Highest spectral channel selected. [def all] \item xxx\-BIF OBIT\_\-int (1,1,1) First \char`\"{}IF\char`\"{} selected. [def all] \item xxx\-EIF OBIT\_\-int (1,1,1) Highest \char`\"{}IF\char`\"{} selected. [def all] \item xxxdo\-Pol OBIT\_\-int (1,1,1) $>$0 -$>$ calibrate polarization. \item xxxdo\-Calib OBIT\_\-int (1,1,1) $>$0 -$>$ calibrate, 2=$>$ also calibrate Weights \item xxxgain\-Use OBIT\_\-int (1,1,1) SN/CL table version number, 0-$>$ use highest \item xxxflag\-Ver OBIT\_\-int (1,1,1) Flag table version, 0-$>$ use highest, $<$0-$>$ none \item xxx\-BLVer OBIT\_\-int (1,1,1) BL table version, 0$>$ use highest, $<$0-$>$ none \item xxx\-BPVer OBIT\_\-int (1,1,1) Band pass (BP) table version, 0-$>$ use highest \item xxx\-Subarray OBIT\_\-int (1,1,1) Selected subarray, $<$=0-$>$all [default all] \item xxxdrop\-Sub\-A OBIT\_\-bool (1,1,1) Drop subarray info? \item xxx\-Freq\-ID OBIT\_\-int (1,1,1) Selected Frequency ID, $<$=0-$>$all [default all] \item xxxtime\-Range OBIT\_\-float (2,1,1) Selected timerange in days. \item xxx\-UVRange OBIT\_\-float (2,1,1) Selected UV range in kilowavelengths. \item xxx\-Input\-Avg\-Time OBIT\_\-float (1,1,1) Input data averaging time (sec). used for fringe rate decorrelation correction. \item xxx\-Sources OBIT\_\-string (?,?,1) Source names selected unless any starts with a '-' in which case all are deselected (with '-' stripped). \item xxxsou\-Code OBIT\_\-string (4,1,1) Source Cal code desired, ' ' =$>$ any code selected '$\ast$ ' =$>$ any non blank code (calibrators only) '-CAL' =$>$ blank codes only (no calibrators) \item xxx\-Qual Obit\_\-int (1,1,1) Source qualifier, -1 [default] = any \item xxx\-Antennas OBIT\_\-int (?,1,1) a list of selected antenna numbers, if any is negative then the absolute values are used and the specified antennas are deselected. \item xxxcorr\-Type OBIT\_\-int (1,1,1) Correlation type, 0=cross corr only, 1=both, 2=auto only. \item xxxpass\-Al l OBIT\_\-bool (1,1,1) If True, pass along all data when selecting/calibration even if it's all flagged, data deselected by time, source, antenna etc. is not passed. \item xxxdo\-Band OBIT\_\-int (1,1,1) Band pass application type $<$0-$>$ none (1) if = 1 then all the bandpass data for each antenna will be averaged to form a composite bandpass spectrum, this will then be used to correct the data. (2) if = 2 the bandpass spectra nearest in time (in a weighted sense) to the uv data point will be used to correct the data. (3) if = 3 the bandpass data will be interpolated in time using the solution weights to form a composite bandpass spectrum, this interpolated spectrum will then be used to correct the data. (4) if = 4 the bandpass spectra nearest in time (neglecting weights) to the uv data point will be used to correct the data. (5) if = 5 the bandpass data will be interpolated in time ignoring weights to form a composite bandpass spectrum, this interpolated spectrum will then be used to correct the data. \item xxx\-Smooth OBIT\_\-float (3,1,1) specifies the type of spectral smoothing Smooth(1) = type of smoothing to apply: 0 =$>$ no smoothing 1 =$>$ Hanning 2 =$>$ Gaussian 3 =$>$ Boxcar 4 =$>$ Sinc (i.e. sin(x)/x) Smooth(2) = the \char`\"{}diameter\char`\"{} of the function, i.e. width between first nulls of Hanning triangle and sinc function, FWHM of Gaussian, width of Boxcar. Defaults (if $<$ 0.1) are 4, 2, 2 and 3 channels for Smooth(1) = 1 - 4. Smooth(3) = the diameter over which the convolving function has value - in channels. Defaults: 1, 3, 1, 4 times Smooth(2) used when \item xxx\-Alpha OBIT\_\-float (1,1,1) Spectral index to apply, 0=none \item xxx\-Sub\-Scan\-Time Obit\_\-float scalar [Optional] if given, this is the desired time (days) of a sub scan. This is used by the selector to suggest a value close to this which will evenly divide the current scan. See {\bf Obit\-UVSel\-Sub\-Scan}{\rm (p.\,\pageref{ObitUVSel_8c_a21})} 0 =$>$ Use scan average. This is only useful for Read\-Select operations on indexed Obit\-UVs. \end{itemize}
\item[{\em err}]{\bf Obit\-Err}{\rm (p.\,\pageref{structObitErr})} for reporting errors. \end{description}
\end{Desc}
\index{ObitIOUVAIPS.h@{Obit\-IOUVAIPS.h}!ObitIOUVAIPSOpen@{ObitIOUVAIPSOpen}}
\index{ObitIOUVAIPSOpen@{ObitIOUVAIPSOpen}!ObitIOUVAIPS.h@{Obit\-IOUVAIPS.h}}
\subsubsection{\setlength{\rightskip}{0pt plus 5cm}Obit\-IOCode Obit\-IOUVAIPSOpen ({\bf Obit\-IOUVAIPS} $\ast$ {\em in}, Obit\-IOAccess {\em access}, {\bf Obit\-Info\-List} $\ast$ {\em info}, {\bf Obit\-Err} $\ast$ {\em err})}\label{ObitIOUVAIPS_8h_a10}


Public: Open. 

The file etc. info should have been stored in the {\bf Obit\-Info\-List}{\rm (p.\,\pageref{structObitInfoList})}. \begin{Desc}
\item[Parameters:]
\begin{description}
\item[{\em in}]Pointer to object to be opened. \item[{\em access}]access (OBIT\_\-IO\_\-Read\-Only,OBIT\_\-IO\_\-Read\-Write, OBIT\_\-IO\_\-Read\-Cal). \item[{\em info}]{\bf Obit\-Info\-List}{\rm (p.\,\pageref{structObitInfoList})} with instructions for opening \item[{\em err}]{\bf Obit\-Err}{\rm (p.\,\pageref{structObitErr})} for reporting errors. \end{description}
\end{Desc}
\begin{Desc}
\item[Returns:]return code, 0=$>$ OK \end{Desc}
\index{ObitIOUVAIPS.h@{Obit\-IOUVAIPS.h}!ObitIOUVAIPSRead@{ObitIOUVAIPSRead}}
\index{ObitIOUVAIPSRead@{ObitIOUVAIPSRead}!ObitIOUVAIPS.h@{Obit\-IOUVAIPS.h}}
\subsubsection{\setlength{\rightskip}{0pt plus 5cm}Obit\-IOCode Obit\-IOUVAIPSRead ({\bf Obit\-IOUVAIPS} $\ast$ {\em in}, {\bf ofloat} $\ast$ {\em data}, {\bf Obit\-Err} $\ast$ {\em err})}\label{ObitIOUVAIPS_8h_a13}


Public: Read. 

The number read will be my\-Sel-$>$n\-Vis\-PIO (until the end of the selected range of visibilities in which case it will be smaller). The first visibility number after a read is my\-Desc-$>$first\-Vis and the number of visibilities attempted is my\-Sel-$>$num\-Vis\-Read; actual value saved as my\-Desc-$>$num\-Vis\-Buff. When OBIT\_\-IO\_\-EOF is returned all data has been read (then is no new data in data) and the I/O has been closed. \begin{Desc}
\item[Parameters:]
\begin{description}
\item[{\em in}]Pointer to object to be read. \item[{\em data}]pointer to buffer to write results. \item[{\em err}]{\bf Obit\-Err}{\rm (p.\,\pageref{structObitErr})} for reporting errors. \end{description}
\end{Desc}
\begin{Desc}
\item[Returns:]return code, 0=$>$ OK \end{Desc}
\index{ObitIOUVAIPS.h@{Obit\-IOUVAIPS.h}!ObitIOUVAIPSReadDescriptor@{ObitIOUVAIPSReadDescriptor}}
\index{ObitIOUVAIPSReadDescriptor@{ObitIOUVAIPSReadDescriptor}!ObitIOUVAIPS.h@{Obit\-IOUVAIPS.h}}
\subsubsection{\setlength{\rightskip}{0pt plus 5cm}Obit\-IOCode Obit\-IOUVAIPSRead\-Descriptor ({\bf Obit\-IOUVAIPS} $\ast$ {\em in}, {\bf Obit\-Err} $\ast$ {\em err})}\label{ObitIOUVAIPS_8h_a21}


Public: Read Descriptor. 

\begin{Desc}
\item[Parameters:]
\begin{description}
\item[{\em in}]Pointer to object with {\bf Obit\-UVDesc}{\rm (p.\,\pageref{structObitUVDesc})} to be read. \item[{\em err}]{\bf Obit\-Err}{\rm (p.\,\pageref{structObitErr})} for reporting errors. \end{description}
\end{Desc}
\begin{Desc}
\item[Returns:]return code, 0=$>$ OK \end{Desc}
\index{ObitIOUVAIPS.h@{Obit\-IOUVAIPS.h}!ObitIOUVAIPSReadMulti@{ObitIOUVAIPSReadMulti}}
\index{ObitIOUVAIPSReadMulti@{ObitIOUVAIPSReadMulti}!ObitIOUVAIPS.h@{Obit\-IOUVAIPS.h}}
\subsubsection{\setlength{\rightskip}{0pt plus 5cm}Obit\-IOCode Obit\-IOUVAIPSRead\-Multi ({\bf olong} {\em n\-Buff}, {\bf Obit\-IOUVAIPS} $\ast$$\ast$ {\em in}, {\bf ofloat} $\ast$$\ast$ {\em data}, {\bf Obit\-Err} $\ast$ {\em err})}\label{ObitIOUVAIPS_8h_a15}


Public: Read to multiple buffers. 

All buffers must be the same size and the underlying dataset the same. The number read will be my\-Sel-$>$n\-Vis\-PIO (until the end of the selected range of visibilities in which case it will be smaller). The first visibility number after a read is my\-Desc-$>$first\-Vis and the number of visibilities attempted is my\-Sel-$>$num\-Vis\-Read; actual value saved as my\-Desc-$>$num\-Vis\-Buff. When OBIT\_\-IO\_\-EOF is returned all data has been read (then is no new data in data) \begin{Desc}
\item[Parameters:]
\begin{description}
\item[{\em n\-Buff}]Number of buffers to be filled \item[{\em in}]Array of pointers to to object to be read; must all be to same underlying data set but with independent calibration \item[{\em data}]array of pointers to buffers to write results. \item[{\em err}]{\bf Obit\-Err}{\rm (p.\,\pageref{structObitErr})} for reporting errors. \end{description}
\end{Desc}
\begin{Desc}
\item[Returns:]return code, OBIT\_\-IO\_\-OK=$>$ OK \end{Desc}
\index{ObitIOUVAIPS.h@{Obit\-IOUVAIPS.h}!ObitIOUVAIPSReadMultiSelect@{ObitIOUVAIPSReadMultiSelect}}
\index{ObitIOUVAIPSReadMultiSelect@{ObitIOUVAIPSReadMultiSelect}!ObitIOUVAIPS.h@{Obit\-IOUVAIPS.h}}
\subsubsection{\setlength{\rightskip}{0pt plus 5cm}Obit\-IOCode Obit\-IOUVAIPSRead\-Multi\-Select ({\bf olong} {\em n\-Buff}, {\bf Obit\-IOUVAIPS} $\ast$$\ast$ {\em in}, {\bf ofloat} $\ast$$\ast$ {\em data}, {\bf Obit\-Err} $\ast$ {\em err})}\label{ObitIOUVAIPS_8h_a17}


Public: Read/cal/select multiple buffers. 

If amp/phase calibration being applied, it is done independently for each buffer using the my\-Cal in the associated in, otherwise, the first buffer is processed and copied to the others. All selected buffer sizes etc must be the same. The number read will be my\-Sel-$>$n\-Vis\-PIO (until the end of the selected range of visibilities in which case it will be smaller). The first visibility number after a read is my\-Desc-$>$first\-Vis and the number of visibilities is my\-Desc-$>$num\-Vis\-Buff (which may be zero). The number attempted in a read is my\-Sel-$>$num\-Vis\-Read. When OBIT\_\-IO\_\-EOF is returned all data has been read (then is no new data in data) and the I/O has been closed. \begin{Desc}
\item[Parameters:]
\begin{description}
\item[{\em n\-Buff}]Number of buffers to be filled \item[{\em in}]Array of pointers to to object to be read; must all be to same underlying data set but with independent calibration \item[{\em data}]array of pointers to buffers to write results. \item[{\em err}]{\bf Obit\-Err}{\rm (p.\,\pageref{structObitErr})} for reporting errors. \end{description}
\end{Desc}
\begin{Desc}
\item[Returns:]return code, OBIT\_\-IO\_\-OK=$>$ OK \end{Desc}
\index{ObitIOUVAIPS.h@{Obit\-IOUVAIPS.h}!ObitIOUVAIPSReadSelect@{ObitIOUVAIPSReadSelect}}
\index{ObitIOUVAIPSReadSelect@{ObitIOUVAIPSReadSelect}!ObitIOUVAIPS.h@{Obit\-IOUVAIPS.h}}
\subsubsection{\setlength{\rightskip}{0pt plus 5cm}Obit\-IOCode Obit\-IOUVAIPSRead\-Select ({\bf Obit\-IOUVAIPS} $\ast$ {\em in}, {\bf ofloat} $\ast$ {\em data}, {\bf Obit\-Err} $\ast$ {\em err})}\label{ObitIOUVAIPS_8h_a14}


Public: Read/cal/select. 

The number read will be my\-Sel-$>$n\-Vis\-PIO (until the end of the selected range of visibilities in which case it will be smaller). The first visibility number after a read is my\-Desc-$>$first\-Vis and the number of visibilities is my\-Desc-$>$num\-Vis\-Buff (which may be zero). The number attempted in a read is my\-Sel-$>$num\-Vis\-Read. When OBIT\_\-IO\_\-EOF is returned all data has been read (then is no new data in data) and the I/O has been closed. \begin{Desc}
\item[Parameters:]
\begin{description}
\item[{\em in}]Pointer to object to be read. \item[{\em data}]pointer to buffer to write results. \item[{\em err}]{\bf Obit\-Err}{\rm (p.\,\pageref{structObitErr})} for reporting errors. \end{description}
\end{Desc}
\begin{Desc}
\item[Returns:]return code, OBIT\_\-IO\_\-OK =$>$ OK \end{Desc}
\index{ObitIOUVAIPS.h@{Obit\-IOUVAIPS.h}!ObitIOUVAIPSRename@{ObitIOUVAIPSRename}}
\index{ObitIOUVAIPSRename@{ObitIOUVAIPSRename}!ObitIOUVAIPS.h@{Obit\-IOUVAIPS.h}}
\subsubsection{\setlength{\rightskip}{0pt plus 5cm}void Obit\-IOUVAIPSRename ({\bf Obit\-IO} $\ast$ {\em in}, {\bf Obit\-Info\-List} $\ast$ {\em info}, {\bf Obit\-Err} $\ast$ {\em err})}\label{ObitIOUVAIPS_8h_a7}


Public: Rename underlying structures. 

New name information is given on the info member: \begin{itemize}
\item \char`\"{}new\-Name\char`\"{} OBIT\_\-string (12,1,1) New AIPS Name absent or Blank = don't change \item \char`\"{}new\-Class\char`\"{} OBIT\_\-string (6,1,1) New AIPS Class absent or Blank = don't change\-O \item \char`\"{}new\-Seq\char`\"{} OBIT\_\-int (1,1,1) New AIPS Sequence 0 =$>$ unique value \begin{Desc}
\item[Parameters:]
\begin{description}
\item[{\em in}]Pointer to object to be zapped. \item[{\em info}]Associated {\bf Obit\-Info\-List}{\rm (p.\,\pageref{structObitInfoList})} \end{description}
\end{Desc}
\item \char`\"{}Disk\char`\"{} OBIT\_\-int (1,1,1) Disk number \item \char`\"{}CNO\char`\"{} OBIT\_\-int (1,1,1) Catalog slot number \item \char`\"{}new\-Name\char`\"{} OBIT\_\-string (12,1,1) New AIPS Name absent or Blank = don't change \item \char`\"{}new\-Class\char`\"{} OBIT\_\-string (6,1,1) New AIPS Class absent or Blank = don't change\-O \item \char`\"{}new\-Seq\char`\"{} OBIT\_\-int (1,1,1) New AIPS Sequence 0 =$>$ unique value \begin{Desc}
\item[Parameters:]
\begin{description}
\item[{\em err}]{\bf Obit\-Err}{\rm (p.\,\pageref{structObitErr})} for reporting errors. \end{description}
\end{Desc}
\end{itemize}
\index{ObitIOUVAIPS.h@{Obit\-IOUVAIPS.h}!ObitIOUVAIPSReReadMulti@{ObitIOUVAIPSReReadMulti}}
\index{ObitIOUVAIPSReReadMulti@{ObitIOUVAIPSReReadMulti}!ObitIOUVAIPS.h@{Obit\-IOUVAIPS.h}}
\subsubsection{\setlength{\rightskip}{0pt plus 5cm}Obit\-IOCode Obit\-IOUVAIPSRe\-Read\-Multi ({\bf olong} {\em n\-Buff}, {\bf Obit\-IOUVAIPS} $\ast$$\ast$ {\em in}, {\bf ofloat} $\ast$$\ast$ {\em data}, {\bf Obit\-Err} $\ast$ {\em err})}\label{ObitIOUVAIPS_8h_a16}


Public: Reread to multiple buffers. 

Retreives data read in a previous call to Obit\-IOUVAIPSRead\-Multi NOTE: this depends on retreiving the data from the first element in in which should be the same as in the call to Obit\-IOUVAIPSRead\-Multi All buffers must be the same size and the underlying dataset the same. The number read will be my\-Sel-$>$n\-Vis\-PIO (until the end of the selected range of visibilities in which case it will be smaller). The first visibility number after a read is my\-Desc-$>$first\-Vis and the number of visibilities attempted is my\-Sel-$>$num\-Vis\-Read; actual value saved as my\-Desc-$>$num\-Vis\-Buff. When OBIT\_\-IO\_\-EOF is returned all data has been read (then is no new data in data) \begin{Desc}
\item[Parameters:]
\begin{description}
\item[{\em n\-Buff}]Number of buffers to be filled \item[{\em in}]Array of pointers to to object to be read; must all be to same underlying data set but with independent calibration \item[{\em data}]array of pointers to buffers to write results. \item[{\em err}]{\bf Obit\-Err}{\rm (p.\,\pageref{structObitErr})} for reporting errors. \end{description}
\end{Desc}
\begin{Desc}
\item[Returns:]return code, OBIT\_\-IO\_\-OK=$>$ OK \end{Desc}
\index{ObitIOUVAIPS.h@{Obit\-IOUVAIPS.h}!ObitIOUVAIPSReReadMultiSelect@{ObitIOUVAIPSReReadMultiSelect}}
\index{ObitIOUVAIPSReReadMultiSelect@{ObitIOUVAIPSReReadMultiSelect}!ObitIOUVAIPS.h@{Obit\-IOUVAIPS.h}}
\subsubsection{\setlength{\rightskip}{0pt plus 5cm}Obit\-IOCode Obit\-IOUVAIPSRe\-Read\-Multi\-Select ({\bf olong} {\em n\-Buff}, {\bf Obit\-IOUVAIPS} $\ast$$\ast$ {\em in}, {\bf ofloat} $\ast$$\ast$ {\em data}, {\bf Obit\-Err} $\ast$ {\em err})}\label{ObitIOUVAIPS_8h_a18}


Public: Reread/cal/select multiple buffers. 

Retreives data read in a previous call to Obit\-IOUVAIPSRead\-Multi\-Select possibly applying new calibration. If amp/phase calibration being applied, it is done independently for each buffer using the my\-Cal in the associated in, otherwise, data from the first buffer copied to the others. NOTE: this depends on retrieving the data from the first element in in which should be the same as in the call to Obit\-IOUVAIPSRead\-Multi\-Select All selected buffer sizes etc must be the same. The number read will be my\-Sel-$>$n\-Vis\-PIO (until the end of the selected range of visibilities in which case it will be smaller). The first visibility number after a read is my\-Desc-$>$first\-Vis and the number of visibilities is my\-Desc-$>$num\-Vis\-Buff (which may be zero). The number attempted in a read is my\-Sel-$>$num\-Vis\-Read. When OBIT\_\-IO\_\-EOF is returned all data has been read (then is no new data in data) and the I/O has been closed. \begin{Desc}
\item[Parameters:]
\begin{description}
\item[{\em n\-Buff}]Number of buffers to be filled \item[{\em in}]Array of pointers to to object to be read; must all be to same underlying data set but with independent calibration \item[{\em data}]array of pointers to buffers to write results. \item[{\em err}]{\bf Obit\-Err}{\rm (p.\,\pageref{structObitErr})} for reporting errors. \end{description}
\end{Desc}
\begin{Desc}
\item[Returns:]return code, OBIT\_\-IO\_\-OK=$>$ OK \end{Desc}
\index{ObitIOUVAIPS.h@{Obit\-IOUVAIPS.h}!ObitIOUVAIPSSame@{ObitIOUVAIPSSame}}
\index{ObitIOUVAIPSSame@{ObitIOUVAIPSSame}!ObitIOUVAIPS.h@{Obit\-IOUVAIPS.h}}
\subsubsection{\setlength{\rightskip}{0pt plus 5cm}gboolean Obit\-IOUVAIPSSame ({\bf Obit\-IO} $\ast$ {\em in}, {\bf Obit\-Info\-List} $\ast$ {\em in1}, {\bf Obit\-Info\-List} $\ast$ {\em in2}, {\bf Obit\-Err} $\ast$ {\em err})}\label{ObitIOUVAIPS_8h_a6}


Public: Are underlying structures the same. 

This test is done using values entered into the {\bf Obit\-Info\-List}{\rm (p.\,\pageref{structObitInfoList})} in case the object has not yet been opened. \begin{Desc}
\item[Parameters:]
\begin{description}
\item[{\em in}]{\bf Obit\-IO}{\rm (p.\,\pageref{structObitIO})} for test \item[{\em in1}]{\bf Obit\-Info\-List}{\rm (p.\,\pageref{structObitInfoList})} for first object to be tested \item[{\em in2}]{\bf Obit\-Info\-List}{\rm (p.\,\pageref{structObitInfoList})} for second object to be tested \item[{\em err}]{\bf Obit\-Err}{\rm (p.\,\pageref{structObitErr})} for reporting errors. \end{description}
\end{Desc}
\begin{Desc}
\item[Returns:]TRUE if to objects have the same underlying structures else FALSE \end{Desc}
\index{ObitIOUVAIPS.h@{Obit\-IOUVAIPS.h}!ObitIOUVAIPSSet@{ObitIOUVAIPSSet}}
\index{ObitIOUVAIPSSet@{ObitIOUVAIPSSet}!ObitIOUVAIPS.h@{Obit\-IOUVAIPS.h}}
\subsubsection{\setlength{\rightskip}{0pt plus 5cm}Obit\-IOCode Obit\-IOUVAIPSSet ({\bf Obit\-IOUVAIPS} $\ast$ {\em in}, {\bf Obit\-Info\-List} $\ast$ {\em info}, {\bf Obit\-Err} $\ast$ {\em err})}\label{ObitIOUVAIPS_8h_a12}


Public: Init I/O. 

\begin{Desc}
\item[Parameters:]
\begin{description}
\item[{\em in}]Pointer to object to be accessed. \item[{\em info}]{\bf Obit\-Info\-List}{\rm (p.\,\pageref{structObitInfoList})} with instructions \item[{\em err}]{\bf Obit\-Err}{\rm (p.\,\pageref{structObitErr})} for reporting errors. \end{description}
\end{Desc}
\begin{Desc}
\item[Returns:]return code, 0=$>$ OK \end{Desc}
\index{ObitIOUVAIPS.h@{Obit\-IOUVAIPS.h}!ObitIOUVAIPSUpdateTables@{ObitIOUVAIPSUpdateTables}}
\index{ObitIOUVAIPSUpdateTables@{ObitIOUVAIPSUpdateTables}!ObitIOUVAIPS.h@{Obit\-IOUVAIPS.h}}
\subsubsection{\setlength{\rightskip}{0pt plus 5cm}Obit\-IOCode Obit\-IOUVAIPSUpdate\-Tables ({\bf Obit\-IOUVAIPS} $\ast$ {\em in}, {\bf Obit\-Info\-List} $\ast$ {\em info}, {\bf Obit\-Err} $\ast$ {\em err})}\label{ObitIOUVAIPS_8h_a25}


Public: Update disk resident tables information. 

\begin{Desc}
\item[Parameters:]
\begin{description}
\item[{\em in}]Pointer to object to be updated. \item[{\em info}]{\bf Obit\-Info\-List}{\rm (p.\,\pageref{structObitInfoList})} of parent object (not used here). \item[{\em err}]{\bf Obit\-Err}{\rm (p.\,\pageref{structObitErr})} for reporting errors. \end{description}
\end{Desc}
\begin{Desc}
\item[Returns:]return code, OBIT\_\-IO\_\-OK=$>$ OK \end{Desc}
\index{ObitIOUVAIPS.h@{Obit\-IOUVAIPS.h}!ObitIOUVAIPSWrite@{ObitIOUVAIPSWrite}}
\index{ObitIOUVAIPSWrite@{ObitIOUVAIPSWrite}!ObitIOUVAIPS.h@{Obit\-IOUVAIPS.h}}
\subsubsection{\setlength{\rightskip}{0pt plus 5cm}Obit\-IOCode Obit\-IOUVAIPSWrite ({\bf Obit\-IOUVAIPS} $\ast$ {\em in}, {\bf ofloat} $\ast$ {\em data}, {\bf Obit\-Err} $\ast$ {\em err})}\label{ObitIOUVAIPS_8h_a19}


Public: Write. 

The data in the buffer will be written starting at visibility my\-Desc-$>$first\-Vis and the number written will be my\-Desc-$>$num\-Vis\-Buff which should not exceed my\-Sel-$>$n\-Vis\-PIO if the internal buffer is used. my\-Desc-$>$first\-Vis will be maintained and need not be changed for sequential writing. \begin{Desc}
\item[Parameters:]
\begin{description}
\item[{\em in}]Pointer to object to be written. \item[{\em data}]pointer to buffer containing input data. \item[{\em err}]{\bf Obit\-Err}{\rm (p.\,\pageref{structObitErr})} for reporting errors. \end{description}
\end{Desc}
\begin{Desc}
\item[Returns:]return code, 0=$>$ OK \end{Desc}
\index{ObitIOUVAIPS.h@{Obit\-IOUVAIPS.h}!ObitIOUVAIPSWriteDescriptor@{ObitIOUVAIPSWriteDescriptor}}
\index{ObitIOUVAIPSWriteDescriptor@{ObitIOUVAIPSWriteDescriptor}!ObitIOUVAIPS.h@{Obit\-IOUVAIPS.h}}
\subsubsection{\setlength{\rightskip}{0pt plus 5cm}Obit\-IOCode Obit\-IOUVAIPSWrite\-Descriptor ({\bf Obit\-IOUVAIPS} $\ast$ {\em in}, {\bf Obit\-Err} $\ast$ {\em err})}\label{ObitIOUVAIPS_8h_a22}


Public: Write Descriptor. 

\begin{Desc}
\item[Parameters:]
\begin{description}
\item[{\em in}]Pointer to object with {\bf Obit\-UVDesc}{\rm (p.\,\pageref{structObitUVDesc})} to be written. \item[{\em err}]{\bf Obit\-Err}{\rm (p.\,\pageref{structObitErr})} for reporting errors. \end{description}
\end{Desc}
\begin{Desc}
\item[Returns:]return code, 0=$>$ OK \end{Desc}
\index{ObitIOUVAIPS.h@{Obit\-IOUVAIPS.h}!ObitIOUVAIPSZap@{ObitIOUVAIPSZap}}
\index{ObitIOUVAIPSZap@{ObitIOUVAIPSZap}!ObitIOUVAIPS.h@{Obit\-IOUVAIPS.h}}
\subsubsection{\setlength{\rightskip}{0pt plus 5cm}void Obit\-IOUVAIPSZap ({\bf Obit\-IOUVAIPS} $\ast$ {\em in}, {\bf Obit\-Err} $\ast$ {\em err})}\label{ObitIOUVAIPS_8h_a8}


Public: Delete underlying structures. 

\begin{Desc}
\item[Parameters:]
\begin{description}
\item[{\em in}]Pointer to object to be zapped. \item[{\em err}]{\bf Obit\-Err}{\rm (p.\,\pageref{structObitErr})} for reporting errors. \end{description}
\end{Desc}
