\section{Obit.c File Reference}
\label{Obit_8c}\index{Obit.c@{Obit.c}}
{\bf Obit}{\rm (p.\,\pageref{structObit})} class function definitions. 

{\tt \#include $<$time.h$>$}\par
{\tt \#include \char`\"{}Obit.h\char`\"{}}\par
{\tt \#include \char`\"{}Obit\-Mem.h\char`\"{}}\par
\subsection*{Functions}
\begin{CompactItemize}
\item 
{\bf Obit} $\ast$ {\bf new\-Obit} (gchar $\ast$name)
\begin{CompactList}\small\item\em Public: Constructor. \item\end{CompactList}\item 
gconstpointer {\bf Obit\-Get\-Class} (void)
\begin{CompactList}\small\item\em Public: Class\-Info pointer. \item\end{CompactList}\item 
{\bf Obit} $\ast$ {\bf Obit\-Copy} ({\bf Obit} $\ast$in, {\bf Obit} $\ast$out, {\bf Obit\-Err} $\ast$err)
\begin{CompactList}\small\item\em Public: Copy (deep) constructor. \item\end{CompactList}\item 
{\bf Obit} $\ast$ {\bf Obit\-Clone} ({\bf Obit} $\ast$in, {\bf Obit} $\ast$out)
\begin{CompactList}\small\item\em Public: Copy (shallow) constructor. \item\end{CompactList}\item 
gpointer {\bf Obit\-Ref} (gpointer in)
\begin{CompactList}\small\item\em Public: Ref pointer, increment reference count, return pointer. \item\end{CompactList}\item 
gpointer {\bf Obit\-Unref} (gpointer inn)
\begin{CompactList}\small\item\em Public: Unref pointer, decrement reference count and destroy if 0. \item\end{CompactList}\item 
gboolean {\bf Obit\-Is\-A} (gpointer in, gconstpointer class)
\begin{CompactList}\small\item\em Public: returns TRUE is object is a member of my\-Class\-Info or a derived class. \item\end{CompactList}\item 
{\bf ofloat} {\bf Obit\-Magic\-F} (void)
\begin{CompactList}\small\item\em Public: returns magic value blanking value. \item\end{CompactList}\item 
void {\bf Obit\-Trim\-Trail} (gchar $\ast$str)
\begin{CompactList}\small\item\em Public: trim trailing blanks from string. \item\end{CompactList}\item 
gboolean {\bf Obit\-Str\-Cmp} (gchar $\ast$str1, gchar $\ast$str2, {\bf olong} maxlen)
\begin{CompactList}\small\item\em Public: compare strings. \item\end{CompactList}\item 
gchar $\ast$ {\bf Obit\-Today} (void)
\begin{CompactList}\small\item\em Public: return today's date as yyyy-mm-dd. \item\end{CompactList}\item 
void {\bf Obit\-Class\-Init} (void)
\begin{CompactList}\small\item\em Public: Class initializer. \item\end{CompactList}\item 
void {\bf Obit\-Class\-Info\-Def\-Fn} (gpointer in\-Class)
\begin{CompactList}\small\item\em Public: Set Class function pointers. \item\end{CompactList}\item 
gboolean {\bf Obit\-Info\-Is\-A} ({\bf Obit\-Class\-Info} $\ast$class, {\bf Obit\-Class\-Info} $\ast$type)
\begin{CompactList}\small\item\em Public: returns TRUE is object is type or a derived class. \item\end{CompactList}\end{CompactItemize}
\subsection*{Variables}
\begin{CompactItemize}
\item 
{\bf Obit\-Class\-Info} {\bf my\-Class\-Info} = \{FALSE\}
\begin{CompactList}\small\item\em Class\-Info structure {\bf Obit\-Class\-Info}{\rm (p.\,\pageref{structObitClassInfo})}. \item\end{CompactList}\end{CompactItemize}


\subsection{Detailed Description}
{\bf Obit}{\rm (p.\,\pageref{structObit})} class function definitions. 



\subsection{Function Documentation}
\index{Obit.c@{Obit.c}!newObit@{newObit}}
\index{newObit@{newObit}!Obit.c@{Obit.c}}
\subsubsection{\setlength{\rightskip}{0pt plus 5cm}{\bf Obit}$\ast$ new\-Obit (gchar $\ast$ {\em name})}\label{Obit_8c_a4}


Public: Constructor. 

Initializes class if needed on first call. \begin{Desc}
\item[Parameters:]
\begin{description}
\item[{\em name}]An optional name for the object. \end{description}
\end{Desc}
\begin{Desc}
\item[Returns:]the new object. \end{Desc}
\index{Obit.c@{Obit.c}!ObitClassInfoDefFn@{ObitClassInfoDefFn}}
\index{ObitClassInfoDefFn@{ObitClassInfoDefFn}!Obit.c@{Obit.c}}
\subsubsection{\setlength{\rightskip}{0pt plus 5cm}void Obit\-Class\-Info\-Def\-Fn (gpointer {\em in\-Class})}\label{Obit_8c_a16}


Public: Set Class function pointers. 

\begin{Desc}
\item[Parameters:]
\begin{description}
\item[{\em in\-Class}]Pointer to Class\-Info structure of the class to be filled. \item[{\em call\-Class}]Pointer to Class\-Info of calling class \end{description}
\end{Desc}
\index{Obit.c@{Obit.c}!ObitClassInit@{ObitClassInit}}
\index{ObitClassInit@{ObitClassInit}!Obit.c@{Obit.c}}
\subsubsection{\setlength{\rightskip}{0pt plus 5cm}void Obit\-Class\-Init (void)}\label{Obit_8c_a15}


Public: Class initializer. 

\index{Obit.c@{Obit.c}!ObitClone@{ObitClone}}
\index{ObitClone@{ObitClone}!Obit.c@{Obit.c}}
\subsubsection{\setlength{\rightskip}{0pt plus 5cm}{\bf Obit}$\ast$ Obit\-Clone ({\bf Obit} $\ast$ {\em in}, {\bf Obit} $\ast$ {\em out})}\label{Obit_8c_a7}


Public: Copy (shallow) constructor. 

The result will have pointers to the more complex members. \begin{Desc}
\item[Parameters:]
\begin{description}
\item[{\em in}]The object to copy \item[{\em out}]An existing object pointer for output or NULL if none exists. \end{description}
\end{Desc}
\begin{Desc}
\item[Returns:]pointer to the new object. \end{Desc}
\index{Obit.c@{Obit.c}!ObitCopy@{ObitCopy}}
\index{ObitCopy@{ObitCopy}!Obit.c@{Obit.c}}
\subsubsection{\setlength{\rightskip}{0pt plus 5cm}{\bf Obit}$\ast$ Obit\-Copy ({\bf Obit} $\ast$ {\em in}, {\bf Obit} $\ast$ {\em out}, {\bf Obit\-Err} $\ast$ {\em err})}\label{Obit_8c_a6}


Public: Copy (deep) constructor. 

Copies are made of complex members such as files; these will be copied applying whatever selection is associated with the input. \begin{Desc}
\item[Parameters:]
\begin{description}
\item[{\em in}]The object to copy \item[{\em out}]An existing object pointer for output or NULL if none exists. \item[{\em err}]Error stack, returns if not empty. \end{description}
\end{Desc}
\begin{Desc}
\item[Returns:]pointer to the new (existing) object. \end{Desc}
\index{Obit.c@{Obit.c}!ObitGetClass@{ObitGetClass}}
\index{ObitGetClass@{ObitGetClass}!Obit.c@{Obit.c}}
\subsubsection{\setlength{\rightskip}{0pt plus 5cm}gconstpointer Obit\-Get\-Class (void)}\label{Obit_8c_a5}


Public: Class\-Info pointer. 

This method MUST be included in each derived class to ensure proper linking and class initialization. Initializes class if needed on first call. \begin{Desc}
\item[Returns:]pointer to the class structure. \end{Desc}
\index{Obit.c@{Obit.c}!ObitInfoIsA@{ObitInfoIsA}}
\index{ObitInfoIsA@{ObitInfoIsA}!Obit.c@{Obit.c}}
\subsubsection{\setlength{\rightskip}{0pt plus 5cm}gboolean Obit\-Info\-Is\-A ({\bf Obit\-Class\-Info} $\ast$ {\em class}, {\bf Obit\-Class\-Info} $\ast$ {\em type})}\label{Obit_8c_a17}


Public: returns TRUE is object is type or a derived class. 

\begin{Desc}
\item[Parameters:]
\begin{description}
\item[{\em in}]Pointer to object to test. \item[{\em class}]Pointer to Class\-Info structure of the class to be tested. \end{description}
\end{Desc}
\begin{Desc}
\item[Returns:]TRUE if test or a derived class, else FALSE. \end{Desc}
\index{Obit.c@{Obit.c}!ObitIsA@{ObitIsA}}
\index{ObitIsA@{ObitIsA}!Obit.c@{Obit.c}}
\subsubsection{\setlength{\rightskip}{0pt plus 5cm}gboolean Obit\-Is\-A (gpointer {\em in}, gconstpointer {\em class})}\label{Obit_8c_a10}


Public: returns TRUE is object is a member of my\-Class\-Info or a derived class. 

Should also work for derived classes. \begin{Desc}
\item[Parameters:]
\begin{description}
\item[{\em in}]Pointer to object to test. \item[{\em class}]Pointer to Class\-Info structure of the class to be tested. \end{description}
\end{Desc}
\begin{Desc}
\item[Returns:]TRUE if member of class or a derived class, else FALSE. \end{Desc}
\index{Obit.c@{Obit.c}!ObitMagicF@{ObitMagicF}}
\index{ObitMagicF@{ObitMagicF}!Obit.c@{Obit.c}}
\subsubsection{\setlength{\rightskip}{0pt plus 5cm}{\bf ofloat} Obit\-Magic\-F (void)}\label{Obit_8c_a11}


Public: returns magic value blanking value. 

\begin{Desc}
\item[Returns:]float magic value \end{Desc}
\index{Obit.c@{Obit.c}!ObitRef@{ObitRef}}
\index{ObitRef@{ObitRef}!Obit.c@{Obit.c}}
\subsubsection{\setlength{\rightskip}{0pt plus 5cm}gpointer Obit\-Ref (gpointer {\em in})}\label{Obit_8c_a8}


Public: Ref pointer, increment reference count, return pointer. 

This function should always be used to copy pointers as this will ensure a proper reference count. Should also work for derived classes \begin{Desc}
\item[Parameters:]
\begin{description}
\item[{\em in}]Pointer to object to link, if Null, just return. \end{description}
\end{Desc}
\begin{Desc}
\item[Returns:]the pointer to in. \end{Desc}
\index{Obit.c@{Obit.c}!ObitStrCmp@{ObitStrCmp}}
\index{ObitStrCmp@{ObitStrCmp}!Obit.c@{Obit.c}}
\subsubsection{\setlength{\rightskip}{0pt plus 5cm}gboolean Obit\-Str\-Cmp (gchar $\ast$ {\em str1}, gchar $\ast$ {\em str2}, {\bf olong} {\em maxlen})}\label{Obit_8c_a13}


Public: compare strings. 

Blanks past the last non blank, non-NULL are considered insignificant \begin{Desc}
\item[Parameters:]
\begin{description}
\item[{\em str1}]First string to compare \item[{\em str2}]Second string to compare \item[{\em maxlen}]Maximum number of characters to compare \end{description}
\end{Desc}
\begin{Desc}
\item[Returns:]True if all significant characters match, else False \end{Desc}
\index{Obit.c@{Obit.c}!ObitToday@{ObitToday}}
\index{ObitToday@{ObitToday}!Obit.c@{Obit.c}}
\subsubsection{\setlength{\rightskip}{0pt plus 5cm}gchar$\ast$ Obit\-Today (void)}\label{Obit_8c_a14}


Public: return today's date as yyyy-mm-dd. 

\begin{Desc}
\item[Returns:]data string, should be g\_\-freeed when done. \end{Desc}
\index{Obit.c@{Obit.c}!ObitTrimTrail@{ObitTrimTrail}}
\index{ObitTrimTrail@{ObitTrimTrail}!Obit.c@{Obit.c}}
\subsubsection{\setlength{\rightskip}{0pt plus 5cm}void Obit\-Trim\-Trail (gchar $\ast$ {\em str})}\label{Obit_8c_a12}


Public: trim trailing blanks from string. 

\begin{Desc}
\item[Parameters:]
\begin{description}
\item[{\em str}]String to trim \end{description}
\end{Desc}
\index{Obit.c@{Obit.c}!ObitUnref@{ObitUnref}}
\index{ObitUnref@{ObitUnref}!Obit.c@{Obit.c}}
\subsubsection{\setlength{\rightskip}{0pt plus 5cm}gpointer Obit\-Unref (gpointer {\em inn})}\label{Obit_8c_a9}


Public: Unref pointer, decrement reference count and destroy if 0. 

if the input pointer is NULL, the reference count is already $<$=0 or the object is not a valid {\bf Obit}{\rm (p.\,\pageref{structObit})} Object, it simply returns. \begin{Desc}
\item[Parameters:]
\begin{description}
\item[{\em in}]Pointer to object to unreference. \end{description}
\end{Desc}
\begin{Desc}
\item[Returns:]NULL pointer. \end{Desc}


\subsection{Variable Documentation}
\index{Obit.c@{Obit.c}!myClassInfo@{myClassInfo}}
\index{myClassInfo@{myClassInfo}!Obit.c@{Obit.c}}
\subsubsection{\setlength{\rightskip}{0pt plus 5cm}{\bf Obit\-Class\-Info} {\bf my\-Class\-Info} = \{FALSE\}}\label{Obit_8c_a1}


Class\-Info structure {\bf Obit\-Class\-Info}{\rm (p.\,\pageref{structObitClassInfo})}. 

This structure is used by class objects to access class functions. 