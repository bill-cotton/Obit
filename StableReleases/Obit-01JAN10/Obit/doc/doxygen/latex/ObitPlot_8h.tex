\section{Obit\-Plot.h File Reference}
\label{ObitPlot_8h}\index{ObitPlot.h@{ObitPlot.h}}
{\bf Obit\-Plot}{\rm (p.\,\pageref{structObitPlot})} {\bf Obit}{\rm (p.\,\pageref{structObit})} graphics class definition. 

{\tt \#include \char`\"{}Obit.h\char`\"{}}\par
{\tt \#include \char`\"{}Obit\-Err.h\char`\"{}}\par
{\tt \#include \char`\"{}Obit\-Thread.h\char`\"{}}\par
{\tt \#include \char`\"{}Obit\-Info\-List.h\char`\"{}}\par
{\tt \#include \char`\"{}Obit\-Image.h\char`\"{}}\par
\subsection*{Classes}
\begin{CompactItemize}
\item 
struct {\bf Obit\-Plot}
\begin{CompactList}\small\item\em Obit\-Plot Class structure. \item\end{CompactList}\item 
struct {\bf Obit\-Plot\-Class\-Info}
\begin{CompactList}\small\item\em Class\-Info Structure. \item\end{CompactList}\end{CompactItemize}
\subsection*{Defines}
\begin{CompactItemize}
\item 
\#define {\bf Obit\-Plot\-Unref}(in)\ Obit\-Unref (in)
\begin{CompactList}\small\item\em Macro to unreference (and possibly destroy) an {\bf Obit\-Plot}{\rm (p.\,\pageref{structObitPlot})} returns a Obit\-Plot$\ast$ (NULL). \item\end{CompactList}\item 
\#define {\bf Obit\-Plot\-Ref}(in)\ Obit\-Ref (in)
\begin{CompactList}\small\item\em Macro to reference (update reference count) an {\bf Obit\-Plot}{\rm (p.\,\pageref{structObitPlot})}. \item\end{CompactList}\item 
\#define {\bf Obit\-Plot\-Is\-A}(in)\ Obit\-Is\-A (in, Obit\-Plot\-Get\-Class())
\begin{CompactList}\small\item\em Macro to determine if an object is the member of this or a derived class. \item\end{CompactList}\end{CompactItemize}
\subsection*{Functions}
\begin{CompactItemize}
\item 
void {\bf Obit\-Plot\-Class\-Init} (void)
\begin{CompactList}\small\item\em Public: Class initializer. \item\end{CompactList}\item 
{\bf Obit\-Plot} $\ast$ {\bf new\-Obit\-Plot} (gchar $\ast$name)
\begin{CompactList}\small\item\em Public: Constructor. \item\end{CompactList}\item 
void {\bf Obit\-Plot\-Init\-Plot} ({\bf Obit\-Plot} $\ast$in, gchar $\ast$output, {\bf olong} color, {\bf olong} nx, {\bf olong} ny, {\bf Obit\-Err} $\ast$err)
\begin{CompactList}\small\item\em Public: Initialize plot. \item\end{CompactList}\item 
void {\bf Obit\-Plot\-Finish\-Plot} ({\bf Obit\-Plot} $\ast$in, {\bf Obit\-Err} $\ast$err)
\begin{CompactList}\small\item\em Public: Finalize plot. \item\end{CompactList}\item 
{\bf Obit\-Plot} $\ast$ {\bf Obit\-Plot\-Copy} ({\bf Obit\-Plot} $\ast$in, {\bf Obit\-Plot} $\ast$out, {\bf Obit\-Err} $\ast$err)
\begin{CompactList}\small\item\em Public: Copy Plot. \item\end{CompactList}\item 
gconstpointer {\bf Obit\-Plot\-Get\-Class} (void)
\begin{CompactList}\small\item\em Public: Return class pointer. \item\end{CompactList}\item 
void {\bf Obit\-Plot\-XYPlot} ({\bf Obit\-Plot} $\ast$in, {\bf olong} symbol, {\bf olong} n, {\bf ofloat} $\ast$x, {\bf ofloat} $\ast$y, {\bf Obit\-Err} $\ast$err)
\begin{CompactList}\small\item\em Public: Simple X-Y plot. \item\end{CompactList}\item 
void {\bf Obit\-Plot\-XYOver} ({\bf Obit\-Plot} $\ast$in, {\bf olong} symbol, {\bf olong} n, {\bf ofloat} $\ast$x, {\bf ofloat} $\ast$y, {\bf Obit\-Err} $\ast$err)
\begin{CompactList}\small\item\em Public: Simple X-Y over plot. \item\end{CompactList}\item 
void {\bf Obit\-Plot\-XYErr} ({\bf Obit\-Plot} $\ast$in, {\bf olong} symbol, {\bf olong} n, {\bf ofloat} $\ast$x, {\bf ofloat} $\ast$y, {\bf ofloat} $\ast$e, {\bf Obit\-Err} $\ast$err)
\begin{CompactList}\small\item\em Public: X-Y plot with error bars. \item\end{CompactList}\item 
void {\bf Obit\-Plot\-Contour} ({\bf Obit\-Plot} $\ast$in, gchar $\ast$label, {\bf Obit\-Image} $\ast$image, {\bf ofloat} lev, {\bf ofloat} cntfac, {\bf Obit\-Err} $\ast$err)
\begin{CompactList}\small\item\em Public: Contour plot. \item\end{CompactList}\item 
void {\bf Obit\-Plot\-Gray\-Scale} ({\bf Obit\-Plot} $\ast$in, gchar $\ast$label, {\bf Obit\-Image} $\ast$image, {\bf Obit\-Err} $\ast$err)
\begin{CompactList}\small\item\em Public: Gray scale plot. \item\end{CompactList}\item 
void {\bf Obit\-Plot\-Mark\-Cross} ({\bf Obit\-Plot} $\ast$in, {\bf Obit\-Image} $\ast$image, {\bf olong} n, {\bf odouble} $\ast$ra, {\bf odouble} $\ast$dec, {\bf ofloat} size, {\bf Obit\-Err} $\ast$err)
\begin{CompactList}\small\item\em Public: Mark positions on Contour plot. \item\end{CompactList}\item 
void {\bf Obit\-Plot\-Set\-Plot} ({\bf Obit\-Plot} $\ast$in, {\bf ofloat} xmin, {\bf ofloat} xmax, {\bf ofloat} ymin, {\bf ofloat} ymax, {\bf olong} just, {\bf olong} axis, {\bf Obit\-Err} $\ast$err)
\begin{CompactList}\small\item\em Public: set window and viewport and draw labeled frame. \item\end{CompactList}\item 
void {\bf Obit\-Plot\-Label} ({\bf Obit\-Plot} $\ast$in, gchar $\ast$xlabel, gchar $\ast$ylabel, gchar $\ast$title, {\bf Obit\-Err} $\ast$err)
\begin{CompactList}\small\item\em Public: write labels for x-axis, y-axis, and top of plot. \item\end{CompactList}\item 
void {\bf Obit\-Plot\-Draw\-Axes} ({\bf Obit\-Plot} $\ast$in, gchar $\ast$xopt, {\bf ofloat} xtick, {\bf olong} nxsub, gchar $\ast$yopt, {\bf ofloat} ytick, {\bf olong} nysub, {\bf Obit\-Err} $\ast$err)
\begin{CompactList}\small\item\em Public: draw labeled frame around viewport. \item\end{CompactList}\item 
void {\bf Obit\-Plot\-Set\-Char\-Size} ({\bf Obit\-Plot} $\ast$in, {\bf ofloat} cscale, {\bf Obit\-Err} $\ast$err)
\begin{CompactList}\small\item\em Public: Scaling for characters. \item\end{CompactList}\item 
void {\bf Obit\-Plot\-Set\-Line\-Width} ({\bf Obit\-Plot} $\ast$in, {\bf olong} lwidth, {\bf Obit\-Err} $\ast$err)
\begin{CompactList}\small\item\em Public: Set line width. \item\end{CompactList}\item 
void {\bf Obit\-Plot\-Set\-Line\-Style} ({\bf Obit\-Plot} $\ast$in, {\bf olong} lstyle, {\bf Obit\-Err} $\ast$err)
\begin{CompactList}\small\item\em Public: Set line style. \item\end{CompactList}\item 
void {\bf Obit\-Plot\-Set\-Color} ({\bf Obit\-Plot} $\ast$in, {\bf olong} color, {\bf Obit\-Err} $\ast$err)
\begin{CompactList}\small\item\em Public: Set foreground color. \item\end{CompactList}\item 
void {\bf Obit\-Plot\-Set\-Page} ({\bf Obit\-Plot} $\ast$in, {\bf olong} sub, {\bf Obit\-Err} $\ast$err)
\begin{CompactList}\small\item\em Public: Set (sub)page. \item\end{CompactList}\item 
void {\bf Obit\-Plot\-Text} ({\bf Obit\-Plot} $\ast$in, {\bf ofloat} x, {\bf ofloat} y, {\bf ofloat} dx, {\bf ofloat} dy, {\bf ofloat} just, gchar $\ast$text, {\bf Obit\-Err} $\ast$err)
\begin{CompactList}\small\item\em Public: Write text. \item\end{CompactList}\item 
void {\bf Obit\-Plot\-Rel\-Text} ({\bf Obit\-Plot} $\ast$in, gchar $\ast$side, {\bf ofloat} disp, {\bf ofloat} coord, {\bf ofloat} fjust, gchar $\ast$text, {\bf Obit\-Err} $\ast$err)
\begin{CompactList}\small\item\em Public: Write text relative to port. \item\end{CompactList}\item 
void {\bf Obit\-Plot\-Draw\-Line} ({\bf Obit\-Plot} $\ast$in, {\bf ofloat} x1, {\bf ofloat} y1, {\bf ofloat} x2, {\bf ofloat} y2, {\bf Obit\-Err} $\ast$err)
\begin{CompactList}\small\item\em Public: Draw a line. \item\end{CompactList}\item 
void {\bf Obit\-Plot\-Draw\-Curve} ({\bf Obit\-Plot} $\ast$in, {\bf olong} n, {\bf ofloat} $\ast$x, {\bf ofloat} $\ast$y, {\bf Obit\-Err} $\ast$err)
\begin{CompactList}\small\item\em Public: Draw a curve. \item\end{CompactList}\item 
void {\bf Obit\-Plot\-Draw\-Symbol} ({\bf Obit\-Plot} $\ast$in, {\bf ofloat} x, {\bf ofloat} y, {\bf olong} symbol, {\bf Obit\-Err} $\ast$err)
\begin{CompactList}\small\item\em Public: Draw a Symbol. \item\end{CompactList}\end{CompactItemize}


\subsection{Detailed Description}
{\bf Obit\-Plot}{\rm (p.\,\pageref{structObitPlot})} {\bf Obit}{\rm (p.\,\pageref{structObit})} graphics class definition. 

This class is derived from the {\bf Obit}{\rm (p.\,\pageref{structObit})} class.

This contains information about the observations and the size and structure of the data. Implementations using both pgplot and plplot with perference given to the second.\subsection{Usage}\label{ObitPlot_8h_ObitPlotUsage}
Instances can be obtained using the {\bf new\-Obit\-Plot}{\rm (p.\,\pageref{ObitPlot_8c_a6})} constructor the {\bf Obit\-Plot\-Copy}{\rm (p.\,\pageref{ObitPlot_8c_a9})} copy constructor or a pointer duplicated using the {\bf Obit\-Plot\-Ref}{\rm (p.\,\pageref{ObitPlot_8h_a1})} function. When an instance is no longer needed, use the {\bf Obit\-Plot\-Unref}{\rm (p.\,\pageref{ObitPlot_8h_a0})} macro to release it.

\subsection{Define Documentation}
\index{ObitPlot.h@{Obit\-Plot.h}!ObitPlotIsA@{ObitPlotIsA}}
\index{ObitPlotIsA@{ObitPlotIsA}!ObitPlot.h@{Obit\-Plot.h}}
\subsubsection{\setlength{\rightskip}{0pt plus 5cm}\#define Obit\-Plot\-Is\-A(in)\ Obit\-Is\-A (in, Obit\-Plot\-Get\-Class())}\label{ObitPlot_8h_a2}


Macro to determine if an object is the member of this or a derived class. 

Returns TRUE if a member, else FALSE in = object to reference \index{ObitPlot.h@{Obit\-Plot.h}!ObitPlotRef@{ObitPlotRef}}
\index{ObitPlotRef@{ObitPlotRef}!ObitPlot.h@{Obit\-Plot.h}}
\subsubsection{\setlength{\rightskip}{0pt plus 5cm}\#define Obit\-Plot\-Ref(in)\ Obit\-Ref (in)}\label{ObitPlot_8h_a1}


Macro to reference (update reference count) an {\bf Obit\-Plot}{\rm (p.\,\pageref{structObitPlot})}. 

returns a Obit\-Plot$\ast$. in = object to reference \index{ObitPlot.h@{Obit\-Plot.h}!ObitPlotUnref@{ObitPlotUnref}}
\index{ObitPlotUnref@{ObitPlotUnref}!ObitPlot.h@{Obit\-Plot.h}}
\subsubsection{\setlength{\rightskip}{0pt plus 5cm}\#define Obit\-Plot\-Unref(in)\ Obit\-Unref (in)}\label{ObitPlot_8h_a0}


Macro to unreference (and possibly destroy) an {\bf Obit\-Plot}{\rm (p.\,\pageref{structObitPlot})} returns a Obit\-Plot$\ast$ (NULL). 

\begin{itemize}
\item in = object to unreference. \end{itemize}


\subsection{Function Documentation}
\index{ObitPlot.h@{Obit\-Plot.h}!newObitPlot@{newObitPlot}}
\index{newObitPlot@{newObitPlot}!ObitPlot.h@{Obit\-Plot.h}}
\subsubsection{\setlength{\rightskip}{0pt plus 5cm}{\bf Obit\-Plot}$\ast$ new\-Obit\-Plot (gchar $\ast$ {\em name})}\label{ObitPlot_8h_a4}


Public: Constructor. 

\begin{Desc}
\item[Returns:]pointer to object created. \end{Desc}
\index{ObitPlot.h@{Obit\-Plot.h}!ObitPlotClassInit@{ObitPlotClassInit}}
\index{ObitPlotClassInit@{ObitPlotClassInit}!ObitPlot.h@{Obit\-Plot.h}}
\subsubsection{\setlength{\rightskip}{0pt plus 5cm}void Obit\-Plot\-Class\-Init (void)}\label{ObitPlot_8h_a3}


Public: Class initializer. 

\index{ObitPlot.h@{Obit\-Plot.h}!ObitPlotContour@{ObitPlotContour}}
\index{ObitPlotContour@{ObitPlotContour}!ObitPlot.h@{Obit\-Plot.h}}
\subsubsection{\setlength{\rightskip}{0pt plus 5cm}void Obit\-Plot\-Contour ({\bf Obit\-Plot} $\ast$ {\em in}, gchar $\ast$ {\em label}, {\bf Obit\-Image} $\ast$ {\em image}, {\bf ofloat} {\em lev}, {\bf ofloat} {\em cntfac}, {\bf Obit\-Err} $\ast$ {\em err})}\label{ObitPlot_8h_a12}


Public: Contour plot. 

\begin{Desc}
\item[Parameters:]
\begin{description}
\item[{\em in}]Pointer to existing {\bf Obit\-Plot}{\rm (p.\,\pageref{structObitPlot})} object. \item[{\em label}]Label for plot \item[{\em image}]Image to plot (first plane in BLC,TRC) Rotated images aren't done quite right \item[{\em lev}]basic contour level (def 0.1 peak) \item[{\em cntfac}]Contour level factor (def sqrt(2) \item[{\em err}]{\bf Obit\-Err}{\rm (p.\,\pageref{structObitErr})} error stack\end{description}
\end{Desc}
Optional parameters on in-$>$info \begin{itemize}
\item XTICK (float) world coordinate interval between major tick marks on X axis. If xtick=0.0 [def], the interval is chosen. \item NXSUB (int) the number of subintervals to divide the major coordinate interval into. If xtick=0.0 or nxsub=0, the number is chosen. [def 0] \item YTICK (float) like xtick for the Y axis. \item NYSUB (int) like nxsub for the Y axis \item CSIZE (int) Scaling factor for characters(default = 1) \item LWIDTH (int) Line width (default = 1) \end{itemize}
\index{ObitPlot.h@{Obit\-Plot.h}!ObitPlotCopy@{ObitPlotCopy}}
\index{ObitPlotCopy@{ObitPlotCopy}!ObitPlot.h@{Obit\-Plot.h}}
\subsubsection{\setlength{\rightskip}{0pt plus 5cm}{\bf Obit\-Plot}$\ast$ Obit\-Plot\-Copy ({\bf Obit\-Plot} $\ast$ {\em in}, {\bf Obit\-Plot} $\ast$ {\em out}, {\bf Obit\-Err} $\ast$ {\em err})}\label{ObitPlot_8h_a7}


Public: Copy Plot. 

\begin{Desc}
\item[Parameters:]
\begin{description}
\item[{\em in}]Pointer to object to be copied. \item[{\em out}]Pointer to object to be written. If NULL then a new structure is created. \item[{\em err}]{\bf Obit\-Err}{\rm (p.\,\pageref{structObitErr})} error stack \end{description}
\end{Desc}
\begin{Desc}
\item[Returns:]Pointer to new object. \end{Desc}
\index{ObitPlot.h@{Obit\-Plot.h}!ObitPlotDrawAxes@{ObitPlotDrawAxes}}
\index{ObitPlotDrawAxes@{ObitPlotDrawAxes}!ObitPlot.h@{Obit\-Plot.h}}
\subsubsection{\setlength{\rightskip}{0pt plus 5cm}void Obit\-Plot\-Draw\-Axes ({\bf Obit\-Plot} $\ast$ {\em in}, gchar $\ast$ {\em xopt}, {\bf ofloat} {\em xtick}, {\bf olong} {\em nxsub}, gchar $\ast$ {\em yopt}, {\bf ofloat} {\em ytick}, {\bf olong} {\em nysub}, {\bf Obit\-Err} $\ast$ {\em err})}\label{ObitPlot_8h_a17}


Public: draw labeled frame around viewport. 

\begin{Desc}
\item[Parameters:]
\begin{description}
\item[{\em in}]Pointer to Plot object. \item[{\em xopt}]string of options for X (horizontal) axis of plot. Options are single letters, and may be in any order (see below). \item[{\em xtick}]world coordinate interval between major tick marks on X axis. If xtick=0.0, the interval is chosen. \item[{\em nxsub}]the number of subintervals to divide the major coordinate interval into. If xtick=0.0 or nxsub=0, the number is chosen. \item[{\em yopt}]string of options for Y (vertical) axis of plot. Coding is the same as for xopt. \item[{\em ytick}]like xtick for the Y axis. \item[{\em nysub}]like nxsub for the Y axis \item[{\em err}]{\bf Obit\-Err}{\rm (p.\,\pageref{structObitErr})} error stack\end{description}
\end{Desc}
Axis options: \begin{itemize}
\item A : draw Axis (X axis is horizontal line Y=0, Y axis is vertical line X=0). \item B : draw bottom (X) or left (Y) edge of frame. \item C : draw top (X) or right (Y) edge of frame. \item G : draw Grid of vertical (X) or horizontal (Y) lines \item I : Invert the tick marks; ie draw them outside the viewport instead of inside. \item L : label axis Logarithmically \item N : write Numeric labels in the conventional location below the viewport (X) or to the left of the viewport (Y). \item M : write numeric labels in the unconventional location above the viewport (X) or to the right of the viewport (Y). \item P : extend (\char`\"{}Project\char`\"{}) major tick marks outside the box (ignored if option I is specified) \item T : draw major Tick marks at the major coordinate interval. \item S : draw minor tick marks (Subticks). \end{itemize}
\index{ObitPlot.h@{Obit\-Plot.h}!ObitPlotDrawCurve@{ObitPlotDrawCurve}}
\index{ObitPlotDrawCurve@{ObitPlotDrawCurve}!ObitPlot.h@{Obit\-Plot.h}}
\subsubsection{\setlength{\rightskip}{0pt plus 5cm}void Obit\-Plot\-Draw\-Curve ({\bf Obit\-Plot} $\ast$ {\em in}, {\bf olong} {\em n}, {\bf ofloat} $\ast$ {\em x}, {\bf ofloat} $\ast$ {\em y}, {\bf Obit\-Err} $\ast$ {\em err})}\label{ObitPlot_8h_a26}


Public: Draw a curve. 

\begin{Desc}
\item[Parameters:]
\begin{description}
\item[{\em in}]Pointer to Plot object. \item[{\em n}]Number of points \item[{\em x}]Array of world x-coordinates of points \item[{\em y}]Array of world y-coordinates of points \item[{\em err}]{\bf Obit\-Err}{\rm (p.\,\pageref{structObitErr})} error stack \end{description}
\end{Desc}
\index{ObitPlot.h@{Obit\-Plot.h}!ObitPlotDrawLine@{ObitPlotDrawLine}}
\index{ObitPlotDrawLine@{ObitPlotDrawLine}!ObitPlot.h@{Obit\-Plot.h}}
\subsubsection{\setlength{\rightskip}{0pt plus 5cm}void Obit\-Plot\-Draw\-Line ({\bf Obit\-Plot} $\ast$ {\em in}, {\bf ofloat} {\em x1}, {\bf ofloat} {\em y1}, {\bf ofloat} {\em x2}, {\bf ofloat} {\em y2}, {\bf Obit\-Err} $\ast$ {\em err})}\label{ObitPlot_8h_a25}


Public: Draw a line. 

\begin{Desc}
\item[Parameters:]
\begin{description}
\item[{\em in}]Pointer to Plot object. \item[{\em x1}]world x-coordinate of the new pen position. \item[{\em y1}]world y-coordinate of the new pen position. \item[{\em x2}]world x-coordinate of the new pen position. \item[{\em y2}]world y-coordinate of the new pen position. \item[{\em err}]{\bf Obit\-Err}{\rm (p.\,\pageref{structObitErr})} error stack \end{description}
\end{Desc}
\index{ObitPlot.h@{Obit\-Plot.h}!ObitPlotDrawSymbol@{ObitPlotDrawSymbol}}
\index{ObitPlotDrawSymbol@{ObitPlotDrawSymbol}!ObitPlot.h@{Obit\-Plot.h}}
\subsubsection{\setlength{\rightskip}{0pt plus 5cm}void Obit\-Plot\-Draw\-Symbol ({\bf Obit\-Plot} $\ast$ {\em in}, {\bf ofloat} {\em x}, {\bf ofloat} {\em y}, {\bf olong} {\em symbol}, {\bf Obit\-Err} $\ast$ {\em err})}\label{ObitPlot_8h_a27}


Public: Draw a Symbol. 

\begin{Desc}
\item[Parameters:]
\begin{description}
\item[{\em in}]Pointer to Plot object. \item[{\em x}]world x-coordinate of the center of the symbol \item[{\em y}]world y-coordinate of the center of the symbol \item[{\em symbol}]Symbol index to use for plotting values in the range [1,12] are usable \begin{itemize}
\item 1 = dot \item 2 = plus \item 3 = $\ast$ \item 4 = open circle \item 5 = x \item 6 = open square \item 7 = open triangle \item 8 = open star \item 9 = filled triangle \item 10 = filled square \item 11 = filled circle \item 12 = filled star \end{itemize}
\item[{\em err}]{\bf Obit\-Err}{\rm (p.\,\pageref{structObitErr})} error stack \end{description}
\end{Desc}
\index{ObitPlot.h@{Obit\-Plot.h}!ObitPlotFinishPlot@{ObitPlotFinishPlot}}
\index{ObitPlotFinishPlot@{ObitPlotFinishPlot}!ObitPlot.h@{Obit\-Plot.h}}
\subsubsection{\setlength{\rightskip}{0pt plus 5cm}void Obit\-Plot\-Finish\-Plot ({\bf Obit\-Plot} $\ast$ {\em in}, {\bf Obit\-Err} $\ast$ {\em err})}\label{ObitPlot_8h_a6}


Public: Finalize plot. 

\begin{Desc}
\item[Parameters:]
\begin{description}
\item[{\em in}]Pointer to Plot object. \item[{\em err}]{\bf Obit\-Err}{\rm (p.\,\pageref{structObitErr})} error stack, return if existing error \end{description}
\end{Desc}
\index{ObitPlot.h@{Obit\-Plot.h}!ObitPlotGetClass@{ObitPlotGetClass}}
\index{ObitPlotGetClass@{ObitPlotGetClass}!ObitPlot.h@{Obit\-Plot.h}}
\subsubsection{\setlength{\rightskip}{0pt plus 5cm}gconstpointer Obit\-Plot\-Get\-Class (void)}\label{ObitPlot_8h_a8}


Public: Return class pointer. 

Initializes class if needed on first call. \begin{Desc}
\item[Returns:]pointer to the class structure. \end{Desc}
\index{ObitPlot.h@{Obit\-Plot.h}!ObitPlotGrayScale@{ObitPlotGrayScale}}
\index{ObitPlotGrayScale@{ObitPlotGrayScale}!ObitPlot.h@{Obit\-Plot.h}}
\subsubsection{\setlength{\rightskip}{0pt plus 5cm}void Obit\-Plot\-Gray\-Scale ({\bf Obit\-Plot} $\ast$ {\em in}, gchar $\ast$ {\em label}, {\bf Obit\-Image} $\ast$ {\em image}, {\bf Obit\-Err} $\ast$ {\em err})}\label{ObitPlot_8h_a13}


Public: Gray scale plot. 

\begin{Desc}
\item[Parameters:]
\begin{description}
\item[{\em in}]Pointer to existing {\bf Obit\-Plot}{\rm (p.\,\pageref{structObitPlot})} object. \item[{\em label}]Label for plot \item[{\em image}]Image to plot (first plane in BLC,TRC) Rotated images aren't done quite right \item[{\em err}]{\bf Obit\-Err}{\rm (p.\,\pageref{structObitErr})} error stack\end{description}
\end{Desc}
Optional parameters on in-$>$info \begin{itemize}
\item XTICK (float) world coordinate interval between major tick marks on X axis. If xtick=0.0 [def], the interval is chosen. \item NXSUB (int) the number of subintervals to divide the major coordinate interval into. If xtick=0.0 or nxsub=0, the number is chosen. [def 0] \item YTICK (float) like xtick for the Y axis. \item NYSUB (int) like nxsub for the Y axis \item CSIZE (int) Scaling factor for characters(default = 1) \item SQRT (bool) If present and true plot sqrt (pixel\_\-value) \item INVERT (bool) If present and true ionvert colors \item COLOR (string) Color scheme \char`\"{}GRAY\char`\"{}, CONTOUR\char`\"{}, \char`\"{}PHLAME\char`\"{} default \char`\"{}GRAY\char`\"{} \item PIX\_\-MAX (float) maximum pixel value [def min in image] \item PIX\_\-MIN (float) minimum pixel value [def max in image] \end{itemize}
\index{ObitPlot.h@{Obit\-Plot.h}!ObitPlotInitPlot@{ObitPlotInitPlot}}
\index{ObitPlotInitPlot@{ObitPlotInitPlot}!ObitPlot.h@{Obit\-Plot.h}}
\subsubsection{\setlength{\rightskip}{0pt plus 5cm}void Obit\-Plot\-Init\-Plot ({\bf Obit\-Plot} $\ast$ {\em in}, gchar $\ast$ {\em output}, {\bf olong} {\em color}, {\bf olong} {\em nx}, {\bf olong} {\em ny}, {\bf Obit\-Err} $\ast$ {\em err})}\label{ObitPlot_8h_a5}


Public: Initialize plot. 

\begin{Desc}
\item[Parameters:]
\begin{description}
\item[{\em in}]Pointer to object to be copied. \item[{\em output}]name and type of output device in form \char`\"{}filename/device\char`\"{} NULL =$>$ ask This doesn't work for pg\-Plot Devices for plplot \begin{itemize}
\item xwin X-Window (Xlib) \item gcw Gnome Canvas Widget \item ps Post\-Script File (monochrome) \item psc Post\-Script File (color) \item xfig Fig file \item png PNG file \item jpeg JPEG file \item gif GIF file \item null Null device \end{itemize}
\item[{\em color}]background Color index, not available for pgplot. 0=black, 1=red(default), 2=yellow, 3=green, 4=aquamarine, 5=pink, 6=wheat, 7=gray, 8=brown, 9=blue, 10=Blue\-Violet, 11=cyan, 12=turquoise 13=magenta, 14=salmon, 15=white \item[{\em nx}]Number of frames in x on page \item[{\em ny}]Number of frames in y on page \item[{\em err}]{\bf Obit\-Err}{\rm (p.\,\pageref{structObitErr})} error stack \end{description}
\end{Desc}
\index{ObitPlot.h@{Obit\-Plot.h}!ObitPlotLabel@{ObitPlotLabel}}
\index{ObitPlotLabel@{ObitPlotLabel}!ObitPlot.h@{Obit\-Plot.h}}
\subsubsection{\setlength{\rightskip}{0pt plus 5cm}void Obit\-Plot\-Label ({\bf Obit\-Plot} $\ast$ {\em in}, gchar $\ast$ {\em xlabel}, gchar $\ast$ {\em ylabel}, gchar $\ast$ {\em title}, {\bf Obit\-Err} $\ast$ {\em err})}\label{ObitPlot_8h_a16}


Public: write labels for x-axis, y-axis, and top of plot. 

write labels for x-axis, y-axis, and top of plot Write labels outside the viewport. This routine is a simple interface to PGMTXT, which should be used if PGLAB is inadequate. \begin{Desc}
\item[Parameters:]
\begin{description}
\item[{\em in}]Pointer to Plot object. \item[{\em xlabel}]a label for the x-axis (centered below the viewport). \item[{\em ylabel}]a label for the y-axis (centered to the left of the viewport, drawn vertically) \item[{\em title}]a label for the entire plot (centered above the viewport) \item[{\em err}]{\bf Obit\-Err}{\rm (p.\,\pageref{structObitErr})} error stack \end{description}
\end{Desc}
\index{ObitPlot.h@{Obit\-Plot.h}!ObitPlotMarkCross@{ObitPlotMarkCross}}
\index{ObitPlotMarkCross@{ObitPlotMarkCross}!ObitPlot.h@{Obit\-Plot.h}}
\subsubsection{\setlength{\rightskip}{0pt plus 5cm}void Obit\-Plot\-Mark\-Cross ({\bf Obit\-Plot} $\ast$ {\em in}, {\bf Obit\-Image} $\ast$ {\em image}, {\bf olong} {\em n}, {\bf odouble} $\ast$ {\em ra}, {\bf odouble} $\ast$ {\em dec}, {\bf ofloat} {\em size}, {\bf Obit\-Err} $\ast$ {\em err})}\label{ObitPlot_8h_a14}


Public: Mark positions on Contour plot. 

\begin{Desc}
\item[Parameters:]
\begin{description}
\item[{\em in}]Pointer to existing {\bf Obit\-Plot}{\rm (p.\,\pageref{structObitPlot})} object. \item[{\em image}]Image plotted Descriptor assumed valid \item[{\em n}]number of positions to plot \item[{\em ra}]RAs (deg) to plot \item[{\em dec}]Declinations to plot \item[{\em size}]size of symbol in pixels \item[{\em err}]{\bf Obit\-Err}{\rm (p.\,\pageref{structObitErr})} error stack\end{description}
\end{Desc}
Optional parameters on in-$>$info \begin{itemize}
\item CSIZE (int) Scaling factor for characters(default = 1) \item LWIDTH (int) Line width (default = 1) \end{itemize}
\index{ObitPlot.h@{Obit\-Plot.h}!ObitPlotRelText@{ObitPlotRelText}}
\index{ObitPlotRelText@{ObitPlotRelText}!ObitPlot.h@{Obit\-Plot.h}}
\subsubsection{\setlength{\rightskip}{0pt plus 5cm}void Obit\-Plot\-Rel\-Text ({\bf Obit\-Plot} $\ast$ {\em in}, gchar $\ast$ {\em side}, {\bf ofloat} {\em disp}, {\bf ofloat} {\em coord}, {\bf ofloat} {\em fjust}, gchar $\ast$ {\em text}, {\bf Obit\-Err} $\ast$ {\em err})}\label{ObitPlot_8h_a24}


Public: Write text relative to port. 

This routine is useful for annotating graphs. The text is written using the current values of attributes color-index, line-width, character-height, and character-font. \begin{Desc}
\item[Parameters:]
\begin{description}
\item[{\em in}]Pointer to Plot object. \item[{\em side}]Must include one of the characters 'B', 'L', 'T', or 'R' signifying the Bottom, Left, Top, or Right margin of the viewport. If it includes 'LV' or 'RV', the string is written perpendicular to the frame rather than parallel to it. \item[{\em disp}]The displacement of the character string from the specified edge of the viewport, measured outwards from the viewport in units of the character height. Use a negative value to write inside the viewport, a positive value to write outside. \item[{\em coord}]The location of the character string along the specified edge of the viewport, as a fraction of the length of the edge. \item[{\em just}]Controls justification of the string parallel to the specified edge of the viewport. If just = 0.0, the left-hand end of the string will be placed at coord; if just = 0.5, the center of the string will be placed at coord; if just = 1.0, the right-hand end of the string will be placed at at coord. Other values between 0 and 1 give inter- mediate placing, but they are not very useful. \item[{\em text}]The text string to be plotted. Trailing spaces are ignored when justifying the string, but leading spaces are significant. \item[{\em err}]{\bf Obit\-Err}{\rm (p.\,\pageref{structObitErr})} error stack \end{description}
\end{Desc}
\index{ObitPlot.h@{Obit\-Plot.h}!ObitPlotSetCharSize@{ObitPlotSetCharSize}}
\index{ObitPlotSetCharSize@{ObitPlotSetCharSize}!ObitPlot.h@{Obit\-Plot.h}}
\subsubsection{\setlength{\rightskip}{0pt plus 5cm}void Obit\-Plot\-Set\-Char\-Size ({\bf Obit\-Plot} $\ast$ {\em in}, {\bf ofloat} {\em cscale}, {\bf Obit\-Err} $\ast$ {\em err})}\label{ObitPlot_8h_a18}


Public: Scaling for characters. 

\begin{Desc}
\item[Parameters:]
\begin{description}
\item[{\em in}]Pointer to Plot object. \item[{\em cscale}]new character size (dimensionless multiple of the default size). \item[{\em err}]{\bf Obit\-Err}{\rm (p.\,\pageref{structObitErr})} error stack \end{description}
\end{Desc}
\index{ObitPlot.h@{Obit\-Plot.h}!ObitPlotSetColor@{ObitPlotSetColor}}
\index{ObitPlotSetColor@{ObitPlotSetColor}!ObitPlot.h@{Obit\-Plot.h}}
\subsubsection{\setlength{\rightskip}{0pt plus 5cm}void Obit\-Plot\-Set\-Color ({\bf Obit\-Plot} $\ast$ {\em in}, {\bf olong} {\em color}, {\bf Obit\-Err} $\ast$ {\em err})}\label{ObitPlot_8h_a21}


Public: Set foreground color. 

\begin{Desc}
\item[Parameters:]
\begin{description}
\item[{\em in}]Pointer to Plot object. \item[{\em color}]color index (1-15) 0=black, 1=red(default), 2=yellow, 3=green, 4=aquamarine, 5=pink, 6=wheat, 7=gray, 8=brown, 9=blue, 10=Blue\-Violet, 11=cyan, 12=turquoise 13=magenta, 14=salmon, 15=white \item[{\em err}]{\bf Obit\-Err}{\rm (p.\,\pageref{structObitErr})} error stack \end{description}
\end{Desc}
\index{ObitPlot.h@{Obit\-Plot.h}!ObitPlotSetLineStyle@{ObitPlotSetLineStyle}}
\index{ObitPlotSetLineStyle@{ObitPlotSetLineStyle}!ObitPlot.h@{Obit\-Plot.h}}
\subsubsection{\setlength{\rightskip}{0pt plus 5cm}void Obit\-Plot\-Set\-Line\-Style ({\bf Obit\-Plot} $\ast$ {\em in}, {\bf olong} {\em lstyle}, {\bf Obit\-Err} $\ast$ {\em err})}\label{ObitPlot_8h_a20}


Public: Set line style. 

\begin{Desc}
\item[Parameters:]
\begin{description}
\item[{\em in}]Pointer to Plot object. Actually applies to all \item[{\em lstyle}]Style of line, 1 = continious, 2 = dashed, 3=dot dash 4 = dotted, 5 = dash dot dot dot \item[{\em err}]{\bf Obit\-Err}{\rm (p.\,\pageref{structObitErr})} error stack \end{description}
\end{Desc}
\index{ObitPlot.h@{Obit\-Plot.h}!ObitPlotSetLineWidth@{ObitPlotSetLineWidth}}
\index{ObitPlotSetLineWidth@{ObitPlotSetLineWidth}!ObitPlot.h@{Obit\-Plot.h}}
\subsubsection{\setlength{\rightskip}{0pt plus 5cm}void Obit\-Plot\-Set\-Line\-Width ({\bf Obit\-Plot} $\ast$ {\em in}, {\bf olong} {\em lwidth}, {\bf Obit\-Err} $\ast$ {\em err})}\label{ObitPlot_8h_a19}


Public: Set line width. 

\begin{Desc}
\item[Parameters:]
\begin{description}
\item[{\em in}]Pointer to Plot object. \item[{\em lwidth}]Width of line, multiple of default \item[{\em err}]{\bf Obit\-Err}{\rm (p.\,\pageref{structObitErr})} error stack \end{description}
\end{Desc}
\index{ObitPlot.h@{Obit\-Plot.h}!ObitPlotSetPage@{ObitPlotSetPage}}
\index{ObitPlotSetPage@{ObitPlotSetPage}!ObitPlot.h@{Obit\-Plot.h}}
\subsubsection{\setlength{\rightskip}{0pt plus 5cm}void Obit\-Plot\-Set\-Page ({\bf Obit\-Plot} $\ast$ {\em in}, {\bf olong} {\em sub}, {\bf Obit\-Err} $\ast$ {\em err})}\label{ObitPlot_8h_a22}


Public: Set (sub)page. 

\begin{Desc}
\item[Parameters:]
\begin{description}
\item[{\em in}]Pointer to Plot object. \item[{\em sub}]if $<$=0 advance page, if $>$0 set current subpage to sub \item[{\em err}]{\bf Obit\-Err}{\rm (p.\,\pageref{structObitErr})} error stack \end{description}
\end{Desc}
\index{ObitPlot.h@{Obit\-Plot.h}!ObitPlotSetPlot@{ObitPlotSetPlot}}
\index{ObitPlotSetPlot@{ObitPlotSetPlot}!ObitPlot.h@{Obit\-Plot.h}}
\subsubsection{\setlength{\rightskip}{0pt plus 5cm}void Obit\-Plot\-Set\-Plot ({\bf Obit\-Plot} $\ast$ {\em in}, {\bf ofloat} {\em xmin}, {\bf ofloat} {\em xmax}, {\bf ofloat} {\em ymin}, {\bf ofloat} {\em ymax}, {\bf olong} {\em just}, {\bf olong} {\em axis}, {\bf Obit\-Err} $\ast$ {\em err})}\label{ObitPlot_8h_a15}


Public: set window and viewport and draw labeled frame. 

\begin{Desc}
\item[Parameters:]
\begin{description}
\item[{\em in}]Pointer to Plot object. \item[{\em xmin}]the world x-coordinate at the bottom left corner of the viewport. \item[{\em xmax}]the world x-coordinate at the top right corner of the viewport (note XMAX may be less than XMIN). \item[{\em ymin}]the world y-coordinate at the bottom left corner of the viewport. \item[{\em ymax}]the world y-coordinate at the top right corner of the viewport (note YMAX may be less than YMIN) \item[{\em just}]if JUST=1, the scales of the x and y axes (in world coordinates per inch) will be equal, otherwise they will be scaled independently. \item[{\em axis}]controls the plotting of axes, tick marks, etc: \begin{itemize}
\item axis = -2 : draw no box, axes or labels; \item axis = -1 : draw box only; \item axis = 0 : draw box and label it with coordinates; \item axis = 1 : same as axis=0, but also draw the coordinate axes (X=0, Y=0); \item axis = 2 : same as axis=1, but also draw grid lines at major increments of the coordinates; \item axis = 10 : draw box and label X-axis logarithmically; \item axis = 20 : draw box and label Y-axis logarithmically; \item axis = 30 : draw box and label both axes logarithmically. \end{itemize}
\item[{\em err}]{\bf Obit\-Err}{\rm (p.\,\pageref{structObitErr})} error stack \end{description}
\end{Desc}
\index{ObitPlot.h@{Obit\-Plot.h}!ObitPlotText@{ObitPlotText}}
\index{ObitPlotText@{ObitPlotText}!ObitPlot.h@{Obit\-Plot.h}}
\subsubsection{\setlength{\rightskip}{0pt plus 5cm}void Obit\-Plot\-Text ({\bf Obit\-Plot} $\ast$ {\em in}, {\bf ofloat} {\em x}, {\bf ofloat} {\em y}, {\bf ofloat} {\em dx}, {\bf ofloat} {\em dy}, {\bf ofloat} {\em just}, gchar $\ast$ {\em text}, {\bf Obit\-Err} $\ast$ {\em err})}\label{ObitPlot_8h_a23}


Public: Write text. 

This routine is useful for annotating graphs. The text is written using the current values of attributes color-index, line-width, character-height, and character-font. \begin{Desc}
\item[Parameters:]
\begin{description}
\item[{\em in}]Pointer to Plot object. \item[{\em x}]Plot x in world coordinates \item[{\em y}]Plot y in world coordinates \item[{\em dx}]x component of inclination \item[{\em dy}]y component of inclination \item[{\em just}]Controls justification of the string parallel to the specified edge of the viewport. If FJUST = 0.0, the left-hand end of the string will be placed at (x,y); if JUST = 0.5, the center of the string will be placed at (x,y); if JUST = 1.0, the right-hand end of the string will be placed at at (x,y). Other values between 0 and 1 give inter- mediate placing, but they are not very useful. \item[{\em text}]The text string to be plotted. Trailing spaces are ignored when justifying the string, but leading spaces are significant. \item[{\em err}]{\bf Obit\-Err}{\rm (p.\,\pageref{structObitErr})} error stack \end{description}
\end{Desc}
\index{ObitPlot.h@{Obit\-Plot.h}!ObitPlotXYErr@{ObitPlotXYErr}}
\index{ObitPlotXYErr@{ObitPlotXYErr}!ObitPlot.h@{Obit\-Plot.h}}
\subsubsection{\setlength{\rightskip}{0pt plus 5cm}void Obit\-Plot\-XYErr ({\bf Obit\-Plot} $\ast$ {\em in}, {\bf olong} {\em symbol}, {\bf olong} {\em n}, {\bf ofloat} $\ast$ {\em x}, {\bf ofloat} $\ast$ {\em y}, {\bf ofloat} $\ast$ {\em e}, {\bf Obit\-Err} $\ast$ {\em err})}\label{ObitPlot_8h_a11}


Public: X-Y plot with error bars. 

\begin{Desc}
\item[Parameters:]
\begin{description}
\item[{\em in}]Pointer to Plot object. \item[{\em in}]Pointer to Plot object. \item[{\em symbol}]Symbol index to use for plotting values in the range [1,12] are usable if negative, use abs value and connect points \begin{itemize}
\item 1 = dot \item 2 = plus \item 3 = $\ast$ \item 4 = open circle \item 5 = x \item 6 = open square \item 7 = open triangle \item 8 = open star \item 9 = filled triangle \item 10 = filled square \item 11 = filled circle \item 12 = filled star\end{itemize}
\item[{\em n}]Number of data points in x, y \item[{\em x}]Independent variable, if NULL use index \item[{\em y}]Dependent variable \item[{\em e}]if non\-NULL, error in y \item[{\em err}]{\bf Obit\-Err}{\rm (p.\,\pageref{structObitErr})} error stack\end{description}
\end{Desc}
Optional parameters on in-$>$info \begin{itemize}
\item XMAX (float) maximum X value (defaults to actual value) \item XMIN (float) minimum X value (defaults to actual value) \item YMAX (float) maximum Y value (defaults to actual value) \item YMIN (float) minimum Y value (defaults to actual value) \item TITLE (string) Label for the plot (defaults to none), max 120 \item XLABEL (string) Label for horizontal axis (defaults to none) \item XOPT (string) Options for horizontal axis (default \char`\"{}BCNTS\char`\"{}) See {\bf Obit\-Plot\-Draw\-Axes}{\rm (p.\,\pageref{ObitPlot_8c_a19})} for details. \item YLABEL (string) Label for vertical axis (defaults to none) \item YOPT (string) Options for vertical axis (default \char`\"{}BCNTS\char`\"{}) See {\bf Obit\-Plot\-Draw\-Axes}{\rm (p.\,\pageref{ObitPlot_8c_a19})} for details. \item XTICK (float) world coordinate interval between major tick marks on X axis. If xtick=0.0 [def], the interval is chosen. \item NXSUB (int) the number of subintervals to divide the major coordinate interval into. If xtick=0.0 or nxsub=0, the number is chosen. [def 0] \item YTICK (float) like xtick for the Y axis. \item NYSUB (int) like nxsub for the Y axis \item CSIZE (int) Scaling factor for characters(default = 1) \item SSIZE (int) Scaling factor for symbols(default = 1) \item LWIDTH (int) Line width (default = 1) \end{itemize}
\index{ObitPlot.h@{Obit\-Plot.h}!ObitPlotXYOver@{ObitPlotXYOver}}
\index{ObitPlotXYOver@{ObitPlotXYOver}!ObitPlot.h@{Obit\-Plot.h}}
\subsubsection{\setlength{\rightskip}{0pt plus 5cm}void Obit\-Plot\-XYOver ({\bf Obit\-Plot} $\ast$ {\em in}, {\bf olong} {\em symbol}, {\bf olong} {\em n}, {\bf ofloat} $\ast$ {\em x}, {\bf ofloat} $\ast$ {\em y}, {\bf Obit\-Err} $\ast$ {\em err})}\label{ObitPlot_8h_a10}


Public: Simple X-Y over plot. 

\begin{Desc}
\item[Parameters:]
\begin{description}
\item[{\em in}]Pointer to Plot object. \item[{\em symbol}]Symbol index to use for plotting values in the range [1,12] are usable if negative, use abs value and connect points \begin{itemize}
\item 0 = line only \item 1 = dot \item 2 = plus \item 3 = $\ast$ \item 4 = open circle \item 5 = x \item 6 = open square \item 7 = open triangle \item 8 = open star \item 9 = filled triangle \item 10 = filled square \item 11 = filled circle \item 12 = filled star\end{itemize}
\item[{\em n}]Number of data points in x, y \item[{\em x}]Independent variable, if NULL use index \item[{\em y}]Dependent variable \item[{\em err}]{\bf Obit\-Err}{\rm (p.\,\pageref{structObitErr})} error stack\end{description}
\end{Desc}
Optional parameters on in-$>$info \begin{itemize}
\item CSIZE (int) Scaling factor for characters(default = 1) \item LWIDTH (int) Line width (default = 1) \end{itemize}
\index{ObitPlot.h@{Obit\-Plot.h}!ObitPlotXYPlot@{ObitPlotXYPlot}}
\index{ObitPlotXYPlot@{ObitPlotXYPlot}!ObitPlot.h@{Obit\-Plot.h}}
\subsubsection{\setlength{\rightskip}{0pt plus 5cm}void Obit\-Plot\-XYPlot ({\bf Obit\-Plot} $\ast$ {\em in}, {\bf olong} {\em symbol}, {\bf olong} {\em n}, {\bf ofloat} $\ast$ {\em x}, {\bf ofloat} $\ast$ {\em y}, {\bf Obit\-Err} $\ast$ {\em err})}\label{ObitPlot_8h_a9}


Public: Simple X-Y plot. 

Plot should be finalized with Obit\-Plot\-Finish\-Plot after all drawing on the current frame is finished. This routine draws the frame and adds labels, to only overplot data on the same frame, use Obit\-Plot\-XYOver \begin{Desc}
\item[Parameters:]
\begin{description}
\item[{\em in}]Pointer to Plot object. \item[{\em symbol}]Symbol index to use for plotting values in the range [1,12] are usable if negative, use abs value and connect points \begin{itemize}
\item 0 = line only \item 1 = dot \item 2 = plus \item 3 = $\ast$ \item 4 = open circle \item 5 = x \item 6 = open square \item 7 = open triangle \item 8 = open star \item 9 = filled triangle \item 10 = filled square \item 11 = filled circle \item 12 = filled star\end{itemize}
\item[{\em n}]Number of data points in x, y \item[{\em x}]Independent variable, if NULL use index \item[{\em y}]Dependent variable \item[{\em err}]{\bf Obit\-Err}{\rm (p.\,\pageref{structObitErr})} error stack\end{description}
\end{Desc}
Optional parameters on in-$>$info \begin{itemize}
\item XMAX (float) maximum X value (defaults to actual value) \item XMIN (float) minimum X value (defaults to actual value) \item YMAX (float) maximum Y value (defaults to actual value) \item YMIN (float) minimum Y value (defaults to actual value) \item TITLE (string) Label for the plot (defaults to none), max 120 \item XLABEL (string) Label for horizontal axis (defaults to none) \item XOPT (string) Option for horizontal axis (default \char`\"{}BCNTS\char`\"{}) See {\bf Obit\-Plot\-Draw\-Axes}{\rm (p.\,\pageref{ObitPlot_8c_a19})} \item YLABEL (string) Label for vertical axis (defaults to none) \item YOPT (string) Option for horizontal axis (default \char`\"{}BCNTS\char`\"{}) See {\bf Obit\-Plot\-Draw\-Axes}{\rm (p.\,\pageref{ObitPlot_8c_a19})} \item XTICK (float) world coordinate interval between major tick marks on X axis. If xtick=0.0 [def], the interval is chosen. \item NXSUB (int) the number of subintervals to divide the major coordinate interval into. If xtick=0.0 or nxsub=0, the number is chosen. [def 0] \item YTICK (float) like xtick for the Y axis. \item NYSUB (int) like nxsub for the Y axis \item CSIZE (int) Scaling factor for characters(default = 1) \item SSIZE (int) Scaling factor for symbols(default = 1) \item LWIDTH (int) Line width (default = 1) \item JUST (int) If !=0 then force X and Y axis scaling to be the same \end{itemize}
