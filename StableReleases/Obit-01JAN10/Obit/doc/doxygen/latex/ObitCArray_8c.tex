\section{Obit\-CArray.c File Reference}
\label{ObitCArray_8c}\index{ObitCArray.c@{ObitCArray.c}}
{\bf Obit\-CArray}{\rm (p.\,\pageref{structObitCArray})} class function definitions. 

{\tt \#include \char`\"{}Obit\-CArray.h\char`\"{}}\par
{\tt \#include \char`\"{}Obit\-Mem.h\char`\"{}}\par
\subsection*{Functions}
\begin{CompactItemize}
\item 
void {\bf Obit\-CArray\-Init} (gpointer in)
\begin{CompactList}\small\item\em Private: Initialize newly instantiated object. \item\end{CompactList}\item 
void {\bf Obit\-CArray\-Clear} (gpointer in)
\begin{CompactList}\small\item\em Private: Deallocate members. \item\end{CompactList}\item 
{\bf Obit\-CArray} $\ast$ {\bf new\-Obit\-CArray} (gchar $\ast$name)
\begin{CompactList}\small\item\em Public: Default Constructor. \item\end{CompactList}\item 
gconstpointer {\bf Obit\-CArray\-Get\-Class} (void)
\begin{CompactList}\small\item\em Public: Class\-Info pointer. \item\end{CompactList}\item 
{\bf Obit\-CArray} $\ast$ {\bf Obit\-CArray\-Copy} ({\bf Obit\-CArray} $\ast$in, {\bf Obit\-CArray} $\ast$out, {\bf Obit\-Err} $\ast$err)
\begin{CompactList}\small\item\em Public: Copy (deep) constructor. \item\end{CompactList}\item 
gboolean {\bf Obit\-CArray\-Is\-Compatable} ({\bf Obit\-CArray} $\ast$in1, {\bf Obit\-CArray} $\ast$in2)
\begin{CompactList}\small\item\em Public: Are two CArrays of compatable geometry. \item\end{CompactList}\item 
{\bf Obit\-CArray} $\ast$ {\bf Obit\-CArray\-Create} (gchar $\ast$name, {\bf olong} ndim, {\bf olong} $\ast$naxis)
\begin{CompactList}\small\item\em Public: Create/initialize {\bf Obit\-CArray}{\rm (p.\,\pageref{structObitCArray})} structures. \item\end{CompactList}\item 
{\bf Obit\-CArray} $\ast$ {\bf Obit\-CArray\-Realloc} ({\bf Obit\-CArray} $\ast$in, {\bf olong} ndim, {\bf olong} $\ast$naxis)
\begin{CompactList}\small\item\em Public: Reallocate/initialize {\bf Obit\-CArray}{\rm (p.\,\pageref{structObitCArray})} structures. \item\end{CompactList}\item 
{\bf ofloat} $\ast$ {\bf Obit\-CArray\-Index} ({\bf Obit\-CArray} $\ast$in, {\bf olong} $\ast$pos)
\begin{CompactList}\small\item\em Public: return pointer to a specified element. \item\end{CompactList}\item 
{\bf ofloat} {\bf Obit\-CArray\-Max\-Abs} ({\bf Obit\-CArray} $\ast$in, {\bf olong} $\ast$pos)
\begin{CompactList}\small\item\em Public: Find Maximum abs value in an {\bf Obit\-CArray}{\rm (p.\,\pageref{structObitCArray})}. \item\end{CompactList}\item 
{\bf ofloat} {\bf Obit\-CArray\-Min} ({\bf Obit\-CArray} $\ast$in, {\bf olong} $\ast$pos)
\begin{CompactList}\small\item\em Public: Find Minimum real or imaginary in an {\bf Obit\-CArray}{\rm (p.\,\pageref{structObitCArray})}. \item\end{CompactList}\item 
void {\bf Obit\-CArray\-Neg} ({\bf Obit\-CArray} $\ast$in)
\begin{CompactList}\small\item\em Public: negate elements of an CArray. \item\end{CompactList}\item 
void {\bf Obit\-CArray\-Conjg} ({\bf Obit\-CArray} $\ast$in)
\begin{CompactList}\small\item\em Public: conjugate elements of an CArray. \item\end{CompactList}\item 
void {\bf Obit\-CArray\-Fill} ({\bf Obit\-CArray} $\ast$in, {\bf ofloat} cmpx[2])
\begin{CompactList}\small\item\em Public: Fill a CArray with a complex scalar. \item\end{CompactList}\item 
void {\bf Obit\-CArray\-SAdd} ({\bf Obit\-CArray} $\ast$in, {\bf ofloat} scalar)
\begin{CompactList}\small\item\em Public: Add a scalar to elements of a CArray. \item\end{CompactList}\item 
void {\bf Obit\-CArray\-SMul} ({\bf Obit\-CArray} $\ast$in, {\bf ofloat} scalar)
\begin{CompactList}\small\item\em Public: Multiply elements of a CArray by a scalar. \item\end{CompactList}\item 
void {\bf Obit\-CArray\-CSAdd} ({\bf Obit\-CArray} $\ast$in, {\bf ofloat} scalar[2])
\begin{CompactList}\small\item\em Public: Add a complex scalar to elements of a CArray. \item\end{CompactList}\item 
void {\bf Obit\-CArray\-CSMul} ({\bf Obit\-CArray} $\ast$in, {\bf ofloat} scalar[2])
\begin{CompactList}\small\item\em Public: Multiply elements of a CArray by a complex scalar. \item\end{CompactList}\item 
void {\bf Obit\-CArray\-Add} ({\bf Obit\-CArray} $\ast$in1, {\bf Obit\-CArray} $\ast$in2, {\bf Obit\-CArray} $\ast$out)
\begin{CompactList}\small\item\em Public: Add elements of two CArrays. \item\end{CompactList}\item 
void {\bf Obit\-CArray\-Sub} ({\bf Obit\-CArray} $\ast$in1, {\bf Obit\-CArray} $\ast$in2, {\bf Obit\-CArray} $\ast$out)
\begin{CompactList}\small\item\em Public: Subtract elements of two CArrays. \item\end{CompactList}\item 
void {\bf Obit\-CArray\-Mul} ({\bf Obit\-CArray} $\ast$in1, {\bf Obit\-CArray} $\ast$in2, {\bf Obit\-CArray} $\ast$out)
\begin{CompactList}\small\item\em Public: Multiply elements of two CArrays. \item\end{CompactList}\item 
void {\bf Obit\-CArray\-Div} ({\bf Obit\-CArray} $\ast$in1, {\bf Obit\-CArray} $\ast$in2, {\bf Obit\-CArray} $\ast$out)
\begin{CompactList}\small\item\em Public: Divide elements of two CArrays. \item\end{CompactList}\item 
{\bf Obit\-FArray} $\ast$ {\bf Obit\-CArray\-Make\-F} ({\bf Obit\-CArray} $\ast$in)
\begin{CompactList}\small\item\em Public: Create an FArray with the same geometry as a CArray. \item\end{CompactList}\item 
{\bf Obit\-CArray} $\ast$ {\bf Obit\-CArray\-Make\-C} ({\bf Obit\-FArray} $\ast$in)
\begin{CompactList}\small\item\em Public: Create a CArray with the same geometry as a FArray. \item\end{CompactList}\item 
gboolean {\bf Obit\-CArray\-Is\-FCompatable} ({\bf Obit\-CArray} $\ast$Cin, {\bf Obit\-FArray} $\ast$Fin)
\begin{CompactList}\small\item\em Public: Are an FArray and a CArray of compatable geometry. \item\end{CompactList}\item 
void {\bf Obit\-CArray\-FMul} ({\bf Obit\-CArray} $\ast$Cin, {\bf Obit\-FArray} $\ast$Fin, {\bf Obit\-CArray} $\ast$out)
\begin{CompactList}\small\item\em Public: Multiply a CArray by an FArray. \item\end{CompactList}\item 
void {\bf Obit\-CArray\-FDiv} ({\bf Obit\-CArray} $\ast$Cin, {\bf Obit\-FArray} $\ast$Fin, {\bf Obit\-CArray} $\ast$out)
\begin{CompactList}\small\item\em Public: Divide a CArray by an FArray. \item\end{CompactList}\item 
void {\bf Obit\-CArray\-FAdd} ({\bf Obit\-CArray} $\ast$Cin, {\bf Obit\-FArray} $\ast$Fin, {\bf Obit\-CArray} $\ast$out)
\begin{CompactList}\small\item\em Public: Add an FArray to a CArray (real part). \item\end{CompactList}\item 
void {\bf Obit\-CArray\-FRot} ({\bf Obit\-CArray} $\ast$Cin, {\bf Obit\-FArray} $\ast$Fin, {\bf Obit\-CArray} $\ast$out)
\begin{CompactList}\small\item\em Public: Rotate phases of a CArray ob phases in an FArray. \item\end{CompactList}\item 
void {\bf Obit\-CArray\-Complex} ({\bf Obit\-FArray} $\ast$Fin1, {\bf Obit\-FArray} $\ast$Fin2, {\bf Obit\-CArray} $\ast$out)
\begin{CompactList}\small\item\em Public: Form A CArray from two FArrays. \item\end{CompactList}\item 
void {\bf Obit\-CArray\-Real} ({\bf Obit\-CArray} $\ast$in, {\bf Obit\-FArray} $\ast$out)
\begin{CompactList}\small\item\em Public: Return the real elements of a CArray in an FArray. \item\end{CompactList}\item 
void {\bf Obit\-CArray\-Imag} ({\bf Obit\-CArray} $\ast$in, {\bf Obit\-FArray} $\ast$out)
\begin{CompactList}\small\item\em Public: Return the imaginary elements of a CArray in an FArray. \item\end{CompactList}\item 
void {\bf Obit\-CArray\-Amp} ({\bf Obit\-CArray} $\ast$in, {\bf Obit\-FArray} $\ast$out)
\begin{CompactList}\small\item\em Public: Return the amplitudes of a CArray in an FArray. \item\end{CompactList}\item 
void {\bf Obit\-CArray\-Phase} ({\bf Obit\-CArray} $\ast$in, {\bf Obit\-FArray} $\ast$out)
\begin{CompactList}\small\item\em Public: Return the phasess of a CArray in an FArray. \item\end{CompactList}\item 
void {\bf Obit\-CArray2DCenter} ({\bf Obit\-CArray} $\ast$in)
\begin{CompactList}\small\item\em Public: Convert a half plane 2D \char`\"{}center at edges\char`\"{} array to proper order. \item\end{CompactList}\item 
void {\bf Obit\-CArray2DCenter\-Full} ({\bf Obit\-CArray} $\ast$in)
\begin{CompactList}\small\item\em Public: Convert a full plane 2D \char`\"{}center at edges\char`\"{} array to proper order. \item\end{CompactList}\item 
{\bf Obit\-CArray} $\ast$ {\bf Obit\-CArray\-Add\-Conjg} ({\bf Obit\-CArray} $\ast$in, {\bf olong} num\-Conj\-Col)
\begin{CompactList}\small\item\em Public: Add conjugate columns to half plane complex image. \item\end{CompactList}\item 
void {\bf Obit\-CArray\-Class\-Init} (void)
\begin{CompactList}\small\item\em Public: Class initializer. \item\end{CompactList}\end{CompactItemize}


\subsection{Detailed Description}
{\bf Obit\-CArray}{\rm (p.\,\pageref{structObitCArray})} class function definitions. 

This class is derived from the {\bf Obit}{\rm (p.\,\pageref{structObit})} base class.

\subsection{Function Documentation}
\index{ObitCArray.c@{Obit\-CArray.c}!newObitCArray@{newObitCArray}}
\index{newObitCArray@{newObitCArray}!ObitCArray.c@{Obit\-CArray.c}}
\subsubsection{\setlength{\rightskip}{0pt plus 5cm}{\bf Obit\-CArray}$\ast$ new\-Obit\-CArray (gchar $\ast$ {\em name})}\label{ObitCArray_8c_a6}


Public: Default Constructor. 

Initializes class if needed on first call. \begin{Desc}
\item[Parameters:]
\begin{description}
\item[{\em name}]An optional name for the object. \end{description}
\end{Desc}
\begin{Desc}
\item[Returns:]the new object. \end{Desc}
\index{ObitCArray.c@{Obit\-CArray.c}!ObitCArray2DCenter@{ObitCArray2DCenter}}
\index{ObitCArray2DCenter@{ObitCArray2DCenter}!ObitCArray.c@{Obit\-CArray.c}}
\subsubsection{\setlength{\rightskip}{0pt plus 5cm}void Obit\-CArray2DCenter ({\bf Obit\-CArray} $\ast$ {\em in})}\label{ObitCArray_8c_a38}


Public: Convert a half plane 2D \char`\"{}center at edges\char`\"{} array to proper order. 

The first and second halves of each column are swaped. This is needed for the peculiar order of FFTs. This uses the FFTW convention that half plane complex arrays have the \char`\"{}short\char`\"{} (i.e. n/2+1) dimension first. \begin{Desc}
\item[Parameters:]
\begin{description}
\item[{\em in}]Half plane complex 2D array to reorder \end{description}
\end{Desc}
\begin{Desc}
\item[Returns:]the new object. \end{Desc}
\index{ObitCArray.c@{Obit\-CArray.c}!ObitCArray2DCenterFull@{ObitCArray2DCenterFull}}
\index{ObitCArray2DCenterFull@{ObitCArray2DCenterFull}!ObitCArray.c@{Obit\-CArray.c}}
\subsubsection{\setlength{\rightskip}{0pt plus 5cm}void Obit\-CArray2DCenter\-Full ({\bf Obit\-CArray} $\ast$ {\em in})}\label{ObitCArray_8c_a39}


Public: Convert a full plane 2D \char`\"{}center at edges\char`\"{} array to proper order. 

The first and second halves of each column are swaped. This is needed for the peculiar order of FFTs. This uses the FFTW convention that half plane complex arrays have the \char`\"{}short\char`\"{} (i.e. n/2+1) dimension first. \begin{Desc}
\item[Parameters:]
\begin{description}
\item[{\em in}]Full plane complex 2D array to reorder \end{description}
\end{Desc}
\begin{Desc}
\item[Returns:]the new object. \end{Desc}
\index{ObitCArray.c@{Obit\-CArray.c}!ObitCArrayAdd@{ObitCArrayAdd}}
\index{ObitCArrayAdd@{ObitCArrayAdd}!ObitCArray.c@{Obit\-CArray.c}}
\subsubsection{\setlength{\rightskip}{0pt plus 5cm}void Obit\-CArray\-Add ({\bf Obit\-CArray} $\ast$ {\em in1}, {\bf Obit\-CArray} $\ast$ {\em in2}, {\bf Obit\-CArray} $\ast$ {\em out})}\label{ObitCArray_8c_a22}


Public: Add elements of two CArrays. 

out = in1 + in2 \begin{Desc}
\item[Parameters:]
\begin{description}
\item[{\em in1}]Input object with data \item[{\em in2}]Input object with data \item[{\em out}]Output object \end{description}
\end{Desc}
\index{ObitCArray.c@{Obit\-CArray.c}!ObitCArrayAddConjg@{ObitCArrayAddConjg}}
\index{ObitCArrayAddConjg@{ObitCArrayAddConjg}!ObitCArray.c@{Obit\-CArray.c}}
\subsubsection{\setlength{\rightskip}{0pt plus 5cm}{\bf Obit\-CArray}$\ast$ Obit\-CArray\-Add\-Conjg ({\bf Obit\-CArray} $\ast$ {\em in}, {\bf olong} {\em num\-Conj\-Col})}\label{ObitCArray_8c_a40}


Public: Add conjugate columns to half plane complex image. 

Only does 2D half plane complex images. This uses the FFTW convention that half plane complex arrays have the \char`\"{}short\char`\"{} (i.e. n/2+1) dimension first. Thus rows must have added columns. \begin{Desc}
\item[Parameters:]
\begin{description}
\item[{\em in}]Input array \item[{\em num\-Conj\-Col}]How many conjugate columns to add. \end{description}
\end{Desc}
\begin{Desc}
\item[Returns:]the new object. \end{Desc}
\index{ObitCArray.c@{Obit\-CArray.c}!ObitCArrayAmp@{ObitCArrayAmp}}
\index{ObitCArrayAmp@{ObitCArrayAmp}!ObitCArray.c@{Obit\-CArray.c}}
\subsubsection{\setlength{\rightskip}{0pt plus 5cm}void Obit\-CArray\-Amp ({\bf Obit\-CArray} $\ast$ {\em in}, {\bf Obit\-FArray} $\ast$ {\em out})}\label{ObitCArray_8c_a36}


Public: Return the amplitudes of a CArray in an FArray. 

\begin{Desc}
\item[Parameters:]
\begin{description}
\item[{\em in}]Input CArray \item[{\em out}]Output FArray \end{description}
\end{Desc}
\index{ObitCArray.c@{Obit\-CArray.c}!ObitCArrayClassInit@{ObitCArrayClassInit}}
\index{ObitCArrayClassInit@{ObitCArrayClassInit}!ObitCArray.c@{Obit\-CArray.c}}
\subsubsection{\setlength{\rightskip}{0pt plus 5cm}void Obit\-CArray\-Class\-Init (void)}\label{ObitCArray_8c_a41}


Public: Class initializer. 

\index{ObitCArray.c@{Obit\-CArray.c}!ObitCArrayClear@{ObitCArrayClear}}
\index{ObitCArrayClear@{ObitCArrayClear}!ObitCArray.c@{Obit\-CArray.c}}
\subsubsection{\setlength{\rightskip}{0pt plus 5cm}void Obit\-CArray\-Clear (gpointer {\em inn})}\label{ObitCArray_8c_a4}


Private: Deallocate members. 

Does (recursive) deallocation of parent class members. For some reason this wasn't build into the GType class. \begin{Desc}
\item[Parameters:]
\begin{description}
\item[{\em inn}]Pointer to the object to deallocate. Actually it should be an Obit\-CArray$\ast$ cast to an Obit$\ast$. \end{description}
\end{Desc}
\index{ObitCArray.c@{Obit\-CArray.c}!ObitCArrayComplex@{ObitCArrayComplex}}
\index{ObitCArrayComplex@{ObitCArrayComplex}!ObitCArray.c@{Obit\-CArray.c}}
\subsubsection{\setlength{\rightskip}{0pt plus 5cm}void Obit\-CArray\-Complex ({\bf Obit\-FArray} $\ast$ {\em Fin1}, {\bf Obit\-FArray} $\ast$ {\em Fin2}, {\bf Obit\-CArray} $\ast$ {\em out})}\label{ObitCArray_8c_a33}


Public: Form A CArray from two FArrays. 

\begin{Desc}
\item[Parameters:]
\begin{description}
\item[{\em Fin1}]Input FArray for real part \item[{\em Fin2}]Input FArray for imaginary part \item[{\em out}]Output CArray \end{description}
\end{Desc}
\index{ObitCArray.c@{Obit\-CArray.c}!ObitCArrayConjg@{ObitCArrayConjg}}
\index{ObitCArrayConjg@{ObitCArrayConjg}!ObitCArray.c@{Obit\-CArray.c}}
\subsubsection{\setlength{\rightskip}{0pt plus 5cm}void Obit\-CArray\-Conjg ({\bf Obit\-CArray} $\ast$ {\em in})}\label{ObitCArray_8c_a16}


Public: conjugate elements of an CArray. 

in = conjg(in). \begin{Desc}
\item[Parameters:]
\begin{description}
\item[{\em in}]Input object with data \end{description}
\end{Desc}
\index{ObitCArray.c@{Obit\-CArray.c}!ObitCArrayCopy@{ObitCArrayCopy}}
\index{ObitCArrayCopy@{ObitCArrayCopy}!ObitCArray.c@{Obit\-CArray.c}}
\subsubsection{\setlength{\rightskip}{0pt plus 5cm}{\bf Obit\-CArray}$\ast$ Obit\-CArray\-Copy ({\bf Obit\-CArray} $\ast$ {\em in}, {\bf Obit\-CArray} $\ast$ {\em out}, {\bf Obit\-Err} $\ast$ {\em err})}\label{ObitCArray_8c_a8}


Public: Copy (deep) constructor. 

\begin{Desc}
\item[Parameters:]
\begin{description}
\item[{\em in}]The object to copy \item[{\em out}]An existing object pointer for output or NULL if none exists. \item[{\em err}]{\bf Obit}{\rm (p.\,\pageref{structObit})} error stack object. \end{description}
\end{Desc}
\begin{Desc}
\item[Returns:]pointer to the new object. \end{Desc}
\index{ObitCArray.c@{Obit\-CArray.c}!ObitCArrayCreate@{ObitCArrayCreate}}
\index{ObitCArrayCreate@{ObitCArrayCreate}!ObitCArray.c@{Obit\-CArray.c}}
\subsubsection{\setlength{\rightskip}{0pt plus 5cm}{\bf Obit\-CArray}$\ast$ Obit\-CArray\-Create (gchar $\ast$ {\em name}, {\bf olong} {\em ndim}, {\bf olong} $\ast$ {\em naxis})}\label{ObitCArray_8c_a10}


Public: Create/initialize {\bf Obit\-CArray}{\rm (p.\,\pageref{structObitCArray})} structures. 

\begin{Desc}
\item[Parameters:]
\begin{description}
\item[{\em name}]An optional name for the object. \item[{\em ndim}]Number of dimensions desired, if $<$=0 data array not created. maximum value = 10. \item[{\em naxis}]Dimensionality along each axis. NULL =$>$ don't create array. \end{description}
\end{Desc}
\begin{Desc}
\item[Returns:]the new object. \end{Desc}
\index{ObitCArray.c@{Obit\-CArray.c}!ObitCArrayCSAdd@{ObitCArrayCSAdd}}
\index{ObitCArrayCSAdd@{ObitCArrayCSAdd}!ObitCArray.c@{Obit\-CArray.c}}
\subsubsection{\setlength{\rightskip}{0pt plus 5cm}void Obit\-CArray\-CSAdd ({\bf Obit\-CArray} $\ast$ {\em in}, {\bf ofloat} {\em scalar}[2])}\label{ObitCArray_8c_a20}


Public: Add a complex scalar to elements of a CArray. 

in = in + scalar \begin{Desc}
\item[Parameters:]
\begin{description}
\item[{\em in}]Input object with data \item[{\em scalar}]Scalar value \end{description}
\end{Desc}
\index{ObitCArray.c@{Obit\-CArray.c}!ObitCArrayCSMul@{ObitCArrayCSMul}}
\index{ObitCArrayCSMul@{ObitCArrayCSMul}!ObitCArray.c@{Obit\-CArray.c}}
\subsubsection{\setlength{\rightskip}{0pt plus 5cm}void Obit\-CArray\-CSMul ({\bf Obit\-CArray} $\ast$ {\em in}, {\bf ofloat} {\em scalar}[2])}\label{ObitCArray_8c_a21}


Public: Multiply elements of a CArray by a complex scalar. 

in = in $\ast$ scalar \begin{Desc}
\item[Parameters:]
\begin{description}
\item[{\em in}]Input object with data \item[{\em scalar}]Scalar value \end{description}
\end{Desc}
\index{ObitCArray.c@{Obit\-CArray.c}!ObitCArrayDiv@{ObitCArrayDiv}}
\index{ObitCArrayDiv@{ObitCArrayDiv}!ObitCArray.c@{Obit\-CArray.c}}
\subsubsection{\setlength{\rightskip}{0pt plus 5cm}void Obit\-CArray\-Div ({\bf Obit\-CArray} $\ast$ {\em in1}, {\bf Obit\-CArray} $\ast$ {\em in2}, {\bf Obit\-CArray} $\ast$ {\em out})}\label{ObitCArray_8c_a25}


Public: Divide elements of two CArrays. 

out = in1 / in2 \begin{Desc}
\item[Parameters:]
\begin{description}
\item[{\em in1}]Input object with data \item[{\em in2}]Input object with data \item[{\em out}]Output object \end{description}
\end{Desc}
\index{ObitCArray.c@{Obit\-CArray.c}!ObitCArrayFAdd@{ObitCArrayFAdd}}
\index{ObitCArrayFAdd@{ObitCArrayFAdd}!ObitCArray.c@{Obit\-CArray.c}}
\subsubsection{\setlength{\rightskip}{0pt plus 5cm}void Obit\-CArray\-FAdd ({\bf Obit\-CArray} $\ast$ {\em Cin}, {\bf Obit\-FArray} $\ast$ {\em Fin}, {\bf Obit\-CArray} $\ast$ {\em out})}\label{ObitCArray_8c_a31}


Public: Add an FArray to a CArray (real part). 

\begin{Desc}
\item[Parameters:]
\begin{description}
\item[{\em Cin}]Input CArray \item[{\em Fin}]Input FArray \item[{\em out}]Output CArray \end{description}
\end{Desc}
\index{ObitCArray.c@{Obit\-CArray.c}!ObitCArrayFDiv@{ObitCArrayFDiv}}
\index{ObitCArrayFDiv@{ObitCArrayFDiv}!ObitCArray.c@{Obit\-CArray.c}}
\subsubsection{\setlength{\rightskip}{0pt plus 5cm}void Obit\-CArray\-FDiv ({\bf Obit\-CArray} $\ast$ {\em Cin}, {\bf Obit\-FArray} $\ast$ {\em Fin}, {\bf Obit\-CArray} $\ast$ {\em out})}\label{ObitCArray_8c_a30}


Public: Divide a CArray by an FArray. 

\begin{Desc}
\item[Parameters:]
\begin{description}
\item[{\em Cin}]Input CArray \item[{\em Fin}]Input FArray \item[{\em out}]Output CArray \end{description}
\end{Desc}
\index{ObitCArray.c@{Obit\-CArray.c}!ObitCArrayFill@{ObitCArrayFill}}
\index{ObitCArrayFill@{ObitCArrayFill}!ObitCArray.c@{Obit\-CArray.c}}
\subsubsection{\setlength{\rightskip}{0pt plus 5cm}void Obit\-CArray\-Fill ({\bf Obit\-CArray} $\ast$ {\em in}, {\bf ofloat} {\em cmpx}[2])}\label{ObitCArray_8c_a17}


Public: Fill a CArray with a complex scalar. 

\begin{Desc}
\item[Parameters:]
\begin{description}
\item[{\em in}]Input object with data \item[{\em cmpx}]Scalar value as (real,imaginary) \end{description}
\end{Desc}
\index{ObitCArray.c@{Obit\-CArray.c}!ObitCArrayFMul@{ObitCArrayFMul}}
\index{ObitCArrayFMul@{ObitCArrayFMul}!ObitCArray.c@{Obit\-CArray.c}}
\subsubsection{\setlength{\rightskip}{0pt plus 5cm}void Obit\-CArray\-FMul ({\bf Obit\-CArray} $\ast$ {\em Cin}, {\bf Obit\-FArray} $\ast$ {\em Fin}, {\bf Obit\-CArray} $\ast$ {\em out})}\label{ObitCArray_8c_a29}


Public: Multiply a CArray by an FArray. 

out = Cin $\ast$ Fin \begin{Desc}
\item[Parameters:]
\begin{description}
\item[{\em Cin}]Input CArray \item[{\em Fin}]Input FArray \item[{\em out}]Output CArray \end{description}
\end{Desc}
\index{ObitCArray.c@{Obit\-CArray.c}!ObitCArrayFRot@{ObitCArrayFRot}}
\index{ObitCArrayFRot@{ObitCArrayFRot}!ObitCArray.c@{Obit\-CArray.c}}
\subsubsection{\setlength{\rightskip}{0pt plus 5cm}void Obit\-CArray\-FRot ({\bf Obit\-CArray} $\ast$ {\em Cin}, {\bf Obit\-FArray} $\ast$ {\em Fin}, {\bf Obit\-CArray} $\ast$ {\em out})}\label{ObitCArray_8c_a32}


Public: Rotate phases of a CArray ob phases in an FArray. 

out = Cin $\ast$ exp(i$\ast$Fin) \begin{Desc}
\item[Parameters:]
\begin{description}
\item[{\em Cin}]Input CArray \item[{\em Fin}]Input FArray \item[{\em out}]Output CArray \end{description}
\end{Desc}
\index{ObitCArray.c@{Obit\-CArray.c}!ObitCArrayGetClass@{ObitCArrayGetClass}}
\index{ObitCArrayGetClass@{ObitCArrayGetClass}!ObitCArray.c@{Obit\-CArray.c}}
\subsubsection{\setlength{\rightskip}{0pt plus 5cm}gconstpointer Obit\-CArray\-Get\-Class (void)}\label{ObitCArray_8c_a7}


Public: Class\-Info pointer. 

\begin{Desc}
\item[Returns:]pointer to the class structure. \end{Desc}
\index{ObitCArray.c@{Obit\-CArray.c}!ObitCArrayImag@{ObitCArrayImag}}
\index{ObitCArrayImag@{ObitCArrayImag}!ObitCArray.c@{Obit\-CArray.c}}
\subsubsection{\setlength{\rightskip}{0pt plus 5cm}void Obit\-CArray\-Imag ({\bf Obit\-CArray} $\ast$ {\em in}, {\bf Obit\-FArray} $\ast$ {\em out})}\label{ObitCArray_8c_a35}


Public: Return the imaginary elements of a CArray in an FArray. 

\begin{Desc}
\item[Parameters:]
\begin{description}
\item[{\em in}]Input CArray \item[{\em out}]Output FArray \end{description}
\end{Desc}
\index{ObitCArray.c@{Obit\-CArray.c}!ObitCArrayIndex@{ObitCArrayIndex}}
\index{ObitCArrayIndex@{ObitCArrayIndex}!ObitCArray.c@{Obit\-CArray.c}}
\subsubsection{\setlength{\rightskip}{0pt plus 5cm}{\bf ofloat}$\ast$ Obit\-CArray\-Index ({\bf Obit\-CArray} $\ast$ {\em in}, {\bf olong} $\ast$ {\em pos})}\label{ObitCArray_8c_a12}


Public: return pointer to a specified element. 

Subsequent data are stored in order of increasing dimension (rows, then columns...). \begin{Desc}
\item[Parameters:]
\begin{description}
\item[{\em in}]Object with data \item[{\em pos}]array of 0-rel pixel numbers on each axis \end{description}
\end{Desc}
\begin{Desc}
\item[Returns:]pointer to specified object; NULL if illegal pixel. \end{Desc}
\index{ObitCArray.c@{Obit\-CArray.c}!ObitCArrayInit@{ObitCArrayInit}}
\index{ObitCArrayInit@{ObitCArrayInit}!ObitCArray.c@{Obit\-CArray.c}}
\subsubsection{\setlength{\rightskip}{0pt plus 5cm}void Obit\-CArray\-Init (gpointer {\em inn})}\label{ObitCArray_8c_a3}


Private: Initialize newly instantiated object. 

Parent classes portions are (recursively) initialized first \begin{Desc}
\item[Parameters:]
\begin{description}
\item[{\em inn}]Pointer to the object to initialize. \end{description}
\end{Desc}
\index{ObitCArray.c@{Obit\-CArray.c}!ObitCArrayIsCompatable@{ObitCArrayIsCompatable}}
\index{ObitCArrayIsCompatable@{ObitCArrayIsCompatable}!ObitCArray.c@{Obit\-CArray.c}}
\subsubsection{\setlength{\rightskip}{0pt plus 5cm}gboolean Obit\-CArray\-Is\-Compatable ({\bf Obit\-CArray} $\ast$ {\em in1}, {\bf Obit\-CArray} $\ast$ {\em in2})}\label{ObitCArray_8c_a9}


Public: Are two CArrays of compatable geometry. 

Must have same number of non degenerate dimensions and each dimension must be the same size. \begin{Desc}
\item[Parameters:]
\begin{description}
\item[{\em in1}]First object to test. \item[{\em in2}]Second object to test. \end{description}
\end{Desc}
\begin{Desc}
\item[Returns:]TRUE if compatable, else FALSE. \end{Desc}
\index{ObitCArray.c@{Obit\-CArray.c}!ObitCArrayIsFCompatable@{ObitCArrayIsFCompatable}}
\index{ObitCArrayIsFCompatable@{ObitCArrayIsFCompatable}!ObitCArray.c@{Obit\-CArray.c}}
\subsubsection{\setlength{\rightskip}{0pt plus 5cm}gboolean Obit\-CArray\-Is\-FCompatable ({\bf Obit\-CArray} $\ast$ {\em Cin}, {\bf Obit\-FArray} $\ast$ {\em Fin})}\label{ObitCArray_8c_a28}


Public: Are an FArray and a CArray of compatable geometry. 

\begin{Desc}
\item[Parameters:]
\begin{description}
\item[{\em Cin}]Input CArray \item[{\em Fin}]Input FArray \end{description}
\end{Desc}
\begin{Desc}
\item[Returns:]TRUE if compatable, else FALSE. \end{Desc}
\index{ObitCArray.c@{Obit\-CArray.c}!ObitCArrayMakeC@{ObitCArrayMakeC}}
\index{ObitCArrayMakeC@{ObitCArrayMakeC}!ObitCArray.c@{Obit\-CArray.c}}
\subsubsection{\setlength{\rightskip}{0pt plus 5cm}{\bf Obit\-CArray}$\ast$ Obit\-CArray\-Make\-C ({\bf Obit\-FArray} $\ast$ {\em in})}\label{ObitCArray_8c_a27}


Public: Create a CArray with the same geometry as a FArray. 

\begin{Desc}
\item[Parameters:]
\begin{description}
\item[{\em in}]Input object with data \end{description}
\end{Desc}
\begin{Desc}
\item[Returns:]the new object. \end{Desc}
\index{ObitCArray.c@{Obit\-CArray.c}!ObitCArrayMakeF@{ObitCArrayMakeF}}
\index{ObitCArrayMakeF@{ObitCArrayMakeF}!ObitCArray.c@{Obit\-CArray.c}}
\subsubsection{\setlength{\rightskip}{0pt plus 5cm}{\bf Obit\-FArray}$\ast$ Obit\-CArray\-Make\-F ({\bf Obit\-CArray} $\ast$ {\em in})}\label{ObitCArray_8c_a26}


Public: Create an FArray with the same geometry as a CArray. 

\begin{Desc}
\item[Parameters:]
\begin{description}
\item[{\em in}]Input object with data \end{description}
\end{Desc}
\begin{Desc}
\item[Returns:]the new object. \end{Desc}
\index{ObitCArray.c@{Obit\-CArray.c}!ObitCArrayMaxAbs@{ObitCArrayMaxAbs}}
\index{ObitCArrayMaxAbs@{ObitCArrayMaxAbs}!ObitCArray.c@{Obit\-CArray.c}}
\subsubsection{\setlength{\rightskip}{0pt plus 5cm}{\bf ofloat} Obit\-CArray\-Max\-Abs ({\bf Obit\-CArray} $\ast$ {\em in}, {\bf olong} $\ast$ {\em pos})}\label{ObitCArray_8c_a13}


Public: Find Maximum abs value in an {\bf Obit\-CArray}{\rm (p.\,\pageref{structObitCArray})}. 

Return value and location in pos. \begin{Desc}
\item[Parameters:]
\begin{description}
\item[{\em in}]Object with data \item[{\em pos}](out) array of 0-rel pixel numbers on each axis \end{description}
\end{Desc}
\begin{Desc}
\item[Returns:]maximum modulus value. \end{Desc}
\index{ObitCArray.c@{Obit\-CArray.c}!ObitCArrayMin@{ObitCArrayMin}}
\index{ObitCArrayMin@{ObitCArrayMin}!ObitCArray.c@{Obit\-CArray.c}}
\subsubsection{\setlength{\rightskip}{0pt plus 5cm}{\bf ofloat} Obit\-CArray\-Min ({\bf Obit\-CArray} $\ast$ {\em in}, {\bf olong} $\ast$ {\em pos})}\label{ObitCArray_8c_a14}


Public: Find Minimum real or imaginary in an {\bf Obit\-CArray}{\rm (p.\,\pageref{structObitCArray})}. 

Return value and location in pos. \begin{Desc}
\item[Parameters:]
\begin{description}
\item[{\em in}]Object with data \item[{\em pos}](out) array of 0-rel pixel numbers on each axis \end{description}
\end{Desc}
\begin{Desc}
\item[Returns:]minimum real or imaginary value. \end{Desc}
\index{ObitCArray.c@{Obit\-CArray.c}!ObitCArrayMul@{ObitCArrayMul}}
\index{ObitCArrayMul@{ObitCArrayMul}!ObitCArray.c@{Obit\-CArray.c}}
\subsubsection{\setlength{\rightskip}{0pt plus 5cm}void Obit\-CArray\-Mul ({\bf Obit\-CArray} $\ast$ {\em in1}, {\bf Obit\-CArray} $\ast$ {\em in2}, {\bf Obit\-CArray} $\ast$ {\em out})}\label{ObitCArray_8c_a24}


Public: Multiply elements of two CArrays. 

Output may be one of the inputs out = in1 $\ast$ in2 \begin{Desc}
\item[Parameters:]
\begin{description}
\item[{\em in1}]Input object with data \item[{\em in2}]Input object with data \item[{\em out}]Output object \end{description}
\end{Desc}
\index{ObitCArray.c@{Obit\-CArray.c}!ObitCArrayNeg@{ObitCArrayNeg}}
\index{ObitCArrayNeg@{ObitCArrayNeg}!ObitCArray.c@{Obit\-CArray.c}}
\subsubsection{\setlength{\rightskip}{0pt plus 5cm}void Obit\-CArray\-Neg ({\bf Obit\-CArray} $\ast$ {\em in})}\label{ObitCArray_8c_a15}


Public: negate elements of an CArray. 

in = -in. \begin{Desc}
\item[Parameters:]
\begin{description}
\item[{\em in}]Input object with data \end{description}
\end{Desc}
\index{ObitCArray.c@{Obit\-CArray.c}!ObitCArrayPhase@{ObitCArrayPhase}}
\index{ObitCArrayPhase@{ObitCArrayPhase}!ObitCArray.c@{Obit\-CArray.c}}
\subsubsection{\setlength{\rightskip}{0pt plus 5cm}void Obit\-CArray\-Phase ({\bf Obit\-CArray} $\ast$ {\em in}, {\bf Obit\-FArray} $\ast$ {\em out})}\label{ObitCArray_8c_a37}


Public: Return the phasess of a CArray in an FArray. 

\begin{Desc}
\item[Parameters:]
\begin{description}
\item[{\em in}]Input CArray \item[{\em out}]Output FArray \end{description}
\end{Desc}
\index{ObitCArray.c@{Obit\-CArray.c}!ObitCArrayReal@{ObitCArrayReal}}
\index{ObitCArrayReal@{ObitCArrayReal}!ObitCArray.c@{Obit\-CArray.c}}
\subsubsection{\setlength{\rightskip}{0pt plus 5cm}void Obit\-CArray\-Real ({\bf Obit\-CArray} $\ast$ {\em in}, {\bf Obit\-FArray} $\ast$ {\em out})}\label{ObitCArray_8c_a34}


Public: Return the real elements of a CArray in an FArray. 

\begin{Desc}
\item[Parameters:]
\begin{description}
\item[{\em in}]Input CArray \item[{\em out}]Output CArray \end{description}
\end{Desc}
\index{ObitCArray.c@{Obit\-CArray.c}!ObitCArrayRealloc@{ObitCArrayRealloc}}
\index{ObitCArrayRealloc@{ObitCArrayRealloc}!ObitCArray.c@{Obit\-CArray.c}}
\subsubsection{\setlength{\rightskip}{0pt plus 5cm}{\bf Obit\-CArray}$\ast$ Obit\-CArray\-Realloc ({\bf Obit\-CArray} $\ast$ {\em in}, {\bf olong} {\em ndim}, {\bf olong} $\ast$ {\em naxis})}\label{ObitCArray_8c_a11}


Public: Reallocate/initialize {\bf Obit\-CArray}{\rm (p.\,\pageref{structObitCArray})} structures. 

\begin{Desc}
\item[Parameters:]
\begin{description}
\item[{\em in}]Object with structures to reallocate. \item[{\em ndim}]Number of dimensions desired, if $<$=0 data array not created. maximum value = 10. \item[{\em naxis}]Dimensionality along each axis. NULL =$>$ don't create array. \end{description}
\end{Desc}
\begin{Desc}
\item[Returns:]the new object. \end{Desc}
\index{ObitCArray.c@{Obit\-CArray.c}!ObitCArraySAdd@{ObitCArraySAdd}}
\index{ObitCArraySAdd@{ObitCArraySAdd}!ObitCArray.c@{Obit\-CArray.c}}
\subsubsection{\setlength{\rightskip}{0pt plus 5cm}void Obit\-CArray\-SAdd ({\bf Obit\-CArray} $\ast$ {\em in}, {\bf ofloat} {\em scalar})}\label{ObitCArray_8c_a18}


Public: Add a scalar to elements of a CArray. 

in = in + scalar \begin{Desc}
\item[Parameters:]
\begin{description}
\item[{\em in}]Input object with data \item[{\em scalar}]Scalar value \end{description}
\end{Desc}
\index{ObitCArray.c@{Obit\-CArray.c}!ObitCArraySMul@{ObitCArraySMul}}
\index{ObitCArraySMul@{ObitCArraySMul}!ObitCArray.c@{Obit\-CArray.c}}
\subsubsection{\setlength{\rightskip}{0pt plus 5cm}void Obit\-CArray\-SMul ({\bf Obit\-CArray} $\ast$ {\em in}, {\bf ofloat} {\em scalar})}\label{ObitCArray_8c_a19}


Public: Multiply elements of a CArray by a scalar. 

in = in $\ast$ scalar \begin{Desc}
\item[Parameters:]
\begin{description}
\item[{\em in}]Input object with data \item[{\em scalar}]Scalar value \end{description}
\end{Desc}
\index{ObitCArray.c@{Obit\-CArray.c}!ObitCArraySub@{ObitCArraySub}}
\index{ObitCArraySub@{ObitCArraySub}!ObitCArray.c@{Obit\-CArray.c}}
\subsubsection{\setlength{\rightskip}{0pt plus 5cm}void Obit\-CArray\-Sub ({\bf Obit\-CArray} $\ast$ {\em in1}, {\bf Obit\-CArray} $\ast$ {\em in2}, {\bf Obit\-CArray} $\ast$ {\em out})}\label{ObitCArray_8c_a23}


Public: Subtract elements of two CArrays. 

out = in1 - in2 \begin{Desc}
\item[Parameters:]
\begin{description}
\item[{\em in1}]Input object with data \item[{\em in2}]Input object with data \item[{\em out}]Output object \end{description}
\end{Desc}
