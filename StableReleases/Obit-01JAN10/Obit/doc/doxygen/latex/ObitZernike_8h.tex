\section{Obit\-Zernike.h File Reference}
\label{ObitZernike_8h}\index{ObitZernike.h@{ObitZernike.h}}
Obit\-Zernike utility module definition. 

{\tt \#include \char`\"{}Obit.h\char`\"{}}\par
\subsection*{Functions}
\begin{CompactItemize}
\item 
{\bf ofloat} {\bf Obit\-Zernike} ({\bf olong} n, {\bf ofloat} x, {\bf ofloat} y)
\begin{CompactList}\small\item\em Return Zernike term N for X and Y. \item\end{CompactList}\item 
{\bf ofloat} {\bf Obit\-Zernike\-Grad\-X} ({\bf olong} n, {\bf ofloat} x, {\bf ofloat} y)
\begin{CompactList}\small\item\em Return Zernike term N gradient in X for X and Y. \item\end{CompactList}\item 
{\bf ofloat} {\bf Obit\-Zernike\-Grad\-Y} ({\bf olong} n, {\bf ofloat} x, {\bf ofloat} y)
\begin{CompactList}\small\item\em Return Zernike term N gradient in Y for X and Y. \item\end{CompactList}\item 
{\bf ofloat} {\bf Obit\-Zernike\-Polar} ({\bf olong} n, {\bf ofloat} rho, {\bf ofloat} phi)
\begin{CompactList}\small\item\em Return Zernike term N for Polar coordinates rho and phi. \item\end{CompactList}\end{CompactItemize}


\subsection{Detailed Description}
Obit\-Zernike utility module definition. 

This utility package supplies Zernike terms and derivatives. Zernike polynomials are used to describe phase errors across an aperature.

\subsection{Function Documentation}
\index{ObitZernike.h@{Obit\-Zernike.h}!ObitZernike@{ObitZernike}}
\index{ObitZernike@{ObitZernike}!ObitZernike.h@{Obit\-Zernike.h}}
\subsubsection{\setlength{\rightskip}{0pt plus 5cm}{\bf ofloat} Obit\-Zernike ({\bf olong} {\em n}, {\bf ofloat} {\em x}, {\bf ofloat} {\em y})}\label{ObitZernike_8h_a0}


Return Zernike term N for X and Y. 

\begin{Desc}
\item[Parameters:]
\begin{description}
\item[{\em N}]1-rel term number, between 1 and 18 supported \item[{\em X}]\char`\"{}X\char`\"{} rectangular coordinate on unit circle \item[{\em Y}]\char`\"{}Y\char`\"{} rectangular coordinate on unit circle \end{description}
\end{Desc}
\begin{Desc}
\item[Returns:]Zernike term \end{Desc}
\index{ObitZernike.h@{Obit\-Zernike.h}!ObitZernikeGradX@{ObitZernikeGradX}}
\index{ObitZernikeGradX@{ObitZernikeGradX}!ObitZernike.h@{Obit\-Zernike.h}}
\subsubsection{\setlength{\rightskip}{0pt plus 5cm}{\bf ofloat} Obit\-Zernike\-Grad\-X ({\bf olong} {\em n}, {\bf ofloat} {\em x}, {\bf ofloat} {\em y})}\label{ObitZernike_8h_a1}


Return Zernike term N gradient in X for X and Y. 

\begin{Desc}
\item[Parameters:]
\begin{description}
\item[{\em N}]1-rel term number, between 1 and 18 supported \item[{\em X}]\char`\"{}X\char`\"{} rectangular coordinate on unit circle \item[{\em Y}]\char`\"{}Y\char`\"{} rectangular coordinate on unit circle \end{description}
\end{Desc}
\begin{Desc}
\item[Returns:]Zernike term \end{Desc}
\index{ObitZernike.h@{Obit\-Zernike.h}!ObitZernikeGradY@{ObitZernikeGradY}}
\index{ObitZernikeGradY@{ObitZernikeGradY}!ObitZernike.h@{Obit\-Zernike.h}}
\subsubsection{\setlength{\rightskip}{0pt plus 5cm}{\bf ofloat} Obit\-Zernike\-Grad\-Y ({\bf olong} {\em n}, {\bf ofloat} {\em x}, {\bf ofloat} {\em y})}\label{ObitZernike_8h_a2}


Return Zernike term N gradient in Y for X and Y. 

\begin{Desc}
\item[Parameters:]
\begin{description}
\item[{\em N}]1-rel term number, between 1 and 18 supported \item[{\em X}]\char`\"{}X\char`\"{} rectangular coordinate on unit circle \item[{\em Y}]\char`\"{}Y\char`\"{} rectangular coordinate on unit circle \end{description}
\end{Desc}
\begin{Desc}
\item[Returns:]Zernike term \end{Desc}
\index{ObitZernike.h@{Obit\-Zernike.h}!ObitZernikePolar@{ObitZernikePolar}}
\index{ObitZernikePolar@{ObitZernikePolar}!ObitZernike.h@{Obit\-Zernike.h}}
\subsubsection{\setlength{\rightskip}{0pt plus 5cm}{\bf ofloat} Obit\-Zernike\-Polar ({\bf olong} {\em n}, {\bf ofloat} {\em rho}, {\bf ofloat} {\em phi})}\label{ObitZernike_8h_a3}


Return Zernike term N for Polar coordinates rho and phi. 

\begin{Desc}
\item[Parameters:]
\begin{description}
\item[{\em N}]1-rel term number, between 1 and 36 supported \item[{\em rho}]radial coordinate on unit circle \item[{\em phi}]azimuthal coordinate on unit circle (radian) \end{description}
\end{Desc}
\begin{Desc}
\item[Returns:]Zernike term \end{Desc}
