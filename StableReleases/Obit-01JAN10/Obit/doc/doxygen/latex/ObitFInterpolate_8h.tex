\section{Obit\-FInterpolate.h File Reference}
\label{ObitFInterpolate_8h}\index{ObitFInterpolate.h@{ObitFInterpolate.h}}
{\bf Obit\-FInterpolate}{\rm (p.\,\pageref{structObitFInterpolate})} does Lagrangian interpolation of positions in an {\bf Obit\-FArray}{\rm (p.\,\pageref{structObitFArray})}. 

{\tt \#include \char`\"{}Obit.h\char`\"{}}\par
{\tt \#include \char`\"{}Obit\-Err.h\char`\"{}}\par
{\tt \#include \char`\"{}Obit\-Thread.h\char`\"{}}\par
{\tt \#include \char`\"{}Obit\-Info\-List.h\char`\"{}}\par
{\tt \#include \char`\"{}Obit\-Image\-Desc.h\char`\"{}}\par
{\tt \#include \char`\"{}Obit\-FArray.h\char`\"{}}\par
\subsection*{Classes}
\begin{CompactItemize}
\item 
struct {\bf Obit\-FInterpolate}
\begin{CompactList}\small\item\em Obit\-FInterpolate Class structure. \item\end{CompactList}\item 
struct {\bf Obit\-FInterpolate\-Class\-Info}
\begin{CompactList}\small\item\em Class\-Info Structure. \item\end{CompactList}\end{CompactItemize}
\subsection*{Defines}
\begin{CompactItemize}
\item 
\#define {\bf Obit\-FInterpolate\-Unref}(in)\ Obit\-Unref (in)
\begin{CompactList}\small\item\em Macro to unreference (and possibly destroy) an {\bf Obit\-FInterpolate}{\rm (p.\,\pageref{structObitFInterpolate})} returns a Obit\-FInterpolate$\ast$. \item\end{CompactList}\item 
\#define {\bf Obit\-FInterpolate\-Ref}(in)\ Obit\-Ref (in)
\begin{CompactList}\small\item\em Macro to reference (update reference count) an {\bf Obit\-FInterpolate}{\rm (p.\,\pageref{structObitFInterpolate})}. \item\end{CompactList}\item 
\#define {\bf Obit\-FInterpolate\-Is\-A}(in)\ Obit\-Is\-A (in, Obit\-FInterpolate\-Get\-Class())
\begin{CompactList}\small\item\em Macro to determine if an object is the member of this or a derived class. \item\end{CompactList}\end{CompactItemize}
\subsection*{Functions}
\begin{CompactItemize}
\item 
void {\bf Obit\-FInterpolate\-Class\-Init} (void)
\begin{CompactList}\small\item\em Public: Class initializer. \item\end{CompactList}\item 
{\bf Obit\-FInterpolate} $\ast$ {\bf new\-Obit\-FInterpolate} (gchar $\ast$name)
\begin{CompactList}\small\item\em Public: Constructor. \item\end{CompactList}\item 
{\bf Obit\-FInterpolate} $\ast$ {\bf new\-Obit\-FInterpolate\-Create} (gchar $\ast$name, {\bf Obit\-FArray} $\ast$array, {\bf Obit\-Image\-Desc} $\ast$desc, {\bf olong} hwidth)
\begin{CompactList}\small\item\em Public: Constructor from value. \item\end{CompactList}\item 
gconstpointer {\bf Obit\-FInterpolate\-Get\-Class} (void)
\begin{CompactList}\small\item\em Public: Class\-Info pointer. \item\end{CompactList}\item 
{\bf Obit\-FInterpolate} $\ast$ {\bf Obit\-FInterpolate\-Copy} ({\bf Obit\-FInterpolate} $\ast$in, {\bf Obit\-FInterpolate} $\ast$out, {\bf Obit\-Err} $\ast$err)
\begin{CompactList}\small\item\em Public: Copy (deep) constructor. \item\end{CompactList}\item 
{\bf Obit\-FInterpolate} $\ast$ {\bf Obit\-FInterpolate\-Clone} ({\bf Obit\-FInterpolate} $\ast$in, {\bf Obit\-FInterpolate} $\ast$out)
\begin{CompactList}\small\item\em Public: Copy (shallow) constructor. \item\end{CompactList}\item 
void {\bf Obit\-FInterpolate\-Replace} ({\bf Obit\-FInterpolate} $\ast$in, {\bf Obit\-FArray} $\ast$new\-Array)
\begin{CompactList}\small\item\em Public: Replace member {\bf Obit\-FArray}{\rm (p.\,\pageref{structObitFArray})}. \item\end{CompactList}\item 
{\bf ofloat} {\bf Obit\-FInterpolate\-Pixel} ({\bf Obit\-FInterpolate} $\ast$in, {\bf ofloat} $\ast$pixel, {\bf Obit\-Err} $\ast$err)
\begin{CompactList}\small\item\em Public: Interpolate Pixel in 2D array. \item\end{CompactList}\item 
{\bf ofloat} {\bf Obit\-FInterpolate1D} ({\bf Obit\-FInterpolate} $\ast$in, {\bf ofloat} pixel)
\begin{CompactList}\small\item\em Public: Interpolate value in 1- array. \item\end{CompactList}\item 
{\bf ofloat} {\bf Obit\-FInterpolate\-Position} ({\bf Obit\-FInterpolate} $\ast$in, {\bf odouble} $\ast$coord, {\bf Obit\-Err} $\ast$err)
\begin{CompactList}\small\item\em Public: Interpolate Position in 2D array. \item\end{CompactList}\item 
{\bf ofloat} {\bf Obit\-FInterpolate\-Offset} ({\bf Obit\-FInterpolate} $\ast$in, {\bf odouble} $\ast$offset, {\bf Obit\-Err} $\ast$err)
\begin{CompactList}\small\item\em Public: Interpolate Offset in 2D array. \item\end{CompactList}\end{CompactItemize}


\subsection{Detailed Description}
{\bf Obit\-FInterpolate}{\rm (p.\,\pageref{structObitFInterpolate})} does Lagrangian interpolation of positions in an {\bf Obit\-FArray}{\rm (p.\,\pageref{structObitFArray})}. 

This class is derived from the {\bf Obit}{\rm (p.\,\pageref{structObit})} class.

\subsection{Define Documentation}
\index{ObitFInterpolate.h@{Obit\-FInterpolate.h}!ObitFInterpolateIsA@{ObitFInterpolateIsA}}
\index{ObitFInterpolateIsA@{ObitFInterpolateIsA}!ObitFInterpolate.h@{Obit\-FInterpolate.h}}
\subsubsection{\setlength{\rightskip}{0pt plus 5cm}\#define Obit\-FInterpolate\-Is\-A(in)\ Obit\-Is\-A (in, Obit\-FInterpolate\-Get\-Class())}\label{ObitFInterpolate_8h_a2}


Macro to determine if an object is the member of this or a derived class. 

Returns TRUE if a member, else FALSE in = object to reference \index{ObitFInterpolate.h@{Obit\-FInterpolate.h}!ObitFInterpolateRef@{ObitFInterpolateRef}}
\index{ObitFInterpolateRef@{ObitFInterpolateRef}!ObitFInterpolate.h@{Obit\-FInterpolate.h}}
\subsubsection{\setlength{\rightskip}{0pt plus 5cm}\#define Obit\-FInterpolate\-Ref(in)\ Obit\-Ref (in)}\label{ObitFInterpolate_8h_a1}


Macro to reference (update reference count) an {\bf Obit\-FInterpolate}{\rm (p.\,\pageref{structObitFInterpolate})}. 

returns a Obit\-FInterpolate$\ast$. in = object to reference \index{ObitFInterpolate.h@{Obit\-FInterpolate.h}!ObitFInterpolateUnref@{ObitFInterpolateUnref}}
\index{ObitFInterpolateUnref@{ObitFInterpolateUnref}!ObitFInterpolate.h@{Obit\-FInterpolate.h}}
\subsubsection{\setlength{\rightskip}{0pt plus 5cm}\#define Obit\-FInterpolate\-Unref(in)\ Obit\-Unref (in)}\label{ObitFInterpolate_8h_a0}


Macro to unreference (and possibly destroy) an {\bf Obit\-FInterpolate}{\rm (p.\,\pageref{structObitFInterpolate})} returns a Obit\-FInterpolate$\ast$. 

in = object to unreference 

\subsection{Function Documentation}
\index{ObitFInterpolate.h@{Obit\-FInterpolate.h}!newObitFInterpolate@{newObitFInterpolate}}
\index{newObitFInterpolate@{newObitFInterpolate}!ObitFInterpolate.h@{Obit\-FInterpolate.h}}
\subsubsection{\setlength{\rightskip}{0pt plus 5cm}{\bf Obit\-FInterpolate}$\ast$ new\-Obit\-FInterpolate (gchar $\ast$ {\em name})}\label{ObitFInterpolate_8h_a4}


Public: Constructor. 

Initializes class if needed on first call. \begin{Desc}
\item[Parameters:]
\begin{description}
\item[{\em name}]An optional name for the object. \end{description}
\end{Desc}
\begin{Desc}
\item[Returns:]the new object. \end{Desc}
\index{ObitFInterpolate.h@{Obit\-FInterpolate.h}!newObitFInterpolateCreate@{newObitFInterpolateCreate}}
\index{newObitFInterpolateCreate@{newObitFInterpolateCreate}!ObitFInterpolate.h@{Obit\-FInterpolate.h}}
\subsubsection{\setlength{\rightskip}{0pt plus 5cm}{\bf Obit\-FInterpolate}$\ast$ new\-Obit\-FInterpolate\-Create (gchar $\ast$ {\em name}, {\bf Obit\-FArray} $\ast$ {\em array}, {\bf Obit\-Image\-Desc} $\ast$ {\em desc}, {\bf olong} {\em hwidth})}\label{ObitFInterpolate_8h_a5}


Public: Constructor from value. 

Initializes class if needed on first call. \begin{Desc}
\item[Parameters:]
\begin{description}
\item[{\em name}]An optional name for the object. \item[{\em array}]The Obit\-Farray to be interpolated. \item[{\em desc}]if non\-NULL, an image descriptor to be attached to the output. \item[{\em hwidth}]Half width of interpolation kernal (range [1,4] allowed). \end{description}
\end{Desc}
\begin{Desc}
\item[Returns:]the new object. \end{Desc}
\index{ObitFInterpolate.h@{Obit\-FInterpolate.h}!ObitFInterpolate1D@{ObitFInterpolate1D}}
\index{ObitFInterpolate1D@{ObitFInterpolate1D}!ObitFInterpolate.h@{Obit\-FInterpolate.h}}
\subsubsection{\setlength{\rightskip}{0pt plus 5cm}{\bf ofloat} Obit\-FInterpolate1D ({\bf Obit\-FInterpolate} $\ast$ {\em in}, {\bf ofloat} {\em pixel})}\label{ObitFInterpolate_8h_a11}


Public: Interpolate value in 1- array. 

\begin{Desc}
\item[Parameters:]
\begin{description}
\item[{\em in}]The object to interpolate \item[{\em pixel}]Pixel location (1-rel) in array \end{description}
\end{Desc}
\begin{Desc}
\item[Returns:]value, blanked if invalid \end{Desc}
\index{ObitFInterpolate.h@{Obit\-FInterpolate.h}!ObitFInterpolateClassInit@{ObitFInterpolateClassInit}}
\index{ObitFInterpolateClassInit@{ObitFInterpolateClassInit}!ObitFInterpolate.h@{Obit\-FInterpolate.h}}
\subsubsection{\setlength{\rightskip}{0pt plus 5cm}void Obit\-FInterpolate\-Class\-Init (void)}\label{ObitFInterpolate_8h_a3}


Public: Class initializer. 

\index{ObitFInterpolate.h@{Obit\-FInterpolate.h}!ObitFInterpolateClone@{ObitFInterpolateClone}}
\index{ObitFInterpolateClone@{ObitFInterpolateClone}!ObitFInterpolate.h@{Obit\-FInterpolate.h}}
\subsubsection{\setlength{\rightskip}{0pt plus 5cm}{\bf Obit\-FInterpolate}$\ast$ Obit\-FInterpolate\-Clone ({\bf Obit\-FInterpolate} $\ast$ {\em in}, {\bf Obit\-FInterpolate} $\ast$ {\em out})}\label{ObitFInterpolate_8h_a8}


Public: Copy (shallow) constructor. 

The result will have pointers to the more complex members. Parent class members are included but any derived class info is ignored. \begin{Desc}
\item[Parameters:]
\begin{description}
\item[{\em in}]The object to copy \item[{\em out}]An existing object pointer for output or NULL if none exists. \end{description}
\end{Desc}
\begin{Desc}
\item[Returns:]pointer to the new object. \end{Desc}
\index{ObitFInterpolate.h@{Obit\-FInterpolate.h}!ObitFInterpolateCopy@{ObitFInterpolateCopy}}
\index{ObitFInterpolateCopy@{ObitFInterpolateCopy}!ObitFInterpolate.h@{Obit\-FInterpolate.h}}
\subsubsection{\setlength{\rightskip}{0pt plus 5cm}{\bf Obit\-FInterpolate}$\ast$ Obit\-FInterpolate\-Copy ({\bf Obit\-FInterpolate} $\ast$ {\em in}, {\bf Obit\-FInterpolate} $\ast$ {\em out}, {\bf Obit\-Err} $\ast$ {\em err})}\label{ObitFInterpolate_8h_a7}


Public: Copy (deep) constructor. 

Copies are made of complex members including disk files; these will be copied applying whatever selection is associated with the input. Parent class members are included but any derived class info is ignored. \begin{Desc}
\item[Parameters:]
\begin{description}
\item[{\em in}]The object to copy \item[{\em out}]An existing object pointer for output or NULL if none exists. \item[{\em err}]Error stack, returns if not empty. \end{description}
\end{Desc}
\begin{Desc}
\item[Returns:]pointer to the new object. \end{Desc}
\index{ObitFInterpolate.h@{Obit\-FInterpolate.h}!ObitFInterpolateGetClass@{ObitFInterpolateGetClass}}
\index{ObitFInterpolateGetClass@{ObitFInterpolateGetClass}!ObitFInterpolate.h@{Obit\-FInterpolate.h}}
\subsubsection{\setlength{\rightskip}{0pt plus 5cm}gconstpointer Obit\-FInterpolate\-Get\-Class (void)}\label{ObitFInterpolate_8h_a6}


Public: Class\-Info pointer. 

\begin{Desc}
\item[Returns:]pointer to the class structure. \end{Desc}
\index{ObitFInterpolate.h@{Obit\-FInterpolate.h}!ObitFInterpolateOffset@{ObitFInterpolateOffset}}
\index{ObitFInterpolateOffset@{ObitFInterpolateOffset}!ObitFInterpolate.h@{Obit\-FInterpolate.h}}
\subsubsection{\setlength{\rightskip}{0pt plus 5cm}{\bf ofloat} Obit\-FInterpolate\-Offset ({\bf Obit\-FInterpolate} $\ast$ {\em in}, {\bf odouble} $\ast$ {\em off}, {\bf Obit\-Err} $\ast$ {\em err})}\label{ObitFInterpolate_8h_a13}


Public: Interpolate Offset in 2D array. 

The object must have an image descriptor to allow determing pixel coordinates. Interpolation between planes is not supported. \begin{Desc}
\item[Parameters:]
\begin{description}
\item[{\em in}]The object to interpolate \item[{\em off}]Coordinate offset in plane \item[{\em err}]Error stack is pixel not inside image. \end{description}
\end{Desc}
\begin{Desc}
\item[Returns:]value, blanked if invalid \end{Desc}
\index{ObitFInterpolate.h@{Obit\-FInterpolate.h}!ObitFInterpolatePixel@{ObitFInterpolatePixel}}
\index{ObitFInterpolatePixel@{ObitFInterpolatePixel}!ObitFInterpolate.h@{Obit\-FInterpolate.h}}
\subsubsection{\setlength{\rightskip}{0pt plus 5cm}{\bf ofloat} Obit\-FInterpolate\-Pixel ({\bf Obit\-FInterpolate} $\ast$ {\em in}, {\bf ofloat} $\ast$ {\em pixel}, {\bf Obit\-Err} $\ast$ {\em err})}\label{ObitFInterpolate_8h_a10}


Public: Interpolate Pixel in 2D array. 

Interpolation between planes is not supported. \begin{Desc}
\item[Parameters:]
\begin{description}
\item[{\em in}]The object to interpolate \item[{\em pixel}]Pixel location (1-rel) in planes and which plane. Should have number of dimensions equal to in. \item[{\em err}]Error stack if pixel not inside image. \end{description}
\end{Desc}
\begin{Desc}
\item[Returns:]value, magic blanked if invalid \end{Desc}
\index{ObitFInterpolate.h@{Obit\-FInterpolate.h}!ObitFInterpolatePosition@{ObitFInterpolatePosition}}
\index{ObitFInterpolatePosition@{ObitFInterpolatePosition}!ObitFInterpolate.h@{Obit\-FInterpolate.h}}
\subsubsection{\setlength{\rightskip}{0pt plus 5cm}{\bf ofloat} Obit\-FInterpolate\-Position ({\bf Obit\-FInterpolate} $\ast$ {\em in}, {\bf odouble} $\ast$ {\em coord}, {\bf Obit\-Err} $\ast$ {\em err})}\label{ObitFInterpolate_8h_a12}


Public: Interpolate Position in 2D array. 

The object must have an image descriptor to allow determing pixel coordinates. Interpolation between planes is not supported. \begin{Desc}
\item[Parameters:]
\begin{description}
\item[{\em in}]The object to interpolate \item[{\em coord}]Coordinate value in plane and which plane. Should have number of dimensions equal to in. \item[{\em err}]Error stack is pixel not inside image. \end{description}
\end{Desc}
\begin{Desc}
\item[Returns:]value, blanked if invalid \end{Desc}
\index{ObitFInterpolate.h@{Obit\-FInterpolate.h}!ObitFInterpolateReplace@{ObitFInterpolateReplace}}
\index{ObitFInterpolateReplace@{ObitFInterpolateReplace}!ObitFInterpolate.h@{Obit\-FInterpolate.h}}
\subsubsection{\setlength{\rightskip}{0pt plus 5cm}void Obit\-FInterpolate\-Replace ({\bf Obit\-FInterpolate} $\ast$ {\em in}, {\bf Obit\-FArray} $\ast$ {\em new\-Array})}\label{ObitFInterpolate_8h_a9}


Public: Replace member {\bf Obit\-FArray}{\rm (p.\,\pageref{structObitFArray})}. 

\begin{Desc}
\item[Parameters:]
\begin{description}
\item[{\em in}]The object to update \item[{\em new\-Array}]The new FArray for in \end{description}
\end{Desc}
