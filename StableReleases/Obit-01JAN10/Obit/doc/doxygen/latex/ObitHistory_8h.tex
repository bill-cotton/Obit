\section{Obit\-History.h File Reference}
\label{ObitHistory_8h}\index{ObitHistory.h@{ObitHistory.h}}
{\bf Obit\-History}{\rm (p.\,\pageref{structObitHistory})} class definition. 

{\tt \#include \char`\"{}Obit.h\char`\"{}}\par
{\tt \#include \char`\"{}Obit\-Err.h\char`\"{}}\par
{\tt \#include \char`\"{}Obit\-Thread.h\char`\"{}}\par
{\tt \#include \char`\"{}Obit\-Info\-List.h\char`\"{}}\par
{\tt \#include \char`\"{}Obit\-IOHistory.h\char`\"{}}\par
\subsection*{Classes}
\begin{CompactItemize}
\item 
struct {\bf Obit\-History}
\begin{CompactList}\small\item\em Obit\-History Class structure. \item\end{CompactList}\item 
struct {\bf Obit\-History\-Class\-Info}
\begin{CompactList}\small\item\em Class\-Info Structure For Table. \item\end{CompactList}\end{CompactItemize}
\subsection*{Defines}
\begin{CompactItemize}
\item 
\#define {\bf Obit\-History\-Unref}(in)\ Obit\-Unref (in)
\begin{CompactList}\small\item\em Macro to unreference (and possibly destroy) an {\bf Obit\-History}{\rm (p.\,\pageref{structObitHistory})} returns an Obit\-History$\ast$. \item\end{CompactList}\item 
\#define {\bf Obit\-History\-Ref}(in)\ Obit\-Ref (in)
\begin{CompactList}\small\item\em Macro to reference (update reference count) an {\bf Obit\-History}{\rm (p.\,\pageref{structObitHistory})}. \item\end{CompactList}\item 
\#define {\bf Obit\-History\-Is\-A}(in)\ Obit\-Is\-A (in, Obit\-History\-Get\-Class())
\begin{CompactList}\small\item\em Macro to determine if an object is the member of this or a derived class. \item\end{CompactList}\item 
\#define {\bf Obit\-History\-Set\-FITS}(in, disk, file, err)
\begin{CompactList}\small\item\em Convenience Macro to define History I/O to a FITS file. \item\end{CompactList}\item 
\#define {\bf Obit\-History\-Set\-AIPS}(in, disk, cno, user, err)
\begin{CompactList}\small\item\em Convenience Macro to define History I/O to an AIPS file. \item\end{CompactList}\end{CompactItemize}
\subsection*{Typedefs}
\begin{CompactItemize}
\item 
typedef {\bf Obit\-History} $\ast$($\ast$ {\bf Obit\-History\-Zap\-FP} )({\bf Obit\-History} $\ast$in, {\bf Obit\-Err} $\ast$err)
\item 
typedef Obit\-IOCode($\ast$ {\bf Obit\-History\-Open\-FP} )({\bf Obit\-History} $\ast$in, Obit\-IOAccess access, {\bf Obit\-Err} $\ast$err)
\item 
typedef Obit\-IOCode($\ast$ {\bf Obit\-History\-Close\-FP} )({\bf Obit\-History} $\ast$in, {\bf Obit\-Err} $\ast$err)
\end{CompactItemize}
\subsection*{Functions}
\begin{CompactItemize}
\item 
void {\bf Obit\-History\-Class\-Init} (void)
\begin{CompactList}\small\item\em Public: Class initializer. \item\end{CompactList}\item 
{\bf Obit\-History} $\ast$ {\bf new\-Obit\-History} (gchar $\ast$name)
\begin{CompactList}\small\item\em Public: Default constructor. \item\end{CompactList}\item 
gconstpointer {\bf Obit\-History\-Get\-Class} (void)
\begin{CompactList}\small\item\em Public: Class\-Info pointer. \item\end{CompactList}\item 
{\bf Obit\-History} $\ast$ {\bf new\-Obit\-History\-Value} (gchar $\ast$name, {\bf Obit\-Info\-List} $\ast$info, {\bf Obit\-Err} $\ast$err)
\begin{CompactList}\small\item\em Public: Constructor from object info\-List. \item\end{CompactList}\item 
{\bf Obit\-History} $\ast$ {\bf Obit\-History\-Zap} ({\bf Obit\-History} $\ast$in, {\bf Obit\-Err} $\ast$err)
\begin{CompactList}\small\item\em Public: Delete underlying structures. \item\end{CompactList}\item 
{\bf Obit\-History} $\ast$ {\bf Obit\-History\-Copy} ({\bf Obit\-History} $\ast$in, {\bf Obit\-History} $\ast$out, {\bf Obit\-Err} $\ast$err)
\begin{CompactList}\small\item\em Public: Deep copy. \item\end{CompactList}\item 
Obit\-IOCode {\bf Obit\-History\-Copy\-Header} ({\bf Obit\-History} $\ast$in, {\bf Obit\-History} $\ast$out, {\bf Obit\-Err} $\ast$err)
\begin{CompactList}\small\item\em Public: Copy history from header (FITS). \item\end{CompactList}\item 
Obit\-IOCode {\bf Obit\-History\-Copy2Header} ({\bf Obit\-History} $\ast$in, {\bf Obit\-History} $\ast$out, {\bf Obit\-Err} $\ast$err)
\begin{CompactList}\small\item\em Public: Copy history to header (FITS). \item\end{CompactList}\item 
Obit\-IOCode {\bf Obit\-History\-Header2Header} ({\bf Obit\-History} $\ast$in, {\bf Obit\-History} $\ast$out, {\bf Obit\-Err} $\ast$err)
\begin{CompactList}\small\item\em Public: Copy history from header (FITS) to header (FITS). \item\end{CompactList}\item 
Obit\-IOCode {\bf Obit\-History\-Open} ({\bf Obit\-History} $\ast$in, Obit\-IOAccess access, {\bf Obit\-Err} $\ast$err)
\begin{CompactList}\small\item\em Public: Create {\bf Obit\-IO}{\rm (p.\,\pageref{structObitIO})} structures and open file. \item\end{CompactList}\item 
Obit\-IOCode {\bf Obit\-History\-Close} ({\bf Obit\-History} $\ast$in, {\bf Obit\-Err} $\ast$err)
\begin{CompactList}\small\item\em Public: Close file and become inactive. \item\end{CompactList}\item 
Obit\-IOCode {\bf Obit\-History\-Read\-Rec} ({\bf Obit\-History} $\ast$in, {\bf olong} recno, gchar hi\-Card[73], {\bf Obit\-Err} $\ast$err)
\begin{CompactList}\small\item\em Public: Read specified Record. \item\end{CompactList}\item 
Obit\-IOCode {\bf Obit\-History\-Write\-Rec} ({\bf Obit\-History} $\ast$in, {\bf olong} rowno, gchar hi\-Card[73], {\bf Obit\-Err} $\ast$err)
\begin{CompactList}\small\item\em Public: Write specified Record. \item\end{CompactList}\item 
Obit\-IOCode {\bf Obit\-History\-Time\-Stamp} ({\bf Obit\-History} $\ast$in, gchar $\ast$label, {\bf Obit\-Err} $\ast$err)
\begin{CompactList}\small\item\em Public: Add time stamp and label. \item\end{CompactList}\item 
Obit\-IOCode {\bf Obit\-History\-Copy\-Info\-List} ({\bf Obit\-History} $\ast$out, gchar $\ast$pgm\-Name, gchar $\ast$list[$\,$], {\bf Obit\-Info\-List} $\ast$info, {\bf Obit\-Err} $\ast$err)
\begin{CompactList}\small\item\em Public: Copy a list of values from an Info\-List to a History. \item\end{CompactList}\item 
{\bf olong} {\bf Obit\-History\-Num\-Rec} ({\bf Obit\-History} $\ast$in)
\begin{CompactList}\small\item\em Public: Tell number of history records. \item\end{CompactList}\end{CompactItemize}


\subsection{Detailed Description}
{\bf Obit\-History}{\rm (p.\,\pageref{structObitHistory})} class definition. 

This class is derived from the {\bf Obit}{\rm (p.\,\pageref{structObit})} class.

This class contains a processing history in tabular form. An {\bf Obit\-History}{\rm (p.\,\pageref{structObitHistory})} is the front end to a persistent disk resident structure. Both FITS (as Tables) and AIPS cataloged data are supported. This class is derived from the {\bf Obit}{\rm (p.\,\pageref{structObit})} class. The AIPS conventions for history records are give the program name followed by parameters in keyword=value form and to preceed any non parsable text by the FITS comment header comment delimiter '/'.\subsection{History\-Table data storage}\label{ObitTableWX_8h_TableDataStorage}
History tables are accessed as tables although the AIPS implementation is a pre-table version. History records are blocked into 70 character fixed strings althought AIPS internally uses 72.\subsection{Access to History}\label{ObitHistory_8h_ObitHistorySpecification}
History records are stored in a system dependent fashion. AIPS history records are stored in an AIPS HI$\ast$ file with 72 characters per record ({\bf Obit}{\rm (p.\,\pageref{structObit})} only uses 70). FOR FITS files, history records are normally kept in A History binary table but can be read or written to tyhe more traditional HISTORY keywords using {\bf Obit\-History\-Copy\-Header}{\rm (p.\,\pageref{ObitHistory_8c_a12})}, or {\bf Obit\-History\-Copy2Header}{\rm (p.\,\pageref{ObitHistory_8c_a13})} for entire collections or {\bf Obit\-File\-FITS}{\rm (p.\,\pageref{structObitFileFITS})} functions {\bf Obit\-File\-FITSRead\-History}{\rm (p.\,\pageref{ObitFileFITS_8c_a21})} and {\bf Obit\-File\-FITSWrite\-History}{\rm (p.\,\pageref{ObitFileFITS_8c_a26})} for individual records. Access to the history component of an object (e.g. Image, UV data) can be obtained using the {\bf Obit\-Info\-List}{\rm (p.\,\pageref{structObitInfoList})} containing the information defining the underlying file. This uses routine {\bf new\-Obit\-History\-Value}{\rm (p.\,\pageref{ObitHistory_8c_a8})}. Then history lines can be read or written one at a time using {\bf Obit\-History\-Open}{\rm (p.\,\pageref{ObitHistory_8c_a15})}, {\bf Obit\-History\-Read\-Rec}{\rm (p.\,\pageref{ObitHistory_8c_a17})}, {\bf Obit\-History\-Write\-Rec}{\rm (p.\,\pageref{ObitHistory_8c_a18})}, {\bf Obit\-History\-Time\-Stamp}{\rm (p.\,\pageref{ObitHistory_8c_a19})}, {\bf Obit\-History\-Close}{\rm (p.\,\pageref{ObitHistory_8c_a16})}. The contents of entire history files may be copied using {\bf Obit\-History\-Copy}{\rm (p.\,\pageref{ObitHistory_8c_a11})}, or {\bf Obit\-History\-Copy\-Header}{\rm (p.\,\pageref{ObitHistory_8c_a12})} to copy HISTORY records from a FITS header or {\bf Obit\-History\-Copy2Header}{\rm (p.\,\pageref{ObitHistory_8c_a13})} to copy a history file to a FITS header.\subsubsection{FITS files}\label{ObitTableWX_8h_TableFITS}
The {\bf Obit}{\rm (p.\,\pageref{structObit})} FITS implementation uses a HISTORY table unlike the standard FITS practice of keeping histories in the main HDU header regular FITS images, gzip compressed files, pipes, shared memory and a number of other input forms. The convenience Macro {\bf Obit\-History\-Set\-FITS}{\rm (p.\,\pageref{ObitHistory_8h_a3})} simplifies specifying the desired data. \begin{itemize}
\item \char`\"{}Disk\char`\"{} OBIT\_\-int (1,1,1) FITS \char`\"{}disk\char`\"{} number. \item \char`\"{}File\-Name\char`\"{} OBIT\_\-string (?,1,1) FITS file name.\end{itemize}
\subsubsection{AIPS files}\label{ObitHistory_8h_ObitHistoryAIPS}
The {\bf Obit\-AIPS}{\rm (p.\,\pageref{structObitAIPS})} class must be initialized before accessing AIPS files; this uses {\bf Obit\-AIPSClass\-Init}{\rm (p.\,\pageref{ObitAIPS_8c_a5})}. The convenience Macro Obit\-History\-Set\-AIPS simplifies specifying the desired data. For accessing AIPS files the following entries in the {\bf Obit\-Info\-List}{\rm (p.\,\pageref{structObitInfoList})} are used: \begin{itemize}
\item \char`\"{}Disk\char`\"{} OBIT\_\-int (1,1,1) AIPS \char`\"{}disk\char`\"{} number. \item \char`\"{}User\char`\"{} OBIT\_\-int (1,1,1) user number. \item \char`\"{}CNO\char`\"{} OBIT\_\-int (1,1,1) AIPS catalog slot number.\end{itemize}
\subsection{Creators and Destructors}\label{ObitHistory_8h_ObitHistoryaccess}
A copy of a pointer to an {\bf Obit\-History}{\rm (p.\,\pageref{structObitHistory})} should always be made using the Obit\-History\-Ref function which updates the reference count in the object. Then whenever freeing an {\bf Obit\-History}{\rm (p.\,\pageref{structObitHistory})} or changing a pointer, the function Obit\-History\-Unref will decrement the reference count and destroy the object when the reference count hits 0.\subsection{I/O}\label{ObitHistory_8h_ObitHistoryUsage}
Visibility data is available after an input object is \char`\"{}Opened\char`\"{} and \char`\"{}Read\char`\"{}. I/O optionally uses a buffer attached to the {\bf Obit\-History}{\rm (p.\,\pageref{structObitHistory})} or some external location. To Write an {\bf Obit\-History}{\rm (p.\,\pageref{structObitHistory})}, create it, open it, and write. The object should be closed to ensure all data is flushed to disk. Deletion of an {\bf Obit\-History}{\rm (p.\,\pageref{structObitHistory})} after its final unreferencing will automatically close it.\subsection{Obit\-History\-Row}\label{ObitHistory_8h_ObitHistoryRow}
An Obit\-History\-Row is used to pass the data for a single table row. It is most useful in derived classes where it includes entries by symbolic names. Obit\-History\-Row class definitions and functions are included in the files defining the associated {\bf Obit\-History}{\rm (p.\,\pageref{structObitHistory})}. Obit\-History\-Rows are derived from basal class {\bf Obit}{\rm (p.\,\pageref{structObit})}.

\subsection{Define Documentation}
\index{ObitHistory.h@{Obit\-History.h}!ObitHistoryIsA@{ObitHistoryIsA}}
\index{ObitHistoryIsA@{ObitHistoryIsA}!ObitHistory.h@{Obit\-History.h}}
\subsubsection{\setlength{\rightskip}{0pt plus 5cm}\#define Obit\-History\-Is\-A(in)\ Obit\-Is\-A (in, Obit\-History\-Get\-Class())}\label{ObitHistory_8h_a2}


Macro to determine if an object is the member of this or a derived class. 

Returns TRUE if a member, else FALSE in = object to reference \index{ObitHistory.h@{Obit\-History.h}!ObitHistoryRef@{ObitHistoryRef}}
\index{ObitHistoryRef@{ObitHistoryRef}!ObitHistory.h@{Obit\-History.h}}
\subsubsection{\setlength{\rightskip}{0pt plus 5cm}\#define Obit\-History\-Ref(in)\ Obit\-Ref (in)}\label{ObitHistory_8h_a1}


Macro to reference (update reference count) an {\bf Obit\-History}{\rm (p.\,\pageref{structObitHistory})}. 

returns an Obit\-History$\ast$. in = object to reference \index{ObitHistory.h@{Obit\-History.h}!ObitHistorySetAIPS@{ObitHistorySetAIPS}}
\index{ObitHistorySetAIPS@{ObitHistorySetAIPS}!ObitHistory.h@{Obit\-History.h}}
\subsubsection{\setlength{\rightskip}{0pt plus 5cm}\#define Obit\-History\-Set\-AIPS(in, disk, cno, user, err)}\label{ObitHistory_8h_a4}


{\bf Value:}

\footnotesize\begin{verbatim}G_STMT_START{  \
       in->info->dim[0]=1; in->info->dim[1]=1; in->info->dim[2]=1;  \
       in->info->dim[3]=1; in->info->dim[4]=1;                      \
       in->info->work[0] = OBIT_IO_AIPS;                            \
       ObitInfoListPut (in->info, "FileType", OBIT_long,             \
                  in->info->dim, (gpointer)&in->info->work[0], err);\
       in->info->dim[0] = 1;                                        \
       ObitInfoListPut (in->info, "Disk", OBIT_long,                 \
                 in->info->dim, (gpointer)&disk, err);              \
       ObitInfoListPut (in->info, "CNO", OBIT_long,                  \
                 in->info->dim, (gpointer)&cno, err);               \
       ObitInfoListPut (in->info, "User", OBIT_long,                 \
                 in->info->dim, (gpointer)&user, err);              \
     }G_STMT_END
\end{verbatim}\normalsize 
Convenience Macro to define History I/O to an AIPS file. 

Sets values on {\bf Obit\-Info\-List}{\rm (p.\,\pageref{structObitInfoList})} on input object. \begin{itemize}
\item in = {\bf Obit}{\rm (p.\,\pageref{structObit})} Object with Info\-List member to specify i/O for. \item disk = AIPS disk number \item cno = catalog slot number \item user = User id number \item err = {\bf Obit\-Err}{\rm (p.\,\pageref{structObitErr})} to receive error messages. \end{itemize}
\index{ObitHistory.h@{Obit\-History.h}!ObitHistorySetFITS@{ObitHistorySetFITS}}
\index{ObitHistorySetFITS@{ObitHistorySetFITS}!ObitHistory.h@{Obit\-History.h}}
\subsubsection{\setlength{\rightskip}{0pt plus 5cm}\#define Obit\-History\-Set\-FITS(in, disk, file, err)}\label{ObitHistory_8h_a3}


{\bf Value:}

\footnotesize\begin{verbatim}G_STMT_START{ \
       in->info->dim[0]=1; in->info->dim[1]=1; in->info->dim[2]=1;  \
       in->info->dim[3]=1; in->info->dim[4]=1;                      \
       in->info->work[0] = OBIT_IO_FITS;                            \
       in->info->work[2] = disk;                                    \
       in->info->dim[0] = 1;                                        \
       ObitInfoListPut (in->info, "Disk", OBIT_long,                 \
                 in->info->dim, (gpointer)&in->info->work[2], err); \
       ObitInfoListPut (in->info, "FileType", OBIT_long,             \
                  in->info->dim, (gpointer)&in->info->work[0], err);\
       in->info->dim[0] = strlen(file);                             \
       ObitInfoListPut (in->info, "FileName", OBIT_string,          \
                 in->info->dim, (gpointer)file, err);               \
     }G_STMT_END
\end{verbatim}\normalsize 
Convenience Macro to define History I/O to a FITS file. 

Sets values on {\bf Obit\-Info\-List}{\rm (p.\,\pageref{structObitInfoList})} on input object. \begin{itemize}
\item in = {\bf Obit}{\rm (p.\,\pageref{structObit})} Object with Info\-List member to specify i/O for. \item disk = FITS disk number \item file = Specified FITS file name. \item err = {\bf Obit\-Err}{\rm (p.\,\pageref{structObitErr})} to receive error messages. \end{itemize}
\index{ObitHistory.h@{Obit\-History.h}!ObitHistoryUnref@{ObitHistoryUnref}}
\index{ObitHistoryUnref@{ObitHistoryUnref}!ObitHistory.h@{Obit\-History.h}}
\subsubsection{\setlength{\rightskip}{0pt plus 5cm}\#define Obit\-History\-Unref(in)\ Obit\-Unref (in)}\label{ObitHistory_8h_a0}


Macro to unreference (and possibly destroy) an {\bf Obit\-History}{\rm (p.\,\pageref{structObitHistory})} returns an Obit\-History$\ast$. 

in = object to unreference 

\subsection{Typedef Documentation}
\index{ObitHistory.h@{Obit\-History.h}!ObitHistoryCloseFP@{ObitHistoryCloseFP}}
\index{ObitHistoryCloseFP@{ObitHistoryCloseFP}!ObitHistory.h@{Obit\-History.h}}
\subsubsection{\setlength{\rightskip}{0pt plus 5cm}typedef Obit\-IOCode($\ast$ {\bf Obit\-History\-Close\-FP})({\bf Obit\-History} $\ast$in, {\bf Obit\-Err} $\ast$err)}\label{ObitHistory_8h_a7}


\index{ObitHistory.h@{Obit\-History.h}!ObitHistoryOpenFP@{ObitHistoryOpenFP}}
\index{ObitHistoryOpenFP@{ObitHistoryOpenFP}!ObitHistory.h@{Obit\-History.h}}
\subsubsection{\setlength{\rightskip}{0pt plus 5cm}typedef Obit\-IOCode($\ast$ {\bf Obit\-History\-Open\-FP})({\bf Obit\-History} $\ast$in, Obit\-IOAccess access, {\bf Obit\-Err} $\ast$err)}\label{ObitHistory_8h_a6}


\index{ObitHistory.h@{Obit\-History.h}!ObitHistoryZapFP@{ObitHistoryZapFP}}
\index{ObitHistoryZapFP@{ObitHistoryZapFP}!ObitHistory.h@{Obit\-History.h}}
\subsubsection{\setlength{\rightskip}{0pt plus 5cm}typedef {\bf Obit\-History}$\ast$($\ast$ {\bf Obit\-History\-Zap\-FP})({\bf Obit\-History} $\ast$in, {\bf Obit\-Err} $\ast$err)}\label{ObitHistory_8h_a5}




\subsection{Function Documentation}
\index{ObitHistory.h@{Obit\-History.h}!newObitHistory@{newObitHistory}}
\index{newObitHistory@{newObitHistory}!ObitHistory.h@{Obit\-History.h}}
\subsubsection{\setlength{\rightskip}{0pt plus 5cm}{\bf Obit\-History}$\ast$ new\-Obit\-History (gchar $\ast$ {\em name})}\label{ObitHistory_8h_a9}


Public: Default constructor. 

Initializes class if needed on first call. \begin{Desc}
\item[Parameters:]
\begin{description}
\item[{\em name}]An optional name for the object. \end{description}
\end{Desc}
\begin{Desc}
\item[Returns:]the new object. \end{Desc}
\index{ObitHistory.h@{Obit\-History.h}!newObitHistoryValue@{newObitHistoryValue}}
\index{newObitHistoryValue@{newObitHistoryValue}!ObitHistory.h@{Obit\-History.h}}
\subsubsection{\setlength{\rightskip}{0pt plus 5cm}{\bf Obit\-History}$\ast$ new\-Obit\-History\-Value (gchar $\ast$ {\em name}, {\bf Obit\-Info\-List} $\ast$ {\em info}, {\bf Obit\-Err} $\ast$ {\em err})}\label{ObitHistory_8h_a11}


Public: Constructor from object info\-List. 

\begin{Desc}
\item[Parameters:]
\begin{description}
\item[{\em name}]An optional name for the object. \item[{\em info}]Parent object list defining the underlying file e.g. File\-Type, disk, name for FITS, disk, user, cno for AIPS. \item[{\em err}]Error stack, returns if not empty. \end{description}
\end{Desc}
\begin{Desc}
\item[Returns:]the new object. \end{Desc}
\index{ObitHistory.h@{Obit\-History.h}!ObitHistoryClassInit@{ObitHistoryClassInit}}
\index{ObitHistoryClassInit@{ObitHistoryClassInit}!ObitHistory.h@{Obit\-History.h}}
\subsubsection{\setlength{\rightskip}{0pt plus 5cm}void Obit\-History\-Class\-Init (void)}\label{ObitHistory_8h_a8}


Public: Class initializer. 

\index{ObitHistory.h@{Obit\-History.h}!ObitHistoryClose@{ObitHistoryClose}}
\index{ObitHistoryClose@{ObitHistoryClose}!ObitHistory.h@{Obit\-History.h}}
\subsubsection{\setlength{\rightskip}{0pt plus 5cm}Obit\-IOCode Obit\-History\-Close ({\bf Obit\-History} $\ast$ {\em in}, {\bf Obit\-Err} $\ast$ {\em err})}\label{ObitHistory_8h_a18}


Public: Close file and become inactive. 

\begin{Desc}
\item[Parameters:]
\begin{description}
\item[{\em in}]Pointer to object to be closed. \item[{\em err}]{\bf Obit\-Err}{\rm (p.\,\pageref{structObitErr})} for reporting errors. \end{description}
\end{Desc}
\begin{Desc}
\item[Returns:]error code, OBIT\_\-IO\_\-OK=$>$ OK \end{Desc}
\index{ObitHistory.h@{Obit\-History.h}!ObitHistoryCopy@{ObitHistoryCopy}}
\index{ObitHistoryCopy@{ObitHistoryCopy}!ObitHistory.h@{Obit\-History.h}}
\subsubsection{\setlength{\rightskip}{0pt plus 5cm}{\bf Obit\-History}$\ast$ Obit\-History\-Copy ({\bf Obit\-History} $\ast$ {\em in}, {\bf Obit\-History} $\ast$ {\em out}, {\bf Obit\-Err} $\ast$ {\em err})}\label{ObitHistory_8h_a13}


Public: Deep copy. 

Both objects should be filly defined. \begin{Desc}
\item[Parameters:]
\begin{description}
\item[{\em in}]The object to copy, if underlying structures don't exist, it merely returns without writing the out History. \item[{\em out}]An existing object pointer for output \item[{\em err}]Error stack, returns if not empty. \end{description}
\end{Desc}
\begin{Desc}
\item[Returns:]pointer to the new object. \end{Desc}
\index{ObitHistory.h@{Obit\-History.h}!ObitHistoryCopy2Header@{ObitHistoryCopy2Header}}
\index{ObitHistoryCopy2Header@{ObitHistoryCopy2Header}!ObitHistory.h@{Obit\-History.h}}
\subsubsection{\setlength{\rightskip}{0pt plus 5cm}Obit\-IOCode Obit\-History\-Copy2Header ({\bf Obit\-History} $\ast$ {\em in}, {\bf Obit\-History} $\ast$ {\em out}, {\bf Obit\-Err} $\ast$ {\em err})}\label{ObitHistory_8h_a15}


Public: Copy history to header (FITS). 

Both objects should be filly defined. \begin{Desc}
\item[Parameters:]
\begin{description}
\item[{\em in}]The object to copy \item[{\em out}]Output object for HISTORY header entries. \item[{\em err}]Error stack, returns if not empty. \end{description}
\end{Desc}
\begin{Desc}
\item[Returns:]pointer to the new object. \end{Desc}
\index{ObitHistory.h@{Obit\-History.h}!ObitHistoryCopyHeader@{ObitHistoryCopyHeader}}
\index{ObitHistoryCopyHeader@{ObitHistoryCopyHeader}!ObitHistory.h@{Obit\-History.h}}
\subsubsection{\setlength{\rightskip}{0pt plus 5cm}Obit\-IOCode Obit\-History\-Copy\-Header ({\bf Obit\-History} $\ast$ {\em in}, {\bf Obit\-History} $\ast$ {\em out}, {\bf Obit\-Err} $\ast$ {\em err})}\label{ObitHistory_8h_a14}


Public: Copy history from header (FITS). 

\begin{Desc}
\item[Parameters:]
\begin{description}
\item[{\em in}]The object to copy \item[{\em out}]An existing object pointer for output \item[{\em err}]Error stack, returns if not empty. \end{description}
\end{Desc}
\begin{Desc}
\item[Returns:]pointer to the new object. \end{Desc}
\index{ObitHistory.h@{Obit\-History.h}!ObitHistoryCopyInfoList@{ObitHistoryCopyInfoList}}
\index{ObitHistoryCopyInfoList@{ObitHistoryCopyInfoList}!ObitHistory.h@{Obit\-History.h}}
\subsubsection{\setlength{\rightskip}{0pt plus 5cm}Obit\-IOCode Obit\-History\-Copy\-Info\-List ({\bf Obit\-History} $\ast$ {\em out}, gchar $\ast$ {\em pgm\-Name}, gchar $\ast$ {\em list}[$\,$], {\bf Obit\-Info\-List} $\ast$ {\em info}, {\bf Obit\-Err} $\ast$ {\em err})}\label{ObitHistory_8h_a22}


Public: Copy a list of values from an Info\-List to a History. 

\begin{Desc}
\item[Parameters:]
\begin{description}
\item[{\em out}]Output object for HISTORY header entries. \item[{\em list}]NULL terminated list of entries in info \item[{\em info}]{\bf Obit\-Info\-List}{\rm (p.\,\pageref{structObitInfoList})} with values to copy \item[{\em err}]Error stack, returns if not empty. \end{description}
\end{Desc}
\begin{Desc}
\item[Returns:]pointer to the new object. \end{Desc}
\index{ObitHistory.h@{Obit\-History.h}!ObitHistoryGetClass@{ObitHistoryGetClass}}
\index{ObitHistoryGetClass@{ObitHistoryGetClass}!ObitHistory.h@{Obit\-History.h}}
\subsubsection{\setlength{\rightskip}{0pt plus 5cm}gconstpointer Obit\-History\-Get\-Class (void)}\label{ObitHistory_8h_a10}


Public: Class\-Info pointer. 

\begin{Desc}
\item[Returns:]pointer to the class structure. \end{Desc}
\index{ObitHistory.h@{Obit\-History.h}!ObitHistoryHeader2Header@{ObitHistoryHeader2Header}}
\index{ObitHistoryHeader2Header@{ObitHistoryHeader2Header}!ObitHistory.h@{Obit\-History.h}}
\subsubsection{\setlength{\rightskip}{0pt plus 5cm}Obit\-IOCode Obit\-History\-Header2Header ({\bf Obit\-History} $\ast$ {\em in}, {\bf Obit\-History} $\ast$ {\em out}, {\bf Obit\-Err} $\ast$ {\em err})}\label{ObitHistory_8h_a16}


Public: Copy history from header (FITS) to header (FITS). 

Both objects should be filly defined. \begin{Desc}
\item[Parameters:]
\begin{description}
\item[{\em in}]The object to copy \item[{\em out}]Output object for HISTORY header entries. \item[{\em err}]Error stack, returns if not empty. \end{description}
\end{Desc}
\begin{Desc}
\item[Returns:]pointer to the new object. \end{Desc}
\index{ObitHistory.h@{Obit\-History.h}!ObitHistoryNumRec@{ObitHistoryNumRec}}
\index{ObitHistoryNumRec@{ObitHistoryNumRec}!ObitHistory.h@{Obit\-History.h}}
\subsubsection{\setlength{\rightskip}{0pt plus 5cm}{\bf olong} Obit\-History\-Num\-Rec ({\bf Obit\-History} $\ast$ {\em in})}\label{ObitHistory_8h_a23}


Public: Tell number of history records. 

\begin{Desc}
\item[Parameters:]
\begin{description}
\item[{\em in}]Pointer to open object to be tested \end{description}
\end{Desc}
\begin{Desc}
\item[Returns:]number of records, $<$0 =$>$ problem \end{Desc}
\index{ObitHistory.h@{Obit\-History.h}!ObitHistoryOpen@{ObitHistoryOpen}}
\index{ObitHistoryOpen@{ObitHistoryOpen}!ObitHistory.h@{Obit\-History.h}}
\subsubsection{\setlength{\rightskip}{0pt plus 5cm}Obit\-IOCode Obit\-History\-Open ({\bf Obit\-History} $\ast$ {\em in}, Obit\-IOAccess {\em access}, {\bf Obit\-Err} $\ast$ {\em err})}\label{ObitHistory_8h_a17}


Public: Create {\bf Obit\-IO}{\rm (p.\,\pageref{structObitIO})} structures and open file. 

The image descriptor is read if OBIT\_\-IO\_\-Read\-Only or OBIT\_\-IO\_\-Read\-Write and written to disk if opened OBIT\_\-IO\_\-Write\-Only. After the file has been opened the member, buffer is initialized for reading/storing the table unless member buffer\-Size is $<$0. If the requested version (\char`\"{}Ver\char`\"{} in Info\-List) is 0 then the highest numbered table of the same type is opened on Read or Read/Write, or a new table is created on on Write. The file etc. info should have been stored in the {\bf Obit\-Info\-List}{\rm (p.\,\pageref{structObitInfoList})}: \begin{itemize}
\item \char`\"{}File\-Type\char`\"{} OBIT\_\-long scalar = OBIT\_\-IO\_\-FITS or OBIT\_\-IO\_\-AIPS for file type (see class documentation for details). \begin{Desc}
\item[Parameters:]
\begin{description}
\item[{\em in}]Pointer to object to be opened. \item[{\em access}]access (OBIT\_\-IO\_\-Read\-Only,OBIT\_\-IO\_\-Read\-Write, or OBIT\_\-IO\_\-Write\-Only). If OBIT\_\-IO\_\-Write\-Only any existing data in the output file will be lost. \item[{\em err}]{\bf Obit\-Err}{\rm (p.\,\pageref{structObitErr})} for reporting errors. \end{description}
\end{Desc}
\begin{Desc}
\item[Returns:]return code, OBIT\_\-IO\_\-OK=$>$ OK \end{Desc}
\end{itemize}
\index{ObitHistory.h@{Obit\-History.h}!ObitHistoryReadRec@{ObitHistoryReadRec}}
\index{ObitHistoryReadRec@{ObitHistoryReadRec}!ObitHistory.h@{Obit\-History.h}}
\subsubsection{\setlength{\rightskip}{0pt plus 5cm}Obit\-IOCode Obit\-History\-Read\-Rec ({\bf Obit\-History} $\ast$ {\em in}, {\bf olong} {\em recno}, gchar {\em hi\-Card}[73], {\bf Obit\-Err} $\ast$ {\em err})}\label{ObitHistory_8h_a19}


Public: Read specified Record. 

\begin{Desc}
\item[Parameters:]
\begin{description}
\item[{\em in}]Pointer to object to be read. \item[{\em recno}]Record number to read, -1 = next; \item[{\em hi\-Card}]Char array to accept line \item[{\em err}]{\bf Obit\-Err}{\rm (p.\,\pageref{structObitErr})} for reporting errors. \end{description}
\end{Desc}
\begin{Desc}
\item[Returns:]return code, OBIT\_\-IO\_\-OK =$>$ OK \end{Desc}
\index{ObitHistory.h@{Obit\-History.h}!ObitHistoryTimeStamp@{ObitHistoryTimeStamp}}
\index{ObitHistoryTimeStamp@{ObitHistoryTimeStamp}!ObitHistory.h@{Obit\-History.h}}
\subsubsection{\setlength{\rightskip}{0pt plus 5cm}Obit\-IOCode Obit\-History\-Time\-Stamp ({\bf Obit\-History} $\ast$ {\em in}, gchar $\ast$ {\em label}, {\bf Obit\-Err} $\ast$ {\em err})}\label{ObitHistory_8h_a21}


Public: Add time stamp and label. 

\begin{Desc}
\item[Parameters:]
\begin{description}
\item[{\em in}]Pointer to object to be written. \item[{\em label}]Label string for record \item[{\em err}]{\bf Obit\-Err}{\rm (p.\,\pageref{structObitErr})} for reporting errors. \end{description}
\end{Desc}
\begin{Desc}
\item[Returns:]return code, OBIT\_\-IO\_\-OK =$>$ OK \end{Desc}
\index{ObitHistory.h@{Obit\-History.h}!ObitHistoryWriteRec@{ObitHistoryWriteRec}}
\index{ObitHistoryWriteRec@{ObitHistoryWriteRec}!ObitHistory.h@{Obit\-History.h}}
\subsubsection{\setlength{\rightskip}{0pt plus 5cm}Obit\-IOCode Obit\-History\-Write\-Rec ({\bf Obit\-History} $\ast$ {\em in}, {\bf olong} {\em recno}, gchar {\em hi\-Card}[73], {\bf Obit\-Err} $\ast$ {\em err})}\label{ObitHistory_8h_a20}


Public: Write specified Record. 

\begin{Desc}
\item[Parameters:]
\begin{description}
\item[{\em in}]Pointer to object to be written. \item[{\em rowno}]Record number to write, -1 = next; \item[{\em err}]{\bf Obit\-Err}{\rm (p.\,\pageref{structObitErr})} for reporting errors. \end{description}
\end{Desc}
\begin{Desc}
\item[Returns:]return code, OBIT\_\-IO\_\-OK =$>$ OK \end{Desc}
\index{ObitHistory.h@{Obit\-History.h}!ObitHistoryZap@{ObitHistoryZap}}
\index{ObitHistoryZap@{ObitHistoryZap}!ObitHistory.h@{Obit\-History.h}}
\subsubsection{\setlength{\rightskip}{0pt plus 5cm}{\bf Obit\-History}$\ast$ Obit\-History\-Zap ({\bf Obit\-History} $\ast$ {\em in}, {\bf Obit\-Err} $\ast$ {\em err})}\label{ObitHistory_8h_a12}


Public: Delete underlying structures. 

\begin{Desc}
\item[Parameters:]
\begin{description}
\item[{\em in}]Pointer to object to be zapped. \item[{\em err}]{\bf Obit\-Err}{\rm (p.\,\pageref{structObitErr})} for reporting errors. \end{description}
\end{Desc}
\begin{Desc}
\item[Returns:]pointer for input object, NULL if deletion successful \end{Desc}
