\section{Obit\-FFT.h File Reference}
\label{ObitFFT_8h}\index{ObitFFT.h@{ObitFFT.h}}
{\bf Obit\-FFT}{\rm (p.\,\pageref{structObitFFT})} Fast Fourier Transform class definition. 

{\tt \#include \char`\"{}Obit.h\char`\"{}}\par
{\tt \#include \char`\"{}Obit\-Thread.h\char`\"{}}\par
{\tt \#include \char`\"{}Obit\-FArray.h\char`\"{}}\par
{\tt \#include \char`\"{}Obit\-CArray.h\char`\"{}}\par
\subsection*{Classes}
\begin{CompactItemize}
\item 
struct {\bf Obit\-FFT}
\begin{CompactList}\small\item\em Obit\-FFT Class structure. \item\end{CompactList}\item 
struct {\bf Obit\-FFTClass\-Info}
\begin{CompactList}\small\item\em Class\-Info Structure. \item\end{CompactList}\end{CompactItemize}
\subsection*{Defines}
\begin{CompactItemize}
\item 
\#define {\bf Obit\-FFTUnref}(in)\ Obit\-Unref (in)
\begin{CompactList}\small\item\em Macro to unreference (and possibly destroy) an {\bf Obit\-FFT}{\rm (p.\,\pageref{structObitFFT})} returns a Obit\-FFT$\ast$. \item\end{CompactList}\item 
\#define {\bf Obit\-FFTRef}(in)\ Obit\-Ref (in)
\begin{CompactList}\small\item\em Macro to reference (update reference count) an {\bf Obit\-FFT}{\rm (p.\,\pageref{structObitFFT})}. \item\end{CompactList}\item 
\#define {\bf Obit\-FFTIs\-A}(in)\ Obit\-Is\-A (in, Obit\-FFTGet\-Class())
\begin{CompactList}\small\item\em Macro to determine if an object is the member of this or a derived class. \item\end{CompactList}\end{CompactItemize}
\subsection*{Typedefs}
\begin{CompactItemize}
\item 
typedef {\bf Obit\-FFT} $\ast$($\ast$ {\bf new\-Obit\-FFTFP} )(gchar $\ast$name, Obit\-FFTdir dir, Obit\-FFTtype type, {\bf olong} rank, {\bf olong} $\ast$dim)
\begin{CompactList}\small\item\em Typedef for definition of class pointer structure. \item\end{CompactList}\item 
typedef void($\ast$ {\bf Obit\-FFTR2CFP} )({\bf Obit\-FFT} $\ast$in, {\bf Obit\-FArray} $\ast$in\-Array, {\bf Obit\-CArray} $\ast$out\-Array)
\item 
typedef void($\ast$ {\bf Obit\-FFTC2RFP} )({\bf Obit\-FFT} $\ast$in, {\bf Obit\-CArray} $\ast$in\-Array, {\bf Obit\-FArray} $\ast$out\-Array)
\item 
typedef void($\ast$ {\bf Obit\-FFTC2CFP} )({\bf Obit\-FFT} $\ast$in, {\bf Obit\-CArray} $\ast$in\-Array, {\bf Obit\-CArray} $\ast$out\-Array)
\end{CompactItemize}
\subsection*{Enumerations}
\begin{CompactItemize}
\item 
enum {\bf obit\-FFTdir} \{ {\bf OBIT\_\-FFT\_\-Forward}, 
{\bf OBIT\_\-FFT\_\-Reverse}
 \}
\item 
enum {\bf obit\-FFTtype} \{ {\bf OBIT\_\-FFT\_\-Full\-Complex}, 
{\bf OBIT\_\-FFT\_\-Half\-Complex}
 \}
\end{CompactItemize}
\subsection*{Functions}
\begin{CompactItemize}
\item 
void {\bf Obit\-FFTClass\-Init} (void)
\begin{CompactList}\small\item\em Public: Class initializer. \item\end{CompactList}\item 
{\bf Obit\-FFT} $\ast$ {\bf new\-Obit\-FFT} (gchar $\ast$name, Obit\-FFTdir dir, Obit\-FFTtype type, {\bf olong} rank, {\bf olong} $\ast$dim)
\begin{CompactList}\small\item\em Public: Constructor. \item\end{CompactList}\item 
gconstpointer {\bf Obit\-FFTGet\-Class} (void)
\begin{CompactList}\small\item\em Public: Class\-Info pointer. \item\end{CompactList}\item 
{\bf olong} {\bf Obit\-FFTSuggest\-Size} ({\bf olong} length)
\begin{CompactList}\small\item\em Public: Suggest efficient size for a transform. \item\end{CompactList}\item 
void {\bf Obit\-FFTR2C} ({\bf Obit\-FFT} $\ast$in, {\bf Obit\-FArray} $\ast$in\-Array, {\bf Obit\-CArray} $\ast$out\-Array)
\begin{CompactList}\small\item\em Public: Real to half Complex. \item\end{CompactList}\item 
void {\bf Obit\-FFTC2R} ({\bf Obit\-FFT} $\ast$in, {\bf Obit\-CArray} $\ast$in\-Array, {\bf Obit\-FArray} $\ast$out\-Array)
\begin{CompactList}\small\item\em Public: Half Complex to Real. \item\end{CompactList}\item 
void {\bf Obit\-FFTC2C} ({\bf Obit\-FFT} $\ast$in, {\bf Obit\-CArray} $\ast$in\-Array, {\bf Obit\-CArray} $\ast$out\-Array)
\begin{CompactList}\small\item\em Public: Full Complex to Complex. \item\end{CompactList}\end{CompactItemize}


\subsection{Detailed Description}
{\bf Obit\-FFT}{\rm (p.\,\pageref{structObitFFT})} Fast Fourier Transform class definition. 

This class is derived from the {\bf Obit}{\rm (p.\,\pageref{structObit})} class.

This class is for performing FFT on memory resident data. This implementation uses the FFTW package if available, else gsl\subsection{Data order}\label{ObitFFT_8h_FFTOrder}
Data passed to and from {\bf Obit\-FFT}{\rm (p.\,\pageref{structObitFFT})} routines are as an {\bf Obit\-FArray}{\rm (p.\,\pageref{structObitFArray})} or {\bf Obit\-CArray}{\rm (p.\,\pageref{structObitCArray})} which are stored in column major (Fortran) order. Data are passed and returned in \char`\"{}center-at-the-edge\char`\"{} (i.e. unnatural) order and there is NO transpose of the array axes. In the half complex form, the first axis is nx/2+1 where nx is the number of real elements. Only even numbers of elements on each axis will work well.\subsection{Creators and Destructors}\label{ObitFFT_8h_ObitFFTaccess}
An {\bf Obit\-FFT}{\rm (p.\,\pageref{structObitFFT})} can be created using new\-Obit\-FFT which allows specifying a name for the object, and the type, size and direction of the transform.

A copy of a pointer to an {\bf Obit\-FFT}{\rm (p.\,\pageref{structObitFFT})} should always be made using the {\bf Obit\-FFTRef}{\rm (p.\,\pageref{ObitFFT_8h_a1})} function which updates the reference count in the object. Then whenever freeing an {\bf Obit\-FFT}{\rm (p.\,\pageref{structObitFFT})} or changing a pointer, the function {\bf Obit\-FFTUnref}{\rm (p.\,\pageref{ObitFFT_8h_a0})} will decrement the reference count and destroy the object when the reference count hits 0. There is no explicit destructor.

\subsection{Define Documentation}
\index{ObitFFT.h@{Obit\-FFT.h}!ObitFFTIsA@{ObitFFTIsA}}
\index{ObitFFTIsA@{ObitFFTIsA}!ObitFFT.h@{Obit\-FFT.h}}
\subsubsection{\setlength{\rightskip}{0pt plus 5cm}\#define Obit\-FFTIs\-A(in)\ Obit\-Is\-A (in, Obit\-FFTGet\-Class())}\label{ObitFFT_8h_a2}


Macro to determine if an object is the member of this or a derived class. 

Returns TRUE if a member, else FALSE in = object to reference \index{ObitFFT.h@{Obit\-FFT.h}!ObitFFTRef@{ObitFFTRef}}
\index{ObitFFTRef@{ObitFFTRef}!ObitFFT.h@{Obit\-FFT.h}}
\subsubsection{\setlength{\rightskip}{0pt plus 5cm}\#define Obit\-FFTRef(in)\ Obit\-Ref (in)}\label{ObitFFT_8h_a1}


Macro to reference (update reference count) an {\bf Obit\-FFT}{\rm (p.\,\pageref{structObitFFT})}. 

returns a Obit\-FFT$\ast$. in = object to reference \index{ObitFFT.h@{Obit\-FFT.h}!ObitFFTUnref@{ObitFFTUnref}}
\index{ObitFFTUnref@{ObitFFTUnref}!ObitFFT.h@{Obit\-FFT.h}}
\subsubsection{\setlength{\rightskip}{0pt plus 5cm}\#define Obit\-FFTUnref(in)\ Obit\-Unref (in)}\label{ObitFFT_8h_a0}


Macro to unreference (and possibly destroy) an {\bf Obit\-FFT}{\rm (p.\,\pageref{structObitFFT})} returns a Obit\-FFT$\ast$. 

in = object to unreference 

\subsection{Typedef Documentation}
\index{ObitFFT.h@{Obit\-FFT.h}!newObitFFTFP@{newObitFFTFP}}
\index{newObitFFTFP@{newObitFFTFP}!ObitFFT.h@{Obit\-FFT.h}}
\subsubsection{\setlength{\rightskip}{0pt plus 5cm}typedef {\bf Obit\-FFT}$\ast$($\ast$ {\bf new\-Obit\-FFTFP})(gchar $\ast$name, Obit\-FFTdir dir, Obit\-FFTtype type, {\bf olong} rank, {\bf olong} $\ast$dim)}\label{ObitFFT_8h_a3}


Typedef for definition of class pointer structure. 

\index{ObitFFT.h@{Obit\-FFT.h}!ObitFFTC2CFP@{ObitFFTC2CFP}}
\index{ObitFFTC2CFP@{ObitFFTC2CFP}!ObitFFT.h@{Obit\-FFT.h}}
\subsubsection{\setlength{\rightskip}{0pt plus 5cm}typedef void($\ast$ {\bf Obit\-FFTC2CFP})({\bf Obit\-FFT} $\ast$in, {\bf Obit\-CArray} $\ast$in\-Array, {\bf Obit\-CArray} $\ast$out\-Array)}\label{ObitFFT_8h_a6}


\index{ObitFFT.h@{Obit\-FFT.h}!ObitFFTC2RFP@{ObitFFTC2RFP}}
\index{ObitFFTC2RFP@{ObitFFTC2RFP}!ObitFFT.h@{Obit\-FFT.h}}
\subsubsection{\setlength{\rightskip}{0pt plus 5cm}typedef void($\ast$ {\bf Obit\-FFTC2RFP})({\bf Obit\-FFT} $\ast$in, {\bf Obit\-CArray} $\ast$in\-Array, {\bf Obit\-FArray} $\ast$out\-Array)}\label{ObitFFT_8h_a5}


\index{ObitFFT.h@{Obit\-FFT.h}!ObitFFTR2CFP@{ObitFFTR2CFP}}
\index{ObitFFTR2CFP@{ObitFFTR2CFP}!ObitFFT.h@{Obit\-FFT.h}}
\subsubsection{\setlength{\rightskip}{0pt plus 5cm}typedef void($\ast$ {\bf Obit\-FFTR2CFP})({\bf Obit\-FFT} $\ast$in, {\bf Obit\-FArray} $\ast$in\-Array, {\bf Obit\-CArray} $\ast$out\-Array)}\label{ObitFFT_8h_a4}




\subsection{Enumeration Type Documentation}
\index{ObitFFT.h@{Obit\-FFT.h}!obitFFTdir@{obitFFTdir}}
\index{obitFFTdir@{obitFFTdir}!ObitFFT.h@{Obit\-FFT.h}}
\subsubsection{\setlength{\rightskip}{0pt plus 5cm}enum {\bf obit\-FFTdir}}\label{ObitFFT_8h_a18}


\begin{Desc}
\item[Enumeration values: ]\par
\begin{description}
\index{OBIT_FFT_Forward@{OBIT\_\-FFT\_\-Forward}!ObitFFT.h@{ObitFFT.h}}\index{ObitFFT.h@{ObitFFT.h}!OBIT_FFT_Forward@{OBIT\_\-FFT\_\-Forward}}\item[{\em 
OBIT\_\-FFT\_\-Forward\label{ObitFFT_8h_a18a7}
}]Sign of exponent in transform = -1, real to complex. \index{OBIT_FFT_Reverse@{OBIT\_\-FFT\_\-Reverse}!ObitFFT.h@{ObitFFT.h}}\index{ObitFFT.h@{ObitFFT.h}!OBIT_FFT_Reverse@{OBIT\_\-FFT\_\-Reverse}}\item[{\em 
OBIT\_\-FFT\_\-Reverse\label{ObitFFT_8h_a18a8}
}]Sign of exponent in transform = +1, complex to real. \end{description}
\end{Desc}

\index{ObitFFT.h@{Obit\-FFT.h}!obitFFTtype@{obitFFTtype}}
\index{obitFFTtype@{obitFFTtype}!ObitFFT.h@{Obit\-FFT.h}}
\subsubsection{\setlength{\rightskip}{0pt plus 5cm}enum {\bf obit\-FFTtype}}\label{ObitFFT_8h_a19}


\begin{Desc}
\item[Enumeration values: ]\par
\begin{description}
\index{OBIT_FFT_FullComplex@{OBIT\_\-FFT\_\-FullComplex}!ObitFFT.h@{ObitFFT.h}}\index{ObitFFT.h@{ObitFFT.h}!OBIT_FFT_FullComplex@{OBIT\_\-FFT\_\-FullComplex}}\item[{\em 
OBIT\_\-FFT\_\-Full\-Complex\label{ObitFFT_8h_a19a9}
}]Full complex to complex transforms. \index{OBIT_FFT_HalfComplex@{OBIT\_\-FFT\_\-HalfComplex}!ObitFFT.h@{ObitFFT.h}}\index{ObitFFT.h@{ObitFFT.h}!OBIT_FFT_HalfComplex@{OBIT\_\-FFT\_\-HalfComplex}}\item[{\em 
OBIT\_\-FFT\_\-Half\-Complex\label{ObitFFT_8h_a19a10}
}]Real to half complex or reverse. \end{description}
\end{Desc}



\subsection{Function Documentation}
\index{ObitFFT.h@{Obit\-FFT.h}!newObitFFT@{newObitFFT}}
\index{newObitFFT@{newObitFFT}!ObitFFT.h@{Obit\-FFT.h}}
\subsubsection{\setlength{\rightskip}{0pt plus 5cm}{\bf Obit\-FFT}$\ast$ new\-Obit\-FFT (gchar $\ast$ {\em name}, Obit\-FFTdir {\em dir}, Obit\-FFTtype {\em type}, {\bf olong} {\em rank}, {\bf olong} $\ast$ {\em dim})}\label{ObitFFT_8h_a12}


Public: Constructor. 

Initializes class if needed on first call. \begin{Desc}
\item[Parameters:]
\begin{description}
\item[{\em name}]An optional name for the object. \item[{\em dir}]The direction of the transform OBIT\_\-FFT\_\-Forward (R2C) or OBIT\_\-FFT\_\-Reverse (C2R). \item[{\em type}]Whether OBIT\_\-FFT\_\-Full\-Complex (full C2C) or OBIT\_\-FFT\_\-Half\-Complex (R2C or C2R). \item[{\em rank}]of matrix range [1,7] \item[{\em dim}]dimensionality of each axis in column major (Fortran) order. If real/half complex is being used, then dim[0] should be the number of reals. \end{description}
\end{Desc}
\begin{Desc}
\item[Returns:]the new object. \end{Desc}
\index{ObitFFT.h@{Obit\-FFT.h}!ObitFFTC2C@{ObitFFTC2C}}
\index{ObitFFTC2C@{ObitFFTC2C}!ObitFFT.h@{Obit\-FFT.h}}
\subsubsection{\setlength{\rightskip}{0pt plus 5cm}void Obit\-FFTC2C ({\bf Obit\-FFT} $\ast$ {\em in}, {\bf Obit\-CArray} $\ast$ {\em in\-Array}, {\bf Obit\-CArray} $\ast$ {\em out\-Array})}\label{ObitFFT_8h_a17}


Public: Full Complex to Complex. 

Must have been created with dir = OBIT\_\-FFT\_\-Reverse have same geometry as constructor call. Transform is in the direction specified in constructor call. \begin{Desc}
\item[Parameters:]
\begin{description}
\item[{\em in}]Object with FFT structures. \item[{\em in\-Array}]Array to be transformed (disturbed on output). \item[{\em out\-Array}]Output array \end{description}
\end{Desc}
\index{ObitFFT.h@{Obit\-FFT.h}!ObitFFTC2R@{ObitFFTC2R}}
\index{ObitFFTC2R@{ObitFFTC2R}!ObitFFT.h@{Obit\-FFT.h}}
\subsubsection{\setlength{\rightskip}{0pt plus 5cm}void Obit\-FFTC2R ({\bf Obit\-FFT} $\ast$ {\em in}, {\bf Obit\-CArray} $\ast$ {\em in\-Array}, {\bf Obit\-FArray} $\ast$ {\em out\-Array})}\label{ObitFFT_8h_a16}


Public: Half Complex to Real. 

Must have been created with dir = OBIT\_\-FFT\_\-Reverse and type = OBIT\_\-FFT\_\-Half\-Complex and have same geometry as constructor call. Note: FFT returned is not normalized. \begin{Desc}
\item[Parameters:]
\begin{description}
\item[{\em in}]Object with FFT structures. \item[{\em in\-Array}]Array to be transformed (disturbed on output). \item[{\em out\-Array}]Output array \end{description}
\end{Desc}
\index{ObitFFT.h@{Obit\-FFT.h}!ObitFFTClassInit@{ObitFFTClassInit}}
\index{ObitFFTClassInit@{ObitFFTClassInit}!ObitFFT.h@{Obit\-FFT.h}}
\subsubsection{\setlength{\rightskip}{0pt plus 5cm}void Obit\-FFTClass\-Init (void)}\label{ObitFFT_8h_a11}


Public: Class initializer. 

\index{ObitFFT.h@{Obit\-FFT.h}!ObitFFTGetClass@{ObitFFTGetClass}}
\index{ObitFFTGetClass@{ObitFFTGetClass}!ObitFFT.h@{Obit\-FFT.h}}
\subsubsection{\setlength{\rightskip}{0pt plus 5cm}gconstpointer Obit\-FFTGet\-Class (void)}\label{ObitFFT_8h_a13}


Public: Class\-Info pointer. 

\begin{Desc}
\item[Returns:]pointer to the class structure. \end{Desc}
\index{ObitFFT.h@{Obit\-FFT.h}!ObitFFTR2C@{ObitFFTR2C}}
\index{ObitFFTR2C@{ObitFFTR2C}!ObitFFT.h@{Obit\-FFT.h}}
\subsubsection{\setlength{\rightskip}{0pt plus 5cm}void Obit\-FFTR2C ({\bf Obit\-FFT} $\ast$ {\em in}, {\bf Obit\-FArray} $\ast$ {\em in\-Array}, {\bf Obit\-CArray} $\ast$ {\em out\-Array})}\label{ObitFFT_8h_a15}


Public: Real to half Complex. 

Must have been created with dir = OBIT\_\-FFT\_\-Forward and type = OBIT\_\-FFT\_\-Half\-Complex and have same geometry as constructor call. \begin{Desc}
\item[Parameters:]
\begin{description}
\item[{\em in}]Object with FFT structures. \item[{\em in\-Array}]Array to be transformed (undisturbed on output). \item[{\em out\-Array}]Output array \end{description}
\end{Desc}
\index{ObitFFT.h@{Obit\-FFT.h}!ObitFFTSuggestSize@{ObitFFTSuggestSize}}
\index{ObitFFTSuggestSize@{ObitFFTSuggestSize}!ObitFFT.h@{Obit\-FFT.h}}
\subsubsection{\setlength{\rightskip}{0pt plus 5cm}{\bf olong} Obit\-FFTSuggest\-Size ({\bf olong} {\em length})}\label{ObitFFT_8h_a14}


Public: Suggest efficient size for a transform. 

\begin{Desc}
\item[Parameters:]
\begin{description}
\item[{\em length}]number of values to be transformed \end{description}
\end{Desc}
\begin{Desc}
\item[Returns:]a number equal or larger than length that will have an efficient transform. \end{Desc}
