\section{Obit.h File Reference}
\label{Obit_8h}\index{Obit.h@{Obit.h}}
Base class of {\bf Obit}{\rm (p.\,\pageref{structObit})} library. 

{\tt \#include $<$glib.h$>$}\par
{\tt \#include $<$string.h$>$}\par
{\tt \#include \char`\"{}memwatch.h\char`\"{}}\par
{\tt \#include $<$stdio.h$>$}\par
{\tt \#include $<$stdlib.h$>$}\par
{\tt \#include $<$math.h$>$}\par
{\tt \#include \char`\"{}Obit\-Types.h\char`\"{}}\par
{\tt \#include \char`\"{}Obit\-Err.h\char`\"{}}\par
\subsection*{Classes}
\begin{CompactItemize}
\item 
struct {\bf Obit}
\begin{CompactList}\small\item\em Obit base class definition. \item\end{CompactList}\item 
struct {\bf Obit\-Class\-Info}
\begin{CompactList}\small\item\em Class\-Info Structure. \item\end{CompactList}\end{CompactItemize}
\subsection*{Defines}
\begin{CompactItemize}
\item 
\#define {\bf OBIT\_\-ID}\ 0x4f626974
\begin{CompactList}\small\item\em {\bf Obit}{\rm (p.\,\pageref{structObit})} object recognition string. \item\end{CompactList}\end{CompactItemize}
\subsection*{Typedefs}
\begin{CompactItemize}
\item 
typedef gpointer($\ast$ {\bf new\-Obit\-FP} )(gchar $\ast$name)
\item 
typedef void($\ast$ {\bf Obit\-Class\-Init\-FP} )(void)
\item 
typedef gconstpointer($\ast$ {\bf Obit\-Get\-Class\-FP} )(void)
\item 
typedef gpointer($\ast$ {\bf Obit\-Copy\-FP} )(gpointer in, gpointer out, {\bf Obit\-Err} $\ast$err)
\item 
typedef gpointer($\ast$ {\bf Obit\-Clone\-FP} )({\bf Obit} $\ast$in, {\bf Obit} $\ast$out)
\item 
typedef gpointer($\ast$ {\bf Obit\-Ref\-FP} )(gpointer in)
\item 
typedef gpointer($\ast$ {\bf Obit\-Unref\-FP} )(gpointer $\ast$in)
\item 
typedef gboolean($\ast$ {\bf Obit\-Is\-AFP} )(gpointer in, gconstpointer class)
\item 
typedef void($\ast$ {\bf Obit\-Init\-FP} )(gpointer in)
\begin{CompactList}\small\item\em Function pointers for private functions. \item\end{CompactList}\item 
typedef void($\ast$ {\bf Obit\-Clear\-FP} )(gpointer in)
\item 
typedef void($\ast$ {\bf Obit\-Class\-Info\-Def\-Fn\-FP} )(gpointer in\-Class)
\item 
typedef gboolean($\ast$ {\bf Obit\-Info\-Is\-AFP} )({\bf Obit\-Class\-Info} $\ast$in, {\bf Obit\-Class\-Info} $\ast$type)
\end{CompactItemize}
\subsection*{Functions}
\begin{CompactItemize}
\item 
{\bf Obit} $\ast$ {\bf new\-Obit} (gchar $\ast$name)
\begin{CompactList}\small\item\em Public: Constructor. \item\end{CompactList}\item 
void {\bf Obit\-Class\-Init} (void)
\begin{CompactList}\small\item\em Public: Class initializer. \item\end{CompactList}\item 
gconstpointer {\bf Obit\-Get\-Class} (void)
\begin{CompactList}\small\item\em Public: Class\-Info pointer. \item\end{CompactList}\item 
{\bf Obit} $\ast$ {\bf Obit\-Copy} ({\bf Obit} $\ast$in, {\bf Obit} $\ast$out, {\bf Obit\-Err} $\ast$err)
\begin{CompactList}\small\item\em Public: Copy (deep) constructor. \item\end{CompactList}\item 
{\bf Obit} $\ast$ {\bf Obit\-Clone} ({\bf Obit} $\ast$in, {\bf Obit} $\ast$out)
\begin{CompactList}\small\item\em Public: Copy (shallow) constructor. \item\end{CompactList}\item 
gpointer {\bf Obit\-Ref} (gpointer in)
\begin{CompactList}\small\item\em Public: Ref pointer, increment reference count, return pointer. \item\end{CompactList}\item 
gpointer {\bf Obit\-Unref} (gpointer in)
\begin{CompactList}\small\item\em Public: Unref pointer, decrement reference count and destroy if 0. \item\end{CompactList}\item 
gboolean {\bf Obit\-Is\-A} (gpointer in, gconstpointer type)
\begin{CompactList}\small\item\em Public: returns TRUE is object is a member of my\-Class\-Info or a derived class. \item\end{CompactList}\item 
{\bf ofloat} {\bf Obit\-Magic\-F} (void)
\begin{CompactList}\small\item\em Public: returns magic value blanking value. \item\end{CompactList}\item 
void {\bf Obit\-Trim\-Trail} (gchar $\ast$str)
\begin{CompactList}\small\item\em Public: trim trailing blanks from string. \item\end{CompactList}\item 
gboolean {\bf Obit\-Str\-Cmp} (gchar $\ast$str1, gchar $\ast$str2, {\bf olong} maxlen)
\begin{CompactList}\small\item\em Public: compare strings. \item\end{CompactList}\item 
gchar $\ast$ {\bf Obit\-Today} (void)
\begin{CompactList}\small\item\em Public: return today's date as yyyy-mm-dd. \item\end{CompactList}\item 
void {\bf Obit\-Class\-Info\-Def\-Fn} (gpointer in\-Class)
\begin{CompactList}\small\item\em Public: Set Class function pointers. \item\end{CompactList}\item 
gboolean {\bf Obit\-Info\-Is\-A} ({\bf Obit\-Class\-Info} $\ast$in, {\bf Obit\-Class\-Info} $\ast$type)
\begin{CompactList}\small\item\em Public: returns TRUE is object is type or a derived class. \item\end{CompactList}\end{CompactItemize}


\subsection{Detailed Description}
Base class of {\bf Obit}{\rm (p.\,\pageref{structObit})} library. 

This class is the virtual base class of most {\bf Obit}{\rm (p.\,\pageref{structObit})} classes. Class hierarchies are generally noted in the names of modules, i.e. {\bf Obit}{\rm (p.\,\pageref{structObit})} is the base class from which (almost) all others are derived. {\bf Obit}{\rm (p.\,\pageref{structObit})} class derivation is by means of nested include files; each class has an include file for the data members and for the class function pointers. These include files include the corresponding includes of their parent class. The functional members are defined in Obit\-Class\-Def.h and the data members in Obit\-Def.h to allow recursive definition of derived classes.\subsection{Usage}\label{Obit_8h_ObitUsage}
No instances should be created of this class.

\subsection{Define Documentation}
\index{Obit.h@{Obit.h}!OBIT_ID@{OBIT\_\-ID}}
\index{OBIT_ID@{OBIT\_\-ID}!Obit.h@{Obit.h}}
\subsubsection{\setlength{\rightskip}{0pt plus 5cm}\#define OBIT\_\-ID\ 0x4f626974}\label{Obit_8h_a0}


{\bf Obit}{\rm (p.\,\pageref{structObit})} object recognition string. 

This is to be the first gint32 of each object. The value in ascii is \char`\"{}Obit\char`\"{} 

\subsection{Typedef Documentation}
\index{Obit.h@{Obit.h}!newObitFP@{newObitFP}}
\index{newObitFP@{newObitFP}!Obit.h@{Obit.h}}
\subsubsection{\setlength{\rightskip}{0pt plus 5cm}typedef gpointer($\ast$ {\bf new\-Obit\-FP})(gchar $\ast$name)}\label{Obit_8h_a1}


\index{Obit.h@{Obit.h}!ObitClassInfoDefFnFP@{ObitClassInfoDefFnFP}}
\index{ObitClassInfoDefFnFP@{ObitClassInfoDefFnFP}!Obit.h@{Obit.h}}
\subsubsection{\setlength{\rightskip}{0pt plus 5cm}typedef void($\ast$ {\bf Obit\-Class\-Info\-Def\-Fn\-FP})(gpointer in\-Class)}\label{Obit_8h_a11}


\index{Obit.h@{Obit.h}!ObitClassInitFP@{ObitClassInitFP}}
\index{ObitClassInitFP@{ObitClassInitFP}!Obit.h@{Obit.h}}
\subsubsection{\setlength{\rightskip}{0pt plus 5cm}typedef void($\ast$ {\bf Obit\-Class\-Init\-FP})(void)}\label{Obit_8h_a2}


\index{Obit.h@{Obit.h}!ObitClearFP@{ObitClearFP}}
\index{ObitClearFP@{ObitClearFP}!Obit.h@{Obit.h}}
\subsubsection{\setlength{\rightskip}{0pt plus 5cm}typedef void($\ast$ {\bf Obit\-Clear\-FP})(gpointer in)}\label{Obit_8h_a10}


\index{Obit.h@{Obit.h}!ObitCloneFP@{ObitCloneFP}}
\index{ObitCloneFP@{ObitCloneFP}!Obit.h@{Obit.h}}
\subsubsection{\setlength{\rightskip}{0pt plus 5cm}typedef gpointer($\ast$ {\bf Obit\-Clone\-FP})({\bf Obit} $\ast$in, {\bf Obit} $\ast$out)}\label{Obit_8h_a5}


\index{Obit.h@{Obit.h}!ObitCopyFP@{ObitCopyFP}}
\index{ObitCopyFP@{ObitCopyFP}!Obit.h@{Obit.h}}
\subsubsection{\setlength{\rightskip}{0pt plus 5cm}typedef gpointer($\ast$ {\bf Obit\-Copy\-FP})(gpointer in, gpointer out, {\bf Obit\-Err} $\ast$err)}\label{Obit_8h_a4}


\index{Obit.h@{Obit.h}!ObitGetClassFP@{ObitGetClassFP}}
\index{ObitGetClassFP@{ObitGetClassFP}!Obit.h@{Obit.h}}
\subsubsection{\setlength{\rightskip}{0pt plus 5cm}typedef gconstpointer($\ast$ {\bf Obit\-Get\-Class\-FP})(void)}\label{Obit_8h_a3}


\index{Obit.h@{Obit.h}!ObitInfoIsAFP@{ObitInfoIsAFP}}
\index{ObitInfoIsAFP@{ObitInfoIsAFP}!Obit.h@{Obit.h}}
\subsubsection{\setlength{\rightskip}{0pt plus 5cm}typedef gboolean($\ast$ {\bf Obit\-Info\-Is\-AFP})({\bf Obit\-Class\-Info} $\ast$in, {\bf Obit\-Class\-Info} $\ast$type)}\label{Obit_8h_a12}


\index{Obit.h@{Obit.h}!ObitInitFP@{ObitInitFP}}
\index{ObitInitFP@{ObitInitFP}!Obit.h@{Obit.h}}
\subsubsection{\setlength{\rightskip}{0pt plus 5cm}typedef void($\ast$ {\bf Obit\-Init\-FP})(gpointer in)}\label{Obit_8h_a9}


Function pointers for private functions. 

\index{Obit.h@{Obit.h}!ObitIsAFP@{ObitIsAFP}}
\index{ObitIsAFP@{ObitIsAFP}!Obit.h@{Obit.h}}
\subsubsection{\setlength{\rightskip}{0pt plus 5cm}typedef gboolean($\ast$ {\bf Obit\-Is\-AFP})(gpointer in, gconstpointer class)}\label{Obit_8h_a8}


\index{Obit.h@{Obit.h}!ObitRefFP@{ObitRefFP}}
\index{ObitRefFP@{ObitRefFP}!Obit.h@{Obit.h}}
\subsubsection{\setlength{\rightskip}{0pt plus 5cm}typedef gpointer($\ast$ {\bf Obit\-Ref\-FP})(gpointer in)}\label{Obit_8h_a6}


\index{Obit.h@{Obit.h}!ObitUnrefFP@{ObitUnrefFP}}
\index{ObitUnrefFP@{ObitUnrefFP}!Obit.h@{Obit.h}}
\subsubsection{\setlength{\rightskip}{0pt plus 5cm}typedef gpointer($\ast$ {\bf Obit\-Unref\-FP})(gpointer $\ast$in)}\label{Obit_8h_a7}




\subsection{Function Documentation}
\index{Obit.h@{Obit.h}!newObit@{newObit}}
\index{newObit@{newObit}!Obit.h@{Obit.h}}
\subsubsection{\setlength{\rightskip}{0pt plus 5cm}{\bf Obit}$\ast$ new\-Obit (gchar $\ast$ {\em name})}\label{Obit_8h_a13}


Public: Constructor. 

Initializes class if needed on first call. \begin{Desc}
\item[Parameters:]
\begin{description}
\item[{\em name}]An optional name for the object. \end{description}
\end{Desc}
\begin{Desc}
\item[Returns:]the new object. \end{Desc}
\index{Obit.h@{Obit.h}!ObitClassInfoDefFn@{ObitClassInfoDefFn}}
\index{ObitClassInfoDefFn@{ObitClassInfoDefFn}!Obit.h@{Obit.h}}
\subsubsection{\setlength{\rightskip}{0pt plus 5cm}void Obit\-Class\-Info\-Def\-Fn (gpointer {\em in\-Class})}\label{Obit_8h_a25}


Public: Set Class function pointers. 

\begin{Desc}
\item[Parameters:]
\begin{description}
\item[{\em in\-Class}]Pointer to Class\-Info structure of the class to be filled. \item[{\em call\-Class}]Pointer to Class\-Info of calling class \end{description}
\end{Desc}
\index{Obit.h@{Obit.h}!ObitClassInit@{ObitClassInit}}
\index{ObitClassInit@{ObitClassInit}!Obit.h@{Obit.h}}
\subsubsection{\setlength{\rightskip}{0pt plus 5cm}void Obit\-Class\-Init (void)}\label{Obit_8h_a14}


Public: Class initializer. 

\index{Obit.h@{Obit.h}!ObitClone@{ObitClone}}
\index{ObitClone@{ObitClone}!Obit.h@{Obit.h}}
\subsubsection{\setlength{\rightskip}{0pt plus 5cm}{\bf Obit}$\ast$ Obit\-Clone ({\bf Obit} $\ast$ {\em in}, {\bf Obit} $\ast$ {\em out})}\label{Obit_8h_a17}


Public: Copy (shallow) constructor. 

The result will have pointers to the more complex members. \begin{Desc}
\item[Parameters:]
\begin{description}
\item[{\em in}]The object to copy \item[{\em out}]An existing object pointer for output or NULL if none exists. \end{description}
\end{Desc}
\begin{Desc}
\item[Returns:]pointer to the new object. \end{Desc}
\index{Obit.h@{Obit.h}!ObitCopy@{ObitCopy}}
\index{ObitCopy@{ObitCopy}!Obit.h@{Obit.h}}
\subsubsection{\setlength{\rightskip}{0pt plus 5cm}{\bf Obit}$\ast$ Obit\-Copy ({\bf Obit} $\ast$ {\em in}, {\bf Obit} $\ast$ {\em out}, {\bf Obit\-Err} $\ast$ {\em err})}\label{Obit_8h_a16}


Public: Copy (deep) constructor. 

Copies are made of complex members such as files; these will be copied applying whatever selection is associated with the input. \begin{Desc}
\item[Parameters:]
\begin{description}
\item[{\em in}]The object to copy \item[{\em out}]An existing object pointer for output or NULL if none exists. \item[{\em err}]Error stack, returns if not empty. \end{description}
\end{Desc}
\begin{Desc}
\item[Returns:]pointer to the new (existing) object. \end{Desc}
\index{Obit.h@{Obit.h}!ObitGetClass@{ObitGetClass}}
\index{ObitGetClass@{ObitGetClass}!Obit.h@{Obit.h}}
\subsubsection{\setlength{\rightskip}{0pt plus 5cm}gconstpointer Obit\-Get\-Class (void)}\label{Obit_8h_a15}


Public: Class\-Info pointer. 

This method MUST be included in each derived class to ensure proper linking and class initialization. Initializes class if needed on first call. \begin{Desc}
\item[Returns:]pointer to the class structure. \end{Desc}
\index{Obit.h@{Obit.h}!ObitInfoIsA@{ObitInfoIsA}}
\index{ObitInfoIsA@{ObitInfoIsA}!Obit.h@{Obit.h}}
\subsubsection{\setlength{\rightskip}{0pt plus 5cm}gboolean Obit\-Info\-Is\-A ({\bf Obit\-Class\-Info} $\ast$ {\em class}, {\bf Obit\-Class\-Info} $\ast$ {\em type})}\label{Obit_8h_a26}


Public: returns TRUE is object is type or a derived class. 

\begin{Desc}
\item[Parameters:]
\begin{description}
\item[{\em in}]Pointer to object to test. \item[{\em class}]Pointer to Class\-Info structure of the class to be tested. \end{description}
\end{Desc}
\begin{Desc}
\item[Returns:]TRUE if test or a derived class, else FALSE. \end{Desc}
\index{Obit.h@{Obit.h}!ObitIsA@{ObitIsA}}
\index{ObitIsA@{ObitIsA}!Obit.h@{Obit.h}}
\subsubsection{\setlength{\rightskip}{0pt plus 5cm}gboolean Obit\-Is\-A (gpointer {\em in}, gconstpointer {\em class})}\label{Obit_8h_a20}


Public: returns TRUE is object is a member of my\-Class\-Info or a derived class. 

Should also work for derived classes. \begin{Desc}
\item[Parameters:]
\begin{description}
\item[{\em in}]Pointer to object to test. \item[{\em class}]Pointer to Class\-Info structure of the class to be tested. \end{description}
\end{Desc}
\begin{Desc}
\item[Returns:]TRUE if member of class or a derived class, else FALSE. \end{Desc}
\index{Obit.h@{Obit.h}!ObitMagicF@{ObitMagicF}}
\index{ObitMagicF@{ObitMagicF}!Obit.h@{Obit.h}}
\subsubsection{\setlength{\rightskip}{0pt plus 5cm}{\bf ofloat} Obit\-Magic\-F (void)}\label{Obit_8h_a21}


Public: returns magic value blanking value. 

\begin{Desc}
\item[Returns:]float magic value \end{Desc}
\index{Obit.h@{Obit.h}!ObitRef@{ObitRef}}
\index{ObitRef@{ObitRef}!Obit.h@{Obit.h}}
\subsubsection{\setlength{\rightskip}{0pt plus 5cm}gpointer Obit\-Ref (gpointer {\em in})}\label{Obit_8h_a18}


Public: Ref pointer, increment reference count, return pointer. 

This function should always be used to copy pointers as this will ensure a proper reference count. Should also work for derived classes \begin{Desc}
\item[Parameters:]
\begin{description}
\item[{\em in}]Pointer to object to link, if Null, just return. \end{description}
\end{Desc}
\begin{Desc}
\item[Returns:]the pointer to in. \end{Desc}
\index{Obit.h@{Obit.h}!ObitStrCmp@{ObitStrCmp}}
\index{ObitStrCmp@{ObitStrCmp}!Obit.h@{Obit.h}}
\subsubsection{\setlength{\rightskip}{0pt plus 5cm}gboolean Obit\-Str\-Cmp (gchar $\ast$ {\em str1}, gchar $\ast$ {\em str2}, {\bf olong} {\em maxlen})}\label{Obit_8h_a23}


Public: compare strings. 

Blanks past the last non blank, non-NULL are considered insignificant \begin{Desc}
\item[Parameters:]
\begin{description}
\item[{\em str1}]First string to compare \item[{\em str2}]Second string to compare \item[{\em maxlen}]Maximum number of characters to compare \end{description}
\end{Desc}
\begin{Desc}
\item[Returns:]True if all significant characters match, else False \end{Desc}
\index{Obit.h@{Obit.h}!ObitToday@{ObitToday}}
\index{ObitToday@{ObitToday}!Obit.h@{Obit.h}}
\subsubsection{\setlength{\rightskip}{0pt plus 5cm}gchar$\ast$ Obit\-Today (void)}\label{Obit_8h_a24}


Public: return today's date as yyyy-mm-dd. 

\begin{Desc}
\item[Returns:]data string, should be g\_\-freeed when done. \end{Desc}
\index{Obit.h@{Obit.h}!ObitTrimTrail@{ObitTrimTrail}}
\index{ObitTrimTrail@{ObitTrimTrail}!Obit.h@{Obit.h}}
\subsubsection{\setlength{\rightskip}{0pt plus 5cm}void Obit\-Trim\-Trail (gchar $\ast$ {\em str})}\label{Obit_8h_a22}


Public: trim trailing blanks from string. 

\begin{Desc}
\item[Parameters:]
\begin{description}
\item[{\em str}]String to trim \end{description}
\end{Desc}
\index{Obit.h@{Obit.h}!ObitUnref@{ObitUnref}}
\index{ObitUnref@{ObitUnref}!Obit.h@{Obit.h}}
\subsubsection{\setlength{\rightskip}{0pt plus 5cm}gpointer Obit\-Unref (gpointer {\em inn})}\label{Obit_8h_a19}


Public: Unref pointer, decrement reference count and destroy if 0. 

if the input pointer is NULL, the reference count is already $<$=0 or the object is not a valid {\bf Obit}{\rm (p.\,\pageref{structObit})} Object, it simply returns. \begin{Desc}
\item[Parameters:]
\begin{description}
\item[{\em in}]Pointer to object to unreference. \end{description}
\end{Desc}
\begin{Desc}
\item[Returns:]NULL pointer. \end{Desc}
