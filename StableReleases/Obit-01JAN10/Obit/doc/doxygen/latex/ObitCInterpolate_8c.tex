\section{Obit\-CInterpolate.c File Reference}
\label{ObitCInterpolate_8c}\index{ObitCInterpolate.c@{ObitCInterpolate.c}}
{\bf Obit\-CInterpolate}{\rm (p.\,\pageref{structObitCInterpolate})} class function definitions. 

{\tt \#include \char`\"{}Obit\-CInterpolate.h\char`\"{}}\par
{\tt \#include \char`\"{}Obit\-Position.h\char`\"{}}\par
\subsection*{Functions}
\begin{CompactItemize}
\item 
void {\bf Obit\-CInterpolate\-Init} (gpointer in)
\begin{CompactList}\small\item\em Private: Initialize newly instantiated object. \item\end{CompactList}\item 
void {\bf Obit\-CInterpolate\-Clear} (gpointer in)
\begin{CompactList}\small\item\em Private: Deallocate members. \item\end{CompactList}\item 
{\bf Obit\-CInterpolate} $\ast$ {\bf new\-Obit\-CInterpolate} (gchar $\ast$name)
\begin{CompactList}\small\item\em Public: Constructor. \item\end{CompactList}\item 
{\bf Obit\-CInterpolate} $\ast$ {\bf new\-Obit\-CInterpolate\-Create} (gchar $\ast$name, {\bf Obit\-CArray} $\ast$array, {\bf Obit\-Image\-Desc} $\ast$desc, {\bf ofloat} OSX, {\bf ofloat} OSY, {\bf olong} num\-Conj\-Col, {\bf olong} hwidth, {\bf Obit\-Err} $\ast$err)
\begin{CompactList}\small\item\em Public: Constructor from value. \item\end{CompactList}\item 
gconstpointer {\bf Obit\-CInterpolate\-Get\-Class} (void)
\begin{CompactList}\small\item\em Public: Class\-Info pointer. \item\end{CompactList}\item 
{\bf Obit\-CInterpolate} $\ast$ {\bf Obit\-CInterpolate\-Copy} ({\bf Obit\-CInterpolate} $\ast$in, {\bf Obit\-CInterpolate} $\ast$out, {\bf Obit\-Err} $\ast$err)
\begin{CompactList}\small\item\em Public: Copy (deep) constructor. \item\end{CompactList}\item 
{\bf Obit\-CInterpolate} $\ast$ {\bf Obit\-CInterpolate\-Clone} ({\bf Obit\-CInterpolate} $\ast$in, {\bf Obit\-CInterpolate} $\ast$out)
\begin{CompactList}\small\item\em Public: Copy (shallow) constructor. \item\end{CompactList}\item 
void {\bf Obit\-CInterpolate\-Replace} ({\bf Obit\-CInterpolate} $\ast$in, {\bf Obit\-CArray} $\ast$new\-Array)
\begin{CompactList}\small\item\em Public: Replace member {\bf Obit\-CArray}{\rm (p.\,\pageref{structObitCArray})}. \item\end{CompactList}\item 
void {\bf Obit\-CInterpolate\-Pixel} ({\bf Obit\-CInterpolate} $\ast$in, {\bf ofloat} $\ast$pixel, {\bf ofloat} out[2], {\bf Obit\-Err} $\ast$err)
\begin{CompactList}\small\item\em Public: Interpolate Pixel in 2D array. \item\end{CompactList}\item 
void {\bf Obit\-CInterpolate1D} ({\bf Obit\-CInterpolate} $\ast$in, {\bf ofloat} pixel, {\bf ofloat} out[2])
\begin{CompactList}\small\item\em Public: Interpolate value in 1- array. \item\end{CompactList}\item 
void {\bf Obit\-CInterpolate\-Position} ({\bf Obit\-CInterpolate} $\ast$in, {\bf odouble} $\ast$coord, {\bf ofloat} out[2], {\bf Obit\-Err} $\ast$err)
\begin{CompactList}\small\item\em Public: Interpolate Position in 2D array. \item\end{CompactList}\item 
void {\bf Obit\-CInterpolate\-Offset} ({\bf Obit\-CInterpolate} $\ast$in, {\bf ofloat} $\ast$off, {\bf ofloat} out[2], {\bf Obit\-Err} $\ast$err)
\begin{CompactList}\small\item\em Public: Interpolate Offset in 2D array. \item\end{CompactList}\item 
void {\bf Obit\-CInterpolate\-Class\-Init} (void)
\begin{CompactList}\small\item\em Public: Class initializer. \item\end{CompactList}\end{CompactItemize}


\subsection{Detailed Description}
{\bf Obit\-CInterpolate}{\rm (p.\,\pageref{structObitCInterpolate})} class function definitions. 

This class is derived from the {\bf Obit}{\rm (p.\,\pageref{structObit})} base class. This class supports 1 and 2-D interpolation in Obit\-CArrays using Lagrange interpolation.

\subsection{Function Documentation}
\index{ObitCInterpolate.c@{Obit\-CInterpolate.c}!newObitCInterpolate@{newObitCInterpolate}}
\index{newObitCInterpolate@{newObitCInterpolate}!ObitCInterpolate.c@{Obit\-CInterpolate.c}}
\subsubsection{\setlength{\rightskip}{0pt plus 5cm}{\bf Obit\-CInterpolate}$\ast$ new\-Obit\-CInterpolate (gchar $\ast$ {\em name})}\label{ObitCInterpolate_8c_a9}


Public: Constructor. 

Initializes class if needed on first call. \begin{Desc}
\item[Parameters:]
\begin{description}
\item[{\em name}]An optional name for the object. \end{description}
\end{Desc}
\begin{Desc}
\item[Returns:]the new object. \end{Desc}
\index{ObitCInterpolate.c@{Obit\-CInterpolate.c}!newObitCInterpolateCreate@{newObitCInterpolateCreate}}
\index{newObitCInterpolateCreate@{newObitCInterpolateCreate}!ObitCInterpolate.c@{Obit\-CInterpolate.c}}
\subsubsection{\setlength{\rightskip}{0pt plus 5cm}{\bf Obit\-CInterpolate}$\ast$ new\-Obit\-CInterpolate\-Create (gchar $\ast$ {\em name}, {\bf Obit\-CArray} $\ast$ {\em array}, {\bf Obit\-Image\-Desc} $\ast$ {\em desc}, {\bf ofloat} {\em OSX}, {\bf ofloat} {\em OSY}, {\bf olong} {\em num\-Conj\-Col}, {\bf olong} {\em hwidth}, {\bf Obit\-Err} $\ast$ {\em err})}\label{ObitCInterpolate_8c_a10}


Public: Constructor from value. 

Initializes class if needed on first call. This class is intended for use interpolating complex values in a UV grid which is the Fourier transform of an image. This is potentially modified by OSX, OSY and herm. Only 1 and 2D arrays are handled. Note: The FFTW convention for halfplane complex images is different from AIPS, the \char`\"{}short\char`\"{} = n/2+1 axis is the first one rather than the second. \begin{Desc}
\item[Parameters:]
\begin{description}
\item[{\em name}]An optional name for the object. \item[{\em array}]The {\bf Obit\-CArray}{\rm (p.\,\pageref{structObitCArray})} to be interpolated. \item[{\em desc}]if non\-NULL, an image descriptor describing the image whose Fourier transform is to be interpolated. \item[{\em OSX}]Oversampling factor in X/U, use 1.0 for unpadded image \item[{\em OSY}]Oversampling factor in Y/V \item[{\em num\-Conj\-Col}]Number of conjugate (neg V) columns in array, should be at least hwidth \item[{\em hwidth}]Half width of interpolation kernal (range [1,8] allowed). \end{description}
\end{Desc}
\begin{Desc}
\item[Returns:]the new object. \end{Desc}
\index{ObitCInterpolate.c@{Obit\-CInterpolate.c}!ObitCInterpolate1D@{ObitCInterpolate1D}}
\index{ObitCInterpolate1D@{ObitCInterpolate1D}!ObitCInterpolate.c@{Obit\-CInterpolate.c}}
\subsubsection{\setlength{\rightskip}{0pt plus 5cm}void Obit\-CInterpolate1D ({\bf Obit\-CInterpolate} $\ast$ {\em in}, {\bf ofloat} {\em pixel}, {\bf ofloat} {\em out}[2])}\label{ObitCInterpolate_8c_a16}


Public: Interpolate value in 1- array. 

\begin{Desc}
\item[Parameters:]
\begin{description}
\item[{\em in}]The object to interpolate \item[{\em pixel}]Pixel location (1-rel) in array \item[{\em out}]Complex interpolated value as (real,Imag), magic value blanked \end{description}
\end{Desc}
\index{ObitCInterpolate.c@{Obit\-CInterpolate.c}!ObitCInterpolateClassInit@{ObitCInterpolateClassInit}}
\index{ObitCInterpolateClassInit@{ObitCInterpolateClassInit}!ObitCInterpolate.c@{Obit\-CInterpolate.c}}
\subsubsection{\setlength{\rightskip}{0pt plus 5cm}void Obit\-CInterpolate\-Class\-Init (void)}\label{ObitCInterpolate_8c_a19}


Public: Class initializer. 

\index{ObitCInterpolate.c@{Obit\-CInterpolate.c}!ObitCInterpolateClear@{ObitCInterpolateClear}}
\index{ObitCInterpolateClear@{ObitCInterpolateClear}!ObitCInterpolate.c@{Obit\-CInterpolate.c}}
\subsubsection{\setlength{\rightskip}{0pt plus 5cm}void Obit\-CInterpolate\-Clear (gpointer {\em inn})}\label{ObitCInterpolate_8c_a4}


Private: Deallocate members. 

Does (recursive) deallocation of parent class members. \begin{Desc}
\item[Parameters:]
\begin{description}
\item[{\em in}]Pointer to the object to deallocate. Actually it should be an Obit\-CInterpolate$\ast$ cast to an Obit$\ast$. \end{description}
\end{Desc}
\index{ObitCInterpolate.c@{Obit\-CInterpolate.c}!ObitCInterpolateClone@{ObitCInterpolateClone}}
\index{ObitCInterpolateClone@{ObitCInterpolateClone}!ObitCInterpolate.c@{Obit\-CInterpolate.c}}
\subsubsection{\setlength{\rightskip}{0pt plus 5cm}{\bf Obit\-CInterpolate}$\ast$ Obit\-CInterpolate\-Clone ({\bf Obit\-CInterpolate} $\ast$ {\em in}, {\bf Obit\-CInterpolate} $\ast$ {\em out})}\label{ObitCInterpolate_8c_a13}


Public: Copy (shallow) constructor. 

The result will have pointers to the more complex members. Parent class members are included but any derived class info is ignored. \begin{Desc}
\item[Parameters:]
\begin{description}
\item[{\em in}]The object to copy \item[{\em out}]An existing object pointer for output or NULL if none exists. \end{description}
\end{Desc}
\begin{Desc}
\item[Returns:]pointer to the new object. \end{Desc}
\index{ObitCInterpolate.c@{Obit\-CInterpolate.c}!ObitCInterpolateCopy@{ObitCInterpolateCopy}}
\index{ObitCInterpolateCopy@{ObitCInterpolateCopy}!ObitCInterpolate.c@{Obit\-CInterpolate.c}}
\subsubsection{\setlength{\rightskip}{0pt plus 5cm}{\bf Obit\-CInterpolate}$\ast$ Obit\-CInterpolate\-Copy ({\bf Obit\-CInterpolate} $\ast$ {\em in}, {\bf Obit\-CInterpolate} $\ast$ {\em out}, {\bf Obit\-Err} $\ast$ {\em err})}\label{ObitCInterpolate_8c_a12}


Public: Copy (deep) constructor. 

Copies are made of complex members including disk files; these will be copied applying whatever selection is associated with the input. Parent class members are included but any derived class info is ignored. \begin{Desc}
\item[Parameters:]
\begin{description}
\item[{\em in}]The object to copy \item[{\em out}]An existing object pointer for output or NULL if none exists. \item[{\em err}]Error stack, returns if not empty. \end{description}
\end{Desc}
\begin{Desc}
\item[Returns:]pointer to the new object. \end{Desc}
\index{ObitCInterpolate.c@{Obit\-CInterpolate.c}!ObitCInterpolateGetClass@{ObitCInterpolateGetClass}}
\index{ObitCInterpolateGetClass@{ObitCInterpolateGetClass}!ObitCInterpolate.c@{Obit\-CInterpolate.c}}
\subsubsection{\setlength{\rightskip}{0pt plus 5cm}gconstpointer Obit\-CInterpolate\-Get\-Class (void)}\label{ObitCInterpolate_8c_a11}


Public: Class\-Info pointer. 

\begin{Desc}
\item[Returns:]pointer to the class structure. \end{Desc}
\index{ObitCInterpolate.c@{Obit\-CInterpolate.c}!ObitCInterpolateInit@{ObitCInterpolateInit}}
\index{ObitCInterpolateInit@{ObitCInterpolateInit}!ObitCInterpolate.c@{Obit\-CInterpolate.c}}
\subsubsection{\setlength{\rightskip}{0pt plus 5cm}void Obit\-CInterpolate\-Init (gpointer {\em inn})}\label{ObitCInterpolate_8c_a3}


Private: Initialize newly instantiated object. 

Parent classes portions are (recursively) initialized first \begin{Desc}
\item[Parameters:]
\begin{description}
\item[{\em in}]Pointer to the object to initialize. \end{description}
\end{Desc}
\index{ObitCInterpolate.c@{Obit\-CInterpolate.c}!ObitCInterpolateOffset@{ObitCInterpolateOffset}}
\index{ObitCInterpolateOffset@{ObitCInterpolateOffset}!ObitCInterpolate.c@{Obit\-CInterpolate.c}}
\subsubsection{\setlength{\rightskip}{0pt plus 5cm}void Obit\-CInterpolate\-Offset ({\bf Obit\-CInterpolate} $\ast$ {\em in}, {\bf ofloat} $\ast$ {\em off}, {\bf ofloat} {\em out}[2], {\bf Obit\-Err} $\ast$ {\em err})}\label{ObitCInterpolate_8c_a18}


Public: Interpolate Offset in 2D array. 

The object must have an image descriptor to allow determing pixel coordinates. Interpolation between planes is not supported. \begin{Desc}
\item[Parameters:]
\begin{description}
\item[{\em in}]The object to interpolate \item[{\em off}]Coordinate offset in plane \item[{\em out}]Complex interpolated value as (real,Imag), magic value blanked \item[{\em err}]Error stack is pixel not inside image. \end{description}
\end{Desc}
\index{ObitCInterpolate.c@{Obit\-CInterpolate.c}!ObitCInterpolatePixel@{ObitCInterpolatePixel}}
\index{ObitCInterpolatePixel@{ObitCInterpolatePixel}!ObitCInterpolate.c@{Obit\-CInterpolate.c}}
\subsubsection{\setlength{\rightskip}{0pt plus 5cm}void Obit\-CInterpolate\-Pixel ({\bf Obit\-CInterpolate} $\ast$ {\em in}, {\bf ofloat} $\ast$ {\em pixel}, {\bf ofloat} {\em out}[2], {\bf Obit\-Err} $\ast$ {\em err})}\label{ObitCInterpolate_8c_a15}


Public: Interpolate Pixel in 2D array. 

Interpolation between planes is not supported. \begin{Desc}
\item[Parameters:]
\begin{description}
\item[{\em in}]The object to interpolate \item[{\em pixel}]Pixel location (1-rel) in planes and which plane. Should have number of dimensions equal to in. \item[{\em out}]Complex interpolated value as (real,Imag), magic value blanked \item[{\em err}]Error stack if pixel not inside image. \end{description}
\end{Desc}
\index{ObitCInterpolate.c@{Obit\-CInterpolate.c}!ObitCInterpolatePosition@{ObitCInterpolatePosition}}
\index{ObitCInterpolatePosition@{ObitCInterpolatePosition}!ObitCInterpolate.c@{Obit\-CInterpolate.c}}
\subsubsection{\setlength{\rightskip}{0pt plus 5cm}void Obit\-CInterpolate\-Position ({\bf Obit\-CInterpolate} $\ast$ {\em in}, {\bf odouble} $\ast$ {\em coord}, {\bf ofloat} {\em out}[2], {\bf Obit\-Err} $\ast$ {\em err})}\label{ObitCInterpolate_8c_a17}


Public: Interpolate Position in 2D array. 

The object must have an image descriptor to allow determing pixel coordinates and the coordinates are assumed linear. Interpolation between planes is not supported. \begin{Desc}
\item[Parameters:]
\begin{description}
\item[{\em in}]The object to interpolate \item[{\em coord}]Coordinate value in plane and which plane. Should have number of dimensions equal to in. \item[{\em out}]Complex interpolated value as (real,Imag), magic value blanked \item[{\em err}]Error stack is pixel not inside image. \end{description}
\end{Desc}
\index{ObitCInterpolate.c@{Obit\-CInterpolate.c}!ObitCInterpolateReplace@{ObitCInterpolateReplace}}
\index{ObitCInterpolateReplace@{ObitCInterpolateReplace}!ObitCInterpolate.c@{Obit\-CInterpolate.c}}
\subsubsection{\setlength{\rightskip}{0pt plus 5cm}void Obit\-CInterpolate\-Replace ({\bf Obit\-CInterpolate} $\ast$ {\em in}, {\bf Obit\-CArray} $\ast$ {\em new\-Array})}\label{ObitCInterpolate_8c_a14}


Public: Replace member {\bf Obit\-CArray}{\rm (p.\,\pageref{structObitCArray})}. 

\begin{Desc}
\item[Parameters:]
\begin{description}
\item[{\em in}]The object to update \item[{\em new\-Array}]The new CArray for in \end{description}
\end{Desc}
