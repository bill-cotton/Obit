\section{Obit\-IO.h File Reference}
\label{ObitIO_8h}\index{ObitIO.h@{ObitIO.h}}
{\bf Obit\-IO}{\rm (p.\,\pageref{structObitIO})} base class definition. 

{\tt \#include \char`\"{}Obit.h\char`\"{}}\par
{\tt \#include \char`\"{}Obit\-Err.h\char`\"{}}\par
{\tt \#include \char`\"{}Obit\-Thread.h\char`\"{}}\par
{\tt \#include \char`\"{}Obit\-Info\-List.h\char`\"{}}\par
{\tt \#include \char`\"{}Obit\-Image\-Desc.h\char`\"{}}\par
{\tt \#include \char`\"{}Obit\-UVDesc.h\char`\"{}}\par
{\tt \#include \char`\"{}Obit\-UVCal.h\char`\"{}}\par
\subsection*{Classes}
\begin{CompactItemize}
\item 
struct {\bf Obit\-IO}
\begin{CompactList}\small\item\em Obit\-IO Class. \item\end{CompactList}\item 
struct {\bf Obit\-IOClass\-Info}
\begin{CompactList}\small\item\em Class\-Info Structure. \item\end{CompactList}\end{CompactItemize}
\subsection*{Defines}
\begin{CompactItemize}
\item 
\#define {\bf Obit\-IOUnref}(in)\ Obit\-Unref (in)
\begin{CompactList}\small\item\em Macro to unreference (and possibly destroy) an {\bf Obit\-IO}{\rm (p.\,\pageref{structObitIO})} returns a Obit\-IO$\ast$. \item\end{CompactList}\item 
\#define {\bf Obit\-IORef}(in)\ Obit\-Ref (in)
\begin{CompactList}\small\item\em Macro to reference (update reference count) an {\bf Obit\-IO}{\rm (p.\,\pageref{structObitIO})}. \item\end{CompactList}\item 
\#define {\bf Obit\-IOIs\-A}(in)\ Obit\-Is\-A (in, Obit\-IOGet\-Class())
\begin{CompactList}\small\item\em Macro to determine if an object is the member of this or a derived class. \item\end{CompactList}\end{CompactItemize}
\subsection*{Typedefs}
\begin{CompactItemize}
\item 
typedef {\bf Obit\-IO} $\ast$($\ast$ {\bf new\-Obit\-IOFP} )(gchar $\ast$name, {\bf Obit\-Info\-List} $\ast$info, {\bf Obit\-Err} $\ast$err)
\begin{CompactList}\small\item\em define type for Class\-Info structure \item\end{CompactList}\item 
typedef gboolean($\ast$ {\bf Obit\-IOSame\-FP} )({\bf Obit\-IO} $\ast$in, {\bf Obit\-Info\-List} $\ast$in1, {\bf Obit\-Info\-List} $\ast$in2, {\bf Obit\-Err} $\ast$err)
\item 
typedef void($\ast$ {\bf Obit\-IORename\-FP} )({\bf Obit\-IO} $\ast$in, {\bf Obit\-Info\-List} $\ast$info, {\bf Obit\-Err} $\ast$err)
\item 
typedef void($\ast$ {\bf Obit\-IOZap\-FP} )({\bf Obit\-IO} $\ast$in, {\bf Obit\-Err} $\ast$err)
\item 
typedef Obit\-IOCode($\ast$ {\bf Obit\-IOOpen\-FP} )({\bf Obit\-IO} $\ast$in, Obit\-IOAccess access, {\bf Obit\-Info\-List} $\ast$info, {\bf Obit\-Err} $\ast$err)
\item 
typedef Obit\-IOCode($\ast$ {\bf Obit\-IOClose\-FP} )({\bf Obit\-IO} $\ast$in, {\bf Obit\-Err} $\ast$err)
\item 
typedef Obit\-IOCode($\ast$ {\bf Obit\-IOSet\-FP} )({\bf Obit\-IO} $\ast$in, {\bf Obit\-Info\-List} $\ast$info, {\bf Obit\-Err} $\ast$err)
\item 
typedef Obit\-IOCode($\ast$ {\bf Obit\-IORead\-FP} )({\bf Obit\-IO} $\ast$in, {\bf ofloat} $\ast$data, {\bf Obit\-Err} $\ast$err)
\item 
typedef Obit\-IOCode($\ast$ {\bf Obit\-IORead\-Multi\-FP} )({\bf olong} n\-Buff, {\bf Obit\-IO} $\ast$$\ast$in, {\bf ofloat} $\ast$$\ast$data, {\bf Obit\-Err} $\ast$err)
\item 
typedef Obit\-IOCode($\ast$ {\bf Obit\-IORead\-Row\-FP} )({\bf Obit\-IO} $\ast$in, {\bf olong} rowno, {\bf ofloat} $\ast$data, {\bf Obit\-Err} $\ast$err)
\item 
typedef Obit\-IOCode($\ast$ {\bf Obit\-IORead\-Select\-FP} )({\bf Obit\-IO} $\ast$in, {\bf ofloat} $\ast$data, {\bf Obit\-Err} $\ast$err)
\item 
typedef Obit\-IOCode($\ast$ {\bf Obit\-IORead\-Multi\-Select\-FP} )({\bf olong} n\-Buff, {\bf Obit\-IO} $\ast$$\ast$in, {\bf ofloat} $\ast$$\ast$data, {\bf Obit\-Err} $\ast$err)
\item 
typedef Obit\-IOCode($\ast$ {\bf Obit\-IORe\-Read\-Multi\-FP} )({\bf olong} n\-Buff, {\bf Obit\-IO} $\ast$$\ast$in, {\bf ofloat} $\ast$$\ast$data, {\bf Obit\-Err} $\ast$err)
\item 
typedef Obit\-IOCode($\ast$ {\bf Obit\-IORe\-Read\-Multi\-Select\-FP} )({\bf olong} n\-Buff, {\bf Obit\-IO} $\ast$$\ast$in, {\bf ofloat} $\ast$$\ast$data, {\bf Obit\-Err} $\ast$err)
\item 
typedef Obit\-IOCode($\ast$ {\bf Obit\-IORead\-Row\-Select\-FP} )({\bf Obit\-IO} $\ast$in, {\bf olong} rowno, {\bf ofloat} $\ast$data, {\bf Obit\-Err} $\ast$err)
\item 
typedef Obit\-IOCode($\ast$ {\bf Obit\-IOWrite\-FP} )({\bf Obit\-IO} $\ast$in, {\bf ofloat} $\ast$data, {\bf Obit\-Err} $\ast$err)
\item 
typedef Obit\-IOCode($\ast$ {\bf Obit\-IOWrite\-Row\-FP} )({\bf Obit\-IO} $\ast$in, {\bf olong} rowno, {\bf ofloat} $\ast$data, {\bf Obit\-Err} $\ast$err)
\item 
typedef Obit\-IOCode($\ast$ {\bf Obit\-IOFlush\-FP} )({\bf Obit\-IO} $\ast$in, {\bf Obit\-Err} $\ast$err)
\item 
typedef Obit\-IOCode($\ast$ {\bf Obit\-IORead\-Descriptor\-FP} )({\bf Obit\-IO} $\ast$in, {\bf Obit\-Err} $\ast$err)
\item 
typedef Obit\-IOCode($\ast$ {\bf Obit\-IOWrite\-Descriptor\-FP} )({\bf Obit\-IO} $\ast$in, {\bf Obit\-Err} $\ast$err)
\item 
typedef void($\ast$ {\bf Obit\-IOCreate\-Buffer\-FP} )({\bf ofloat} $\ast$$\ast$data, {\bf olong} $\ast$size, {\bf Obit\-IO} $\ast$in, {\bf Obit\-Info\-List} $\ast$info, {\bf Obit\-Err} $\ast$err)
\item 
typedef void($\ast$ {\bf Obit\-IOFree\-Buffer\-FP} )({\bf ofloat} $\ast$buffer)
\item 
typedef {\bf Obit} $\ast$($\ast$ {\bf new\-Obit\-IOTable\-FP} )({\bf Obit\-IO} $\ast$in, Obit\-IOAccess access, gchar $\ast$tab\-Type, {\bf olong} $\ast$tabver, {\bf Obit\-Err} $\ast$err)
\item 
typedef Obit\-IOCode($\ast$ {\bf Obit\-IOUpdate\-Tables\-FP} )({\bf Obit\-IO} $\ast$in, {\bf Obit\-Info\-List} $\ast$info, {\bf Obit\-Err} $\ast$err)
\item 
typedef void($\ast$ {\bf Obit\-IOGet\-File\-Info\-FP} )({\bf Obit\-IO} $\ast$in, {\bf Obit\-Info\-List} $\ast$my\-Info, gchar $\ast$prefix, {\bf Obit\-Info\-List} $\ast$out\-List, {\bf Obit\-Err} $\ast$err)
\end{CompactItemize}
\subsection*{Functions}
\begin{CompactItemize}
\item 
void {\bf Obit\-IOClass\-Init} (void)
\begin{CompactList}\small\item\em Public: Class initializer. \item\end{CompactList}\item 
{\bf Obit\-IO} $\ast$ {\bf new\-Obit\-IO} (gchar $\ast$name, {\bf Obit\-Info\-List} $\ast$info, {\bf Obit\-Err} $\ast$err)
\begin{CompactList}\small\item\em Public: Constructor. \item\end{CompactList}\item 
gconstpointer {\bf Obit\-IOGet\-Class} (void)
\begin{CompactList}\small\item\em Public: Class\-Info pointer. \item\end{CompactList}\item 
gboolean {\bf Obit\-IOSame} ({\bf Obit\-IO} $\ast$in, {\bf Obit\-Info\-List} $\ast$in1, {\bf Obit\-Info\-List} $\ast$in2, {\bf Obit\-Err} $\ast$err)
\begin{CompactList}\small\item\em Public: Are underlying structures the same. \item\end{CompactList}\item 
void {\bf Obit\-IORename} ({\bf Obit\-IO} $\ast$in, {\bf Obit\-Info\-List} $\ast$info, {\bf Obit\-Err} $\ast$err)
\begin{CompactList}\small\item\em Public: Rename underlying structures. \item\end{CompactList}\item 
void {\bf Obit\-IOZap} ({\bf Obit\-IO} $\ast$in, {\bf Obit\-Err} $\ast$err)
\begin{CompactList}\small\item\em Public: Delete underlying structures. \item\end{CompactList}\item 
{\bf Obit\-IO} $\ast$ {\bf Obit\-IOCopy} ({\bf Obit\-IO} $\ast$in, {\bf Obit\-IO} $\ast$out, {\bf Obit\-Err} $\ast$err)
\begin{CompactList}\small\item\em Public: Copy constructor. \item\end{CompactList}\item 
Obit\-IOCode {\bf Obit\-IOOpen} ({\bf Obit\-IO} $\ast$in, Obit\-IOAccess access, {\bf Obit\-Info\-List} $\ast$info, {\bf Obit\-Err} $\ast$err)
\begin{CompactList}\small\item\em Public: Open. \item\end{CompactList}\item 
Obit\-IOCode {\bf Obit\-IOClose} ({\bf Obit\-IO} $\ast$in, {\bf Obit\-Err} $\ast$err)
\begin{CompactList}\small\item\em Public: Close. \item\end{CompactList}\item 
Obit\-IOCode {\bf Obit\-IOSet} ({\bf Obit\-IO} $\ast$in, {\bf Obit\-Info\-List} $\ast$info, {\bf Obit\-Err} $\ast$err)
\begin{CompactList}\small\item\em Public: Init I/O. \item\end{CompactList}\item 
Obit\-IOCode {\bf Obit\-IORead} ({\bf Obit\-IO} $\ast$in, {\bf ofloat} $\ast$data, {\bf Obit\-Err} $\ast$err)
\begin{CompactList}\small\item\em Public: Read. \item\end{CompactList}\item 
Obit\-IOCode {\bf Obit\-IORead\-Multi} ({\bf olong} n\-Buff, {\bf Obit\-IO} $\ast$$\ast$in, {\bf ofloat} $\ast$$\ast$data, {\bf Obit\-Err} $\ast$err)
\begin{CompactList}\small\item\em Public: Read to multiple buffers. \item\end{CompactList}\item 
Obit\-IOCode {\bf Obit\-IORead\-Row} ({\bf Obit\-IO} $\ast$in, {\bf olong} rowno, {\bf ofloat} $\ast$data, {\bf Obit\-Err} $\ast$err)
\begin{CompactList}\small\item\em Public: Read Row. \item\end{CompactList}\item 
Obit\-IOCode {\bf Obit\-IORead\-Select} ({\bf Obit\-IO} $\ast$in, {\bf ofloat} $\ast$data, {\bf Obit\-Err} $\ast$err)
\begin{CompactList}\small\item\em Public: Read with selection. \item\end{CompactList}\item 
Obit\-IOCode {\bf Obit\-IORead\-Multi\-Select} ({\bf olong} n\-Buff, {\bf Obit\-IO} $\ast$$\ast$in, {\bf ofloat} $\ast$$\ast$data, {\bf Obit\-Err} $\ast$err)
\begin{CompactList}\small\item\em Public: Read with selection to multiple buffers. \item\end{CompactList}\item 
Obit\-IOCode {\bf Obit\-IORe\-Read\-Multi} ({\bf olong} n\-Buff, {\bf Obit\-IO} $\ast$$\ast$in, {\bf ofloat} $\ast$$\ast$data, {\bf Obit\-Err} $\ast$err)
\begin{CompactList}\small\item\em Public: Reread with selection. \item\end{CompactList}\item 
Obit\-IOCode {\bf Obit\-IORe\-Read\-Multi\-Select} ({\bf olong} n\-Buff, {\bf Obit\-IO} $\ast$$\ast$in, {\bf ofloat} $\ast$$\ast$data, {\bf Obit\-Err} $\ast$err)
\begin{CompactList}\small\item\em Public: Reread with selection to multiple buffers. \item\end{CompactList}\item 
Obit\-IOCode {\bf Obit\-IORead\-Row\-Select} ({\bf Obit\-IO} $\ast$in, {\bf olong} rowno, {\bf ofloat} $\ast$data, {\bf Obit\-Err} $\ast$err)
\begin{CompactList}\small\item\em Public: Read Row with selection. \item\end{CompactList}\item 
Obit\-IOCode {\bf Obit\-IOWrite} ({\bf Obit\-IO} $\ast$in, {\bf ofloat} $\ast$data, {\bf Obit\-Err} $\ast$err)
\begin{CompactList}\small\item\em Public: Write. \item\end{CompactList}\item 
Obit\-IOCode {\bf Obit\-IOWrite\-Row} ({\bf Obit\-IO} $\ast$in, {\bf olong} rowno, {\bf ofloat} $\ast$data, {\bf Obit\-Err} $\ast$err)
\begin{CompactList}\small\item\em Public: Write Row. \item\end{CompactList}\item 
Obit\-IOCode {\bf Obit\-IOFlush} ({\bf Obit\-IO} $\ast$in, {\bf Obit\-Err} $\ast$err)
\begin{CompactList}\small\item\em Public: Flush. \item\end{CompactList}\item 
Obit\-IOCode {\bf Obit\-IORead\-Descriptor} ({\bf Obit\-IO} $\ast$in, {\bf Obit\-Err} $\ast$err)
\begin{CompactList}\small\item\em Public: Read Descriptor. \item\end{CompactList}\item 
Obit\-IOCode {\bf Obit\-IOWrite\-Descriptor} ({\bf Obit\-IO} $\ast$in, {\bf Obit\-Err} $\ast$err)
\begin{CompactList}\small\item\em Public: Write Descriptor. \item\end{CompactList}\item 
void {\bf Obit\-IOCreate\-Buffer} ({\bf ofloat} $\ast$$\ast$data, {\bf olong} $\ast$size, {\bf Obit\-IO} $\ast$in, {\bf Obit\-Info\-List} $\ast$info, {\bf Obit\-Err} $\ast$err)
\begin{CompactList}\small\item\em Public: Create buffer. \item\end{CompactList}\item 
void {\bf Obit\-IOFree\-Buffer} ({\bf ofloat} $\ast$buffer)
\begin{CompactList}\small\item\em Public: Destroy buffer. \item\end{CompactList}\item 
{\bf Obit} $\ast$ {\bf new\-Obit\-IOTable} ({\bf Obit\-IO} $\ast$in, Obit\-IOAccess access, gchar $\ast$tab\-Type, {\bf olong} $\ast$tabver, {\bf Obit\-Err} $\ast$err)
\begin{CompactList}\small\item\em Public: Create an associated Table Typed as base class to avoid problems. \item\end{CompactList}\item 
Obit\-IOCode {\bf Obit\-IOUpdate\-Tables} ({\bf Obit\-IO} $\ast$in, {\bf Obit\-Info\-List} $\ast$info, {\bf Obit\-Err} $\ast$err)
\begin{CompactList}\small\item\em Public: Update disk resident tables information. \item\end{CompactList}\item 
void {\bf Obit\-IOGet\-File\-Info} ({\bf Obit\-IO} $\ast$in, {\bf Obit\-Info\-List} $\ast$my\-Info, gchar $\ast$prefix, {\bf Obit\-Info\-List} $\ast$out\-List, {\bf Obit\-Err} $\ast$err)
\begin{CompactList}\small\item\em Public: Extract information about underlying file. \item\end{CompactList}\end{CompactItemize}


\subsection{Detailed Description}
{\bf Obit\-IO}{\rm (p.\,\pageref{structObitIO})} base class definition. 

This class is derived from the {\bf Obit}{\rm (p.\,\pageref{structObit})} class.

This is a virtual base class and should never be directly instantiated, However, its functions mshould be called and the correct version will be run. Derived classes provide an I/O interface to various underlying disk structures. The structure is also defined in Obit\-IODef.h to allow recursive definition in derived classes.\subsection{Usage}\label{ObitIO_8h_ObitIOUsage}
No instances should be created of this class but the class member functions, given a derived type, will invoke the correct function.

\subsection{Define Documentation}
\index{ObitIO.h@{Obit\-IO.h}!ObitIOIsA@{ObitIOIsA}}
\index{ObitIOIsA@{ObitIOIsA}!ObitIO.h@{Obit\-IO.h}}
\subsubsection{\setlength{\rightskip}{0pt plus 5cm}\#define Obit\-IOIs\-A(in)\ Obit\-Is\-A (in, Obit\-IOGet\-Class())}\label{ObitIO_8h_a2}


Macro to determine if an object is the member of this or a derived class. 

Returns TRUE if a member, else FALSE in = object to reference \index{ObitIO.h@{Obit\-IO.h}!ObitIORef@{ObitIORef}}
\index{ObitIORef@{ObitIORef}!ObitIO.h@{Obit\-IO.h}}
\subsubsection{\setlength{\rightskip}{0pt plus 5cm}\#define Obit\-IORef(in)\ Obit\-Ref (in)}\label{ObitIO_8h_a1}


Macro to reference (update reference count) an {\bf Obit\-IO}{\rm (p.\,\pageref{structObitIO})}. 

returns a Obit\-IO$\ast$. in = object to reference \index{ObitIO.h@{Obit\-IO.h}!ObitIOUnref@{ObitIOUnref}}
\index{ObitIOUnref@{ObitIOUnref}!ObitIO.h@{Obit\-IO.h}}
\subsubsection{\setlength{\rightskip}{0pt plus 5cm}\#define Obit\-IOUnref(in)\ Obit\-Unref (in)}\label{ObitIO_8h_a0}


Macro to unreference (and possibly destroy) an {\bf Obit\-IO}{\rm (p.\,\pageref{structObitIO})} returns a Obit\-IO$\ast$. 

in = object to unreference 

\subsection{Typedef Documentation}
\index{ObitIO.h@{Obit\-IO.h}!newObitIOFP@{newObitIOFP}}
\index{newObitIOFP@{newObitIOFP}!ObitIO.h@{Obit\-IO.h}}
\subsubsection{\setlength{\rightskip}{0pt plus 5cm}typedef {\bf Obit\-IO}$\ast$($\ast$ {\bf new\-Obit\-IOFP})(gchar $\ast$name, {\bf Obit\-Info\-List} $\ast$info, {\bf Obit\-Err} $\ast$err)}\label{ObitIO_8h_a3}


define type for Class\-Info structure 

\index{ObitIO.h@{Obit\-IO.h}!newObitIOTableFP@{newObitIOTableFP}}
\index{newObitIOTableFP@{newObitIOTableFP}!ObitIO.h@{Obit\-IO.h}}
\subsubsection{\setlength{\rightskip}{0pt plus 5cm}typedef {\bf Obit}$\ast$($\ast$ {\bf new\-Obit\-IOTable\-FP})({\bf Obit\-IO} $\ast$in, Obit\-IOAccess access, gchar $\ast$tab\-Type, {\bf olong} $\ast$tabver, {\bf Obit\-Err} $\ast$err)}\label{ObitIO_8h_a25}


\index{ObitIO.h@{Obit\-IO.h}!ObitIOCloseFP@{ObitIOCloseFP}}
\index{ObitIOCloseFP@{ObitIOCloseFP}!ObitIO.h@{Obit\-IO.h}}
\subsubsection{\setlength{\rightskip}{0pt plus 5cm}typedef Obit\-IOCode($\ast$ {\bf Obit\-IOClose\-FP})({\bf Obit\-IO} $\ast$in, {\bf Obit\-Err} $\ast$err)}\label{ObitIO_8h_a8}


\index{ObitIO.h@{Obit\-IO.h}!ObitIOCreateBufferFP@{ObitIOCreateBufferFP}}
\index{ObitIOCreateBufferFP@{ObitIOCreateBufferFP}!ObitIO.h@{Obit\-IO.h}}
\subsubsection{\setlength{\rightskip}{0pt plus 5cm}typedef void($\ast$ {\bf Obit\-IOCreate\-Buffer\-FP})({\bf ofloat} $\ast$$\ast$data, {\bf olong} $\ast$size, {\bf Obit\-IO} $\ast$in, {\bf Obit\-Info\-List} $\ast$info, {\bf Obit\-Err} $\ast$err)}\label{ObitIO_8h_a23}


\index{ObitIO.h@{Obit\-IO.h}!ObitIOFlushFP@{ObitIOFlushFP}}
\index{ObitIOFlushFP@{ObitIOFlushFP}!ObitIO.h@{Obit\-IO.h}}
\subsubsection{\setlength{\rightskip}{0pt plus 5cm}typedef Obit\-IOCode($\ast$ {\bf Obit\-IOFlush\-FP})({\bf Obit\-IO} $\ast$in, {\bf Obit\-Err} $\ast$err)}\label{ObitIO_8h_a20}


\index{ObitIO.h@{Obit\-IO.h}!ObitIOFreeBufferFP@{ObitIOFreeBufferFP}}
\index{ObitIOFreeBufferFP@{ObitIOFreeBufferFP}!ObitIO.h@{Obit\-IO.h}}
\subsubsection{\setlength{\rightskip}{0pt plus 5cm}typedef void($\ast$ {\bf Obit\-IOFree\-Buffer\-FP})({\bf ofloat} $\ast$buffer)}\label{ObitIO_8h_a24}


\index{ObitIO.h@{Obit\-IO.h}!ObitIOGetFileInfoFP@{ObitIOGetFileInfoFP}}
\index{ObitIOGetFileInfoFP@{ObitIOGetFileInfoFP}!ObitIO.h@{Obit\-IO.h}}
\subsubsection{\setlength{\rightskip}{0pt plus 5cm}typedef void($\ast$ {\bf Obit\-IOGet\-File\-Info\-FP})({\bf Obit\-IO} $\ast$in, {\bf Obit\-Info\-List} $\ast$my\-Info, gchar $\ast$prefix, {\bf Obit\-Info\-List} $\ast$out\-List, {\bf Obit\-Err} $\ast$err)}\label{ObitIO_8h_a27}


\index{ObitIO.h@{Obit\-IO.h}!ObitIOOpenFP@{ObitIOOpenFP}}
\index{ObitIOOpenFP@{ObitIOOpenFP}!ObitIO.h@{Obit\-IO.h}}
\subsubsection{\setlength{\rightskip}{0pt plus 5cm}typedef Obit\-IOCode($\ast$ {\bf Obit\-IOOpen\-FP})({\bf Obit\-IO} $\ast$in, Obit\-IOAccess access, {\bf Obit\-Info\-List} $\ast$info, {\bf Obit\-Err} $\ast$err)}\label{ObitIO_8h_a7}


\index{ObitIO.h@{Obit\-IO.h}!ObitIOReadDescriptorFP@{ObitIOReadDescriptorFP}}
\index{ObitIOReadDescriptorFP@{ObitIOReadDescriptorFP}!ObitIO.h@{Obit\-IO.h}}
\subsubsection{\setlength{\rightskip}{0pt plus 5cm}typedef Obit\-IOCode($\ast$ {\bf Obit\-IORead\-Descriptor\-FP})({\bf Obit\-IO} $\ast$in, {\bf Obit\-Err} $\ast$err)}\label{ObitIO_8h_a21}


\index{ObitIO.h@{Obit\-IO.h}!ObitIOReadFP@{ObitIOReadFP}}
\index{ObitIOReadFP@{ObitIOReadFP}!ObitIO.h@{Obit\-IO.h}}
\subsubsection{\setlength{\rightskip}{0pt plus 5cm}typedef Obit\-IOCode($\ast$ {\bf Obit\-IORead\-FP})({\bf Obit\-IO} $\ast$in, {\bf ofloat} $\ast$data, {\bf Obit\-Err} $\ast$err)}\label{ObitIO_8h_a10}


\index{ObitIO.h@{Obit\-IO.h}!ObitIOReadMultiFP@{ObitIOReadMultiFP}}
\index{ObitIOReadMultiFP@{ObitIOReadMultiFP}!ObitIO.h@{Obit\-IO.h}}
\subsubsection{\setlength{\rightskip}{0pt plus 5cm}typedef Obit\-IOCode($\ast$ {\bf Obit\-IORead\-Multi\-FP})({\bf olong} n\-Buff, {\bf Obit\-IO} $\ast$$\ast$in, {\bf ofloat} $\ast$$\ast$data, {\bf Obit\-Err} $\ast$err)}\label{ObitIO_8h_a11}


\index{ObitIO.h@{Obit\-IO.h}!ObitIOReadMultiSelectFP@{ObitIOReadMultiSelectFP}}
\index{ObitIOReadMultiSelectFP@{ObitIOReadMultiSelectFP}!ObitIO.h@{Obit\-IO.h}}
\subsubsection{\setlength{\rightskip}{0pt plus 5cm}typedef Obit\-IOCode($\ast$ {\bf Obit\-IORead\-Multi\-Select\-FP})({\bf olong} n\-Buff, {\bf Obit\-IO} $\ast$$\ast$in, {\bf ofloat} $\ast$$\ast$data, {\bf Obit\-Err} $\ast$err)}\label{ObitIO_8h_a14}


\index{ObitIO.h@{Obit\-IO.h}!ObitIOReadRowFP@{ObitIOReadRowFP}}
\index{ObitIOReadRowFP@{ObitIOReadRowFP}!ObitIO.h@{Obit\-IO.h}}
\subsubsection{\setlength{\rightskip}{0pt plus 5cm}typedef Obit\-IOCode($\ast$ {\bf Obit\-IORead\-Row\-FP})({\bf Obit\-IO} $\ast$in, {\bf olong} rowno, {\bf ofloat} $\ast$data, {\bf Obit\-Err} $\ast$err)}\label{ObitIO_8h_a12}


\index{ObitIO.h@{Obit\-IO.h}!ObitIOReadRowSelectFP@{ObitIOReadRowSelectFP}}
\index{ObitIOReadRowSelectFP@{ObitIOReadRowSelectFP}!ObitIO.h@{Obit\-IO.h}}
\subsubsection{\setlength{\rightskip}{0pt plus 5cm}typedef Obit\-IOCode($\ast$ {\bf Obit\-IORead\-Row\-Select\-FP})({\bf Obit\-IO} $\ast$in, {\bf olong} rowno, {\bf ofloat} $\ast$data, {\bf Obit\-Err} $\ast$err)}\label{ObitIO_8h_a17}


\index{ObitIO.h@{Obit\-IO.h}!ObitIOReadSelectFP@{ObitIOReadSelectFP}}
\index{ObitIOReadSelectFP@{ObitIOReadSelectFP}!ObitIO.h@{Obit\-IO.h}}
\subsubsection{\setlength{\rightskip}{0pt plus 5cm}typedef Obit\-IOCode($\ast$ {\bf Obit\-IORead\-Select\-FP})({\bf Obit\-IO} $\ast$in, {\bf ofloat} $\ast$data, {\bf Obit\-Err} $\ast$err)}\label{ObitIO_8h_a13}


\index{ObitIO.h@{Obit\-IO.h}!ObitIORenameFP@{ObitIORenameFP}}
\index{ObitIORenameFP@{ObitIORenameFP}!ObitIO.h@{Obit\-IO.h}}
\subsubsection{\setlength{\rightskip}{0pt plus 5cm}typedef void($\ast$ {\bf Obit\-IORename\-FP})({\bf Obit\-IO} $\ast$in, {\bf Obit\-Info\-List} $\ast$info, {\bf Obit\-Err} $\ast$err)}\label{ObitIO_8h_a5}


\index{ObitIO.h@{Obit\-IO.h}!ObitIOReReadMultiFP@{ObitIOReReadMultiFP}}
\index{ObitIOReReadMultiFP@{ObitIOReReadMultiFP}!ObitIO.h@{Obit\-IO.h}}
\subsubsection{\setlength{\rightskip}{0pt plus 5cm}typedef Obit\-IOCode($\ast$ {\bf Obit\-IORe\-Read\-Multi\-FP})({\bf olong} n\-Buff, {\bf Obit\-IO} $\ast$$\ast$in, {\bf ofloat} $\ast$$\ast$data, {\bf Obit\-Err} $\ast$err)}\label{ObitIO_8h_a15}


\index{ObitIO.h@{Obit\-IO.h}!ObitIOReReadMultiSelectFP@{ObitIOReReadMultiSelectFP}}
\index{ObitIOReReadMultiSelectFP@{ObitIOReReadMultiSelectFP}!ObitIO.h@{Obit\-IO.h}}
\subsubsection{\setlength{\rightskip}{0pt plus 5cm}typedef Obit\-IOCode($\ast$ {\bf Obit\-IORe\-Read\-Multi\-Select\-FP})({\bf olong} n\-Buff, {\bf Obit\-IO} $\ast$$\ast$in, {\bf ofloat} $\ast$$\ast$data, {\bf Obit\-Err} $\ast$err)}\label{ObitIO_8h_a16}


\index{ObitIO.h@{Obit\-IO.h}!ObitIOSameFP@{ObitIOSameFP}}
\index{ObitIOSameFP@{ObitIOSameFP}!ObitIO.h@{Obit\-IO.h}}
\subsubsection{\setlength{\rightskip}{0pt plus 5cm}typedef gboolean($\ast$ {\bf Obit\-IOSame\-FP})({\bf Obit\-IO} $\ast$in, {\bf Obit\-Info\-List} $\ast$in1, {\bf Obit\-Info\-List} $\ast$in2, {\bf Obit\-Err} $\ast$err)}\label{ObitIO_8h_a4}


\index{ObitIO.h@{Obit\-IO.h}!ObitIOSetFP@{ObitIOSetFP}}
\index{ObitIOSetFP@{ObitIOSetFP}!ObitIO.h@{Obit\-IO.h}}
\subsubsection{\setlength{\rightskip}{0pt plus 5cm}typedef Obit\-IOCode($\ast$ {\bf Obit\-IOSet\-FP})({\bf Obit\-IO} $\ast$in, {\bf Obit\-Info\-List} $\ast$info, {\bf Obit\-Err} $\ast$err)}\label{ObitIO_8h_a9}


\index{ObitIO.h@{Obit\-IO.h}!ObitIOUpdateTablesFP@{ObitIOUpdateTablesFP}}
\index{ObitIOUpdateTablesFP@{ObitIOUpdateTablesFP}!ObitIO.h@{Obit\-IO.h}}
\subsubsection{\setlength{\rightskip}{0pt plus 5cm}typedef Obit\-IOCode($\ast$ {\bf Obit\-IOUpdate\-Tables\-FP})({\bf Obit\-IO} $\ast$in, {\bf Obit\-Info\-List} $\ast$info, {\bf Obit\-Err} $\ast$err)}\label{ObitIO_8h_a26}


\index{ObitIO.h@{Obit\-IO.h}!ObitIOWriteDescriptorFP@{ObitIOWriteDescriptorFP}}
\index{ObitIOWriteDescriptorFP@{ObitIOWriteDescriptorFP}!ObitIO.h@{Obit\-IO.h}}
\subsubsection{\setlength{\rightskip}{0pt plus 5cm}typedef Obit\-IOCode($\ast$ {\bf Obit\-IOWrite\-Descriptor\-FP})({\bf Obit\-IO} $\ast$in, {\bf Obit\-Err} $\ast$err)}\label{ObitIO_8h_a22}


\index{ObitIO.h@{Obit\-IO.h}!ObitIOWriteFP@{ObitIOWriteFP}}
\index{ObitIOWriteFP@{ObitIOWriteFP}!ObitIO.h@{Obit\-IO.h}}
\subsubsection{\setlength{\rightskip}{0pt plus 5cm}typedef Obit\-IOCode($\ast$ {\bf Obit\-IOWrite\-FP})({\bf Obit\-IO} $\ast$in, {\bf ofloat} $\ast$data, {\bf Obit\-Err} $\ast$err)}\label{ObitIO_8h_a18}


\index{ObitIO.h@{Obit\-IO.h}!ObitIOWriteRowFP@{ObitIOWriteRowFP}}
\index{ObitIOWriteRowFP@{ObitIOWriteRowFP}!ObitIO.h@{Obit\-IO.h}}
\subsubsection{\setlength{\rightskip}{0pt plus 5cm}typedef Obit\-IOCode($\ast$ {\bf Obit\-IOWrite\-Row\-FP})({\bf Obit\-IO} $\ast$in, {\bf olong} rowno, {\bf ofloat} $\ast$data, {\bf Obit\-Err} $\ast$err)}\label{ObitIO_8h_a19}


\index{ObitIO.h@{Obit\-IO.h}!ObitIOZapFP@{ObitIOZapFP}}
\index{ObitIOZapFP@{ObitIOZapFP}!ObitIO.h@{Obit\-IO.h}}
\subsubsection{\setlength{\rightskip}{0pt plus 5cm}typedef void($\ast$ {\bf Obit\-IOZap\-FP})({\bf Obit\-IO} $\ast$in, {\bf Obit\-Err} $\ast$err)}\label{ObitIO_8h_a6}




\subsection{Function Documentation}
\index{ObitIO.h@{Obit\-IO.h}!newObitIO@{newObitIO}}
\index{newObitIO@{newObitIO}!ObitIO.h@{Obit\-IO.h}}
\subsubsection{\setlength{\rightskip}{0pt plus 5cm}{\bf Obit\-IO}$\ast$ new\-Obit\-IO (gchar $\ast$ {\em name}, {\bf Obit\-Info\-List} $\ast$ {\em info}, {\bf Obit\-Err} $\ast$ {\em err})}\label{ObitIO_8h_a29}


Public: Constructor. 

Initializes class if needed on first call. \begin{Desc}
\item[Returns:]the new object. \end{Desc}
\index{ObitIO.h@{Obit\-IO.h}!newObitIOTable@{newObitIOTable}}
\index{newObitIOTable@{newObitIOTable}!ObitIO.h@{Obit\-IO.h}}
\subsubsection{\setlength{\rightskip}{0pt plus 5cm}{\bf Obit}$\ast$ new\-Obit\-IOTable ({\bf Obit\-IO} $\ast$ {\em in}, Obit\-IOAccess {\em access}, gchar $\ast$ {\em tab\-Type}, {\bf olong} $\ast$ {\em tab\-Ver}, {\bf Obit\-Err} $\ast$ {\em err})}\label{ObitIO_8h_a53}


Public: Create an associated Table Typed as base class to avoid problems. 

If such an object exists, a reference to it is returned, else a new object is created and entered in the {\bf Obit\-Table\-List}{\rm (p.\,\pageref{structObitTableList})}. Returned object is typed an {\bf Obit}{\rm (p.\,\pageref{structObit})} to prevent circular definitions between the {\bf Obit\-Table}{\rm (p.\,\pageref{structObitTable})} and the {\bf Obit\-IO}{\rm (p.\,\pageref{structObitIO})} classes. \begin{Desc}
\item[Parameters:]
\begin{description}
\item[{\em in}]Pointer to object with associated tables. This MUST have been opened before this call. \item[{\em access}]access (OBIT\_\-IO\_\-Read\-Only,OBIT\_\-IO\_\-Read\-Write, or OBIT\_\-IO\_\-Write\-Only). This is used to determine defaulted version number and a different value may be used for the actual Open. \item[{\em tab\-Type}]The table type (e.g. \char`\"{}AIPS CC\char`\"{}). \item[{\em tab\-Ver}]Desired version number, may be zero in which case the highest extant version is returned for read and the highest+1 for write. \item[{\em err}]{\bf Obit\-Err}{\rm (p.\,\pageref{structObitErr})} for reporting errors. \end{description}
\end{Desc}
\begin{Desc}
\item[Returns:]pointer to created {\bf Obit\-Table}{\rm (p.\,\pageref{structObitTable})}, NULL on failure. \end{Desc}
\index{ObitIO.h@{Obit\-IO.h}!ObitIOClassInit@{ObitIOClassInit}}
\index{ObitIOClassInit@{ObitIOClassInit}!ObitIO.h@{Obit\-IO.h}}
\subsubsection{\setlength{\rightskip}{0pt plus 5cm}void Obit\-IOClass\-Init (void)}\label{ObitIO_8h_a28}


Public: Class initializer. 

\index{ObitIO.h@{Obit\-IO.h}!ObitIOClose@{ObitIOClose}}
\index{ObitIOClose@{ObitIOClose}!ObitIO.h@{Obit\-IO.h}}
\subsubsection{\setlength{\rightskip}{0pt plus 5cm}Obit\-IOCode Obit\-IOClose ({\bf Obit\-IO} $\ast$ {\em in}, {\bf Obit\-Err} $\ast$ {\em err})}\label{ObitIO_8h_a36}


Public: Close. 

\begin{Desc}
\item[Parameters:]
\begin{description}
\item[{\em in}]Pointer to object to be closed. \item[{\em err}]{\bf Obit\-Err}{\rm (p.\,\pageref{structObitErr})} for reporting errors. \end{description}
\end{Desc}
\begin{Desc}
\item[Returns:]error code, 0=$>$ OK \end{Desc}
\index{ObitIO.h@{Obit\-IO.h}!ObitIOCopy@{ObitIOCopy}}
\index{ObitIOCopy@{ObitIOCopy}!ObitIO.h@{Obit\-IO.h}}
\subsubsection{\setlength{\rightskip}{0pt plus 5cm}{\bf Obit\-IO}$\ast$ Obit\-IOCopy ({\bf Obit\-IO} $\ast$ {\em in}, {\bf Obit\-IO} $\ast$ {\em out}, {\bf Obit\-Err} $\ast$ {\em err})}\label{ObitIO_8h_a34}


Public: Copy constructor. 

The result will have pointers to the more complex members. Parent class members are included but any derived class info is ignored. \begin{Desc}
\item[Parameters:]
\begin{description}
\item[{\em in}]The object to copy \item[{\em out}]An existing object pointer for output or NULL if none exists. \item[{\em err}]{\bf Obit}{\rm (p.\,\pageref{structObit})} error stack object. \end{description}
\end{Desc}
\begin{Desc}
\item[Returns:]pointer to the new object. \end{Desc}
\index{ObitIO.h@{Obit\-IO.h}!ObitIOCreateBuffer@{ObitIOCreateBuffer}}
\index{ObitIOCreateBuffer@{ObitIOCreateBuffer}!ObitIO.h@{Obit\-IO.h}}
\subsubsection{\setlength{\rightskip}{0pt plus 5cm}void Obit\-IOCreate\-Buffer ({\bf ofloat} $\ast$$\ast$ {\em data}, {\bf olong} $\ast$ {\em size}, {\bf Obit\-IO} $\ast$ {\em in}, {\bf Obit\-Info\-List} $\ast$ {\em info}, {\bf Obit\-Err} $\ast$ {\em err})}\label{ObitIO_8h_a51}


Public: Create buffer. 

\begin{Desc}
\item[Parameters:]
\begin{description}
\item[{\em data}](output) pointer to data array \item[{\em size}](output) size of data array in floats. \item[{\em in}]Pointer to object to be accessed. \item[{\em info}]{\bf Obit\-Info\-List}{\rm (p.\,\pageref{structObitInfoList})} with instructions \item[{\em err}]{\bf Obit\-Err}{\rm (p.\,\pageref{structObitErr})} for reporting errors. \end{description}
\end{Desc}
\index{ObitIO.h@{Obit\-IO.h}!ObitIOFlush@{ObitIOFlush}}
\index{ObitIOFlush@{ObitIOFlush}!ObitIO.h@{Obit\-IO.h}}
\subsubsection{\setlength{\rightskip}{0pt plus 5cm}Obit\-IOCode Obit\-IOFlush ({\bf Obit\-IO} $\ast$ {\em in}, {\bf Obit\-Err} $\ast$ {\em err})}\label{ObitIO_8h_a48}


Public: Flush. 

\begin{Desc}
\item[Parameters:]
\begin{description}
\item[{\em in}]Pointer to object to be accessed. \item[{\em err}]{\bf Obit\-Err}{\rm (p.\,\pageref{structObitErr})} for reporting errors. \end{description}
\end{Desc}
\begin{Desc}
\item[Returns:]return code, 0=$>$ OK \end{Desc}
\index{ObitIO.h@{Obit\-IO.h}!ObitIOFreeBuffer@{ObitIOFreeBuffer}}
\index{ObitIOFreeBuffer@{ObitIOFreeBuffer}!ObitIO.h@{Obit\-IO.h}}
\subsubsection{\setlength{\rightskip}{0pt plus 5cm}void Obit\-IOFree\-Buffer ({\bf ofloat} $\ast$ {\em buffer})}\label{ObitIO_8h_a52}


Public: Destroy buffer. 

\begin{Desc}
\item[Parameters:]
\begin{description}
\item[{\em buffer}]Pointer to buffer to destroy. \end{description}
\end{Desc}
\index{ObitIO.h@{Obit\-IO.h}!ObitIOGetClass@{ObitIOGetClass}}
\index{ObitIOGetClass@{ObitIOGetClass}!ObitIO.h@{Obit\-IO.h}}
\subsubsection{\setlength{\rightskip}{0pt plus 5cm}gconstpointer Obit\-IOGet\-Class (void)}\label{ObitIO_8h_a30}


Public: Class\-Info pointer. 

Initializes class if needed on first call. \begin{Desc}
\item[Returns:]pointer to the class structure. \end{Desc}
\index{ObitIO.h@{Obit\-IO.h}!ObitIOGetFileInfo@{ObitIOGetFileInfo}}
\index{ObitIOGetFileInfo@{ObitIOGetFileInfo}!ObitIO.h@{Obit\-IO.h}}
\subsubsection{\setlength{\rightskip}{0pt plus 5cm}void Obit\-IOGet\-File\-Info ({\bf Obit\-IO} $\ast$ {\em in}, {\bf Obit\-Info\-List} $\ast$ {\em my\-Info}, gchar $\ast$ {\em prefix}, {\bf Obit\-Info\-List} $\ast$ {\em out\-List}, {\bf Obit\-Err} $\ast$ {\em err})}\label{ObitIO_8h_a55}


Public: Extract information about underlying file. 

\begin{Desc}
\item[Parameters:]
\begin{description}
\item[{\em in}]Object of interest. \item[{\em my\-Info}]Info\-List on basic object with selection \item[{\em prefix}]If Non\-Null, string to be added to beginning of out\-List entry name \item[{\em out\-List}]Info\-List to write entries into Following entries for AIPS files (\char`\"{}xxx\char`\"{} = prefix): \begin{itemize}
\item xxx\-Name OBIT\_\-string AIPS file name \item xxx\-Class OBIT\_\-string AIPS file class \item xxx\-Disk OBIT\_\-oint AIPS file disk number \item xxx\-Seq OBIT\_\-oint AIPS file Sequence number \item AIPSuser OBIT\_\-oint AIPS User number \item xxx\-CNO OBIT\_\-oint AIPS Catalog slot number \item xxx\-Dir OBIT\_\-string Directory name for xxx\-Disk\end{itemize}
Following entries for FITS files (\char`\"{}xxx\char`\"{} = prefix): \begin{itemize}
\item xxx\-File\-Name OBIT\_\-string FITS file name \item xxx\-Disk OBIT\_\-oint FITS file disk number \item xxx\-Dir OBIT\_\-string Directory name for xxx\-Disk\end{itemize}
For all File types: \begin{itemize}
\item xxx\-File\-Type OBIT\_\-string \char`\"{}UV\char`\"{} = UV data, \char`\"{}MA\char`\"{}=$>$image, \char`\"{}Table\char`\"{}=Table, \char`\"{}OTF\char`\"{}=OTF, etc \item xxx\-Data\-Type OBIT\_\-string \char`\"{}AIPS\char`\"{}, \char`\"{}FITS\char`\"{}\end{itemize}
For xxx\-Data\-Type = \char`\"{}Table\char`\"{} \begin{itemize}
\item xxx\-Tab OBIT\_\-string (Tables only) Table type (e.g. \char`\"{}AIPS CC\char`\"{}) \item xxx\-Ver OBIT\_\-oint (Tables Only) Table version number\end{itemize}
For xxx\-Data\-Type = \char`\"{}MA\char`\"{} \begin{itemize}
\item xxx\-BLC OBIT\_\-oint[7] (Images only) 1-rel bottom-left corner pixel \item xxx\-TRC OBIT\_\-oint[7] (Images Only) 1-rel top-right corner pixel\end{itemize}
For xxx\-Data\-Type = \char`\"{}OTF\char`\"{} \begin{itemize}
\item xxxn\-Rec\-PIO OBIT\_\-int (1,1,1) Number of vis. records per IO call\end{itemize}
For xxx\-Data\-Type = \char`\"{}UV\char`\"{} \begin{itemize}
\item xxxn\-Vis\-PIO OBIT\_\-int (1,1,1) Number of vis. records per IO call \item xxxdo\-Cal\-Select OBIT\_\-bool (1,1,1) Select/calibrate/edit data? \item xxx\-Stokes OBIT\_\-string (4,1,1) Selected output Stokes parameters: \char`\"{}    \char`\"{}=$>$ no translation,\char`\"{}I   \char`\"{},\char`\"{}V   \char`\"{},\char`\"{}Q   \char`\"{}, \char`\"{}U   \char`\"{}, \char`\"{}IQU \char`\"{}, \char`\"{}IQUV\char`\"{}, \char`\"{}IV  \char`\"{}, \char`\"{}RR  \char`\"{}, \char`\"{}LL  \char`\"{}, \char`\"{}RL  \char`\"{}, \char`\"{}LR  \char`\"{}, \char`\"{}HALF\char`\"{} = RR,LL, \char`\"{}FULL\char`\"{}=RR,LL,RL,LR. [default \char`\"{}    \char`\"{}] In the above 'F' can substitute for \char`\"{}formal\char`\"{} 'I' (both RR+LL). \item xxx\-BChan OBIT\_\-int (1,1,1) First spectral channel selected. [def all] \item xxx\-EChan OBIT\_\-int (1,1,1) Highest spectral channel selected. [def all] \item xxx\-BIF OBIT\_\-int (1,1,1) First \char`\"{}IF\char`\"{} selected. [def all] \item xxx\-EIF OBIT\_\-int (1,1,1) Highest \char`\"{}IF\char`\"{} selected. [def all] \item xxxdo\-Pol OBIT\_\-int (1,1,1) $>$0 -$>$ calibrate polarization. \item xxxdo\-Calib OBIT\_\-int (1,1,1) $>$0 -$>$ calibrate, 2=$>$ also calibrate Weights \item xxxgain\-Use OBIT\_\-int (1,1,1) SN/CL table version number, 0-$>$ use highest \item xxxflag\-Ver OBIT\_\-int (1,1,1) Flag table version, 0-$>$ use highest, $<$0-$>$ none \item xxx\-BLVer OBIT\_\-int (1,1,1) BL table version, 0$>$ use highest, $<$0-$>$ none \item xxx\-BPVer OBIT\_\-int (1,1,1) Band pass (BP) table version, 0-$>$ use highest \item xxx\-Subarray OBIT\_\-int (1,1,1) Selected subarray, $<$=0-$>$all [default all] \item xxxdrop\-Sub\-A OBIT\_\-bool (1,1,1) Drop subarray info? \item xxx\-Freq\-ID OBIT\_\-int (1,1,1) Selected Frequency ID, $<$=0-$>$all [default all] \item xxxtime\-Range OBIT\_\-float (2,1,1) Selected timerange in days. \item xxx\-UVRange OBIT\_\-float (2,1,1) Selected UV range in kilowavelengths. \item xxx\-Input\-Avg\-Time OBIT\_\-float (1,1,1) Input data averaging time (sec). used for fringe rate decorrelation correction. \item xxx\-Sources OBIT\_\-string (?,?,1) Source names selected unless any starts with a '-' in which case all are deselected (with '-' stripped). \item xxxsou\-Code OBIT\_\-string (4,1,1) Source Cal code desired, ' ' =$>$ any code selected '$\ast$ ' =$>$ any non blank code (calibrators only) '-CAL' =$>$ blank codes only (no calibrators) \item xxx\-Qual Obit\_\-int (1,1,1) Source qualifier, -1 [default] = any \item xxx\-Antennas OBIT\_\-int (?,1,1) a list of selected antenna numbers, if any is negative then the absolute values are used and the specified antennas are deselected. \item xxxcorr\-Type OBIT\_\-int (1,1,1) Correlation type, 0=cross corr only, 1=both, 2=auto only. \item xxxpass\-Al l OBIT\_\-bool (1,1,1) If True, pass along all data when selecting/calibration even if it's all flagged, data deselected by time, source, antenna etc. is not passed. \item xxxdo\-Band OBIT\_\-int (1,1,1) Band pass application type $<$0-$>$ none (1) if = 1 then all the bandpass data for each antenna will be averaged to form a composite bandpass spectrum, this will then be used to correct the data. (2) if = 2 the bandpass spectra nearest in time (in a weighted sense) to the uv data point will be used to correct the data. (3) if = 3 the bandpass data will be interpolated in time using the solution weights to form a composite bandpass spectrum, this interpolated spectrum will then be used to correct the data. (4) if = 4 the bandpass spectra nearest in time (neglecting weights) to the uv data point will be used to correct the data. (5) if = 5 the bandpass data will be interpolated in time ignoring weights to form a composite bandpass spectrum, this interpolated spectrum will then be used to correct the data. \item xxx\-Smooth OBIT\_\-float (3,1,1) specifies the type of spectral smoothing Smooth(1) = type of smoothing to apply: 0 =$>$ no smoothing 1 =$>$ Hanning 2 =$>$ Gaussian 3 =$>$ Boxcar 4 =$>$ Sinc (i.e. sin(x)/x) Smooth(2) = the \char`\"{}diameter\char`\"{} of the function, i.e. width between first nulls of Hanning triangle and sinc function, FWHM of Gaussian, width of Boxcar. Defaults (if $<$ 0.1) are 4, 2, 2 and 3 channels for Smooth(1) = 1 - 4. Smooth(3) = the diameter over which the convolving function has value - in channels. Defaults: 1, 3, 1, 4 times Smooth(2) used when \item xxx\-Sub\-Scan\-Time Obit\_\-float scalar [Optional] if given, this is the desired time (days) of a sub scan. This is used by the selector to suggest a value close to this which will evenly divide the current scan. See {\bf Obit\-UVSel\-Sub\-Scan}{\rm (p.\,\pageref{ObitUVSel_8c_a21})} 0 =$>$ Use scan average. This is only useful for Read\-Select operations on indexed Obit\-UVs. \end{itemize}
\item[{\em err}]{\bf Obit\-Err}{\rm (p.\,\pageref{structObitErr})} for reporting errors. \end{description}
\end{Desc}
\index{ObitIO.h@{Obit\-IO.h}!ObitIOOpen@{ObitIOOpen}}
\index{ObitIOOpen@{ObitIOOpen}!ObitIO.h@{Obit\-IO.h}}
\subsubsection{\setlength{\rightskip}{0pt plus 5cm}Obit\-IOCode Obit\-IOOpen ({\bf Obit\-IO} $\ast$ {\em in}, Obit\-IOAccess {\em access}, {\bf Obit\-Info\-List} $\ast$ {\em info}, {\bf Obit\-Err} $\ast$ {\em err})}\label{ObitIO_8h_a35}


Public: Open. 

The file and selection info member should have been stored in the {\bf Obit\-Info\-List}{\rm (p.\,\pageref{structObitInfoList})} prior to calling. See derived classes for details. \begin{Desc}
\item[Parameters:]
\begin{description}
\item[{\em in}]Pointer to object to be opened. \item[{\em access}]access (OBIT\_\-IO\_\-Read\-Only,OBIT\_\-IO\_\-Read\-Write) \item[{\em info}]{\bf Obit\-Info\-List}{\rm (p.\,\pageref{structObitInfoList})} with instructions for opening \item[{\em err}]{\bf Obit\-Err}{\rm (p.\,\pageref{structObitErr})} for reporting errors. \end{description}
\end{Desc}
\begin{Desc}
\item[Returns:]return code, 0=$>$ OK \end{Desc}
\index{ObitIO.h@{Obit\-IO.h}!ObitIORead@{ObitIORead}}
\index{ObitIORead@{ObitIORead}!ObitIO.h@{Obit\-IO.h}}
\subsubsection{\setlength{\rightskip}{0pt plus 5cm}Obit\-IOCode Obit\-IORead ({\bf Obit\-IO} $\ast$ {\em in}, {\bf ofloat} $\ast$ {\em data}, {\bf Obit\-Err} $\ast$ {\em err})}\label{ObitIO_8h_a38}


Public: Read. 

\begin{Desc}
\item[Parameters:]
\begin{description}
\item[{\em in}]Pointer to object to be read. \item[{\em data}]pointer to buffer to write results. \item[{\em err}]{\bf Obit\-Err}{\rm (p.\,\pageref{structObitErr})} for reporting errors. \end{description}
\end{Desc}
\begin{Desc}
\item[Returns:]return code, 0=$>$ OK \end{Desc}
\index{ObitIO.h@{Obit\-IO.h}!ObitIOReadDescriptor@{ObitIOReadDescriptor}}
\index{ObitIOReadDescriptor@{ObitIOReadDescriptor}!ObitIO.h@{Obit\-IO.h}}
\subsubsection{\setlength{\rightskip}{0pt plus 5cm}Obit\-IOCode Obit\-IORead\-Descriptor ({\bf Obit\-IO} $\ast$ {\em in}, {\bf Obit\-Err} $\ast$ {\em err})}\label{ObitIO_8h_a49}


Public: Read Descriptor. 

\begin{Desc}
\item[Parameters:]
\begin{description}
\item[{\em in}]Pointer to object with Obit\-Image\-Descto be read. \item[{\em err}]{\bf Obit\-Err}{\rm (p.\,\pageref{structObitErr})} for reporting errors. \end{description}
\end{Desc}
\begin{Desc}
\item[Returns:]return code, 0=$>$ OK \end{Desc}
\index{ObitIO.h@{Obit\-IO.h}!ObitIOReadMulti@{ObitIOReadMulti}}
\index{ObitIOReadMulti@{ObitIOReadMulti}!ObitIO.h@{Obit\-IO.h}}
\subsubsection{\setlength{\rightskip}{0pt plus 5cm}Obit\-IOCode Obit\-IORead\-Multi ({\bf olong} {\em n\-Buff}, {\bf Obit\-IO} $\ast$$\ast$ {\em in}, {\bf ofloat} $\ast$$\ast$ {\em data}, {\bf Obit\-Err} $\ast$ {\em err})}\label{ObitIO_8h_a39}


Public: Read to multiple buffers. 

\begin{Desc}
\item[Parameters:]
\begin{description}
\item[{\em n\-Buff}]Number of buffers to be filled \item[{\em in}]Array of pointers to to object to be read; must all be to same underlying data set but with independent calibration \item[{\em data}]array of pointers to buffers to write results. \item[{\em err}]{\bf Obit\-Err}{\rm (p.\,\pageref{structObitErr})} for reporting errors. \end{description}
\end{Desc}
\begin{Desc}
\item[Returns:]return code, OBIT\_\-IO\_\-OK=$>$ OK \end{Desc}
\index{ObitIO.h@{Obit\-IO.h}!ObitIOReadMultiSelect@{ObitIOReadMultiSelect}}
\index{ObitIOReadMultiSelect@{ObitIOReadMultiSelect}!ObitIO.h@{Obit\-IO.h}}
\subsubsection{\setlength{\rightskip}{0pt plus 5cm}Obit\-IOCode Obit\-IORead\-Multi\-Select ({\bf olong} {\em n\-Buff}, {\bf Obit\-IO} $\ast$$\ast$ {\em in}, {\bf ofloat} $\ast$$\ast$ {\em data}, {\bf Obit\-Err} $\ast$ {\em err})}\label{ObitIO_8h_a42}


Public: Read with selection to multiple buffers. 

\begin{Desc}
\item[Parameters:]
\begin{description}
\item[{\em n\-Buff}]Number of buffers to be filled \item[{\em in}]Array of pointers to to object to be read; must all be to same underlying data set but with independent calibration \item[{\em data}]array of pointers to buffers to write results. \item[{\em err}]{\bf Obit\-Err}{\rm (p.\,\pageref{structObitErr})} for reporting errors. \end{description}
\end{Desc}
\begin{Desc}
\item[Returns:]return code, OBIT\_\-IO\_\-OK=$>$ OK \end{Desc}
\index{ObitIO.h@{Obit\-IO.h}!ObitIOReadRow@{ObitIOReadRow}}
\index{ObitIOReadRow@{ObitIOReadRow}!ObitIO.h@{Obit\-IO.h}}
\subsubsection{\setlength{\rightskip}{0pt plus 5cm}Obit\-IOCode Obit\-IORead\-Row ({\bf Obit\-IO} $\ast$ {\em in}, {\bf olong} {\em rowno}, {\bf ofloat} $\ast$ {\em data}, {\bf Obit\-Err} $\ast$ {\em err})}\label{ObitIO_8h_a40}


Public: Read Row. 

\begin{Desc}
\item[Parameters:]
\begin{description}
\item[{\em in}]Pointer to object to be read. \item[{\em rowno}]Starting row number (1-rel) -1=$>$ next. \item[{\em data}]pointer to buffer to write results. \item[{\em err}]{\bf Obit\-Err}{\rm (p.\,\pageref{structObitErr})} for reporting errors. \end{description}
\end{Desc}
\begin{Desc}
\item[Returns:]return code, 0=$>$ OK \end{Desc}
\index{ObitIO.h@{Obit\-IO.h}!ObitIOReadRowSelect@{ObitIOReadRowSelect}}
\index{ObitIOReadRowSelect@{ObitIOReadRowSelect}!ObitIO.h@{Obit\-IO.h}}
\subsubsection{\setlength{\rightskip}{0pt plus 5cm}Obit\-IOCode Obit\-IORead\-Row\-Select ({\bf Obit\-IO} $\ast$ {\em in}, {\bf olong} {\em rowno}, {\bf ofloat} $\ast$ {\em data}, {\bf Obit\-Err} $\ast$ {\em err})}\label{ObitIO_8h_a45}


Public: Read Row with selection. 

\begin{Desc}
\item[Parameters:]
\begin{description}
\item[{\em in}]Pointer to object to be read. \item[{\em rowno}]Starting row number (1-rel) -1=$>$ next. \item[{\em data}]pointer to buffer to write results. \item[{\em err}]{\bf Obit\-Err}{\rm (p.\,\pageref{structObitErr})} for reporting errors. \end{description}
\end{Desc}
\begin{Desc}
\item[Returns:]return code, 0=$>$ OK \end{Desc}
\index{ObitIO.h@{Obit\-IO.h}!ObitIOReadSelect@{ObitIOReadSelect}}
\index{ObitIOReadSelect@{ObitIOReadSelect}!ObitIO.h@{Obit\-IO.h}}
\subsubsection{\setlength{\rightskip}{0pt plus 5cm}Obit\-IOCode Obit\-IORead\-Select ({\bf Obit\-IO} $\ast$ {\em in}, {\bf ofloat} $\ast$ {\em data}, {\bf Obit\-Err} $\ast$ {\em err})}\label{ObitIO_8h_a41}


Public: Read with selection. 

\begin{Desc}
\item[Parameters:]
\begin{description}
\item[{\em in}]Pointer to object to be read. \item[{\em data}]pointer to buffer to write results. \item[{\em err}]{\bf Obit\-Err}{\rm (p.\,\pageref{structObitErr})} for reporting errors. \end{description}
\end{Desc}
\begin{Desc}
\item[Returns:]return code, 0=$>$ OK \end{Desc}
\index{ObitIO.h@{Obit\-IO.h}!ObitIORename@{ObitIORename}}
\index{ObitIORename@{ObitIORename}!ObitIO.h@{Obit\-IO.h}}
\subsubsection{\setlength{\rightskip}{0pt plus 5cm}void Obit\-IORename ({\bf Obit\-IO} $\ast$ {\em in}, {\bf Obit\-Info\-List} $\ast$ {\em info}, {\bf Obit\-Err} $\ast$ {\em err})}\label{ObitIO_8h_a32}


Public: Rename underlying structures. 

New name information depends on the underlying file type and is given on the info member. \begin{Desc}
\item[Parameters:]
\begin{description}
\item[{\em in}]Pointer to object to be zapped. \item[{\em info}]Associated {\bf Obit\-Info\-List}{\rm (p.\,\pageref{structObitInfoList})} For FITS files: \begin{itemize}
\item \char`\"{}new\-File\-Name\char`\"{} OBIT\_\-string (?,1,1) New Name of disk file.\end{itemize}
For AIPS: \begin{itemize}
\item \char`\"{}new\-Name\char`\"{} OBIT\_\-string (12,1,1) New AIPS Name absent or Blank = don't change \item \char`\"{}new\-Class\char`\"{} OBIT\_\-string (6,1,1) New AIPS Class absent or Blank = don't change\-O \item \char`\"{}new\-Seq\char`\"{} OBIT\_\-int (1,1,1) New AIPS Sequence 0 =$>$ unique value \end{itemize}
\item[{\em err}]{\bf Obit\-Err}{\rm (p.\,\pageref{structObitErr})} for reporting errors. \end{description}
\end{Desc}
\index{ObitIO.h@{Obit\-IO.h}!ObitIOReReadMulti@{ObitIOReReadMulti}}
\index{ObitIOReReadMulti@{ObitIOReReadMulti}!ObitIO.h@{Obit\-IO.h}}
\subsubsection{\setlength{\rightskip}{0pt plus 5cm}Obit\-IOCode Obit\-IORe\-Read\-Multi ({\bf olong} {\em n\-Buff}, {\bf Obit\-IO} $\ast$$\ast$ {\em in}, {\bf ofloat} $\ast$$\ast$ {\em data}, {\bf Obit\-Err} $\ast$ {\em err})}\label{ObitIO_8h_a43}


Public: Reread with selection. 

Retreives data read in a previous call to Obit\-IORead\-Multi NOTE: this depends on retreiving the data from the first element in in \begin{Desc}
\item[Parameters:]
\begin{description}
\item[{\em n\-Buff}]Number of buffers to be filled \item[{\em in}]Array of pointers to to object to be read; must all be to same underlying data set but with independent calibration \item[{\em data}]array of pointers to buffers to write results. \item[{\em err}]{\bf Obit\-Err}{\rm (p.\,\pageref{structObitErr})} for reporting errors. \end{description}
\end{Desc}
\begin{Desc}
\item[Returns:]return code, OBIT\_\-IO\_\-OK=$>$ OK \end{Desc}
\index{ObitIO.h@{Obit\-IO.h}!ObitIOReReadMultiSelect@{ObitIOReReadMultiSelect}}
\index{ObitIOReReadMultiSelect@{ObitIOReReadMultiSelect}!ObitIO.h@{Obit\-IO.h}}
\subsubsection{\setlength{\rightskip}{0pt plus 5cm}Obit\-IOCode Obit\-IORe\-Read\-Multi\-Select ({\bf olong} {\em n\-Buff}, {\bf Obit\-IO} $\ast$$\ast$ {\em in}, {\bf ofloat} $\ast$$\ast$ {\em data}, {\bf Obit\-Err} $\ast$ {\em err})}\label{ObitIO_8h_a44}


Public: Reread with selection to multiple buffers. 

Retreives data read in a previous call to Obit\-IORead\-Multi\-Select possibly applying new calibration. \begin{Desc}
\item[Parameters:]
\begin{description}
\item[{\em n\-Buff}]Number of buffers to be filled \item[{\em in}]Array of pointers to to object to be read; must all be to same underlying data set but with independent calibration \item[{\em data}]array of pointers to buffers to write results. \item[{\em err}]{\bf Obit\-Err}{\rm (p.\,\pageref{structObitErr})} for reporting errors. \end{description}
\end{Desc}
\begin{Desc}
\item[Returns:]return code, OBIT\_\-IO\_\-OK=$>$ OK \end{Desc}
\index{ObitIO.h@{Obit\-IO.h}!ObitIOSame@{ObitIOSame}}
\index{ObitIOSame@{ObitIOSame}!ObitIO.h@{Obit\-IO.h}}
\subsubsection{\setlength{\rightskip}{0pt plus 5cm}gboolean Obit\-IOSame ({\bf Obit\-IO} $\ast$ {\em in}, {\bf Obit\-Info\-List} $\ast$ {\em in1}, {\bf Obit\-Info\-List} $\ast$ {\em in2}, {\bf Obit\-Err} $\ast$ {\em err})}\label{ObitIO_8h_a31}


Public: Are underlying structures the same. 

This test is done using values entered into the {\bf Obit\-Info\-List}{\rm (p.\,\pageref{structObitInfoList})} in case the object has not yet been opened. \begin{Desc}
\item[Parameters:]
\begin{description}
\item[{\em in1}]{\bf Obit\-Info\-List}{\rm (p.\,\pageref{structObitInfoList})} for first object to be tested \item[{\em in2}]{\bf Obit\-Info\-List}{\rm (p.\,\pageref{structObitInfoList})} for second object to be tested \item[{\em err}]{\bf Obit\-Err}{\rm (p.\,\pageref{structObitErr})} for reporting errors. \end{description}
\end{Desc}
\begin{Desc}
\item[Returns:]TRUE if to objects have the same underlying structures else FALSE \end{Desc}
\index{ObitIO.h@{Obit\-IO.h}!ObitIOSet@{ObitIOSet}}
\index{ObitIOSet@{ObitIOSet}!ObitIO.h@{Obit\-IO.h}}
\subsubsection{\setlength{\rightskip}{0pt plus 5cm}Obit\-IOCode Obit\-IOSet ({\bf Obit\-IO} $\ast$ {\em in}, {\bf Obit\-Info\-List} $\ast$ {\em info}, {\bf Obit\-Err} $\ast$ {\em err})}\label{ObitIO_8h_a37}


Public: Init I/O. 

\begin{Desc}
\item[Parameters:]
\begin{description}
\item[{\em in}]Pointer to object to be accessed. \item[{\em info}]{\bf Obit\-Info\-List}{\rm (p.\,\pageref{structObitInfoList})} with instructions \item[{\em err}]{\bf Obit\-Err}{\rm (p.\,\pageref{structObitErr})} for reporting errors. \end{description}
\end{Desc}
\begin{Desc}
\item[Returns:]return code, 0=$>$ OK \end{Desc}
\index{ObitIO.h@{Obit\-IO.h}!ObitIOUpdateTables@{ObitIOUpdateTables}}
\index{ObitIOUpdateTables@{ObitIOUpdateTables}!ObitIO.h@{Obit\-IO.h}}
\subsubsection{\setlength{\rightskip}{0pt plus 5cm}Obit\-IOCode Obit\-IOUpdate\-Tables ({\bf Obit\-IO} $\ast$ {\em in}, {\bf Obit\-Info\-List} $\ast$ {\em info}, {\bf Obit\-Err} $\ast$ {\em err})}\label{ObitIO_8h_a54}


Public: Update disk resident tables information. 

\begin{Desc}
\item[Parameters:]
\begin{description}
\item[{\em in}]Pointer to object to be updated. \item[{\em info}]{\bf Obit\-Info\-List}{\rm (p.\,\pageref{structObitInfoList})} of parent object. \item[{\em err}]{\bf Obit\-Err}{\rm (p.\,\pageref{structObitErr})} for reporting errors. \end{description}
\end{Desc}
\begin{Desc}
\item[Returns:]return code, OBIT\_\-IO\_\-OK=$>$ OK \end{Desc}
\index{ObitIO.h@{Obit\-IO.h}!ObitIOWrite@{ObitIOWrite}}
\index{ObitIOWrite@{ObitIOWrite}!ObitIO.h@{Obit\-IO.h}}
\subsubsection{\setlength{\rightskip}{0pt plus 5cm}Obit\-IOCode Obit\-IOWrite ({\bf Obit\-IO} $\ast$ {\em in}, {\bf ofloat} $\ast$ {\em data}, {\bf Obit\-Err} $\ast$ {\em err})}\label{ObitIO_8h_a46}


Public: Write. 

Writes row in-$>$my\-Desc-$>$row + 1; plane in-$>$my\-Desc-$>$plane + 1 \begin{Desc}
\item[Parameters:]
\begin{description}
\item[{\em in}]Pointer to object to be written. \item[{\em data}]pointer to buffer containing input data. \item[{\em err}]{\bf Obit\-Err}{\rm (p.\,\pageref{structObitErr})} for reporting errors. \end{description}
\end{Desc}
\begin{Desc}
\item[Returns:]return code, 0=$>$ OK \end{Desc}
\index{ObitIO.h@{Obit\-IO.h}!ObitIOWriteDescriptor@{ObitIOWriteDescriptor}}
\index{ObitIOWriteDescriptor@{ObitIOWriteDescriptor}!ObitIO.h@{Obit\-IO.h}}
\subsubsection{\setlength{\rightskip}{0pt plus 5cm}Obit\-IOCode Obit\-IOWrite\-Descriptor ({\bf Obit\-IO} $\ast$ {\em in}, {\bf Obit\-Err} $\ast$ {\em err})}\label{ObitIO_8h_a50}


Public: Write Descriptor. 

\begin{Desc}
\item[Parameters:]
\begin{description}
\item[{\em in}]Pointer to object with {\bf Obit\-Image\-Desc}{\rm (p.\,\pageref{structObitImageDesc})} to be written. \item[{\em err}]{\bf Obit\-Err}{\rm (p.\,\pageref{structObitErr})} for reporting errors. \end{description}
\end{Desc}
\begin{Desc}
\item[Returns:]return code, 0=$>$ OK \end{Desc}
\index{ObitIO.h@{Obit\-IO.h}!ObitIOWriteRow@{ObitIOWriteRow}}
\index{ObitIOWriteRow@{ObitIOWriteRow}!ObitIO.h@{Obit\-IO.h}}
\subsubsection{\setlength{\rightskip}{0pt plus 5cm}Obit\-IOCode Obit\-IOWrite\-Row ({\bf Obit\-IO} $\ast$ {\em in}, {\bf olong} {\em rowno}, {\bf ofloat} $\ast$ {\em data}, {\bf Obit\-Err} $\ast$ {\em err})}\label{ObitIO_8h_a47}


Public: Write Row. 

Writes row in-$>$my\-Desc-$>$row + 1; plane in-$>$my\-Desc-$>$plane + 1 \begin{Desc}
\item[Parameters:]
\begin{description}
\item[{\em in}]Pointer to object to be written. \item[{\em rowno}]Starting row number (1-rel) -1=$>$ next. \item[{\em data}]pointer to buffer containing input data. \item[{\em err}]{\bf Obit\-Err}{\rm (p.\,\pageref{structObitErr})} for reporting errors. \end{description}
\end{Desc}
\begin{Desc}
\item[Returns:]return code, 0=$>$ OK \end{Desc}
\index{ObitIO.h@{Obit\-IO.h}!ObitIOZap@{ObitIOZap}}
\index{ObitIOZap@{ObitIOZap}!ObitIO.h@{Obit\-IO.h}}
\subsubsection{\setlength{\rightskip}{0pt plus 5cm}void Obit\-IOZap ({\bf Obit\-IO} $\ast$ {\em in}, {\bf Obit\-Err} $\ast$ {\em err})}\label{ObitIO_8h_a33}


Public: Delete underlying structures. 

\begin{Desc}
\item[Parameters:]
\begin{description}
\item[{\em in}]Pointer to object to be zapped. \item[{\em err}]{\bf Obit\-Err}{\rm (p.\,\pageref{structObitErr})} for reporting errors. \end{description}
\end{Desc}
