\section{Obit\-Parser.h File Reference}
\label{ObitParser_8h}\index{ObitParser.h@{ObitParser.h}}
Input utility parser. 

{\tt \#include \char`\"{}Obit.h\char`\"{}}\par
{\tt \#include \char`\"{}Obit\-File.h\char`\"{}}\par
{\tt \#include \char`\"{}Obit\-Err.h\char`\"{}}\par
{\tt \#include \char`\"{}Obit\-Info\-List.h\char`\"{}}\par
\subsection*{Functions}
\begin{CompactItemize}
\item 
Obit\-IOCode {\bf Obit\-Parser\-Parse} (gchar $\ast$infile, {\bf Obit\-Info\-List} $\ast$list, {\bf Obit\-Err} $\ast$err)
\begin{CompactList}\small\item\em Public: Parse text file to {\bf Obit\-Info\-List}{\rm (p.\,\pageref{structObitInfoList})}. \item\end{CompactList}\end{CompactItemize}


\subsection{Detailed Description}
Input utility parser. 

This file contains utility functions for parsing program input parameters from an input text file and storing in an {\bf Obit\-Info\-List}{\rm (p.\,\pageref{structObitInfoList})}. Order of the entries in the text file is arbitrary. A set of defined value names with default values are given to the parser. Supported types are: \begin{itemize}
\item integers (as oint) \item boolean \item floats (as gdouble) \item character strings\end{itemize}
To use do to following: \begin{itemize}
\item Create an {\bf Obit\-Info\-List}{\rm (p.\,\pageref{structObitInfoList})} and fill with default values \item Use Obit\-Parser\-Parse to read text file and store in {\bf Obit\-Info\-List}{\rm (p.\,\pageref{structObitInfoList})} \item Obtain values from {\bf Obit\-Info\-List}{\rm (p.\,\pageref{structObitInfoList})}\end{itemize}
\subsection{input file}\label{ObitParser_8h_input_file}
The text input file is a free form with keyword=value form. Comments are preceeded by \# and an entire line may be a comment. An entry consists of a header line beginning with \char`\"{}\$Key = \char`\"{}, followed by the name of the entry, a data type code and the dimensionality as a Fortran dimensionality, (n,m) = n x m array with elementson the n axis varying most rapidly. Values start on the second line of an entry and take as many lines as necessary with a list of blank separated values. Character strings are one per line. The type codes are: \begin{itemize}
\item Str String, first element of dimension is number of characters in each string, Note: these are NOT NULL terminated. \item Int Integer as oint. \item Boo Boolean, 'T' = true, 'F' = false \item Flt Floating as gdouble, Example Input file 

\footnotesize\begin{verbatim}
  $Key = stringD Str (48)
  somefile.fits  # string input
  $Key = integerD Int (1)
  13
  $Key = floatD Flt (2)
  5.6789 3.478e5
  $Key = BoolD Boo (1)
  T
 \end{verbatim}
\normalsize
\end{itemize}


\subsection{Function Documentation}
\index{ObitParser.h@{Obit\-Parser.h}!ObitParserParse@{ObitParserParse}}
\index{ObitParserParse@{ObitParserParse}!ObitParser.h@{Obit\-Parser.h}}
\subsubsection{\setlength{\rightskip}{0pt plus 5cm}Obit\-IOCode Obit\-Parser\-Parse (gchar $\ast$ {\em infile}, {\bf Obit\-Info\-List} $\ast$ {\em list}, {\bf Obit\-Err} $\ast$ {\em err})}\label{ObitParser_8h_a0}


Public: Parse text file to {\bf Obit\-Info\-List}{\rm (p.\,\pageref{structObitInfoList})}. 

\begin{Desc}
\item[Parameters:]
\begin{description}
\item[{\em infile}]Name of the input text file to parse \item[{\em list}]{\bf Obit\-Info\-List}{\rm (p.\,\pageref{structObitInfoList})} to accept values. \item[{\em err}]{\bf Obit\-Err}{\rm (p.\,\pageref{structObitErr})} for reporting errors. \end{description}
\end{Desc}
\begin{Desc}
\item[Returns:]return code, OBIT\_\-IO\_\-OK =$>$ OK \end{Desc}
