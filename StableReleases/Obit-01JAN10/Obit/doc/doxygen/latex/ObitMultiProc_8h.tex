\section{Obit\-Multi\-Proc.h File Reference}
\label{ObitMultiProc_8h}\index{ObitMultiProc.h@{ObitMultiProc.h}}
{\bf Obit\-Multi\-Proc}{\rm (p.\,\pageref{structObitMultiProc})} template for classes derived from {\bf Obit}{\rm (p.\,\pageref{structObit})}. 

{\tt \#include \char`\"{}Obit\-RPC.h\char`\"{}}\par
{\tt \#include \char`\"{}Obit.h\char`\"{}}\par
{\tt \#include \char`\"{}Obit\-Err.h\char`\"{}}\par
{\tt \#include \char`\"{}Obit\-Thread.h\char`\"{}}\par
{\tt \#include \char`\"{}Obit\-File.h\char`\"{}}\par
\subsection*{Classes}
\begin{CompactItemize}
\item 
struct {\bf Obit\-Multi\-Proc\-Func\-Arg}
\begin{CompactList}\small\item\em Multi\-Proc function argument. \item\end{CompactList}\item 
struct {\bf Obit\-Multi\-Proc}
\begin{CompactList}\small\item\em Obit\-Multi\-Proc Class structure. \item\end{CompactList}\item 
struct {\bf Obit\-Multi\-Proc\-Class\-Info}
\begin{CompactList}\small\item\em Class\-Info Structure. \item\end{CompactList}\end{CompactItemize}
\subsection*{Defines}
\begin{CompactItemize}
\item 
\#define {\bf Obit\-Multi\-Proc\-Unref}(in)\ Obit\-Unref (in)
\begin{CompactList}\small\item\em Macro to unreference (and possibly destroy) an {\bf Obit\-Multi\-Proc}{\rm (p.\,\pageref{structObitMultiProc})} returns a Obit\-Multi\-Proc$\ast$. \item\end{CompactList}\item 
\#define {\bf Obit\-Multi\-Proc\-Ref}(in)\ Obit\-Ref (in)
\begin{CompactList}\small\item\em Macro to reference (update reference count) an {\bf Obit\-Multi\-Proc}{\rm (p.\,\pageref{structObitMultiProc})}. \item\end{CompactList}\item 
\#define {\bf Obit\-Multi\-Proc\-Is\-A}(in)\ Obit\-Is\-A (in, Obit\-Multi\-Proc\-Get\-Class())
\begin{CompactList}\small\item\em Macro to determine if an object is the member of this or a derived class. \item\end{CompactList}\end{CompactItemize}
\subsection*{Typedefs}
\begin{CompactItemize}
\item 
typedef {\bf Obit\-Info\-List} $\ast$($\ast$ {\bf Obit\-Multi\-Proc\-Func} )({\bf Obit\-Info\-List} $\ast$args, {\bf Obit\-Err} $\ast$err)
\begin{CompactList}\small\item\em Typedef for Multi\-Proc enabled functions. \item\end{CompactList}\item 
typedef {\bf Obit\-Multi\-Proc} $\ast$($\ast$ {\bf Obit\-Multi\-Proc\-Create\-FP} )(gchar $\ast$name, {\bf olong} njobs, gchar $\ast$MPfunc, {\bf Obit\-Multi\-Proc\-Func} local\-Func, {\bf Obit\-Err} $\ast$err)
\begin{CompactList}\small\item\em Typedef for definition of class pointer structure. \item\end{CompactList}\end{CompactItemize}
\subsection*{Functions}
\begin{CompactItemize}
\item 
void {\bf Obit\-Multi\-Proc\-Class\-Init} (void)
\begin{CompactList}\small\item\em Public: Class initializer. \item\end{CompactList}\item 
{\bf Obit\-Multi\-Proc} $\ast$ {\bf new\-Obit\-Multi\-Proc} (gchar $\ast$name)
\begin{CompactList}\small\item\em Public: Default Constructor. \item\end{CompactList}\item 
{\bf Obit\-Multi\-Proc} $\ast$ {\bf Obit\-Multi\-Proc\-Create} (gchar $\ast$name, {\bf olong} njobs, gchar $\ast$MPfunc, {\bf Obit\-Multi\-Proc\-Func} local\-Func, {\bf Obit\-Err} $\ast$err)
\begin{CompactList}\small\item\em Public: Create/initialize {\bf Obit\-Multi\-Proc}{\rm (p.\,\pageref{structObitMultiProc})} structures. \item\end{CompactList}\item 
void {\bf Obit\-Multi\-Proc\-Start} ({\bf Obit\-Info\-List} $\ast$my\-Input, {\bf Obit\-Err} $\ast$err)
\begin{CompactList}\small\item\em Start auxillary Func\-Container processes based on values in Inputs list. \item\end{CompactList}\item 
void {\bf Obit\-Multi\-Proc\-Shutdown} ({\bf Obit\-Err} $\ast$err)
\begin{CompactList}\small\item\em Shutdown auxillary Func\-Container processes. \item\end{CompactList}\item 
void {\bf Obit\-Multi\-Proc\-Set\-Func\-Arg} ({\bf Obit\-Multi\-Proc} $\ast$in, {\bf olong} job\-No, {\bf Obit\-Info\-List} $\ast$arg)
\begin{CompactList}\small\item\em Copies argument values for a given job of threading. \item\end{CompactList}\item 
void {\bf Obit\-Multi\-Proc\-Set\-Exec\-Flag} ({\bf Obit\-Multi\-Proc} $\ast$in, {\bf olong} job\-No, gboolean flag)
\begin{CompactList}\small\item\em Set execution flag for a given job. \item\end{CompactList}\item 
void {\bf Obit\-Multi\-Proc\-Execute} ({\bf Obit\-Multi\-Proc} $\ast$in, {\bf ofloat} timeout, {\bf Obit\-Err} $\ast$err)
\begin{CompactList}\small\item\em Execute selected jobs If the URLs are specified in in-$>$args[$\ast$]-$>$URL, they are used, otherwise the first my\-Class\-Info.n\-Processor are assigned the list of URL in the class global structure. \item\end{CompactList}\item 
{\bf Obit\-Info\-List} $\ast$ {\bf Obit\-Multi\-Proc\-Get\-Func\-Ret} ({\bf Obit\-Multi\-Proc} $\ast$in, {\bf olong} job\-No)
\begin{CompactList}\small\item\em Return reference to job return value {\bf Obit\-Info\-List}{\rm (p.\,\pageref{structObitInfoList})}. \item\end{CompactList}\item 
gconstpointer {\bf Obit\-Multi\-Proc\-Get\-Class} (void)
\begin{CompactList}\small\item\em Public: Class\-Info pointer. \item\end{CompactList}\item 
{\bf Obit\-Multi\-Proc} $\ast$ {\bf Obit\-Multi\-Proc\-Copy} ({\bf Obit\-Multi\-Proc} $\ast$in, {\bf Obit\-Multi\-Proc} $\ast$out, {\bf Obit\-Err} $\ast$err)
\begin{CompactList}\small\item\em Public: Copy (deep) constructor. \item\end{CompactList}\item 
void {\bf Obit\-Multi\-Proc\-Clone} ({\bf Obit\-Multi\-Proc} $\ast$in, {\bf Obit\-Multi\-Proc} $\ast$out, {\bf Obit\-Err} $\ast$err)
\begin{CompactList}\small\item\em Public: Copy structure. \item\end{CompactList}\end{CompactItemize}


\subsection{Detailed Description}
{\bf Obit\-Multi\-Proc}{\rm (p.\,\pageref{structObitMultiProc})} template for classes derived from {\bf Obit}{\rm (p.\,\pageref{structObit})}. 

The {\bf Obit\-Multi\-Proc}{\rm (p.\,\pageref{structObitMultiProc})} class enables parallel processing using multiple asynchronous processes. These processes are started by the master process and execute Func\-Container, an XMLRPC based compute server. If multiple processing is not enabled (or only a single job is to be performed) the execution is sequential and in the main task address space. When the master tasks starts the asynchronous Func\-Container processes, the critical aspects of the environment (e.g. AIPS directories) are passed. Pieces of computing to be be done in a potentially parallel process are referred to as \char`\"{}jobs\char`\"{}.

Function arguments are passed to Func\-Container processes and return values, status and logging is returned by means of Obit\-Info\-Lists. Functions for which parallel executions are enabled have a simple (Obit\-Thread\-Func) interface which is passed a single pointer to a structure containing the detailed arguments. These functions should be callable either locally or by the appropriate RPC function in a Func\-Container process. The Obit\-Info\-Lists contain the parameters of the work to be performed.

Implementation is based on the availibity of GThreads. The asynchronous processes are each started in a thread. Threads are then used to manage the interface between the main task and the auxillary processes.\subsection{Creators and Destructors}\label{ObitMultiProc_8h_ObitMultiProcaccess}
An {\bf Obit\-Multi\-Proc}{\rm (p.\,\pageref{structObitMultiProc})} will usually be created using Obit\-Multi\-Proc\-Create which allows specifying a name for the object as well as other information.

A copy of a pointer to an {\bf Obit\-Multi\-Proc}{\rm (p.\,\pageref{structObitMultiProc})} should always be made using the {\bf Obit\-Multi\-Proc\-Ref}{\rm (p.\,\pageref{ObitMultiProc_8h_a1})} function which updates the reference count in the object. Then whenever freeing an {\bf Obit\-Multi\-Proc}{\rm (p.\,\pageref{structObitMultiProc})} or changing a pointer, the function {\bf Obit\-Multi\-Proc\-Unref}{\rm (p.\,\pageref{ObitMultiProc_8h_a0})} will decrement the reference count and destroy the object when the reference count hits 0. There is no explicit destructor.

\subsection{Define Documentation}
\index{ObitMultiProc.h@{Obit\-Multi\-Proc.h}!ObitMultiProcIsA@{ObitMultiProcIsA}}
\index{ObitMultiProcIsA@{ObitMultiProcIsA}!ObitMultiProc.h@{Obit\-Multi\-Proc.h}}
\subsubsection{\setlength{\rightskip}{0pt plus 5cm}\#define Obit\-Multi\-Proc\-Is\-A(in)\ Obit\-Is\-A (in, Obit\-Multi\-Proc\-Get\-Class())}\label{ObitMultiProc_8h_a2}


Macro to determine if an object is the member of this or a derived class. 

Returns TRUE if a member, else FALSE in = object to reference \index{ObitMultiProc.h@{Obit\-Multi\-Proc.h}!ObitMultiProcRef@{ObitMultiProcRef}}
\index{ObitMultiProcRef@{ObitMultiProcRef}!ObitMultiProc.h@{Obit\-Multi\-Proc.h}}
\subsubsection{\setlength{\rightskip}{0pt plus 5cm}\#define Obit\-Multi\-Proc\-Ref(in)\ Obit\-Ref (in)}\label{ObitMultiProc_8h_a1}


Macro to reference (update reference count) an {\bf Obit\-Multi\-Proc}{\rm (p.\,\pageref{structObitMultiProc})}. 

returns a Obit\-Multi\-Proc$\ast$. in = object to reference \index{ObitMultiProc.h@{Obit\-Multi\-Proc.h}!ObitMultiProcUnref@{ObitMultiProcUnref}}
\index{ObitMultiProcUnref@{ObitMultiProcUnref}!ObitMultiProc.h@{Obit\-Multi\-Proc.h}}
\subsubsection{\setlength{\rightskip}{0pt plus 5cm}\#define Obit\-Multi\-Proc\-Unref(in)\ Obit\-Unref (in)}\label{ObitMultiProc_8h_a0}


Macro to unreference (and possibly destroy) an {\bf Obit\-Multi\-Proc}{\rm (p.\,\pageref{structObitMultiProc})} returns a Obit\-Multi\-Proc$\ast$. 

in = object to unreference 

\subsection{Typedef Documentation}
\index{ObitMultiProc.h@{Obit\-Multi\-Proc.h}!ObitMultiProcCreateFP@{ObitMultiProcCreateFP}}
\index{ObitMultiProcCreateFP@{ObitMultiProcCreateFP}!ObitMultiProc.h@{Obit\-Multi\-Proc.h}}
\subsubsection{\setlength{\rightskip}{0pt plus 5cm}typedef {\bf Obit\-Multi\-Proc}$\ast$($\ast$ {\bf Obit\-Multi\-Proc\-Create\-FP})(gchar $\ast$name, {\bf olong} njobs, gchar $\ast$MPfunc, {\bf Obit\-Multi\-Proc\-Func} local\-Func, {\bf Obit\-Err} $\ast$err)}\label{ObitMultiProc_8h_a4}


Typedef for definition of class pointer structure. 

\index{ObitMultiProc.h@{Obit\-Multi\-Proc.h}!ObitMultiProcFunc@{ObitMultiProcFunc}}
\index{ObitMultiProcFunc@{ObitMultiProcFunc}!ObitMultiProc.h@{Obit\-Multi\-Proc.h}}
\subsubsection{\setlength{\rightskip}{0pt plus 5cm}typedef {\bf Obit\-Info\-List}$\ast$($\ast$ {\bf Obit\-Multi\-Proc\-Func})({\bf Obit\-Info\-List} $\ast$args, {\bf Obit\-Err} $\ast$err)}\label{ObitMultiProc_8h_a3}


Typedef for Multi\-Proc enabled functions. 



\subsection{Function Documentation}
\index{ObitMultiProc.h@{Obit\-Multi\-Proc.h}!newObitMultiProc@{newObitMultiProc}}
\index{newObitMultiProc@{newObitMultiProc}!ObitMultiProc.h@{Obit\-Multi\-Proc.h}}
\subsubsection{\setlength{\rightskip}{0pt plus 5cm}{\bf Obit\-Multi\-Proc}$\ast$ new\-Obit\-Multi\-Proc (gchar $\ast$ {\em name})}\label{ObitMultiProc_8h_a6}


Public: Default Constructor. 

Initializes class if needed on first call. \begin{Desc}
\item[Parameters:]
\begin{description}
\item[{\em name}]An optional name for the object. \end{description}
\end{Desc}
\begin{Desc}
\item[Returns:]the new object. \end{Desc}
\index{ObitMultiProc.h@{Obit\-Multi\-Proc.h}!ObitMultiProcClassInit@{ObitMultiProcClassInit}}
\index{ObitMultiProcClassInit@{ObitMultiProcClassInit}!ObitMultiProc.h@{Obit\-Multi\-Proc.h}}
\subsubsection{\setlength{\rightskip}{0pt plus 5cm}void Obit\-Multi\-Proc\-Class\-Init (void)}\label{ObitMultiProc_8h_a5}


Public: Class initializer. 

\index{ObitMultiProc.h@{Obit\-Multi\-Proc.h}!ObitMultiProcClone@{ObitMultiProcClone}}
\index{ObitMultiProcClone@{ObitMultiProcClone}!ObitMultiProc.h@{Obit\-Multi\-Proc.h}}
\subsubsection{\setlength{\rightskip}{0pt plus 5cm}void Obit\-Multi\-Proc\-Clone ({\bf Obit\-Multi\-Proc} $\ast$ {\em in}, {\bf Obit\-Multi\-Proc} $\ast$ {\em out}, {\bf Obit\-Err} $\ast$ {\em err})}\label{ObitMultiProc_8h_a16}


Public: Copy structure. 

\begin{Desc}
\item[Parameters:]
\begin{description}
\item[{\em in}]The object to copy \item[{\em out}]An existing object pointer for output, must be defined. \item[{\em err}]{\bf Obit}{\rm (p.\,\pageref{structObit})} error stack object. \end{description}
\end{Desc}
\index{ObitMultiProc.h@{Obit\-Multi\-Proc.h}!ObitMultiProcCopy@{ObitMultiProcCopy}}
\index{ObitMultiProcCopy@{ObitMultiProcCopy}!ObitMultiProc.h@{Obit\-Multi\-Proc.h}}
\subsubsection{\setlength{\rightskip}{0pt plus 5cm}{\bf Obit\-Multi\-Proc}$\ast$ Obit\-Multi\-Proc\-Copy ({\bf Obit\-Multi\-Proc} $\ast$ {\em in}, {\bf Obit\-Multi\-Proc} $\ast$ {\em out}, {\bf Obit\-Err} $\ast$ {\em err})}\label{ObitMultiProc_8h_a15}


Public: Copy (deep) constructor. 

\begin{Desc}
\item[Parameters:]
\begin{description}
\item[{\em in}]The object to copy \item[{\em out}]An existing object pointer for output or NULL if none exists. \item[{\em err}]{\bf Obit}{\rm (p.\,\pageref{structObit})} error stack object. \end{description}
\end{Desc}
\begin{Desc}
\item[Returns:]pointer to the new object. \end{Desc}
\index{ObitMultiProc.h@{Obit\-Multi\-Proc.h}!ObitMultiProcCreate@{ObitMultiProcCreate}}
\index{ObitMultiProcCreate@{ObitMultiProcCreate}!ObitMultiProc.h@{Obit\-Multi\-Proc.h}}
\subsubsection{\setlength{\rightskip}{0pt plus 5cm}{\bf Obit\-Multi\-Proc}$\ast$ Obit\-Multi\-Proc\-Create (gchar $\ast$ {\em name}, {\bf olong} {\em njobs}, gchar $\ast$ {\em MPfunc}, {\bf Obit\-Multi\-Proc\-Func} {\em local\-Func}, {\bf Obit\-Err} $\ast$ {\em err})}\label{ObitMultiProc_8h_a7}


Public: Create/initialize {\bf Obit\-Multi\-Proc}{\rm (p.\,\pageref{structObitMultiProc})} structures. 

\begin{Desc}
\item[Parameters:]
\begin{description}
\item[{\em name}]An optional name for the object. \item[{\em njobs}]Number of \char`\"{}jobs\char`\"{} to be processed \item[{\em MPfunc}]Name of remote Func\-Container function \item[{\em local\-Func}]Local pointer to job function which should be the same as that executed in Func\-Container. \end{description}
\end{Desc}
\begin{Desc}
\item[Returns:]the new object. \end{Desc}
\index{ObitMultiProc.h@{Obit\-Multi\-Proc.h}!ObitMultiProcExecute@{ObitMultiProcExecute}}
\index{ObitMultiProcExecute@{ObitMultiProcExecute}!ObitMultiProc.h@{Obit\-Multi\-Proc.h}}
\subsubsection{\setlength{\rightskip}{0pt plus 5cm}void Obit\-Multi\-Proc\-Execute ({\bf Obit\-Multi\-Proc} $\ast$ {\em in}, {\bf ofloat} {\em timeout}, {\bf Obit\-Err} $\ast$ {\em err})}\label{ObitMultiProc_8h_a12}


Execute selected jobs If the URLs are specified in in-$>$args[$\ast$]-$>$URL, they are used, otherwise the first my\-Class\-Info.n\-Processor are assigned the list of URL in the class global structure. 

Can run synchronously or asynchronously if enabled. \begin{Desc}
\item[Parameters:]
\begin{description}
\item[{\em in}]Multi\-Proc object to execute \item[{\em timeout}]Timeout (min) per function call, $<$= 0 -$>$ forever \item[{\em err}]{\bf Obit}{\rm (p.\,\pageref{structObit})} error/message structure to be used for process messages and error handling \end{description}
\end{Desc}
\index{ObitMultiProc.h@{Obit\-Multi\-Proc.h}!ObitMultiProcGetClass@{ObitMultiProcGetClass}}
\index{ObitMultiProcGetClass@{ObitMultiProcGetClass}!ObitMultiProc.h@{Obit\-Multi\-Proc.h}}
\subsubsection{\setlength{\rightskip}{0pt plus 5cm}gconstpointer Obit\-Multi\-Proc\-Get\-Class (void)}\label{ObitMultiProc_8h_a14}


Public: Class\-Info pointer. 

\begin{Desc}
\item[Returns:]pointer to the class structure. \end{Desc}
\index{ObitMultiProc.h@{Obit\-Multi\-Proc.h}!ObitMultiProcGetFuncRet@{ObitMultiProcGetFuncRet}}
\index{ObitMultiProcGetFuncRet@{ObitMultiProcGetFuncRet}!ObitMultiProc.h@{Obit\-Multi\-Proc.h}}
\subsubsection{\setlength{\rightskip}{0pt plus 5cm}{\bf Obit\-Info\-List}$\ast$ Obit\-Multi\-Proc\-Get\-Func\-Ret ({\bf Obit\-Multi\-Proc} $\ast$ {\em in}, {\bf olong} {\em job\-No})}\label{ObitMultiProc_8h_a13}


Return reference to job return value {\bf Obit\-Info\-List}{\rm (p.\,\pageref{structObitInfoList})}. 

\begin{Desc}
\item[Parameters:]
\begin{description}
\item[{\em in}]Multi\-Proc object \item[{\em job\-No}]0-rel job number \end{description}
\end{Desc}
\begin{Desc}
\item[Returns:]pointer to return value {\bf Obit\-Info\-List}{\rm (p.\,\pageref{structObitInfoList})}, NULL on error \end{Desc}
\index{ObitMultiProc.h@{Obit\-Multi\-Proc.h}!ObitMultiProcSetExecFlag@{ObitMultiProcSetExecFlag}}
\index{ObitMultiProcSetExecFlag@{ObitMultiProcSetExecFlag}!ObitMultiProc.h@{Obit\-Multi\-Proc.h}}
\subsubsection{\setlength{\rightskip}{0pt plus 5cm}void Obit\-Multi\-Proc\-Set\-Exec\-Flag ({\bf Obit\-Multi\-Proc} $\ast$ {\em in}, {\bf olong} {\em job\-No}, gboolean {\em flag})}\label{ObitMultiProc_8h_a11}


Set execution flag for a given job. 

\begin{Desc}
\item[Parameters:]
\begin{description}
\item[{\em in}]Multi\-Proc object \item[{\em job\-No}]0-rel job number \item[{\em flag}]If TRUE (default) this this job will be executed by Obit\-Multi\-Proc\-Execute, otherwise, not. \end{description}
\end{Desc}
\index{ObitMultiProc.h@{Obit\-Multi\-Proc.h}!ObitMultiProcSetFuncArg@{ObitMultiProcSetFuncArg}}
\index{ObitMultiProcSetFuncArg@{ObitMultiProcSetFuncArg}!ObitMultiProc.h@{Obit\-Multi\-Proc.h}}
\subsubsection{\setlength{\rightskip}{0pt plus 5cm}void Obit\-Multi\-Proc\-Set\-Func\-Arg ({\bf Obit\-Multi\-Proc} $\ast$ {\em in}, {\bf olong} {\em job\-No}, {\bf Obit\-Info\-List} $\ast$ {\em arg})}\label{ObitMultiProc_8h_a10}


Copies argument values for a given job of threading. 

\begin{Desc}
\item[Parameters:]
\begin{description}
\item[{\em in}]Multi\-Proc object \item[{\em job\-No}]0-rel job number \item[{\em arg}]Argument list to copy \end{description}
\end{Desc}
\index{ObitMultiProc.h@{Obit\-Multi\-Proc.h}!ObitMultiProcShutdown@{ObitMultiProcShutdown}}
\index{ObitMultiProcShutdown@{ObitMultiProcShutdown}!ObitMultiProc.h@{Obit\-Multi\-Proc.h}}
\subsubsection{\setlength{\rightskip}{0pt plus 5cm}void Obit\-Multi\-Proc\-Shutdown ({\bf Obit\-Err} $\ast$ {\em err})}\label{ObitMultiProc_8h_a9}


Shutdown auxillary Func\-Container processes. 

Sequential processing using the {\bf Obit\-Multi\-Proc}{\rm (p.\,\pageref{structObitMultiProc})} will still be supported. \begin{Desc}
\item[Parameters:]
\begin{description}
\item[{\em err}]{\bf Obit}{\rm (p.\,\pageref{structObit})} error stack object. \end{description}
\end{Desc}
\begin{Desc}
\item[Returns:]NULL \end{Desc}
\index{ObitMultiProc.h@{Obit\-Multi\-Proc.h}!ObitMultiProcStart@{ObitMultiProcStart}}
\index{ObitMultiProcStart@{ObitMultiProcStart}!ObitMultiProc.h@{Obit\-Multi\-Proc.h}}
\subsubsection{\setlength{\rightskip}{0pt plus 5cm}void Obit\-Multi\-Proc\-Start ({\bf Obit\-Info\-List} $\ast$ {\em my\-Input}, {\bf Obit\-Err} $\ast$ {\em err})}\label{ObitMultiProc_8h_a8}


Start auxillary Func\-Container processes based on values in Inputs list. 

Starts auxillary executions of Func\-Container and saves the necessary information in class structures where they are available to any {\bf Obit\-Multi\-Proc}{\rm (p.\,\pageref{structObitMultiProc})} instance. \begin{Desc}
\item[Parameters:]
\begin{description}
\item[{\em my\-Input}]Task parameter input list. Significant entries: \begin{itemize}
\item \char`\"{}RPCURL\char`\"{} If present and has more than one nonblank entry and RPCURL has no non-blank entries then a set processes running Func\-Container will be started the on each URL watching each port in the URL.. \item \char`\"{}RPCports\char`\"{} If present and has more than one positive entry and RPCURL has no non-blank entries then a set processes running Func\-Container will be started on localhost watching each port listed. \item \char`\"{}AIPSuser\char`\"{} AIPS user number \item \char`\"{}n\-AIPS\char`\"{} Number of AIPS data directories \item \char`\"{}AIPSdirs\char`\"{} List of AIPS data directories \item \char`\"{}n\-FITS\char`\"{} Number of FITS data directories \item \char`\"{}FITSdirs\char`\"{} List of FITS data directories \item \char`\"{}n\-Threads\char`\"{} Number of threads functions are allowed to use. \end{itemize}
\item[{\em err}]{\bf Obit}{\rm (p.\,\pageref{structObit})} error stack object. \end{description}
\end{Desc}
