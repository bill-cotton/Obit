\section{Obit\-UVImager\-Squint.h File Reference}
\label{ObitUVImagerSquint_8h}\index{ObitUVImagerSquint.h@{ObitUVImagerSquint.h}}
{\bf Obit\-UVImager\-Squint}{\rm (p.\,\pageref{structObitUVImagerSquint})} Class for imaging UV data applying Beam squint corrections. 

{\tt \#include \char`\"{}Obit.h\char`\"{}}\par
{\tt \#include \char`\"{}Obit\-Err.h\char`\"{}}\par
{\tt \#include \char`\"{}Obit\-UV.h\char`\"{}}\par
{\tt \#include \char`\"{}Obit\-Image\-Mosaic.h\char`\"{}}\par
{\tt \#include \char`\"{}Obit\-Table\-NI.h\char`\"{}}\par
{\tt \#include \char`\"{}Obit\-UVImager.h\char`\"{}}\par
\subsection*{Classes}
\begin{CompactItemize}
\item 
struct {\bf Obit\-UVImager\-Squint}
\begin{CompactList}\small\item\em Obit\-UVImager\-Squint Class structure. \item\end{CompactList}\item 
struct {\bf Obit\-UVImager\-Squint\-Class\-Info}
\begin{CompactList}\small\item\em Class\-Info Structure. \item\end{CompactList}\end{CompactItemize}
\subsection*{Defines}
\begin{CompactItemize}
\item 
\#define {\bf Obit\-UVImager\-Squint\-Unref}(in)\ Obit\-Unref (in)
\begin{CompactList}\small\item\em Macro to unreference (and possibly destroy) an {\bf Obit\-UVImager\-Squint}{\rm (p.\,\pageref{structObitUVImagerSquint})} returns a Obit\-UVImager\-Squint$\ast$. \item\end{CompactList}\item 
\#define {\bf Obit\-UVImager\-Squint\-Ref}(in)\ Obit\-Ref (in)
\begin{CompactList}\small\item\em Macro to reference (update reference count) an {\bf Obit\-UVImager\-Squint}{\rm (p.\,\pageref{structObitUVImagerSquint})}. \item\end{CompactList}\item 
\#define {\bf Obit\-UVImager\-Squint\-Is\-A}(in)\ Obit\-Is\-A (in, Obit\-UVImager\-Squint\-Get\-Class())
\begin{CompactList}\small\item\em Macro to determine if an object is the member of this or a derived class. \item\end{CompactList}\end{CompactItemize}
\subsection*{Functions}
\begin{CompactItemize}
\item 
void {\bf Obit\-UVImager\-Squint\-Class\-Init} (void)
\begin{CompactList}\small\item\em Public: Class initializer. \item\end{CompactList}\item 
{\bf Obit\-UVImager\-Squint} $\ast$ {\bf new\-Obit\-UVImager\-Squint} (gchar $\ast$name)
\begin{CompactList}\small\item\em Public: Default Constructor. \item\end{CompactList}\item 
void {\bf Obit\-UVImager\-Squint\-From\-Info} ({\bf Obit\-UVImager} $\ast$out, gchar $\ast$prefix, {\bf Obit\-Info\-List} $\ast$in\-List, {\bf Obit\-Err} $\ast$err)
\begin{CompactList}\small\item\em Public: Init UVImager object from description in an {\bf Obit\-Info\-List}{\rm (p.\,\pageref{structObitInfoList})}. \item\end{CompactList}\item 
gconstpointer {\bf Obit\-UVImager\-Squint\-Get\-Class} (void)
\begin{CompactList}\small\item\em Public: Class\-Info pointer. \item\end{CompactList}\item 
{\bf Obit\-UVImager\-Squint} $\ast$ {\bf Obit\-UVImager\-Squint\-Copy} ({\bf Obit\-UVImager\-Squint} $\ast$in, {\bf Obit\-UVImager\-Squint} $\ast$out, {\bf Obit\-Err} $\ast$err)
\begin{CompactList}\small\item\em Public: Copy (deep) constructor. \item\end{CompactList}\item 
void {\bf Obit\-UVImager\-Squint\-Clone} ({\bf Obit\-UVImager\-Squint} $\ast$in, {\bf Obit\-UVImager\-Squint} $\ast$out, {\bf Obit\-Err} $\ast$err)
\begin{CompactList}\small\item\em Public: Copy structure. \item\end{CompactList}\item 
{\bf Obit\-UVImager\-Squint} $\ast$ {\bf Obit\-UVImager\-Squint\-Create} (gchar $\ast$name, {\bf Obit\-UV} $\ast$uvdata, {\bf Obit\-Err} $\ast$err)
\begin{CompactList}\small\item\em Public: Create/initialize {\bf Obit\-UVImager\-Squint}{\rm (p.\,\pageref{structObitUVImagerSquint})} structures. \item\end{CompactList}\item 
{\bf Obit\-UVImager\-Squint} $\ast$ {\bf Obit\-UVImager\-Squint\-Create2} (gchar $\ast$name, {\bf Obit\-UV} $\ast$uvdata, {\bf Obit\-Image\-Mosaic} $\ast$mosaic, {\bf Obit\-Err} $\ast$err)
\begin{CompactList}\small\item\em Public: Create/initialize {\bf Obit\-UVImager\-Squint}{\rm (p.\,\pageref{structObitUVImagerSquint})} structures given mosaic. \item\end{CompactList}\item 
void {\bf Obit\-UVImager\-Squint\-Weight} ({\bf Obit\-UVImager} $\ast$in, {\bf Obit\-Err} $\ast$err)
\begin{CompactList}\small\item\em Public: Weight data. \item\end{CompactList}\item 
void {\bf Obit\-UVImager\-Squint\-Image} ({\bf Obit\-UVImager} $\ast$in, {\bf olong} field, gboolean do\-Weight, gboolean do\-Beam, gboolean do\-Flatten, {\bf Obit\-Err} $\ast$err)
\begin{CompactList}\small\item\em Public: Form Image. \item\end{CompactList}\item 
void {\bf Obit\-UVImager\-Squint\-Get\-Info} ({\bf Obit\-UVImager} $\ast$in, gchar $\ast$prefix, {\bf Obit\-Info\-List} $\ast$out\-List, {\bf Obit\-Err} $\ast$err)
\begin{CompactList}\small\item\em Public: Extract information about underlying structures to {\bf Obit\-Info\-List}{\rm (p.\,\pageref{structObitInfoList})}. \item\end{CompactList}\end{CompactItemize}


\subsection{Detailed Description}
{\bf Obit\-UVImager\-Squint}{\rm (p.\,\pageref{structObitUVImagerSquint})} Class for imaging UV data applying Beam squint corrections. 

This class is derived from the {\bf Obit\-UVImager}{\rm (p.\,\pageref{structObitUVImager})} class.

This class presents a uniform interface to the UV data imaging routines. These go from an input uv data and allow optional selection, calibration and editing and then conversion into a dirty image allowing a number of processing parameters. The calibration consists of generating an SN table for each mosaic center the calibration NI table and applying it in generating that facet. The result is an Image\-Mosaic which may optionally be flattened into a single plane.\subsection{Creators and Destructors}\label{ObitUVImagerSquint_8h_ObitUVImagerSquintaccess}
An {\bf Obit\-UVImager\-Squint}{\rm (p.\,\pageref{structObitUVImagerSquint})} will usually be created using Obit\-UVImager\-Squint\-Create which allows specifying a name for the object as well as other information.

A copy of a pointer to an {\bf Obit\-UVImager\-Squint}{\rm (p.\,\pageref{structObitUVImagerSquint})} should always be made using the {\bf Obit\-UVImager\-Squint\-Ref}{\rm (p.\,\pageref{ObitUVImagerSquint_8h_a1})} function which updates the reference count in the object. Then whenever freeing an {\bf Obit\-UVImager\-Squint}{\rm (p.\,\pageref{structObitUVImagerSquint})} or changing a pointer, the function {\bf Obit\-UVImager\-Squint\-Unref}{\rm (p.\,\pageref{ObitUVImagerSquint_8h_a0})} will decrement the reference count and destroy the object when the reference count hits 0. There is no explicit destructor.\subsection{Control Parameters}\label{ObitUVImagerSquint_8h_ObitUVImagerSquintparameters}
The imaging control parameters are passed through the info object on the uv data, these control both the output image files and the processing parameters. Output images: \begin{itemize}
\item \char`\"{}Type OBIT\_\-int (1,1,1) Underlying file type, 1=FITS, 2=AIPS \item \char`\"{}Name\char`\"{} OBIT\_\-string (?,1,1) Name of image, used as AIPS name or to derive FITS filename \item \char`\"{}Class\char`\"{} OBIT\_\-string (?,1,1) Root of class, used as AIPS class or to derive FITS filename \item \char`\"{}Seq\char`\"{} OBIT\_\-int (1,1,1) Sequence number \item \char`\"{}Disk\char`\"{} OBIT\_\-int (1,1,1) Disk number for underlying files\end{itemize}
UVData selection/calibration/editing control \begin{itemize}
\item \char`\"{}Max\-Baseline\char`\"{} OBIT\_\-float scalar = maximum baseline length in wavelengths. Default = 1.0e15. Output data not flagged by this criteria. \item \char`\"{}Min\-Baseline\char`\"{} OBIT\_\-float scalar = minimum baseline length in wavelengths. Default = 1.0e15.Output data not flagged by this criteria. \item \char`\"{}Stokes\char`\"{} OBIT\_\-string (4,1,1) Selected output Stokes parameters: \char`\"{}    \char`\"{}=$>$ no translation,\char`\"{}I   \char`\"{},\char`\"{}V   \char`\"{},\char`\"{}Q   \char`\"{}, \char`\"{}U   \char`\"{}, \char`\"{}IQU \char`\"{}, \char`\"{}IQUV\char`\"{}, \char`\"{}IV  \char`\"{}, \char`\"{}RR  \char`\"{}, \char`\"{}LL  \char`\"{}, \char`\"{}RL  \char`\"{}, \char`\"{}LR  \char`\"{}, \char`\"{}HALF\char`\"{} = RR,LL, \char`\"{}FULL\char`\"{}=RR,LL,RL,LR. [default \char`\"{}    \char`\"{}] In the above 'F' can substitute for \char`\"{}formal\char`\"{} 'I' (both RR+LL). \item \char`\"{}BChan\char`\"{} OBIT\_\-int (1,1,1) First spectral channel selected. [def all] \item \char`\"{}EChan\char`\"{} OBIT\_\-int (1,1,1) Highest spectral channel selected. [def all] \item \char`\"{}BIF\char`\"{} OBIT\_\-int (1,1,1) First \char`\"{}IF\char`\"{} selected. [def all] \item \char`\"{}EIF\char`\"{} OBIT\_\-int (1,1,1) Highest \char`\"{}IF\char`\"{} selected. [def all] \item \char`\"{}do\-Pol\char`\"{} OBIT\_\-int (1,1,1) $>$0 -$>$ calibrate polarization. \item \char`\"{}ion\-Ver\char`\"{} OBIT\_\-int (1,1,1) NI table version number, 0-$>$ use highest, def=1 \item \char`\"{}flagver\char`\"{} OBIT\_\-int (1,1,1) Flag table version, 0-$>$ use highest, $<$0-$>$ none \item \char`\"{}BLVer\char`\"{} OBIT\_\-int (1,1,1) BL table version, 0$>$ use highest, $<$0-$>$ none \item \char`\"{}BPVer\char`\"{} OBIT\_\-int (1,1,1) Band pass (BP) table version, 0-$>$ use highest \item \char`\"{}Subarray\char`\"{} OBIT\_\-int (1,1,1) Selected subarray, $<$=0-$>$all [default all] \item \char`\"{}freq\-ID\char`\"{} OBIT\_\-int (1,1,1) Selected Frequency ID, $<$=0-$>$all [default all] \item \char`\"{}time\-Range\char`\"{} OBIT\_\-float (2,1,1) Selected timerange in days. \item \char`\"{}UVRange\char`\"{} OBIT\_\-float (2,1,1) Selected UV range in kilowavelengths. \item \char`\"{}Sources\char`\"{} OBIT\_\-string (?,?,1) Source names selected unless any starts with a '-' in which cse all are deselected (with '-' stripped). \item \char`\"{}Antennas\char`\"{} OBIT\_\-int (?,1,1) a list of selected antenna numbers, if any is negative then the absolute values are used and the specified antennas are deselected. \item \char`\"{}corr\-Type\char`\"{} OBIT\_\-int (1,1,1) Correlation type, 0=cross corr only, 1=both, 2=auto only. \item \char`\"{}do\-Band\char`\"{} OBIT\_\-int (1,1,1) Band pass application type $<$0-$>$ none (1) if = 1 then all the bandpass data for each antenna will be averaged to form a composite bandpass spectrum, this will then be used to correct the data. (2) if = 2 the bandpass spectra nearest in time (in a weighted sense) to the uv data point will be used to correct the data. (3) if = 3 the bandpass data will be interpolated in time using the solution weights to form a composite bandpass spectrum, this interpolated spectrum will then be used to correct the data. (4) if = 4 the bandpass spectra nearest in time (neglecting weights) to the uv data point will be used to correct the data. (5) if = 5 the bandpass data will be interpolated in time ignoring weights to form a composite bandpass spectrum, this interpolated spectrum will then be used to correct the data. \item \char`\"{}Smooth\char`\"{} OBIT\_\-float (3,1,1) specifies the type of spectral smoothing Smooth(1) = type of smoothing to apply: 0 =$>$ no smoothing 1 =$>$ Hanning 2 =$>$ Gaussian 3 =$>$ Boxcar 4 =$>$ Sinc (i.e. sin(x)/x) Smooth(2) = the \char`\"{}diameter\char`\"{} of the function, i.e. width between first nulls of Hanning triangle and sinc function, FWHM of Gaussian, width of Boxcar. Defaults (if $<$ 0.1) are 4, 2, 2 and 3 channels for Smooth(1) = 1 - 4. Smooth(3) = the diameter over which the convolving function has value - in channels. Defaults: 1, 3, 1, 4 times Smooth(2) used when\end{itemize}
Imaging parameters: \begin{itemize}
\item \char`\"{}FOV\char`\"{} OBIT\_\-float (1,1,1) Field of view (deg) for Mosaic If $>$ 0.0 then a mosaic of images will be added to cover this region. Note: these are in addition to the NField fields added by other parameters \item \char`\"{}do\-Full\char`\"{} OBIT\_\-boolean (1,1,1) if TRUE, create full field image to cover FOV [def. FALSE] \item \char`\"{}NField\char`\"{} OBIT\_\-int (1,1,1) Number of fields defined in input, if unspecified derive from data and FOV \item \char`\"{}x\-Cells\char`\"{} OBIT\_\-float (?,1,1) Cell spacing in X (asec) for all images, if unspecified derive from data \item \char`\"{}y\-Cells\char`\"{} OBIT\_\-float (?,1,1) Cell spacing in Y (asec) for all images, if unspecified derive from data \item \char`\"{}nx\char`\"{} OBIT\_\-int (?,1,1) Minimum number of cells in X for NField images if unspecified derive from data \item \char`\"{}ny\char`\"{} OBIT\_\-int (?,1,1) Minimum number of cells in Y for NField images if unspecified derive from data \item \char`\"{}RAShift\char`\"{} OBIT\_\-float (?,1,1) Right ascension shift (AIPS convention) for each field if unspecified derive from FOV and data \item \char`\"{}Dec\-Shift\char`\"{} OBIT\_\-float (?,1,1) Declination for each field if unspecified derive from FOV and data\end{itemize}
Outliers to be added: \begin{itemize}
\item \char`\"{}Catalog\char`\"{} OBIT\_\-string (?,1,1) = AIPSVZ format catalog for defining outliers, 'None'=don't use [default] 'Default' = use default catalog. Assumed in FITSdata disk 1. \item \char`\"{}Outlier\-Dist\char`\"{} OBIT\_\-float (1,1,1) Maximum distance (deg) from center to include outlier fields from Catalog. [default 1 deg] \item \char`\"{}Outlier\-Flux\char`\"{} OBIT\_\-float (1,1,1) Minimum estimated flux density include outlier fields from Catalog. [default 0.1 Jy ] \item \char`\"{}Outlier\-SI\char`\"{} OBIT\_\-float (1,1,1) Spectral index to use to convert catalog flux density to observed frequency. [default = -0.75] \item \char`\"{}Outlier\-Size\char`\"{} OBIT\_\-int (?,1,1) Width of outlier field in pixels. [default 50]\end{itemize}
Weighting parameters on in\-UV: \begin{itemize}
\item \char`\"{}nu\-Grid\char`\"{} OBIT\_\-long (1,1,1) = Number of \char`\"{}U\char`\"{} pixels in weighting grid. [defaults to \char`\"{}nx\char`\"{}] \item \char`\"{}nv\-Grid\char`\"{} OBIT\_\-int (1,1,1) Number of \char`\"{}V\char`\"{} pixels in weighting grid. \item \char`\"{}Wt\-Box\char`\"{} OBIT\_\-int (1,1,1) Size of weighting box in cells [def 1] \item \char`\"{}Wt\-Func\char`\"{} OBIT\_\-int (1,1,1) Weighting convolution function [def. 1] 1=Pill box, 2=linear, 3=exponential, 4=Gaussian if positive, function is of radius, negative in u and v. \item \char`\"{}x\-Cells\char`\"{} OBIT\_\-float (1,1,1) Image cell spacing in X in asec. \item \char`\"{}y\-Cells\char`\"{} OBIT\_\-float (1,1,1) Image cell spacing in Y in asec. \item \char`\"{}UVTaper\char`\"{}OBIT\_\-float (1,1,1) UV taper width in kilowavelengths. [def. no taper]. NB: If the taper is applied her is should not also be applied in the imaging step as the taper will be applied to the output data. \item \char`\"{}Robust\char`\"{} OBIT\_\-float (1,1,1) Briggs robust parameter. [def. 0.0] $<$ -7 -$>$ Pure Uniform weight, $>$7 -$>$ Pure natural weight. Uses AIPS rather than Briggs definition of Robust. \item \char`\"{}Wt\-Power\char`\"{} OBIT\_\-float (1,1,1) Power to raise weights to. [def = 1.0] Note: a power of 0.0 sets all the output weights to 1 as modified by uniform/Tapering weighting. Applied in determinng weights as well as after. \item \char`\"{}prt\-Lv\char`\"{} OBIT\_\-int (1,1,1) message level [def 0] 0=none, 1=summary, higher numbers for diagnostics\end{itemize}


\subsection{Define Documentation}
\index{ObitUVImagerSquint.h@{Obit\-UVImager\-Squint.h}!ObitUVImagerSquintIsA@{ObitUVImagerSquintIsA}}
\index{ObitUVImagerSquintIsA@{ObitUVImagerSquintIsA}!ObitUVImagerSquint.h@{Obit\-UVImager\-Squint.h}}
\subsubsection{\setlength{\rightskip}{0pt plus 5cm}\#define Obit\-UVImager\-Squint\-Is\-A(in)\ Obit\-Is\-A (in, Obit\-UVImager\-Squint\-Get\-Class())}\label{ObitUVImagerSquint_8h_a2}


Macro to determine if an object is the member of this or a derived class. 

Returns TRUE if a member, else FALSE in = object to reference \index{ObitUVImagerSquint.h@{Obit\-UVImager\-Squint.h}!ObitUVImagerSquintRef@{ObitUVImagerSquintRef}}
\index{ObitUVImagerSquintRef@{ObitUVImagerSquintRef}!ObitUVImagerSquint.h@{Obit\-UVImager\-Squint.h}}
\subsubsection{\setlength{\rightskip}{0pt plus 5cm}\#define Obit\-UVImager\-Squint\-Ref(in)\ Obit\-Ref (in)}\label{ObitUVImagerSquint_8h_a1}


Macro to reference (update reference count) an {\bf Obit\-UVImager\-Squint}{\rm (p.\,\pageref{structObitUVImagerSquint})}. 

returns a Obit\-UVImager\-Squint$\ast$. in = object to reference \index{ObitUVImagerSquint.h@{Obit\-UVImager\-Squint.h}!ObitUVImagerSquintUnref@{ObitUVImagerSquintUnref}}
\index{ObitUVImagerSquintUnref@{ObitUVImagerSquintUnref}!ObitUVImagerSquint.h@{Obit\-UVImager\-Squint.h}}
\subsubsection{\setlength{\rightskip}{0pt plus 5cm}\#define Obit\-UVImager\-Squint\-Unref(in)\ Obit\-Unref (in)}\label{ObitUVImagerSquint_8h_a0}


Macro to unreference (and possibly destroy) an {\bf Obit\-UVImager\-Squint}{\rm (p.\,\pageref{structObitUVImagerSquint})} returns a Obit\-UVImager\-Squint$\ast$. 

in = object to unreference 

\subsection{Function Documentation}
\index{ObitUVImagerSquint.h@{Obit\-UVImager\-Squint.h}!newObitUVImagerSquint@{newObitUVImagerSquint}}
\index{newObitUVImagerSquint@{newObitUVImagerSquint}!ObitUVImagerSquint.h@{Obit\-UVImager\-Squint.h}}
\subsubsection{\setlength{\rightskip}{0pt plus 5cm}{\bf Obit\-UVImager\-Squint}$\ast$ new\-Obit\-UVImager\-Squint (gchar $\ast$ {\em name})}\label{ObitUVImagerSquint_8h_a4}


Public: Default Constructor. 

Initializes class if needed on first call. \begin{Desc}
\item[Parameters:]
\begin{description}
\item[{\em name}]An optional name for the object. \end{description}
\end{Desc}
\begin{Desc}
\item[Returns:]the new object. \end{Desc}
\index{ObitUVImagerSquint.h@{Obit\-UVImager\-Squint.h}!ObitUVImagerSquintClassInit@{ObitUVImagerSquintClassInit}}
\index{ObitUVImagerSquintClassInit@{ObitUVImagerSquintClassInit}!ObitUVImagerSquint.h@{Obit\-UVImager\-Squint.h}}
\subsubsection{\setlength{\rightskip}{0pt plus 5cm}void Obit\-UVImager\-Squint\-Class\-Init (void)}\label{ObitUVImagerSquint_8h_a3}


Public: Class initializer. 

\index{ObitUVImagerSquint.h@{Obit\-UVImager\-Squint.h}!ObitUVImagerSquintClone@{ObitUVImagerSquintClone}}
\index{ObitUVImagerSquintClone@{ObitUVImagerSquintClone}!ObitUVImagerSquint.h@{Obit\-UVImager\-Squint.h}}
\subsubsection{\setlength{\rightskip}{0pt plus 5cm}void Obit\-UVImager\-Squint\-Clone ({\bf Obit\-UVImager\-Squint} $\ast$ {\em in}, {\bf Obit\-UVImager\-Squint} $\ast$ {\em out}, {\bf Obit\-Err} $\ast$ {\em err})}\label{ObitUVImagerSquint_8h_a8}


Public: Copy structure. 

\begin{Desc}
\item[Parameters:]
\begin{description}
\item[{\em in}]The object to copy \item[{\em out}]An existing object pointer for output, must be defined. \item[{\em err}]{\bf Obit}{\rm (p.\,\pageref{structObit})} error stack object. \end{description}
\end{Desc}
\index{ObitUVImagerSquint.h@{Obit\-UVImager\-Squint.h}!ObitUVImagerSquintCopy@{ObitUVImagerSquintCopy}}
\index{ObitUVImagerSquintCopy@{ObitUVImagerSquintCopy}!ObitUVImagerSquint.h@{Obit\-UVImager\-Squint.h}}
\subsubsection{\setlength{\rightskip}{0pt plus 5cm}{\bf Obit\-UVImager\-Squint}$\ast$ Obit\-UVImager\-Squint\-Copy ({\bf Obit\-UVImager\-Squint} $\ast$ {\em in}, {\bf Obit\-UVImager\-Squint} $\ast$ {\em out}, {\bf Obit\-Err} $\ast$ {\em err})}\label{ObitUVImagerSquint_8h_a7}


Public: Copy (deep) constructor. 

\begin{Desc}
\item[Parameters:]
\begin{description}
\item[{\em in}]The object to copy \item[{\em out}]An existing object pointer for output or NULL if none exists. \item[{\em err}]{\bf Obit}{\rm (p.\,\pageref{structObit})} error stack object. \end{description}
\end{Desc}
\begin{Desc}
\item[Returns:]pointer to the new object. \end{Desc}
\index{ObitUVImagerSquint.h@{Obit\-UVImager\-Squint.h}!ObitUVImagerSquintCreate@{ObitUVImagerSquintCreate}}
\index{ObitUVImagerSquintCreate@{ObitUVImagerSquintCreate}!ObitUVImagerSquint.h@{Obit\-UVImager\-Squint.h}}
\subsubsection{\setlength{\rightskip}{0pt plus 5cm}{\bf Obit\-UVImager\-Squint}$\ast$ Obit\-UVImager\-Squint\-Create (gchar $\ast$ {\em name}, {\bf Obit\-UV} $\ast$ {\em uvdata}, {\bf Obit\-Err} $\ast$ {\em err})}\label{ObitUVImagerSquint_8h_a9}


Public: Create/initialize {\bf Obit\-UVImager\-Squint}{\rm (p.\,\pageref{structObitUVImagerSquint})} structures. 

The output Image\-Mosaic member is created \begin{Desc}
\item[Parameters:]
\begin{description}
\item[{\em name}]An optional name for the object. \item[{\em uvdata}]{\bf Obit\-UV}{\rm (p.\,\pageref{structObitUV})} object with info member containng the output image specifications and all processing parameters. \item[{\em err}]{\bf Obit}{\rm (p.\,\pageref{structObit})} error stack object. \end{description}
\end{Desc}
\begin{Desc}
\item[Returns:]the new object. \end{Desc}
\index{ObitUVImagerSquint.h@{Obit\-UVImager\-Squint.h}!ObitUVImagerSquintCreate2@{ObitUVImagerSquintCreate2}}
\index{ObitUVImagerSquintCreate2@{ObitUVImagerSquintCreate2}!ObitUVImagerSquint.h@{Obit\-UVImager\-Squint.h}}
\subsubsection{\setlength{\rightskip}{0pt plus 5cm}{\bf Obit\-UVImager\-Squint}$\ast$ Obit\-UVImager\-Squint\-Create2 (gchar $\ast$ {\em name}, {\bf Obit\-UV} $\ast$ {\em uvdata}, {\bf Obit\-Image\-Mosaic} $\ast$ {\em mosaic}, {\bf Obit\-Err} $\ast$ {\em err})}\label{ObitUVImagerSquint_8h_a10}


Public: Create/initialize {\bf Obit\-UVImager\-Squint}{\rm (p.\,\pageref{structObitUVImagerSquint})} structures given mosaic. 

\begin{Desc}
\item[Parameters:]
\begin{description}
\item[{\em name}]An optional name for the object. \item[{\em uvdata}]{\bf Obit\-UV}{\rm (p.\,\pageref{structObitUV})} object with info member containng the output image specifications and all processing parameters. \item[{\em mosaic}]Image\-Mosaic to use \item[{\em err}]{\bf Obit}{\rm (p.\,\pageref{structObit})} error stack object. \end{description}
\end{Desc}
\begin{Desc}
\item[Returns:]the new object. \end{Desc}
\index{ObitUVImagerSquint.h@{Obit\-UVImager\-Squint.h}!ObitUVImagerSquintFromInfo@{ObitUVImagerSquintFromInfo}}
\index{ObitUVImagerSquintFromInfo@{ObitUVImagerSquintFromInfo}!ObitUVImagerSquint.h@{Obit\-UVImager\-Squint.h}}
\subsubsection{\setlength{\rightskip}{0pt plus 5cm}void Obit\-UVImager\-Squint\-From\-Info ({\bf Obit\-UVImager} $\ast$ {\em out}, gchar $\ast$ {\em prefix}, {\bf Obit\-Info\-List} $\ast$ {\em in\-List}, {\bf Obit\-Err} $\ast$ {\em err})}\label{ObitUVImagerSquint_8h_a5}


Public: Init UVImager object from description in an {\bf Obit\-Info\-List}{\rm (p.\,\pageref{structObitInfoList})}. 

Initializes class if needed on first call. \begin{Desc}
\item[Parameters:]
\begin{description}
\item[{\em out}]the new object.to be initialized \item[{\em prefix}]If Non\-Null, string to be added to beginning of in\-List entry name \char`\"{}xxx\char`\"{} in the following \item[{\em in\-List}]Info\-List to extract object information from \begin{itemize}
\item \char`\"{}xxx\-Class\-Type\char`\"{} string UVImager type, \char`\"{}Squint\char`\"{} for this class \item \char`\"{}xxx\-UVData\char`\"{} prefix for uvdata member, entry with value \char`\"{}None\char`\"{} =$>$ doesn't exist \item \char`\"{}xxx\-UVWork\char`\"{} prefix for uvwork member, entry with value \char`\"{}None\char`\"{} =$>$ doesn't exist \item \char`\"{}xxx\-Mosaic\char`\"{} prefix for mosaic member, entry with value \char`\"{}None\char`\"{} =$>$ doesn't exist \end{itemize}
\item[{\em err}]{\bf Obit\-Err}{\rm (p.\,\pageref{structObitErr})} for reporting errors. \end{description}
\end{Desc}
\index{ObitUVImagerSquint.h@{Obit\-UVImager\-Squint.h}!ObitUVImagerSquintGetClass@{ObitUVImagerSquintGetClass}}
\index{ObitUVImagerSquintGetClass@{ObitUVImagerSquintGetClass}!ObitUVImagerSquint.h@{Obit\-UVImager\-Squint.h}}
\subsubsection{\setlength{\rightskip}{0pt plus 5cm}gconstpointer Obit\-UVImager\-Squint\-Get\-Class (void)}\label{ObitUVImagerSquint_8h_a6}


Public: Class\-Info pointer. 

\begin{Desc}
\item[Returns:]pointer to the class structure. \end{Desc}
\index{ObitUVImagerSquint.h@{Obit\-UVImager\-Squint.h}!ObitUVImagerSquintGetInfo@{ObitUVImagerSquintGetInfo}}
\index{ObitUVImagerSquintGetInfo@{ObitUVImagerSquintGetInfo}!ObitUVImagerSquint.h@{Obit\-UVImager\-Squint.h}}
\subsubsection{\setlength{\rightskip}{0pt plus 5cm}void Obit\-UVImager\-Squint\-Get\-Info ({\bf Obit\-UVImager} $\ast$ {\em inn}, gchar $\ast$ {\em prefix}, {\bf Obit\-Info\-List} $\ast$ {\em out\-List}, {\bf Obit\-Err} $\ast$ {\em err})}\label{ObitUVImagerSquint_8h_a13}


Public: Extract information about underlying structures to {\bf Obit\-Info\-List}{\rm (p.\,\pageref{structObitInfoList})}. 

\begin{Desc}
\item[Parameters:]
\begin{description}
\item[{\em in}]Object of interest. \item[{\em prefix}]If Non\-Null, string to be added to beginning of out\-List entry name \char`\"{}xxx\char`\"{} in the following \item[{\em out\-List}]Info\-List to write entries into \begin{itemize}
\item \char`\"{}xxx\-Class\-Type\char`\"{} string UVImager type, \char`\"{}Squint\char`\"{} for this class \item \char`\"{}xxx\-UVData\char`\"{} prefix for uvdata member, entry with value \char`\"{}None\char`\"{} =$>$ doesn't exist \item \char`\"{}xxx\-UVWork\char`\"{} prefix for uvwork member, entry with value \char`\"{}None\char`\"{} =$>$ doesn't exist \item \char`\"{}xxx\-Mosaic\char`\"{} prefix for mosaic member, entry with value \char`\"{}None\char`\"{} =$>$ doesn't exist \end{itemize}
\item[{\em err}]{\bf Obit\-Err}{\rm (p.\,\pageref{structObitErr})} for reporting errors. \end{description}
\end{Desc}
\index{ObitUVImagerSquint.h@{Obit\-UVImager\-Squint.h}!ObitUVImagerSquintImage@{ObitUVImagerSquintImage}}
\index{ObitUVImagerSquintImage@{ObitUVImagerSquintImage}!ObitUVImagerSquint.h@{Obit\-UVImager\-Squint.h}}
\subsubsection{\setlength{\rightskip}{0pt plus 5cm}void Obit\-UVImager\-Squint\-Image ({\bf Obit\-UVImager} $\ast$ {\em in}, {\bf olong} {\em field}, gboolean {\em do\-Weight}, gboolean {\em do\-Beam}, gboolean {\em do\-Flatten}, {\bf Obit\-Err} $\ast$ {\em err})}\label{ObitUVImagerSquint_8h_a12}


Public: Form Image. 

\index{ObitUVImagerSquint.h@{Obit\-UVImager\-Squint.h}!ObitUVImagerSquintWeight@{ObitUVImagerSquintWeight}}
\index{ObitUVImagerSquintWeight@{ObitUVImagerSquintWeight}!ObitUVImagerSquint.h@{Obit\-UVImager\-Squint.h}}
\subsubsection{\setlength{\rightskip}{0pt plus 5cm}void Obit\-UVImager\-Squint\-Weight ({\bf Obit\-UVImager} $\ast$ {\em in}, {\bf Obit\-Err} $\ast$ {\em err})}\label{ObitUVImagerSquint_8h_a11}


Public: Weight data. 

\begin{Desc}
\item[Parameters:]
\begin{description}
\item[{\em in}]The input object \item[{\em err}]{\bf Obit}{\rm (p.\,\pageref{structObit})} error stack object. \end{description}
\end{Desc}
