\section{Obit\-UVSel.h File Reference}
\label{ObitUVSel_8h}\index{ObitUVSel.h@{ObitUVSel.h}}
{\bf Obit\-UVSel}{\rm (p.\,\pageref{structObitUVSel})} {\bf Obit}{\rm (p.\,\pageref{structObit})} uv data selector class definition. 

{\tt \#include \char`\"{}Obit.h\char`\"{}}\par
{\tt \#include \char`\"{}Obit\-UVDesc.h\char`\"{}}\par
{\tt \#include \char`\"{}Obit\-Err.h\char`\"{}}\par
\subsection*{Classes}
\begin{CompactItemize}
\item 
struct {\bf Obit\-UVSel}
\begin{CompactList}\small\item\em Obit\-UVSel Class structure. \item\end{CompactList}\item 
struct {\bf Obit\-UVSel\-Class\-Info}
\begin{CompactList}\small\item\em Class\-Info Structure. \item\end{CompactList}\end{CompactItemize}
\subsection*{Defines}
\begin{CompactItemize}
\item 
\#define {\bf Obit\-UVSel\-Unref}(in)\ Obit\-Unref (in)
\begin{CompactList}\small\item\em Macro to unreference (and possibly destroy) an {\bf Obit\-UVSel}{\rm (p.\,\pageref{structObitUVSel})} returns a Obit\-UVSel$\ast$ (NULL). \item\end{CompactList}\item 
\#define {\bf Obit\-UVSel\-Ref}(in)\ Obit\-Ref (in)
\begin{CompactList}\small\item\em Macro to reference (update reference count) an {\bf Obit\-UVSel}{\rm (p.\,\pageref{structObitUVSel})}. \item\end{CompactList}\item 
\#define {\bf Obit\-UVSel\-Is\-A}(in)\ Obit\-Is\-A (in, Obit\-UVSel\-Get\-Class())
\begin{CompactList}\small\item\em Macro to determine if an object is the member of this or a derived class. \item\end{CompactList}\end{CompactItemize}
\subsection*{Functions}
\begin{CompactItemize}
\item 
void {\bf Obit\-UVSel\-Class\-Init} (void)
\begin{CompactList}\small\item\em Public: Class initializer. \item\end{CompactList}\item 
{\bf Obit\-UVSel} $\ast$ {\bf new\-Obit\-UVSel} (gchar $\ast$name)
\begin{CompactList}\small\item\em Public: Constructor. \item\end{CompactList}\item 
gconstpointer {\bf Obit\-UVSel\-Get\-Class} (void)
\begin{CompactList}\small\item\em Public: Return class pointer. \item\end{CompactList}\item 
{\bf Obit\-UVSel} $\ast$ {\bf Obit\-UVSel\-Copy} ({\bf Obit\-UVSel} $\ast$in, {\bf Obit\-UVSel} $\ast$out, {\bf Obit\-Err} $\ast$err)
\begin{CompactList}\small\item\em Public: Copy UVSel. \item\end{CompactList}\item 
{\bf olong} {\bf Obit\-UVSel\-Buffer\-Size} ({\bf Obit\-UVDesc} $\ast$desc, {\bf Obit\-UVSel} $\ast$sel)
\begin{CompactList}\small\item\em Public: How big a buffer is needed for a data transfer? \item\end{CompactList}\item 
void {\bf Obit\-UVSel\-Default} ({\bf Obit\-UVDesc} $\ast$in, {\bf Obit\-UVSel} $\ast$sel)
\begin{CompactList}\small\item\em Public: Enforces defaults in inaxes, blc, trc. \item\end{CompactList}\item 
void {\bf Obit\-UVSel\-Get\-Desc} ({\bf Obit\-UVDesc} $\ast$in, {\bf Obit\-UVSel} $\ast$sel, {\bf Obit\-UVDesc} $\ast$out, {\bf Obit\-Err} $\ast$err)
\begin{CompactList}\small\item\em Public: Applies selection to a Descriptor for writing. \item\end{CompactList}\item 
void {\bf Obit\-UVSel\-Set\-Desc} ({\bf Obit\-UVDesc} $\ast$in, {\bf Obit\-UVSel} $\ast$sel, {\bf Obit\-UVDesc} $\ast$out, {\bf Obit\-Err} $\ast$err)
\begin{CompactList}\small\item\em Public: Applies selection to a Descriptor for reading. \item\end{CompactList}\item 
void {\bf Obit\-UVSel\-Next\-Init} ({\bf Obit\-UVSel} $\ast$in, {\bf Obit\-UVDesc} $\ast$desc, {\bf Obit\-Err} $\ast$err)
\begin{CompactList}\small\item\em See if an NX table exists and if so initialize it to use in deciding which visibilities to read. \item\end{CompactList}\item 
gboolean {\bf Obit\-UVSel\-Next} ({\bf Obit\-UVSel} $\ast$in, {\bf Obit\-UVDesc} $\ast$desc, {\bf Obit\-Err} $\ast$err)
\begin{CompactList}\small\item\em Uses selector member to decide which visibilities to read next. \item\end{CompactList}\item 
void {\bf Obit\-UVSel\-Shutdown} ({\bf Obit\-UVSel} $\ast$in, {\bf Obit\-Err} $\ast$err)
\begin{CompactList}\small\item\em Close NX table if open . \item\end{CompactList}\item 
void {\bf Obit\-UVSel\-Set\-Sour} ({\bf Obit\-UVSel} $\ast$sel, gpointer in\-Data, {\bf olong} Qual, gchar $\ast$sou\-Code, gchar $\ast$Sources, {\bf olong} lsou, {\bf olong} nsou, {\bf Obit\-Err} $\ast$err)
\begin{CompactList}\small\item\em Set selector for source selection. \item\end{CompactList}\item 
void {\bf Obit\-UVSel\-Set\-Ant} ({\bf Obit\-UVSel} $\ast$sel, {\bf olong} $\ast$Antennas, {\bf olong} nant)
\begin{CompactList}\small\item\em Set selector for antenna selection. \item\end{CompactList}\item 
gboolean {\bf Obit\-UVSel\-Want\-Sour} ({\bf Obit\-UVSel} $\ast$sel, {\bf olong} Sour\-ID)
\begin{CompactList}\small\item\em Determine if a given source is selected. \item\end{CompactList}\item 
gboolean {\bf Obit\-UVSel\-Want\-Ant} ({\bf Obit\-UVSel} $\ast$sel, {\bf olong} ant)
\begin{CompactList}\small\item\em Determine if a given antenna is selected. \item\end{CompactList}\item 
{\bf ofloat} {\bf Obit\-UVSel\-Sub\-Scan} ({\bf Obit\-UVSel} $\ast$sel)
\begin{CompactList}\small\item\em Suggest a length for a sub interval of the current scan such that the scan is evenly divided. \item\end{CompactList}\end{CompactItemize}


\subsection{Detailed Description}
{\bf Obit\-UVSel}{\rm (p.\,\pageref{structObitUVSel})} {\bf Obit}{\rm (p.\,\pageref{structObit})} uv data selector class definition. 

This class is derived from the {\bf Obit}{\rm (p.\,\pageref{structObit})} class.

This contains the descriptions of data selection and calibration.\subsection{Usage}\label{ObitUVSel_8h_ObitUVSelUsage}
Instances can be obtained using the {\bf new\-Obit\-UVSel}{\rm (p.\,\pageref{ObitUVSel_8c_a7})} constructor the {\bf Obit\-UVSel\-Copy}{\rm (p.\,\pageref{ObitUVSel_8c_a9})} copy constructor or a pointer duplicated using the {\bf Obit\-UVSel\-Ref}{\rm (p.\,\pageref{ObitUVSel_8h_a1})} function. When an instance is no longer needed, use the {\bf Obit\-UVSel\-Unref}{\rm (p.\,\pageref{ObitUVSel_8h_a0})} macro to release it.\subsection{Data selection and Calibration}\label{ObitUVSel_8h_ObitUVSelCalibration}
The {\bf Obit\-UVSel}{\rm (p.\,\pageref{structObitUVSel})} member of a {\bf Obit\-UV}{\rm (p.\,\pageref{structObitUV})} is used to pass information the the data selection and calibration routines. This information is stored on the {\bf Obit\-Info\-List}{\rm (p.\,\pageref{structObitInfoList})} of the {\bf Obit\-UV}{\rm (p.\,\pageref{structObitUV})} data before it is opened with access OBIT\_\-IO\_\-Read\-Cal. Subsequent calls to Obit\-UVRead\-Select will apply the data selection and calibration requested. The calibration/selection paramters are described in the following list. \begin{itemize}
\item \char`\"{}do\-Cal\-Select\char`\"{} OBIT\_\-bool (1,1,1) Select/calibrate/edit data? \item \char`\"{}Stokes\char`\"{} OBIT\_\-string (4,1,1) Selected output Stokes parameters: \char`\"{}    \char`\"{}=$>$ no translation,\char`\"{}I   \char`\"{},\char`\"{}V   \char`\"{},\char`\"{}Q   \char`\"{}, \char`\"{}U   \char`\"{}, \char`\"{}IQU \char`\"{}, \char`\"{}IQUV\char`\"{}, \char`\"{}IV  \char`\"{}, \char`\"{}RR  \char`\"{}, \char`\"{}LL  \char`\"{}, \char`\"{}RL  \char`\"{}, \char`\"{}LR  \char`\"{}, \char`\"{}HALF\char`\"{} = RR,LL, \char`\"{}FULL\char`\"{}=RR,LL,RL,LR. [default \char`\"{}    \char`\"{}] In the above 'F' can substitute for \char`\"{}formal\char`\"{} 'I' (both RR+LL). \item \char`\"{}BChan\char`\"{} OBIT\_\-int (1,1,1) First spectral channel selected. [def all] \item \char`\"{}EChan\char`\"{} OBIT\_\-int (1,1,1) Highest spectral channel selected. [def all] \item \char`\"{}BIF\char`\"{} OBIT\_\-int (1,1,1) First \char`\"{}IF\char`\"{} selected. [def all] \item \char`\"{}EIF\char`\"{} OBIT\_\-int (1,1,1) Highest \char`\"{}IF\char`\"{} selected. [def all] \item \char`\"{}do\-Pol\char`\"{} OBIT\_\-int (1,1,1) $>$0 -$>$ calibrate polarization. \item \char`\"{}do\-Calib\char`\"{} OBIT\_\-int (1,1,1) $>$0 -$>$ calibrate, 2=$>$ also calibrate Weights \item \char`\"{}gain\-Use\char`\"{} OBIT\_\-int (1,1,1) SN/CL table version number, 0-$>$ use highest \item \char`\"{}flag\-Ver\char`\"{} OBIT\_\-int (1,1,1) Flag table version, 0-$>$ use highest, $<$0-$>$ none \item \char`\"{}BLVer\char`\"{} OBIT\_\-int (1,1,1) BL table version, 0$>$ use highest, $<$0-$>$ none \item \char`\"{}BPVer\char`\"{} OBIT\_\-int (1,1,1) Band pass (BP) table version, 0-$>$ use highest \item \char`\"{}Subarray\char`\"{} OBIT\_\-int (1,1,1) Selected subarray, $<$=0-$>$all [default all] \item \char`\"{}drop\-Sub\-A\char`\"{} OBIT\_\-bool (1,1,1) Drop subarray info? \item \char`\"{}Freq\-ID\char`\"{} OBIT\_\-int (1,1,1) Selected Frequency ID, $<$=0-$>$all [default all] \item \char`\"{}time\-Range\char`\"{} OBIT\_\-float (2,1,1) Selected timerange in days. \item \char`\"{}time\-Range\char`\"{} OBIT\_\-float (2,1,1) Selected timerange in days. \item \char`\"{}UVRange\char`\"{} OBIT\_\-float (2,1,1) Selected UV range in kilowavelengths. \item \char`\"{}Input\-Avg\-Time\char`\"{} OBIT\_\-float (1,1,1) Input data averaging time (sec). used for fringe rate decorrelation correction. \item \char`\"{}Sources\char`\"{} OBIT\_\-string (?,?,1) Source names selected unless any starts with a '-' in which case all are deselected (with '-' stripped). \item \char`\"{}sou\-Code\char`\"{} OBIT\_\-string (4,1,1) Source Cal code desired, ' ' =$>$ any code selected '$\ast$ ' =$>$ any non blank code (calibrators only) '-CAL' =$>$ blank codes only (no calibrators) \item \char`\"{}Qual\char`\"{} Obit\_\-int (1,1,1) Source qualifier, -1 [default] = any \item \char`\"{}Antennas\char`\"{} OBIT\_\-int (?,1,1) a list of selected antenna numbers, if any is negative then the absolute values are used and the specified antennas are deselected. \item \char`\"{}corr\-Type\char`\"{} OBIT\_\-int (1,1,1) Correlation type, 0=cross corr only, 1=both, 2=auto only. \item \char`\"{}pass\-All\char`\"{} OBIT\_\-bool (1,1,1) If True, pass along all data when selecting/calibration even if it's all flagged, data deselected by time, source, antenna etc. is not passed. \item \char`\"{}do\-Band\char`\"{} OBIT\_\-int (1,1,1) Band pass application type $<$0-$>$ none (1) if = 1 then all the bandpass data for each antenna will be averaged to form a composite bandpass spectrum, this will then be used to correct the data. (2) if = 2 the bandpass spectra nearest in time (in a weighted sense) to the uv data point will be used to correct the data. (3) if = 3 the bandpass data will be interpolated in time using the solution weights to form a composite bandpass spectrum, this interpolated spectrum will then be used to correct the data. (4) if = 4 the bandpass spectra nearest in time (neglecting weights) to the uv data point will be used to correct the data. (5) if = 5 the bandpass data will be interpolated in time ignoring weights to form a composite bandpass spectrum, this interpolated spectrum will then be used to correct the data. \item \char`\"{}Smooth\char`\"{} OBIT\_\-float (3,1,1) specifies the type of spectral smoothing Smooth(1) = type of smoothing to apply: 0 =$>$ no smoothing 1 =$>$ Hanning 2 =$>$ Gaussian 3 =$>$ Boxcar 4 =$>$ Sinc (i.e. sin(x)/x) Smooth(2) = the \char`\"{}diameter\char`\"{} of the function, i.e. width between first nulls of Hanning triangle and sinc function, FWHM of Gaussian, width of Boxcar. Defaults (if $<$ 0.1) are 4, 2, 2 and 3 channels for Smooth(1) = 1 - 4. Smooth(3) = the diameter over which the convolving function has value - in channels. Defaults: 1, 3, 1, 4 times Smooth(2) used when \item \char`\"{}Alpha\char`\"{} OBIT\_\-float (1,1,1) Spectral index to apply \item \char`\"{}Sub\-Scan\-Time\char`\"{} Obit\_\-float scalar [Optional] if given, this is the desired time (days) of a sub scan. This is used by the selector to suggest a value close to this which will evenly divide the current scan. See {\bf Obit\-UVSel\-Sub\-Scan}{\rm (p.\,\pageref{ObitUVSel_8c_a21})} 0 =$>$ Use scan average. This is only useful for Read\-Select operations on indexed Obit\-UVs.\end{itemize}


\subsection{Define Documentation}
\index{ObitUVSel.h@{Obit\-UVSel.h}!ObitUVSelIsA@{ObitUVSelIsA}}
\index{ObitUVSelIsA@{ObitUVSelIsA}!ObitUVSel.h@{Obit\-UVSel.h}}
\subsubsection{\setlength{\rightskip}{0pt plus 5cm}\#define Obit\-UVSel\-Is\-A(in)\ Obit\-Is\-A (in, Obit\-UVSel\-Get\-Class())}\label{ObitUVSel_8h_a2}


Macro to determine if an object is the member of this or a derived class. 

Returns TRUE if a member, else FALSE in = object to reference \index{ObitUVSel.h@{Obit\-UVSel.h}!ObitUVSelRef@{ObitUVSelRef}}
\index{ObitUVSelRef@{ObitUVSelRef}!ObitUVSel.h@{Obit\-UVSel.h}}
\subsubsection{\setlength{\rightskip}{0pt plus 5cm}\#define Obit\-UVSel\-Ref(in)\ Obit\-Ref (in)}\label{ObitUVSel_8h_a1}


Macro to reference (update reference count) an {\bf Obit\-UVSel}{\rm (p.\,\pageref{structObitUVSel})}. 

returns a Obit\-UVSel$\ast$. in = object to reference \index{ObitUVSel.h@{Obit\-UVSel.h}!ObitUVSelUnref@{ObitUVSelUnref}}
\index{ObitUVSelUnref@{ObitUVSelUnref}!ObitUVSel.h@{Obit\-UVSel.h}}
\subsubsection{\setlength{\rightskip}{0pt plus 5cm}\#define Obit\-UVSel\-Unref(in)\ Obit\-Unref (in)}\label{ObitUVSel_8h_a0}


Macro to unreference (and possibly destroy) an {\bf Obit\-UVSel}{\rm (p.\,\pageref{structObitUVSel})} returns a Obit\-UVSel$\ast$ (NULL). 

\begin{itemize}
\item in = object to unreference. \end{itemize}


\subsection{Function Documentation}
\index{ObitUVSel.h@{Obit\-UVSel.h}!newObitUVSel@{newObitUVSel}}
\index{newObitUVSel@{newObitUVSel}!ObitUVSel.h@{Obit\-UVSel.h}}
\subsubsection{\setlength{\rightskip}{0pt plus 5cm}{\bf Obit\-UVSel}$\ast$ new\-Obit\-UVSel (gchar $\ast$ {\em name})}\label{ObitUVSel_8h_a4}


Public: Constructor. 

\begin{Desc}
\item[Returns:]pointer to object created. \end{Desc}
\index{ObitUVSel.h@{Obit\-UVSel.h}!ObitUVSelBufferSize@{ObitUVSelBufferSize}}
\index{ObitUVSelBufferSize@{ObitUVSelBufferSize}!ObitUVSel.h@{Obit\-UVSel.h}}
\subsubsection{\setlength{\rightskip}{0pt plus 5cm}{\bf olong} Obit\-UVSel\-Buffer\-Size ({\bf Obit\-UVDesc} $\ast$ {\em desc}, {\bf Obit\-UVSel} $\ast$ {\em sel})}\label{ObitUVSel_8h_a7}


Public: How big a buffer is needed for a data transfer? 

The buffer is intended for the uncompressed versions of uv data records. \begin{Desc}
\item[Parameters:]
\begin{description}
\item[{\em desc}]Pointer input descriptor. \item[{\em sel}]UV selector. \end{description}
\end{Desc}
\begin{Desc}
\item[Returns:]size in floats needed for I/O. \end{Desc}
\index{ObitUVSel.h@{Obit\-UVSel.h}!ObitUVSelClassInit@{ObitUVSelClassInit}}
\index{ObitUVSelClassInit@{ObitUVSelClassInit}!ObitUVSel.h@{Obit\-UVSel.h}}
\subsubsection{\setlength{\rightskip}{0pt plus 5cm}void Obit\-UVSel\-Class\-Init (void)}\label{ObitUVSel_8h_a3}


Public: Class initializer. 

\index{ObitUVSel.h@{Obit\-UVSel.h}!ObitUVSelCopy@{ObitUVSelCopy}}
\index{ObitUVSelCopy@{ObitUVSelCopy}!ObitUVSel.h@{Obit\-UVSel.h}}
\subsubsection{\setlength{\rightskip}{0pt plus 5cm}{\bf Obit\-UVSel}$\ast$ Obit\-UVSel\-Copy ({\bf Obit\-UVSel} $\ast$ {\em in}, {\bf Obit\-UVSel} $\ast$ {\em out}, {\bf Obit\-Err} $\ast$ {\em err})}\label{ObitUVSel_8h_a6}


Public: Copy UVSel. 

\begin{Desc}
\item[Parameters:]
\begin{description}
\item[{\em in}]Pointer to object to be copied. \item[{\em out}]Pointer to object to be written. If NULL then a new structure is created. \item[{\em err}]{\bf Obit\-Err}{\rm (p.\,\pageref{structObitErr})} error stack \end{description}
\end{Desc}
\begin{Desc}
\item[Returns:]Pointer to new object. \end{Desc}
\index{ObitUVSel.h@{Obit\-UVSel.h}!ObitUVSelDefault@{ObitUVSelDefault}}
\index{ObitUVSelDefault@{ObitUVSelDefault}!ObitUVSel.h@{Obit\-UVSel.h}}
\subsubsection{\setlength{\rightskip}{0pt plus 5cm}void Obit\-UVSel\-Default ({\bf Obit\-UVDesc} $\ast$ {\em in}, {\bf Obit\-UVSel} $\ast$ {\em sel})}\label{ObitUVSel_8h_a8}


Public: Enforces defaults in inaxes, blc, trc. 

\begin{Desc}
\item[Parameters:]
\begin{description}
\item[{\em in}]Pointer to descriptor. \item[{\em sel}]UV selector, output vis descriptor changed if needed. \end{description}
\end{Desc}
\index{ObitUVSel.h@{Obit\-UVSel.h}!ObitUVSelGetClass@{ObitUVSelGetClass}}
\index{ObitUVSelGetClass@{ObitUVSelGetClass}!ObitUVSel.h@{Obit\-UVSel.h}}
\subsubsection{\setlength{\rightskip}{0pt plus 5cm}gconstpointer Obit\-UVSel\-Get\-Class (void)}\label{ObitUVSel_8h_a5}


Public: Return class pointer. 

Initializes class if needed on first call. \begin{Desc}
\item[Returns:]pointer to the class structure. \end{Desc}
\index{ObitUVSel.h@{Obit\-UVSel.h}!ObitUVSelGetDesc@{ObitUVSelGetDesc}}
\index{ObitUVSelGetDesc@{ObitUVSelGetDesc}!ObitUVSel.h@{Obit\-UVSel.h}}
\subsubsection{\setlength{\rightskip}{0pt plus 5cm}void Obit\-UVSel\-Get\-Desc ({\bf Obit\-UVDesc} $\ast$ {\em in}, {\bf Obit\-UVSel} $\ast$ {\em sel}, {\bf Obit\-UVDesc} $\ast$ {\em out}, {\bf Obit\-Err} $\ast$ {\em err})}\label{ObitUVSel_8h_a9}


Public: Applies selection to a Descriptor for writing. 

\begin{Desc}
\item[Parameters:]
\begin{description}
\item[{\em in}]Pointer to input descriptor, this describes the data as they appear in memory. \item[{\em sel}]UV selector, blc, trc members changed if needed. \item[{\em out}]Pointer to output descriptor, describing form on disk. \item[{\em err}]{\bf Obit}{\rm (p.\,\pageref{structObit})} error stack \end{description}
\end{Desc}
\index{ObitUVSel.h@{Obit\-UVSel.h}!ObitUVSelNext@{ObitUVSelNext}}
\index{ObitUVSelNext@{ObitUVSelNext}!ObitUVSel.h@{Obit\-UVSel.h}}
\subsubsection{\setlength{\rightskip}{0pt plus 5cm}gboolean Obit\-UVSel\-Next ({\bf Obit\-UVSel} $\ast$ {\em in}, {\bf Obit\-UVDesc} $\ast$ {\em desc}, {\bf Obit\-Err} $\ast$ {\em err})}\label{ObitUVSel_8h_a12}


Uses selector member to decide which visibilities to read next. 

If do\-Index is TRUE, then visibilities are selected from the NX table. \begin{Desc}
\item[Parameters:]
\begin{description}
\item[{\em in}]Pointer to the object. \item[{\em desc}]UV descriptor from IO where the next visibility to read and the number will be stored. 0 causes an initialization.\&nleft) \item[{\em err}]Error stack \end{description}
\end{Desc}
\begin{Desc}
\item[Returns:]TRUE is finished, else FALSE \end{Desc}
\index{ObitUVSel.h@{Obit\-UVSel.h}!ObitUVSelNextInit@{ObitUVSelNextInit}}
\index{ObitUVSelNextInit@{ObitUVSelNextInit}!ObitUVSel.h@{Obit\-UVSel.h}}
\subsubsection{\setlength{\rightskip}{0pt plus 5cm}void Obit\-UVSel\-Next\-Init ({\bf Obit\-UVSel} $\ast$ {\em in}, {\bf Obit\-UVDesc} $\ast$ {\em desc}, {\bf Obit\-Err} $\ast$ {\em err})}\label{ObitUVSel_8h_a11}


See if an NX table exists and if so initialize it to use in deciding which visibilities to read. 

\begin{Desc}
\item[Parameters:]
\begin{description}
\item[{\em in}]Pointer to the object. \item[{\em desc}]UV descriptor from IO where the next visibility to read and the number will be stored. \item[{\em err}]Error stack \end{description}
\end{Desc}
\begin{Desc}
\item[Returns:]TRUE is finished, else FALSE \end{Desc}
\index{ObitUVSel.h@{Obit\-UVSel.h}!ObitUVSelSetAnt@{ObitUVSelSetAnt}}
\index{ObitUVSelSetAnt@{ObitUVSelSetAnt}!ObitUVSel.h@{Obit\-UVSel.h}}
\subsubsection{\setlength{\rightskip}{0pt plus 5cm}void Obit\-UVSel\-Set\-Ant ({\bf Obit\-UVSel} $\ast$ {\em sel}, {\bf olong} $\ast$ {\em Antennas}, {\bf olong} {\em nant})}\label{ObitUVSel_8h_a15}


Set selector for antenna selection. 

\begin{Desc}
\item[Parameters:]
\begin{description}
\item[{\em sel}]UV selector. \item[{\em Antennas}]List of selected Antennas, NULL or all 0 =$>$ all, zero entries after first non zero are ignored. Any negative values means all named are deselected \item[{\em nant}]Number of entries in Antennas \end{description}
\end{Desc}
\index{ObitUVSel.h@{Obit\-UVSel.h}!ObitUVSelSetDesc@{ObitUVSelSetDesc}}
\index{ObitUVSelSetDesc@{ObitUVSelSetDesc}!ObitUVSel.h@{Obit\-UVSel.h}}
\subsubsection{\setlength{\rightskip}{0pt plus 5cm}void Obit\-UVSel\-Set\-Desc ({\bf Obit\-UVDesc} $\ast$ {\em in}, {\bf Obit\-UVSel} $\ast$ {\em sel}, {\bf Obit\-UVDesc} $\ast$ {\em out}, {\bf Obit\-Err} $\ast$ {\em err})}\label{ObitUVSel_8h_a10}


Public: Applies selection to a Descriptor for reading. 

Note: many operations associated with data selection are done in Obit\-UVCal\-Select\-Init. Also sets previously undefined values on sel. \begin{Desc}
\item[Parameters:]
\begin{description}
\item[{\em in}]Pointer to input descriptor, this describes the data as they appear on disk (possibly compressed). \item[{\em sel}]UV selector, members changed if needed. \item[{\em out}]Pointer to output descriptor, this describes the data after any processing when read, or before any compression on output. \item[{\em err}]{\bf Obit}{\rm (p.\,\pageref{structObit})} error stack \end{description}
\end{Desc}
\index{ObitUVSel.h@{Obit\-UVSel.h}!ObitUVSelSetSour@{ObitUVSelSetSour}}
\index{ObitUVSelSetSour@{ObitUVSelSetSour}!ObitUVSel.h@{Obit\-UVSel.h}}
\subsubsection{\setlength{\rightskip}{0pt plus 5cm}void Obit\-UVSel\-Set\-Sour ({\bf Obit\-UVSel} $\ast$ {\em sel}, gpointer {\em in\-Data}, {\bf olong} {\em Qual}, gchar $\ast$ {\em sou\-Code}, gchar $\ast$ {\em Sources}, {\bf olong} {\em lsou}, {\bf olong} {\em nsou}, {\bf Obit\-Err} $\ast$ {\em err})}\label{ObitUVSel_8h_a14}


Set selector for source selection. 

\begin{Desc}
\item[Parameters:]
\begin{description}
\item[{\em sel}]UV selector. \item[{\em in\-Data}]Associated UV data (as gpointer to avoid recursive definition) \item[{\em Qual}]Source qualifier, -1 =$>$ any \item[{\em sou\-Code}]selection of Source by Calcode,if not specified in Source ' ' =$>$ any calibrator code selected '$\ast$ ' =$>$ any non blank code (cal. only) '-CAL' =$>$ blank codes only (no calibrators) anything else = calibrator code to select. NB: The sou\-Code test is applied in addition to the other tests, i.e. Sources and Qual, in the selection of sources to process \item[{\em Sources}]Selected source names, [0] blank=$>$ any this is passed as a lsou x nsou array of characters \item[{\em lsou}]length of source name in Sources \item[{\em nsou}]maximum number of entries in Sources \item[{\em err}]{\bf Obit}{\rm (p.\,\pageref{structObit})} error/message stack \end{description}
\end{Desc}
\index{ObitUVSel.h@{Obit\-UVSel.h}!ObitUVSelShutdown@{ObitUVSelShutdown}}
\index{ObitUVSelShutdown@{ObitUVSelShutdown}!ObitUVSel.h@{Obit\-UVSel.h}}
\subsubsection{\setlength{\rightskip}{0pt plus 5cm}void Obit\-UVSel\-Shutdown ({\bf Obit\-UVSel} $\ast$ {\em in}, {\bf Obit\-Err} $\ast$ {\em err})}\label{ObitUVSel_8h_a13}


Close NX table if open . 

If do\-Index is TRUE, then visibilities are selected from the NX table. \begin{Desc}
\item[Parameters:]
\begin{description}
\item[{\em in}]Pointer to the Selector. \item[{\em err}]Error stack \end{description}
\end{Desc}
\begin{Desc}
\item[Returns:]TRUE is finished, else FALSE \end{Desc}
\index{ObitUVSel.h@{Obit\-UVSel.h}!ObitUVSelSubScan@{ObitUVSelSubScan}}
\index{ObitUVSelSubScan@{ObitUVSelSubScan}!ObitUVSel.h@{Obit\-UVSel.h}}
\subsubsection{\setlength{\rightskip}{0pt plus 5cm}{\bf ofloat} Obit\-UVSel\-Sub\-Scan ({\bf Obit\-UVSel} $\ast$ {\em sel})}\label{ObitUVSel_8h_a18}


Suggest a length for a sub interval of the current scan such that the scan is evenly divided. 

This is based on the target value Sub\-Scan\-Time. This is only useful for Read\-Select operations on an indexed {\bf Obit\-UV}{\rm (p.\,\pageref{structObitUV})}. \begin{Desc}
\item[Parameters:]
\begin{description}
\item[{\em sel}]UV selector. \end{description}
\end{Desc}
\begin{Desc}
\item[Returns:]suggested subscan length in days; \end{Desc}
\index{ObitUVSel.h@{Obit\-UVSel.h}!ObitUVSelWantAnt@{ObitUVSelWantAnt}}
\index{ObitUVSelWantAnt@{ObitUVSelWantAnt}!ObitUVSel.h@{Obit\-UVSel.h}}
\subsubsection{\setlength{\rightskip}{0pt plus 5cm}gboolean Obit\-UVSel\-Want\-Ant ({\bf Obit\-UVSel} $\ast$ {\em sel}, {\bf olong} {\em ant})}\label{ObitUVSel_8h_a17}


Determine if a given antenna is selected. 

\begin{Desc}
\item[Parameters:]
\begin{description}
\item[{\em sel}]UV selector. \item[{\em ant}]antenna id to test \end{description}
\end{Desc}
\begin{Desc}
\item[Returns:]TRUE if antenna selected. \end{Desc}
\index{ObitUVSel.h@{Obit\-UVSel.h}!ObitUVSelWantSour@{ObitUVSelWantSour}}
\index{ObitUVSelWantSour@{ObitUVSelWantSour}!ObitUVSel.h@{Obit\-UVSel.h}}
\subsubsection{\setlength{\rightskip}{0pt plus 5cm}gboolean Obit\-UVSel\-Want\-Sour ({\bf Obit\-UVSel} $\ast$ {\em sel}, {\bf olong} {\em Sour\-ID})}\label{ObitUVSel_8h_a16}


Determine if a given source is selected. 

\begin{Desc}
\item[Parameters:]
\begin{description}
\item[{\em sel}]UV selector. \item[{\em Sour\-ID}]Source ID to be tested \end{description}
\end{Desc}
\begin{Desc}
\item[Returns:]TRUE if source selected. \end{Desc}
