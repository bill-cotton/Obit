\section{Obit\-IOHistory.h File Reference}
\label{ObitIOHistory_8h}\index{ObitIOHistory.h@{ObitIOHistory.h}}
{\bf Obit\-IOHistory}{\rm (p.\,\pageref{structObitIOHistory})} base class definition. 

{\tt \#include \char`\"{}Obit.h\char`\"{}}\par
{\tt \#include \char`\"{}Obit\-IO.h\char`\"{}}\par
{\tt \#include \char`\"{}Obit\-Err.h\char`\"{}}\par
{\tt \#include \char`\"{}Obit\-Thread.h\char`\"{}}\par
{\tt \#include \char`\"{}Obit\-Info\-List.h\char`\"{}}\par
\subsection*{Classes}
\begin{CompactItemize}
\item 
struct {\bf Obit\-IOHistory}
\begin{CompactList}\small\item\em Obit\-IOHistory Class. \item\end{CompactList}\item 
struct {\bf Obit\-IOHistory\-Class\-Info}
\begin{CompactList}\small\item\em Class\-Info Structure. \item\end{CompactList}\end{CompactItemize}
\subsection*{Defines}
\begin{CompactItemize}
\item 
\#define {\bf Obit\-IOHistory\-Unref}(in)\ Obit\-Unref (in)
\begin{CompactList}\small\item\em Macro to unreference (and possibly destroy) an {\bf Obit\-IOHistory}{\rm (p.\,\pageref{structObitIOHistory})} returns a Obit\-IOHistory$\ast$. \item\end{CompactList}\item 
\#define {\bf Obit\-IOHistory\-Ref}(in)\ Obit\-Ref (in)
\begin{CompactList}\small\item\em Macro to reference (update reference count) an {\bf Obit\-IOHistory}{\rm (p.\,\pageref{structObitIOHistory})}. \item\end{CompactList}\item 
\#define {\bf Obit\-IOHistory\-Is\-A}(in)\ Obit\-Is\-A (in, Obit\-IOHistory\-Get\-Class())
\begin{CompactList}\small\item\em Macro to determine if an object is the member of this or a derived class. \item\end{CompactList}\end{CompactItemize}
\subsection*{Typedefs}
\begin{CompactItemize}
\item 
typedef Obit\-IOCode($\ast$ {\bf Obit\-IOHistory\-Read\-Rec\-FP} )({\bf Obit\-IOHistory} $\ast$in, {\bf olong} recno, gchar $\ast$hi\-Card, {\bf Obit\-Err} $\ast$err)
\item 
typedef Obit\-IOCode($\ast$ {\bf Obit\-IOHistory\-Write\-Rec\-FP} )({\bf Obit\-IOHistory} $\ast$in, {\bf olong} recno, gchar $\ast$hi\-Card, {\bf Obit\-Err} $\ast$err)
\item 
typedef {\bf olong}($\ast$ {\bf Obit\-IOHistory\-Num\-Rec\-FP} )({\bf Obit\-IOHistory} $\ast$in)
\end{CompactItemize}
\subsection*{Functions}
\begin{CompactItemize}
\item 
void {\bf Obit\-IOHistory\-Class\-Init} (void)
\begin{CompactList}\small\item\em Public: Class initializer. \item\end{CompactList}\item 
{\bf Obit\-IOHistory} $\ast$ {\bf new\-Obit\-IOHistory} (gchar $\ast$name, {\bf Obit\-Info\-List} $\ast$info, {\bf Obit\-Err} $\ast$err)
\begin{CompactList}\small\item\em Public: Constructor. \item\end{CompactList}\item 
gconstpointer {\bf Obit\-IOHistory\-Get\-Class} (void)
\begin{CompactList}\small\item\em Public: Class\-Info pointer. \item\end{CompactList}\item 
gboolean {\bf Obit\-IOHistory\-Same} ({\bf Obit\-IOHistory} $\ast$in, {\bf Obit\-Info\-List} $\ast$in1, {\bf Obit\-Info\-List} $\ast$in2, {\bf Obit\-Err} $\ast$err)
\begin{CompactList}\small\item\em Public: Are underlying structures the same. \item\end{CompactList}\item 
void {\bf Obit\-IOHistory\-Zap} ({\bf Obit\-IOHistory} $\ast$in, {\bf Obit\-Err} $\ast$err)
\begin{CompactList}\small\item\em Public: Delete underlying structures. \item\end{CompactList}\item 
{\bf Obit\-IOHistory} $\ast$ {\bf Obit\-IOHistory\-Copy} ({\bf Obit\-IOHistory} $\ast$in, {\bf Obit\-IOHistory} $\ast$out, {\bf Obit\-Err} $\ast$err)
\begin{CompactList}\small\item\em Public: Copy constructor. \item\end{CompactList}\item 
Obit\-IOCode {\bf Obit\-IOHistory\-Open} ({\bf Obit\-IOHistory} $\ast$in, Obit\-IOAccess access, {\bf Obit\-Info\-List} $\ast$info, {\bf Obit\-Err} $\ast$err)
\begin{CompactList}\small\item\em Public: Open. \item\end{CompactList}\item 
Obit\-IOCode {\bf Obit\-IOHistory\-Close} ({\bf Obit\-IOHistory} $\ast$in, {\bf Obit\-Err} $\ast$err)
\begin{CompactList}\small\item\em Public: Close. \item\end{CompactList}\item 
Obit\-IOCode {\bf Obit\-IOHistory\-Set} ({\bf Obit\-IOHistory} $\ast$in, {\bf Obit\-Info\-List} $\ast$info, {\bf Obit\-Err} $\ast$err)
\begin{CompactList}\small\item\em Public: Init I/O. \item\end{CompactList}\item 
Obit\-IOCode {\bf Obit\-IOHistory\-Read\-Rec} ({\bf Obit\-IOHistory} $\ast$in, {\bf olong} recno, gchar $\ast$hi\-Card, {\bf Obit\-Err} $\ast$err)
\begin{CompactList}\small\item\em Public: Read Record. \item\end{CompactList}\item 
Obit\-IOCode {\bf Obit\-IOHistory\-Write\-Rec} ({\bf Obit\-IOHistory} $\ast$in, {\bf olong} recno, gchar $\ast$hi\-Card, {\bf Obit\-Err} $\ast$err)
\begin{CompactList}\small\item\em Public: Write Record. \item\end{CompactList}\item 
Obit\-IOCode {\bf Obit\-IOHistory\-Flush} ({\bf Obit\-IOHistory} $\ast$in, {\bf Obit\-Err} $\ast$err)
\begin{CompactList}\small\item\em Public: Flush. \item\end{CompactList}\item 
Obit\-IOCode {\bf Obit\-IOHistory\-Read\-Descriptor} ({\bf Obit\-IOHistory} $\ast$in, {\bf Obit\-Err} $\ast$err)
\begin{CompactList}\small\item\em Public: Read Descriptor. \item\end{CompactList}\item 
Obit\-IOCode {\bf Obit\-IOHistory\-Write\-Descriptor} ({\bf Obit\-IOHistory} $\ast$in, {\bf Obit\-Err} $\ast$err)
\begin{CompactList}\small\item\em Public: Write Descriptor. \item\end{CompactList}\item 
{\bf olong} {\bf Obit\-IOHistory\-Num\-Rec} ({\bf Obit\-IOHistory} $\ast$in)
\begin{CompactList}\small\item\em Public: number of records. \item\end{CompactList}\end{CompactItemize}


\subsection{Detailed Description}
{\bf Obit\-IOHistory}{\rm (p.\,\pageref{structObitIOHistory})} base class definition. 

This class is derived from the {\bf Obit\-IO}{\rm (p.\,\pageref{structObitIO})} class.

This is a virtual base class and should never be directly instantiated, However, its functions should be called and the correct version will be run. This class is the base for specific History access functions. Derived classes provide an I/O interface to various underlying disk structures. The structure is also defined in Obit\-IOHistory\-Def.h to allow recursive definition in derived classes.\subsection{Usage}\label{ObitIOHistory_8h_ObitIOHistoryUsage}
No instances should be created of this class but the class member functions, given a derived type, will invoke the correct function.

\subsection{Define Documentation}
\index{ObitIOHistory.h@{Obit\-IOHistory.h}!ObitIOHistoryIsA@{ObitIOHistoryIsA}}
\index{ObitIOHistoryIsA@{ObitIOHistoryIsA}!ObitIOHistory.h@{Obit\-IOHistory.h}}
\subsubsection{\setlength{\rightskip}{0pt plus 5cm}\#define Obit\-IOHistory\-Is\-A(in)\ Obit\-Is\-A (in, Obit\-IOHistory\-Get\-Class())}\label{ObitIOHistory_8h_a2}


Macro to determine if an object is the member of this or a derived class. 

Returns TRUE if a member, else FALSE in = object to reference \index{ObitIOHistory.h@{Obit\-IOHistory.h}!ObitIOHistoryRef@{ObitIOHistoryRef}}
\index{ObitIOHistoryRef@{ObitIOHistoryRef}!ObitIOHistory.h@{Obit\-IOHistory.h}}
\subsubsection{\setlength{\rightskip}{0pt plus 5cm}\#define Obit\-IOHistory\-Ref(in)\ Obit\-Ref (in)}\label{ObitIOHistory_8h_a1}


Macro to reference (update reference count) an {\bf Obit\-IOHistory}{\rm (p.\,\pageref{structObitIOHistory})}. 

returns a Obit\-IOHistory$\ast$. in = object to reference \index{ObitIOHistory.h@{Obit\-IOHistory.h}!ObitIOHistoryUnref@{ObitIOHistoryUnref}}
\index{ObitIOHistoryUnref@{ObitIOHistoryUnref}!ObitIOHistory.h@{Obit\-IOHistory.h}}
\subsubsection{\setlength{\rightskip}{0pt plus 5cm}\#define Obit\-IOHistory\-Unref(in)\ Obit\-Unref (in)}\label{ObitIOHistory_8h_a0}


Macro to unreference (and possibly destroy) an {\bf Obit\-IOHistory}{\rm (p.\,\pageref{structObitIOHistory})} returns a Obit\-IOHistory$\ast$. 

in = object to unreference 

\subsection{Typedef Documentation}
\index{ObitIOHistory.h@{Obit\-IOHistory.h}!ObitIOHistoryNumRecFP@{ObitIOHistoryNumRecFP}}
\index{ObitIOHistoryNumRecFP@{ObitIOHistoryNumRecFP}!ObitIOHistory.h@{Obit\-IOHistory.h}}
\subsubsection{\setlength{\rightskip}{0pt plus 5cm}typedef {\bf olong}($\ast$ {\bf Obit\-IOHistory\-Num\-Rec\-FP})({\bf Obit\-IOHistory} $\ast$in)}\label{ObitIOHistory_8h_a5}


\index{ObitIOHistory.h@{Obit\-IOHistory.h}!ObitIOHistoryReadRecFP@{ObitIOHistoryReadRecFP}}
\index{ObitIOHistoryReadRecFP@{ObitIOHistoryReadRecFP}!ObitIOHistory.h@{Obit\-IOHistory.h}}
\subsubsection{\setlength{\rightskip}{0pt plus 5cm}typedef Obit\-IOCode($\ast$ {\bf Obit\-IOHistory\-Read\-Rec\-FP})({\bf Obit\-IOHistory} $\ast$in, {\bf olong} recno, gchar $\ast$hi\-Card, {\bf Obit\-Err} $\ast$err)}\label{ObitIOHistory_8h_a3}


\index{ObitIOHistory.h@{Obit\-IOHistory.h}!ObitIOHistoryWriteRecFP@{ObitIOHistoryWriteRecFP}}
\index{ObitIOHistoryWriteRecFP@{ObitIOHistoryWriteRecFP}!ObitIOHistory.h@{Obit\-IOHistory.h}}
\subsubsection{\setlength{\rightskip}{0pt plus 5cm}typedef Obit\-IOCode($\ast$ {\bf Obit\-IOHistory\-Write\-Rec\-FP})({\bf Obit\-IOHistory} $\ast$in, {\bf olong} recno, gchar $\ast$hi\-Card, {\bf Obit\-Err} $\ast$err)}\label{ObitIOHistory_8h_a4}




\subsection{Function Documentation}
\index{ObitIOHistory.h@{Obit\-IOHistory.h}!newObitIOHistory@{newObitIOHistory}}
\index{newObitIOHistory@{newObitIOHistory}!ObitIOHistory.h@{Obit\-IOHistory.h}}
\subsubsection{\setlength{\rightskip}{0pt plus 5cm}{\bf Obit\-IOHistory}$\ast$ new\-Obit\-IOHistory (gchar $\ast$ {\em name}, {\bf Obit\-Info\-List} $\ast$ {\em info}, {\bf Obit\-Err} $\ast$ {\em err})}\label{ObitIOHistory_8h_a7}


Public: Constructor. 

Initializes class if needed on first call. \begin{Desc}
\item[Parameters:]
\begin{description}
\item[{\em name}]Name [optional] for object \item[{\em info}]Info\-List defining file \item[{\em err}]{\bf Obit\-Err}{\rm (p.\,\pageref{structObitErr})} for reporting errors. \end{description}
\end{Desc}
\begin{Desc}
\item[Returns:]the new object. \end{Desc}
\index{ObitIOHistory.h@{Obit\-IOHistory.h}!ObitIOHistoryClassInit@{ObitIOHistoryClassInit}}
\index{ObitIOHistoryClassInit@{ObitIOHistoryClassInit}!ObitIOHistory.h@{Obit\-IOHistory.h}}
\subsubsection{\setlength{\rightskip}{0pt plus 5cm}void Obit\-IOHistory\-Class\-Init (void)}\label{ObitIOHistory_8h_a6}


Public: Class initializer. 

\index{ObitIOHistory.h@{Obit\-IOHistory.h}!ObitIOHistoryClose@{ObitIOHistoryClose}}
\index{ObitIOHistoryClose@{ObitIOHistoryClose}!ObitIOHistory.h@{Obit\-IOHistory.h}}
\subsubsection{\setlength{\rightskip}{0pt plus 5cm}Obit\-IOCode Obit\-IOHistory\-Close ({\bf Obit\-IOHistory} $\ast$ {\em in}, {\bf Obit\-Err} $\ast$ {\em err})}\label{ObitIOHistory_8h_a13}


Public: Close. 

\begin{Desc}
\item[Parameters:]
\begin{description}
\item[{\em in}]Pointer to object to be closed. \item[{\em err}]{\bf Obit\-Err}{\rm (p.\,\pageref{structObitErr})} for reporting errors. \end{description}
\end{Desc}
\begin{Desc}
\item[Returns:]error code, OBIT\_\-IO\_\-OK=$>$ OK \end{Desc}
\index{ObitIOHistory.h@{Obit\-IOHistory.h}!ObitIOHistoryCopy@{ObitIOHistoryCopy}}
\index{ObitIOHistoryCopy@{ObitIOHistoryCopy}!ObitIOHistory.h@{Obit\-IOHistory.h}}
\subsubsection{\setlength{\rightskip}{0pt plus 5cm}{\bf Obit\-IOHistory}$\ast$ Obit\-IOHistory\-Copy ({\bf Obit\-IOHistory} $\ast$ {\em in}, {\bf Obit\-IOHistory} $\ast$ {\em out}, {\bf Obit\-Err} $\ast$ {\em err})}\label{ObitIOHistory_8h_a11}


Public: Copy constructor. 

The result will have pointers to the more complex members. Parent class members are included but any derived class info is ignored. \begin{Desc}
\item[Parameters:]
\begin{description}
\item[{\em in}]The object to copy \item[{\em out}]An existing object pointer for output or NULL if none exists. \item[{\em err}]{\bf Obit}{\rm (p.\,\pageref{structObit})} error stack object. \end{description}
\end{Desc}
\begin{Desc}
\item[Returns:]pointer to the new object. \end{Desc}
\index{ObitIOHistory.h@{Obit\-IOHistory.h}!ObitIOHistoryFlush@{ObitIOHistoryFlush}}
\index{ObitIOHistoryFlush@{ObitIOHistoryFlush}!ObitIOHistory.h@{Obit\-IOHistory.h}}
\subsubsection{\setlength{\rightskip}{0pt plus 5cm}Obit\-IOCode Obit\-IOHistory\-Flush ({\bf Obit\-IOHistory} $\ast$ {\em in}, {\bf Obit\-Err} $\ast$ {\em err})}\label{ObitIOHistory_8h_a17}


Public: Flush. 

\begin{Desc}
\item[Parameters:]
\begin{description}
\item[{\em in}]Pointer to object to be accessed. \item[{\em err}]{\bf Obit\-Err}{\rm (p.\,\pageref{structObitErr})} for reporting errors. \end{description}
\end{Desc}
\begin{Desc}
\item[Returns:]return code, OBIT\_\-IO\_\-OK=$>$ OK \end{Desc}
\index{ObitIOHistory.h@{Obit\-IOHistory.h}!ObitIOHistoryGetClass@{ObitIOHistoryGetClass}}
\index{ObitIOHistoryGetClass@{ObitIOHistoryGetClass}!ObitIOHistory.h@{Obit\-IOHistory.h}}
\subsubsection{\setlength{\rightskip}{0pt plus 5cm}gconstpointer Obit\-IOHistory\-Get\-Class (void)}\label{ObitIOHistory_8h_a8}


Public: Class\-Info pointer. 

Initializes class if needed on first call. \begin{Desc}
\item[Returns:]pointer to the class structure. \end{Desc}
\index{ObitIOHistory.h@{Obit\-IOHistory.h}!ObitIOHistoryNumRec@{ObitIOHistoryNumRec}}
\index{ObitIOHistoryNumRec@{ObitIOHistoryNumRec}!ObitIOHistory.h@{Obit\-IOHistory.h}}
\subsubsection{\setlength{\rightskip}{0pt plus 5cm}{\bf olong} Obit\-IOHistory\-Num\-Rec ({\bf Obit\-IOHistory} $\ast$ {\em in})}\label{ObitIOHistory_8h_a20}


Public: number of records. 

\begin{Desc}
\item[Parameters:]
\begin{description}
\item[{\em in}]Pointer to open object to be tested \end{description}
\end{Desc}
\begin{Desc}
\item[Returns:]number of records, $<$0 =$>$ problem \end{Desc}
\index{ObitIOHistory.h@{Obit\-IOHistory.h}!ObitIOHistoryOpen@{ObitIOHistoryOpen}}
\index{ObitIOHistoryOpen@{ObitIOHistoryOpen}!ObitIOHistory.h@{Obit\-IOHistory.h}}
\subsubsection{\setlength{\rightskip}{0pt plus 5cm}Obit\-IOCode Obit\-IOHistory\-Open ({\bf Obit\-IOHistory} $\ast$ {\em in}, Obit\-IOAccess {\em access}, {\bf Obit\-Info\-List} $\ast$ {\em info}, {\bf Obit\-Err} $\ast$ {\em err})}\label{ObitIOHistory_8h_a12}


Public: Open. 

The file and selection info member should have been stored in the {\bf Obit\-Info\-List}{\rm (p.\,\pageref{structObitInfoList})} prior to calling. See derived classes for details. \begin{Desc}
\item[Parameters:]
\begin{description}
\item[{\em in}]Pointer to object to be opened. \item[{\em access}]access (OBIT\_\-IO\_\-Read\-Only,OBIT\_\-IO\_\-Read\-Write) \item[{\em info}]{\bf Obit\-Info\-List}{\rm (p.\,\pageref{structObitInfoList})} with instructions for opening \item[{\em err}]{\bf Obit\-Err}{\rm (p.\,\pageref{structObitErr})} for reporting errors. \end{description}
\end{Desc}
\begin{Desc}
\item[Returns:]return code, OBIT\_\-IO\_\-OK=$>$ OK \end{Desc}
\index{ObitIOHistory.h@{Obit\-IOHistory.h}!ObitIOHistoryReadDescriptor@{ObitIOHistoryReadDescriptor}}
\index{ObitIOHistoryReadDescriptor@{ObitIOHistoryReadDescriptor}!ObitIOHistory.h@{Obit\-IOHistory.h}}
\subsubsection{\setlength{\rightskip}{0pt plus 5cm}Obit\-IOCode Obit\-IOHistory\-Read\-Descriptor ({\bf Obit\-IOHistory} $\ast$ {\em in}, {\bf Obit\-Err} $\ast$ {\em err})}\label{ObitIOHistory_8h_a18}


Public: Read Descriptor. 

\begin{Desc}
\item[Parameters:]
\begin{description}
\item[{\em in}]Pointer to object with Obit\-Image\-Descto be read. \item[{\em err}]{\bf Obit\-Err}{\rm (p.\,\pageref{structObitErr})} for reporting errors. \end{description}
\end{Desc}
\begin{Desc}
\item[Returns:]return code, OBIT\_\-IO\_\-OK=$>$ OK \end{Desc}
\index{ObitIOHistory.h@{Obit\-IOHistory.h}!ObitIOHistoryReadRec@{ObitIOHistoryReadRec}}
\index{ObitIOHistoryReadRec@{ObitIOHistoryReadRec}!ObitIOHistory.h@{Obit\-IOHistory.h}}
\subsubsection{\setlength{\rightskip}{0pt plus 5cm}Obit\-IOCode Obit\-IOHistory\-Read\-Rec ({\bf Obit\-IOHistory} $\ast$ {\em in}, {\bf olong} {\em recno}, gchar $\ast$ {\em hi\-Card}, {\bf Obit\-Err} $\ast$ {\em err})}\label{ObitIOHistory_8h_a15}


Public: Read Record. 

\begin{Desc}
\item[Parameters:]
\begin{description}
\item[{\em in}]Pointer to object to be read. \item[{\em recno}]record number (1-rel) -1=$>$ next. \item[{\em hi\-Card}]output history record (70 char) \item[{\em err}]{\bf Obit\-Err}{\rm (p.\,\pageref{structObitErr})} for reporting errors. \end{description}
\end{Desc}
\begin{Desc}
\item[Returns:]return code, OBIT\_\-IO\_\-OK=$>$ OK \end{Desc}
\index{ObitIOHistory.h@{Obit\-IOHistory.h}!ObitIOHistorySame@{ObitIOHistorySame}}
\index{ObitIOHistorySame@{ObitIOHistorySame}!ObitIOHistory.h@{Obit\-IOHistory.h}}
\subsubsection{\setlength{\rightskip}{0pt plus 5cm}gboolean Obit\-IOHistory\-Same ({\bf Obit\-IOHistory} $\ast$ {\em in}, {\bf Obit\-Info\-List} $\ast$ {\em in1}, {\bf Obit\-Info\-List} $\ast$ {\em in2}, {\bf Obit\-Err} $\ast$ {\em err})}\label{ObitIOHistory_8h_a9}


Public: Are underlying structures the same. 

This test is done using values entered into the {\bf Obit\-Info\-List}{\rm (p.\,\pageref{structObitInfoList})} in case the object has not yet been opened. \begin{Desc}
\item[Parameters:]
\begin{description}
\item[{\em in}]{\bf Obit\-IO}{\rm (p.\,\pageref{structObitIO})} for test \item[{\em in1}]{\bf Obit\-Info\-List}{\rm (p.\,\pageref{structObitInfoList})} for first object to be tested \item[{\em in2}]{\bf Obit\-Info\-List}{\rm (p.\,\pageref{structObitInfoList})} for second object to be tested \item[{\em err}]{\bf Obit\-Err}{\rm (p.\,\pageref{structObitErr})} for reporting errors. \end{description}
\end{Desc}
\begin{Desc}
\item[Returns:]TRUE if to objects have the same underlying structures else FALSE \end{Desc}
\index{ObitIOHistory.h@{Obit\-IOHistory.h}!ObitIOHistorySet@{ObitIOHistorySet}}
\index{ObitIOHistorySet@{ObitIOHistorySet}!ObitIOHistory.h@{Obit\-IOHistory.h}}
\subsubsection{\setlength{\rightskip}{0pt plus 5cm}Obit\-IOCode Obit\-IOHistory\-Set ({\bf Obit\-IOHistory} $\ast$ {\em in}, {\bf Obit\-Info\-List} $\ast$ {\em info}, {\bf Obit\-Err} $\ast$ {\em err})}\label{ObitIOHistory_8h_a14}


Public: Init I/O. 

\begin{Desc}
\item[Parameters:]
\begin{description}
\item[{\em in}]Pointer to object to be accessed. \item[{\em info}]{\bf Obit\-Info\-List}{\rm (p.\,\pageref{structObitInfoList})} with instructions \item[{\em err}]{\bf Obit\-Err}{\rm (p.\,\pageref{structObitErr})} for reporting errors. \end{description}
\end{Desc}
\begin{Desc}
\item[Returns:]return code, OBIT\_\-IO\_\-OK=$>$ OK \end{Desc}
\index{ObitIOHistory.h@{Obit\-IOHistory.h}!ObitIOHistoryWriteDescriptor@{ObitIOHistoryWriteDescriptor}}
\index{ObitIOHistoryWriteDescriptor@{ObitIOHistoryWriteDescriptor}!ObitIOHistory.h@{Obit\-IOHistory.h}}
\subsubsection{\setlength{\rightskip}{0pt plus 5cm}Obit\-IOCode Obit\-IOHistory\-Write\-Descriptor ({\bf Obit\-IOHistory} $\ast$ {\em in}, {\bf Obit\-Err} $\ast$ {\em err})}\label{ObitIOHistory_8h_a19}


Public: Write Descriptor. 

\begin{Desc}
\item[Parameters:]
\begin{description}
\item[{\em in}]Pointer to object with {\bf Obit\-Image\-Desc}{\rm (p.\,\pageref{structObitImageDesc})} to be written. \item[{\em err}]{\bf Obit\-Err}{\rm (p.\,\pageref{structObitErr})} for reporting errors. \end{description}
\end{Desc}
\begin{Desc}
\item[Returns:]return code, OBIT\_\-IO\_\-OK=$>$ OK \end{Desc}
\index{ObitIOHistory.h@{Obit\-IOHistory.h}!ObitIOHistoryWriteRec@{ObitIOHistoryWriteRec}}
\index{ObitIOHistoryWriteRec@{ObitIOHistoryWriteRec}!ObitIOHistory.h@{Obit\-IOHistory.h}}
\subsubsection{\setlength{\rightskip}{0pt plus 5cm}Obit\-IOCode Obit\-IOHistory\-Write\-Rec ({\bf Obit\-IOHistory} $\ast$ {\em in}, {\bf olong} {\em recno}, gchar $\ast$ {\em hi\-Card}, {\bf Obit\-Err} $\ast$ {\em err})}\label{ObitIOHistory_8h_a16}


Public: Write Record. 

\begin{Desc}
\item[Parameters:]
\begin{description}
\item[{\em in}]Pointer to object to be written. \item[{\em recno}]Record number (1-rel) -1=$>$ next, overwrites any existing \item[{\em hi\-Card}]input history record (70 char) \item[{\em err}]{\bf Obit\-Err}{\rm (p.\,\pageref{structObitErr})} for reporting errors. \end{description}
\end{Desc}
\begin{Desc}
\item[Returns:]return code, OBIT\_\-IO\_\-OK=$>$ OK \end{Desc}
\index{ObitIOHistory.h@{Obit\-IOHistory.h}!ObitIOHistoryZap@{ObitIOHistoryZap}}
\index{ObitIOHistoryZap@{ObitIOHistoryZap}!ObitIOHistory.h@{Obit\-IOHistory.h}}
\subsubsection{\setlength{\rightskip}{0pt plus 5cm}void Obit\-IOHistory\-Zap ({\bf Obit\-IOHistory} $\ast$ {\em in}, {\bf Obit\-Err} $\ast$ {\em err})}\label{ObitIOHistory_8h_a10}


Public: Delete underlying structures. 

\begin{Desc}
\item[Parameters:]
\begin{description}
\item[{\em in}]Pointer to object to be zapped. \item[{\em err}]{\bf Obit\-Err}{\rm (p.\,\pageref{structObitErr})} for reporting errors. \end{description}
\end{Desc}
