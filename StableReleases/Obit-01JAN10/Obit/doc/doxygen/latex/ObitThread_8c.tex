\section{Obit\-Thread.c File Reference}
\label{ObitThread_8c}\index{ObitThread.c@{ObitThread.c}}
{\bf Obit\-Thread}{\rm (p.\,\pageref{structObitThread})} (for multi threading) class function definitions. 

{\tt \#include $<$string.h$>$}\par
{\tt \#include \char`\"{}Obit\-Thread.h\char`\"{}}\par
\subsection*{Functions}
\begin{CompactItemize}
\item 
{\bf Obit\-Thread} $\ast$ {\bf new\-Obit\-Thread} (void)
\begin{CompactList}\small\item\em Public: Constructor. \item\end{CompactList}\item 
{\bf Obit\-Thread} $\ast$ {\bf free\-Obit\-Thread} ({\bf Obit\-Thread} $\ast$in)
\begin{CompactList}\small\item\em Public: Destructor. \item\end{CompactList}\item 
{\bf Obit\-Thread} $\ast$ {\bf Obit\-Thread\-Copy} ({\bf Obit\-Thread} $\ast$in)
\begin{CompactList}\small\item\em Public: Copy Thread. \item\end{CompactList}\item 
{\bf Obit\-Thread} $\ast$ {\bf Obit\-Thread\-Ref} ({\bf Obit\-Thread} $\ast$in)
\begin{CompactList}\small\item\em Public: Reference Thread pointer. \item\end{CompactList}\item 
{\bf Obit\-Thread} $\ast$ {\bf Obit\-Thread\-Unref} ({\bf Obit\-Thread} $\ast$in)
\begin{CompactList}\small\item\em Public: Unreference Thread pointer. \item\end{CompactList}\item 
void {\bf Obit\-Thread\-Lock} ({\bf Obit\-Thread} $\ast$in)
\begin{CompactList}\small\item\em Public: Lock (mutex) object. \item\end{CompactList}\item 
gboolean {\bf Obit\-Thread\-Try\-Lock} ({\bf Obit\-Thread} $\ast$in)
\begin{CompactList}\small\item\em Public: Test Lock (mutex) object. \item\end{CompactList}\item 
void {\bf Obit\-Thread\-Unlock} ({\bf Obit\-Thread} $\ast$in)
\begin{CompactList}\small\item\em Public: Unlock (mutex) object. \item\end{CompactList}\item 
void {\bf Obit\-Thread\-RWRead\-Lock} ({\bf Obit\-Thread} $\ast$in)
\begin{CompactList}\small\item\em Public: Lock (RWLock) object for reader. \item\end{CompactList}\item 
gboolean {\bf Obit\-Thread\-RWRead\-Try\-Lock} ({\bf Obit\-Thread} $\ast$in)
\begin{CompactList}\small\item\em Public: Test Lock (RWLock) object for reader. \item\end{CompactList}\item 
void {\bf Obit\-Thread\-RWRead\-Unlock} ({\bf Obit\-Thread} $\ast$in)
\begin{CompactList}\small\item\em Public: Unlock (RWLock) object for reader. \item\end{CompactList}\item 
void {\bf Obit\-Thread\-RWWrite\-Lock} ({\bf Obit\-Thread} $\ast$in)
\begin{CompactList}\small\item\em Public: Lock (RWLock) object for writer. \item\end{CompactList}\item 
gboolean {\bf Obit\-Thread\-RWWrite\-Try\-Lock} ({\bf Obit\-Thread} $\ast$in)
\begin{CompactList}\small\item\em Public: Test Lock (RWLock) object for writer. \item\end{CompactList}\item 
void {\bf Obit\-Thread\-RWWrite\-Unlock} ({\bf Obit\-Thread} $\ast$in)
\begin{CompactList}\small\item\em Public: Unlock (RWLock) object for writer. \item\end{CompactList}\item 
void {\bf Obit\-Thread\-Join} ({\bf Obit\-Thread} $\ast$in)
\begin{CompactList}\small\item\em Public: Join thread. \item\end{CompactList}\item 
gboolean {\bf Obit\-Thread\-Is\-A} ({\bf Obit\-Thread} $\ast$in)
\begin{CompactList}\small\item\em Public: Returns true if input is a {\bf Obit\-Thread}{\rm (p.\,\pageref{structObitThread})}. \item\end{CompactList}\item 
gboolean {\bf Obit\-Thread\-Have\-Threads} ({\bf Obit\-Thread} $\ast$in)
\begin{CompactList}\small\item\em Public: Returns true if Threads are enabled. \item\end{CompactList}\item 
void {\bf Obit\-Thread\-Init} ({\bf Obit\-Info\-List} $\ast$my\-Input)
\begin{CompactList}\small\item\em Public: Initialize Threading. \item\end{CompactList}\item 
void {\bf Obit\-Thread\-Allow\-Threads} ({\bf Obit\-Thread} $\ast$in, {\bf olong} n\-Threads)
\begin{CompactList}\small\item\em Public: Allows Threads and sets number of processors which can be multithreaded. \item\end{CompactList}\item 
{\bf olong} {\bf Obit\-Thread\-Num\-Proc} ({\bf Obit\-Thread} $\ast$in)
\begin{CompactList}\small\item\em Public: Returns number of processors which can be multithreaded. \item\end{CompactList}\item 
void {\bf Obit\-Thread\-Pool\-Init} ({\bf Obit\-Thread} $\ast$in, {\bf olong} nthreads, {\bf Obit\-Thread\-Func} func, gpointer $\ast$$\ast$args)
\begin{CompactList}\small\item\em Public: Initializes Thread Pool. \item\end{CompactList}\item 
gboolean {\bf Obit\-Thread\-Iterator} ({\bf Obit\-Thread} $\ast$in, {\bf olong} nthreads, {\bf Obit\-Thread\-Func} func, gpointer $\ast$$\ast$args)
\begin{CompactList}\small\item\em Public: Runs multiple copies of a function in different threads. \item\end{CompactList}\item 
void {\bf Obit\-Thread\-Pool\-Done} ({\bf Obit\-Thread} $\ast$in, gpointer arg)
\begin{CompactList}\small\item\em Public: Indicates that a thread function is done. \item\end{CompactList}\item 
void {\bf Obit\-Thread\-Pool\-Free} ({\bf Obit\-Thread} $\ast$in)
\begin{CompactList}\small\item\em Public: Shuts down Thread Pool. \item\end{CompactList}\item 
void {\bf Obit\-Thread\-Start1} ({\bf Obit\-Thread} $\ast$in, {\bf Obit\-Thread\-Func} func, gpointer args)
\begin{CompactList}\small\item\em Public: Starts single thread. \item\end{CompactList}\item 
gpointer {\bf Obit\-Thread\-Join1} ({\bf Obit\-Thread} $\ast$in)
\begin{CompactList}\small\item\em Public: Waits for single thread. \item\end{CompactList}\item 
void {\bf Obit\-Thread\-Queue\-Init} ({\bf Obit\-Thread} $\ast$in)
\begin{CompactList}\small\item\em Public: Initializes Thread message queue. \item\end{CompactList}\item 
gpointer {\bf Obit\-Thread\-Queue\-Check} ({\bf Obit\-Thread} $\ast$in, {\bf olong} add\_\-time)
\begin{CompactList}\small\item\em Public: Check for messages in queue. \item\end{CompactList}\item 
void {\bf Obit\-Thread\-Queue\-Free} ({\bf Obit\-Thread} $\ast$in)
\begin{CompactList}\small\item\em Public: Shuts down message queue. \item\end{CompactList}\end{CompactItemize}


\subsection{Detailed Description}
{\bf Obit\-Thread}{\rm (p.\,\pageref{structObitThread})} (for multi threading) class function definitions. 



\subsection{Function Documentation}
\index{ObitThread.c@{Obit\-Thread.c}!freeObitThread@{freeObitThread}}
\index{freeObitThread@{freeObitThread}!ObitThread.c@{Obit\-Thread.c}}
\subsubsection{\setlength{\rightskip}{0pt plus 5cm}{\bf Obit\-Thread}$\ast$ free\-Obit\-Thread ({\bf Obit\-Thread} $\ast$ {\em in})}\label{ObitThread_8c_a3}


Public: Destructor. 

\begin{Desc}
\item[Parameters:]
\begin{description}
\item[{\em in}]Pointer to object to be destroyed. \end{description}
\end{Desc}
\begin{Desc}
\item[Returns:]Null pointer. \end{Desc}
\index{ObitThread.c@{Obit\-Thread.c}!newObitThread@{newObitThread}}
\index{newObitThread@{newObitThread}!ObitThread.c@{Obit\-Thread.c}}
\subsubsection{\setlength{\rightskip}{0pt plus 5cm}{\bf Obit\-Thread}$\ast$ new\-Obit\-Thread (void)}\label{ObitThread_8c_a2}


Public: Constructor. 

\begin{Desc}
\item[Returns:]pointer to object created. \end{Desc}
\index{ObitThread.c@{Obit\-Thread.c}!ObitThreadAllowThreads@{ObitThreadAllowThreads}}
\index{ObitThreadAllowThreads@{ObitThreadAllowThreads}!ObitThread.c@{Obit\-Thread.c}}
\subsubsection{\setlength{\rightskip}{0pt plus 5cm}void Obit\-Thread\-Allow\-Threads ({\bf Obit\-Thread} $\ast$ {\em in}, {\bf olong} {\em n\-Threads})}\label{ObitThread_8c_a20}


Public: Allows Threads and sets number of processors which can be multithreaded. 

No change is made if compilations did not have OBIT\_\-THREADS\_\-ENABLED \begin{Desc}
\item[Parameters:]
\begin{description}
\item[{\em in}]Pointer to a {\bf Obit\-Thread}{\rm (p.\,\pageref{structObitThread})} object. \item[{\em n\-Proc}]Number of threads \end{description}
\end{Desc}
\index{ObitThread.c@{Obit\-Thread.c}!ObitThreadCopy@{ObitThreadCopy}}
\index{ObitThreadCopy@{ObitThreadCopy}!ObitThread.c@{Obit\-Thread.c}}
\subsubsection{\setlength{\rightskip}{0pt plus 5cm}{\bf Obit\-Thread}$\ast$ Obit\-Thread\-Copy ({\bf Obit\-Thread} $\ast$ {\em in})}\label{ObitThread_8c_a4}


Public: Copy Thread. 

\begin{Desc}
\item[Parameters:]
\begin{description}
\item[{\em in}]Pointer to object to be copied. \end{description}
\end{Desc}
\begin{Desc}
\item[Returns:]Pointer to new object. \end{Desc}
\index{ObitThread.c@{Obit\-Thread.c}!ObitThreadHaveThreads@{ObitThreadHaveThreads}}
\index{ObitThreadHaveThreads@{ObitThreadHaveThreads}!ObitThread.c@{Obit\-Thread.c}}
\subsubsection{\setlength{\rightskip}{0pt plus 5cm}gboolean Obit\-Thread\-Have\-Threads ({\bf Obit\-Thread} $\ast$ {\em in})}\label{ObitThread_8c_a18}


Public: Returns true if Threads are enabled. 

\begin{Desc}
\item[Parameters:]
\begin{description}
\item[{\em in}]Pointer to a thread object. \end{description}
\end{Desc}
\begin{Desc}
\item[Returns:]TRUE if threads are enabled \end{Desc}
\index{ObitThread.c@{Obit\-Thread.c}!ObitThreadInit@{ObitThreadInit}}
\index{ObitThreadInit@{ObitThreadInit}!ObitThread.c@{Obit\-Thread.c}}
\subsubsection{\setlength{\rightskip}{0pt plus 5cm}void Obit\-Thread\-Init ({\bf Obit\-Info\-List} $\ast$ {\em my\-Input})}\label{ObitThread_8c_a19}


Public: Initialize Threading. 

No change is made if compilations did not have OBIT\_\-THREADS\_\-ENABLED \begin{Desc}
\item[Parameters:]
\begin{description}
\item[{\em my\-Input}]an {\bf Obit\-Info\-List}{\rm (p.\,\pageref{structObitInfoList})} possible containing \begin{itemize}
\item \char`\"{}n\-Threads\char`\"{} OBIT\_\-long (1,1,1) Number of threads to attempt per pool. \end{itemize}
\end{description}
\end{Desc}
\index{ObitThread.c@{Obit\-Thread.c}!ObitThreadIsA@{ObitThreadIsA}}
\index{ObitThreadIsA@{ObitThreadIsA}!ObitThread.c@{Obit\-Thread.c}}
\subsubsection{\setlength{\rightskip}{0pt plus 5cm}gboolean Obit\-Thread\-Is\-A ({\bf Obit\-Thread} $\ast$ {\em in})}\label{ObitThread_8c_a17}


Public: Returns true if input is a {\bf Obit\-Thread}{\rm (p.\,\pageref{structObitThread})}. 

\begin{Desc}
\item[Parameters:]
\begin{description}
\item[{\em in}]Pointer to object to test. \end{description}
\end{Desc}
\begin{Desc}
\item[Returns:]TRUE if member else FALSE. \end{Desc}
\index{ObitThread.c@{Obit\-Thread.c}!ObitThreadIterator@{ObitThreadIterator}}
\index{ObitThreadIterator@{ObitThreadIterator}!ObitThread.c@{Obit\-Thread.c}}
\subsubsection{\setlength{\rightskip}{0pt plus 5cm}gboolean Obit\-Thread\-Iterator ({\bf Obit\-Thread} $\ast$ {\em in}, {\bf olong} {\em nthreads}, {\bf Obit\-Thread\-Func} {\em func}, gpointer $\ast$$\ast$ {\em args})}\label{ObitThread_8c_a23}


Public: Runs multiple copies of a function in different threads. 

Waits for operations to finish before returning, 1 min timeout. Initializes Thread pool and asynchronous queue (Obit\-Thread\-Pool\-Init) if not already done. When threaded operations are finished, call Obit\-Thread\-Pool\-Free to release Thread pool. \begin{Desc}
\item[Parameters:]
\begin{description}
\item[{\em in}]Pointer to object \item[{\em nthreads}]Number of threads to create/run \item[{\em func}]Function to call to start thread func should call Obit\-Thread\-Pool\-Done to indicate completion. \item[{\em args}]Array of argument function pointers \end{description}
\end{Desc}
\begin{Desc}
\item[Returns:]TRUE if OK else FALSE. \end{Desc}
\index{ObitThread.c@{Obit\-Thread.c}!ObitThreadJoin@{ObitThreadJoin}}
\index{ObitThreadJoin@{ObitThreadJoin}!ObitThread.c@{Obit\-Thread.c}}
\subsubsection{\setlength{\rightskip}{0pt plus 5cm}void Obit\-Thread\-Join ({\bf Obit\-Thread} $\ast$ {\em in})}\label{ObitThread_8c_a16}


Public: Join thread. 

\begin{Desc}
\item[Parameters:]
\begin{description}
\item[{\em in}]Pointer to object specifying thread to wait for.. \end{description}
\end{Desc}
\index{ObitThread.c@{Obit\-Thread.c}!ObitThreadJoin1@{ObitThreadJoin1}}
\index{ObitThreadJoin1@{ObitThreadJoin1}!ObitThread.c@{Obit\-Thread.c}}
\subsubsection{\setlength{\rightskip}{0pt plus 5cm}gpointer Obit\-Thread\-Join1 ({\bf Obit\-Thread} $\ast$ {\em in})}\label{ObitThread_8c_a27}


Public: Waits for single thread. 

\begin{Desc}
\item[Parameters:]
\begin{description}
\item[{\em in}]Pointer to Thread object \end{description}
\end{Desc}
\begin{Desc}
\item[Returns:]pointer to object returned by function \end{Desc}
\index{ObitThread.c@{Obit\-Thread.c}!ObitThreadLock@{ObitThreadLock}}
\index{ObitThreadLock@{ObitThreadLock}!ObitThread.c@{Obit\-Thread.c}}
\subsubsection{\setlength{\rightskip}{0pt plus 5cm}void Obit\-Thread\-Lock ({\bf Obit\-Thread} $\ast$ {\em in})}\label{ObitThread_8c_a7}


Public: Lock (mutex) object. 

Noop unless compiled with OBIT\_\-THREADS\_\-ENABLED \begin{Desc}
\item[Parameters:]
\begin{description}
\item[{\em in}]Pointer to object to be locked. \end{description}
\end{Desc}
\index{ObitThread.c@{Obit\-Thread.c}!ObitThreadNumProc@{ObitThreadNumProc}}
\index{ObitThreadNumProc@{ObitThreadNumProc}!ObitThread.c@{Obit\-Thread.c}}
\subsubsection{\setlength{\rightskip}{0pt plus 5cm}{\bf olong} Obit\-Thread\-Num\-Proc ({\bf Obit\-Thread} $\ast$ {\em in})}\label{ObitThread_8c_a21}


Public: Returns number of processors which can be multithreaded. 

\begin{Desc}
\item[Parameters:]
\begin{description}
\item[{\em in}]Pointer to a Thread object \end{description}
\end{Desc}
\begin{Desc}
\item[Returns:]number of processors which can be threaaded, $<$=0 =$>$ no threading. \end{Desc}
\index{ObitThread.c@{Obit\-Thread.c}!ObitThreadPoolDone@{ObitThreadPoolDone}}
\index{ObitThreadPoolDone@{ObitThreadPoolDone}!ObitThread.c@{Obit\-Thread.c}}
\subsubsection{\setlength{\rightskip}{0pt plus 5cm}void Obit\-Thread\-Pool\-Done ({\bf Obit\-Thread} $\ast$ {\em in}, gpointer {\em arg})}\label{ObitThread_8c_a24}


Public: Indicates that a thread function is done. 

\begin{Desc}
\item[Parameters:]
\begin{description}
\item[{\em in}]Pointer to Thread object \item[{\em arg}]Pointer to message (CANNOT be NULL) \end{description}
\end{Desc}
\index{ObitThread.c@{Obit\-Thread.c}!ObitThreadPoolFree@{ObitThreadPoolFree}}
\index{ObitThreadPoolFree@{ObitThreadPoolFree}!ObitThread.c@{Obit\-Thread.c}}
\subsubsection{\setlength{\rightskip}{0pt plus 5cm}void Obit\-Thread\-Pool\-Free ({\bf Obit\-Thread} $\ast$ {\em in})}\label{ObitThread_8c_a25}


Public: Shuts down Thread Pool. 

\begin{Desc}
\item[Parameters:]
\begin{description}
\item[{\em in}]Pointer to Thread object \end{description}
\end{Desc}
\index{ObitThread.c@{Obit\-Thread.c}!ObitThreadPoolInit@{ObitThreadPoolInit}}
\index{ObitThreadPoolInit@{ObitThreadPoolInit}!ObitThread.c@{Obit\-Thread.c}}
\subsubsection{\setlength{\rightskip}{0pt plus 5cm}void Obit\-Thread\-Pool\-Init ({\bf Obit\-Thread} $\ast$ {\em in}, {\bf olong} {\em nthreads}, {\bf Obit\-Thread\-Func} {\em func}, gpointer $\ast$$\ast$ {\em args})}\label{ObitThread_8c_a22}


Public: Initializes Thread Pool. 

\begin{Desc}
\item[Parameters:]
\begin{description}
\item[{\em in}]Pointer to Thread object \item[{\em nthreads}]Number of threads to create/run \item[{\em func}]Function to call to start thread \item[{\em args}]Array of argument function pointers \end{description}
\end{Desc}
\index{ObitThread.c@{Obit\-Thread.c}!ObitThreadQueueCheck@{ObitThreadQueueCheck}}
\index{ObitThreadQueueCheck@{ObitThreadQueueCheck}!ObitThread.c@{Obit\-Thread.c}}
\subsubsection{\setlength{\rightskip}{0pt plus 5cm}gpointer Obit\-Thread\-Queue\-Check ({\bf Obit\-Thread} $\ast$ {\em in}, {\bf olong} {\em add\_\-time})}\label{ObitThread_8c_a29}


Public: Check for messages in queue. 

\begin{Desc}
\item[Parameters:]
\begin{description}
\item[{\em in}]Pointer to Thread object \item[{\em add\_\-time}]timeout in microseconds, $<$= -$>$ forever \end{description}
\end{Desc}
\index{ObitThread.c@{Obit\-Thread.c}!ObitThreadQueueFree@{ObitThreadQueueFree}}
\index{ObitThreadQueueFree@{ObitThreadQueueFree}!ObitThread.c@{Obit\-Thread.c}}
\subsubsection{\setlength{\rightskip}{0pt plus 5cm}void Obit\-Thread\-Queue\-Free ({\bf Obit\-Thread} $\ast$ {\em in})}\label{ObitThread_8c_a30}


Public: Shuts down message queue. 

\begin{Desc}
\item[Parameters:]
\begin{description}
\item[{\em in}]Pointer to Thread object \end{description}
\end{Desc}
\index{ObitThread.c@{Obit\-Thread.c}!ObitThreadQueueInit@{ObitThreadQueueInit}}
\index{ObitThreadQueueInit@{ObitThreadQueueInit}!ObitThread.c@{Obit\-Thread.c}}
\subsubsection{\setlength{\rightskip}{0pt plus 5cm}void Obit\-Thread\-Queue\-Init ({\bf Obit\-Thread} $\ast$ {\em in})}\label{ObitThread_8c_a28}


Public: Initializes Thread message queue. 

\begin{Desc}
\item[Parameters:]
\begin{description}
\item[{\em in}]Pointer to Thread object \end{description}
\end{Desc}
\index{ObitThread.c@{Obit\-Thread.c}!ObitThreadRef@{ObitThreadRef}}
\index{ObitThreadRef@{ObitThreadRef}!ObitThread.c@{Obit\-Thread.c}}
\subsubsection{\setlength{\rightskip}{0pt plus 5cm}{\bf Obit\-Thread}$\ast$ Obit\-Thread\-Ref ({\bf Obit\-Thread} $\ast$ {\em in})}\label{ObitThread_8c_a5}


Public: Reference Thread pointer. 

\begin{Desc}
\item[Parameters:]
\begin{description}
\item[{\em in}]Pointer to object to be linked. \end{description}
\end{Desc}
\begin{Desc}
\item[Returns:]Pointer to object. \end{Desc}
\index{ObitThread.c@{Obit\-Thread.c}!ObitThreadRWReadLock@{ObitThreadRWReadLock}}
\index{ObitThreadRWReadLock@{ObitThreadRWReadLock}!ObitThread.c@{Obit\-Thread.c}}
\subsubsection{\setlength{\rightskip}{0pt plus 5cm}void Obit\-Thread\-RWRead\-Lock ({\bf Obit\-Thread} $\ast$ {\em in})}\label{ObitThread_8c_a10}


Public: Lock (RWLock) object for reader. 

Other threads are allowed read access Noop unless compiled with OBIT\_\-THREADS\_\-ENABLED \begin{Desc}
\item[Parameters:]
\begin{description}
\item[{\em in}]Pointer to object to be locked. \end{description}
\end{Desc}
\index{ObitThread.c@{Obit\-Thread.c}!ObitThreadRWReadTryLock@{ObitThreadRWReadTryLock}}
\index{ObitThreadRWReadTryLock@{ObitThreadRWReadTryLock}!ObitThread.c@{Obit\-Thread.c}}
\subsubsection{\setlength{\rightskip}{0pt plus 5cm}gboolean Obit\-Thread\-RWRead\-Try\-Lock ({\bf Obit\-Thread} $\ast$ {\em in})}\label{ObitThread_8c_a11}


Public: Test Lock (RWLock) object for reader. 

If successful returns TRUE, if the object is already locked, returns FALSE immediatly. Noop unless compiled with OBIT\_\-THREADS\_\-ENABLED \begin{Desc}
\item[Parameters:]
\begin{description}
\item[{\em in}]Pointer to object to be locked. \end{description}
\end{Desc}
\begin{Desc}
\item[Returns:]TRUE if successful \end{Desc}
\index{ObitThread.c@{Obit\-Thread.c}!ObitThreadRWReadUnlock@{ObitThreadRWReadUnlock}}
\index{ObitThreadRWReadUnlock@{ObitThreadRWReadUnlock}!ObitThread.c@{Obit\-Thread.c}}
\subsubsection{\setlength{\rightskip}{0pt plus 5cm}void Obit\-Thread\-RWRead\-Unlock ({\bf Obit\-Thread} $\ast$ {\em in})}\label{ObitThread_8c_a12}


Public: Unlock (RWLock) object for reader. 

\begin{Desc}
\item[Parameters:]
\begin{description}
\item[{\em in}]Pointer to object to be unlocked. \end{description}
\end{Desc}
\index{ObitThread.c@{Obit\-Thread.c}!ObitThreadRWWriteLock@{ObitThreadRWWriteLock}}
\index{ObitThreadRWWriteLock@{ObitThreadRWWriteLock}!ObitThread.c@{Obit\-Thread.c}}
\subsubsection{\setlength{\rightskip}{0pt plus 5cm}void Obit\-Thread\-RWWrite\-Lock ({\bf Obit\-Thread} $\ast$ {\em in})}\label{ObitThread_8c_a13}


Public: Lock (RWLock) object for writer. 

Noop unless compiled with OBIT\_\-THREADS\_\-ENABLED \begin{Desc}
\item[Parameters:]
\begin{description}
\item[{\em in}]Pointer to object to be locked. \end{description}
\end{Desc}
\index{ObitThread.c@{Obit\-Thread.c}!ObitThreadRWWriteTryLock@{ObitThreadRWWriteTryLock}}
\index{ObitThreadRWWriteTryLock@{ObitThreadRWWriteTryLock}!ObitThread.c@{Obit\-Thread.c}}
\subsubsection{\setlength{\rightskip}{0pt plus 5cm}gboolean Obit\-Thread\-RWWrite\-Try\-Lock ({\bf Obit\-Thread} $\ast$ {\em in})}\label{ObitThread_8c_a14}


Public: Test Lock (RWLock) object for writer. 

If successful returns TRUE, if the object is already locked, returns FALSE immediatly. Noop unless compiled with OBIT\_\-THREADS\_\-ENABLED \begin{Desc}
\item[Parameters:]
\begin{description}
\item[{\em in}]Pointer to object to be locked. \end{description}
\end{Desc}
\begin{Desc}
\item[Returns:]TRUE if successful \end{Desc}
\index{ObitThread.c@{Obit\-Thread.c}!ObitThreadRWWriteUnlock@{ObitThreadRWWriteUnlock}}
\index{ObitThreadRWWriteUnlock@{ObitThreadRWWriteUnlock}!ObitThread.c@{Obit\-Thread.c}}
\subsubsection{\setlength{\rightskip}{0pt plus 5cm}void Obit\-Thread\-RWWrite\-Unlock ({\bf Obit\-Thread} $\ast$ {\em in})}\label{ObitThread_8c_a15}


Public: Unlock (RWLock) object for writer. 

Noop unless compiled with OBIT\_\-THREADS\_\-ENABLED \begin{Desc}
\item[Parameters:]
\begin{description}
\item[{\em in}]Pointer to object to be unlocked. \end{description}
\end{Desc}
\index{ObitThread.c@{Obit\-Thread.c}!ObitThreadStart1@{ObitThreadStart1}}
\index{ObitThreadStart1@{ObitThreadStart1}!ObitThread.c@{Obit\-Thread.c}}
\subsubsection{\setlength{\rightskip}{0pt plus 5cm}void Obit\-Thread\-Start1 ({\bf Obit\-Thread} $\ast$ {\em in}, {\bf Obit\-Thread\-Func} {\em func}, gpointer {\em args})}\label{ObitThread_8c_a26}


Public: Starts single thread. 

\begin{Desc}
\item[Parameters:]
\begin{description}
\item[{\em in}]Pointer to Thread object \item[{\em func}]Function to call to start thread \item[{\em arg}]Function argument pointer \end{description}
\end{Desc}
\index{ObitThread.c@{Obit\-Thread.c}!ObitThreadTryLock@{ObitThreadTryLock}}
\index{ObitThreadTryLock@{ObitThreadTryLock}!ObitThread.c@{Obit\-Thread.c}}
\subsubsection{\setlength{\rightskip}{0pt plus 5cm}gboolean Obit\-Thread\-Try\-Lock ({\bf Obit\-Thread} $\ast$ {\em in})}\label{ObitThread_8c_a8}


Public: Test Lock (mutex) object. 

If successful returns TRUE, if the object is already locked, returns FALSE immediatly. Noop unless compiled with OBIT\_\-THREADS\_\-ENABLED \begin{Desc}
\item[Parameters:]
\begin{description}
\item[{\em in}]Pointer to object to be locked. \end{description}
\end{Desc}
\begin{Desc}
\item[Returns:]TRUE if successful \end{Desc}
\index{ObitThread.c@{Obit\-Thread.c}!ObitThreadUnlock@{ObitThreadUnlock}}
\index{ObitThreadUnlock@{ObitThreadUnlock}!ObitThread.c@{Obit\-Thread.c}}
\subsubsection{\setlength{\rightskip}{0pt plus 5cm}void Obit\-Thread\-Unlock ({\bf Obit\-Thread} $\ast$ {\em in})}\label{ObitThread_8c_a9}


Public: Unlock (mutex) object. 

Noop unless compiled with OBIT\_\-THREADS\_\-ENABLED \begin{Desc}
\item[Parameters:]
\begin{description}
\item[{\em in}]Pointer to object to be unlocked. \end{description}
\end{Desc}
\index{ObitThread.c@{Obit\-Thread.c}!ObitThreadUnref@{ObitThreadUnref}}
\index{ObitThreadUnref@{ObitThreadUnref}!ObitThread.c@{Obit\-Thread.c}}
\subsubsection{\setlength{\rightskip}{0pt plus 5cm}{\bf Obit\-Thread}$\ast$ Obit\-Thread\-Unref ({\bf Obit\-Thread} $\ast$ {\em in})}\label{ObitThread_8c_a6}


Public: Unreference Thread pointer. 

This function should be used to dismiss an object. \begin{Desc}
\item[Parameters:]
\begin{description}
\item[{\em in}]Pointer to object to be unlinked. \end{description}
\end{Desc}
\begin{Desc}
\item[Returns:]Null Pointer. \end{Desc}
