\section{Obit\-FInterpolate.c File Reference}
\label{ObitFInterpolate_8c}\index{ObitFInterpolate.c@{ObitFInterpolate.c}}
{\bf Obit\-FInterpolate}{\rm (p.\,\pageref{structObitFInterpolate})} class function definitions. 

{\tt \#include \char`\"{}Obit\-FInterpolate.h\char`\"{}}\par
{\tt \#include \char`\"{}Obit\-Position.h\char`\"{}}\par
\subsection*{Functions}
\begin{CompactItemize}
\item 
void {\bf Obit\-FInterpolate\-Init} (gpointer in)
\begin{CompactList}\small\item\em Private: Initialize newly instantiated object. \item\end{CompactList}\item 
void {\bf Obit\-FInterpolate\-Clear} (gpointer in)
\begin{CompactList}\small\item\em Private: Deallocate members. \item\end{CompactList}\item 
{\bf Obit\-FInterpolate} $\ast$ {\bf new\-Obit\-FInterpolate} (gchar $\ast$name)
\begin{CompactList}\small\item\em Public: Constructor. \item\end{CompactList}\item 
{\bf Obit\-FInterpolate} $\ast$ {\bf new\-Obit\-FInterpolate\-Create} (gchar $\ast$name, {\bf Obit\-FArray} $\ast$array, {\bf Obit\-Image\-Desc} $\ast$desc, {\bf olong} hwidth)
\begin{CompactList}\small\item\em Public: Constructor from value. \item\end{CompactList}\item 
gconstpointer {\bf Obit\-FInterpolate\-Get\-Class} (void)
\begin{CompactList}\small\item\em Public: Class\-Info pointer. \item\end{CompactList}\item 
{\bf Obit\-FInterpolate} $\ast$ {\bf Obit\-FInterpolate\-Copy} ({\bf Obit\-FInterpolate} $\ast$in, {\bf Obit\-FInterpolate} $\ast$out, {\bf Obit\-Err} $\ast$err)
\begin{CompactList}\small\item\em Public: Copy (deep) constructor. \item\end{CompactList}\item 
{\bf Obit\-FInterpolate} $\ast$ {\bf Obit\-FInterpolate\-Clone} ({\bf Obit\-FInterpolate} $\ast$in, {\bf Obit\-FInterpolate} $\ast$out)
\begin{CompactList}\small\item\em Public: Copy (shallow) constructor. \item\end{CompactList}\item 
void {\bf Obit\-FInterpolate\-Replace} ({\bf Obit\-FInterpolate} $\ast$in, {\bf Obit\-FArray} $\ast$new\-Array)
\begin{CompactList}\small\item\em Public: Replace member {\bf Obit\-FArray}{\rm (p.\,\pageref{structObitFArray})}. \item\end{CompactList}\item 
{\bf ofloat} {\bf Obit\-FInterpolate\-Pixel} ({\bf Obit\-FInterpolate} $\ast$in, {\bf ofloat} $\ast$pixel, {\bf Obit\-Err} $\ast$err)
\begin{CompactList}\small\item\em Public: Interpolate Pixel in 2D array. \item\end{CompactList}\item 
{\bf ofloat} {\bf Obit\-FInterpolate1D} ({\bf Obit\-FInterpolate} $\ast$in, {\bf ofloat} pixel)
\begin{CompactList}\small\item\em Public: Interpolate value in 1- array. \item\end{CompactList}\item 
{\bf ofloat} {\bf Obit\-FInterpolate\-Position} ({\bf Obit\-FInterpolate} $\ast$in, {\bf odouble} $\ast$coord, {\bf Obit\-Err} $\ast$err)
\begin{CompactList}\small\item\em Public: Interpolate Position in 2D array. \item\end{CompactList}\item 
{\bf ofloat} {\bf Obit\-FInterpolate\-Offset} ({\bf Obit\-FInterpolate} $\ast$in, {\bf odouble} $\ast$off, {\bf Obit\-Err} $\ast$err)
\begin{CompactList}\small\item\em Public: Interpolate Offset in 2D array. \item\end{CompactList}\item 
void {\bf Obit\-FInterpolate\-Class\-Init} (void)
\begin{CompactList}\small\item\em Public: Class initializer. \item\end{CompactList}\end{CompactItemize}


\subsection{Detailed Description}
{\bf Obit\-FInterpolate}{\rm (p.\,\pageref{structObitFInterpolate})} class function definitions. 

This class is derived from the {\bf Obit}{\rm (p.\,\pageref{structObit})} base class. This class supports 1 and 2-D interpolation in Obit\-FArrays using Lagrange interpolation.

\subsection{Function Documentation}
\index{ObitFInterpolate.c@{Obit\-FInterpolate.c}!newObitFInterpolate@{newObitFInterpolate}}
\index{newObitFInterpolate@{newObitFInterpolate}!ObitFInterpolate.c@{Obit\-FInterpolate.c}}
\subsubsection{\setlength{\rightskip}{0pt plus 5cm}{\bf Obit\-FInterpolate}$\ast$ new\-Obit\-FInterpolate (gchar $\ast$ {\em name})}\label{ObitFInterpolate_8c_a7}


Public: Constructor. 

Initializes class if needed on first call. \begin{Desc}
\item[Parameters:]
\begin{description}
\item[{\em name}]An optional name for the object. \end{description}
\end{Desc}
\begin{Desc}
\item[Returns:]the new object. \end{Desc}
\index{ObitFInterpolate.c@{Obit\-FInterpolate.c}!newObitFInterpolateCreate@{newObitFInterpolateCreate}}
\index{newObitFInterpolateCreate@{newObitFInterpolateCreate}!ObitFInterpolate.c@{Obit\-FInterpolate.c}}
\subsubsection{\setlength{\rightskip}{0pt plus 5cm}{\bf Obit\-FInterpolate}$\ast$ new\-Obit\-FInterpolate\-Create (gchar $\ast$ {\em name}, {\bf Obit\-FArray} $\ast$ {\em array}, {\bf Obit\-Image\-Desc} $\ast$ {\em desc}, {\bf olong} {\em hwidth})}\label{ObitFInterpolate_8c_a8}


Public: Constructor from value. 

Initializes class if needed on first call. \begin{Desc}
\item[Parameters:]
\begin{description}
\item[{\em name}]An optional name for the object. \item[{\em array}]The Obit\-Farray to be interpolated. \item[{\em desc}]if non\-NULL, an image descriptor to be attached to the output. \item[{\em hwidth}]Half width of interpolation kernal (range [1,4] allowed). \end{description}
\end{Desc}
\begin{Desc}
\item[Returns:]the new object. \end{Desc}
\index{ObitFInterpolate.c@{Obit\-FInterpolate.c}!ObitFInterpolate1D@{ObitFInterpolate1D}}
\index{ObitFInterpolate1D@{ObitFInterpolate1D}!ObitFInterpolate.c@{Obit\-FInterpolate.c}}
\subsubsection{\setlength{\rightskip}{0pt plus 5cm}{\bf ofloat} Obit\-FInterpolate1D ({\bf Obit\-FInterpolate} $\ast$ {\em in}, {\bf ofloat} {\em pixel})}\label{ObitFInterpolate_8c_a14}


Public: Interpolate value in 1- array. 

\begin{Desc}
\item[Parameters:]
\begin{description}
\item[{\em in}]The object to interpolate \item[{\em pixel}]Pixel location (1-rel) in array \end{description}
\end{Desc}
\begin{Desc}
\item[Returns:]value, blanked if invalid \end{Desc}
\index{ObitFInterpolate.c@{Obit\-FInterpolate.c}!ObitFInterpolateClassInit@{ObitFInterpolateClassInit}}
\index{ObitFInterpolateClassInit@{ObitFInterpolateClassInit}!ObitFInterpolate.c@{Obit\-FInterpolate.c}}
\subsubsection{\setlength{\rightskip}{0pt plus 5cm}void Obit\-FInterpolate\-Class\-Init (void)}\label{ObitFInterpolate_8c_a17}


Public: Class initializer. 

\index{ObitFInterpolate.c@{Obit\-FInterpolate.c}!ObitFInterpolateClear@{ObitFInterpolateClear}}
\index{ObitFInterpolateClear@{ObitFInterpolateClear}!ObitFInterpolate.c@{Obit\-FInterpolate.c}}
\subsubsection{\setlength{\rightskip}{0pt plus 5cm}void Obit\-FInterpolate\-Clear (gpointer {\em inn})}\label{ObitFInterpolate_8c_a4}


Private: Deallocate members. 

Does (recursive) deallocation of parent class members. For some reason this wasn't build into the GType class. \begin{Desc}
\item[Parameters:]
\begin{description}
\item[{\em in}]Pointer to the object to deallocate. Actually it should be an Obit\-FInterpolate$\ast$ cast to an Obit$\ast$. \end{description}
\end{Desc}
\index{ObitFInterpolate.c@{Obit\-FInterpolate.c}!ObitFInterpolateClone@{ObitFInterpolateClone}}
\index{ObitFInterpolateClone@{ObitFInterpolateClone}!ObitFInterpolate.c@{Obit\-FInterpolate.c}}
\subsubsection{\setlength{\rightskip}{0pt plus 5cm}{\bf Obit\-FInterpolate}$\ast$ Obit\-FInterpolate\-Clone ({\bf Obit\-FInterpolate} $\ast$ {\em in}, {\bf Obit\-FInterpolate} $\ast$ {\em out})}\label{ObitFInterpolate_8c_a11}


Public: Copy (shallow) constructor. 

The result will have pointers to the more complex members. Parent class members are included but any derived class info is ignored. \begin{Desc}
\item[Parameters:]
\begin{description}
\item[{\em in}]The object to copy \item[{\em out}]An existing object pointer for output or NULL if none exists. \end{description}
\end{Desc}
\begin{Desc}
\item[Returns:]pointer to the new object. \end{Desc}
\index{ObitFInterpolate.c@{Obit\-FInterpolate.c}!ObitFInterpolateCopy@{ObitFInterpolateCopy}}
\index{ObitFInterpolateCopy@{ObitFInterpolateCopy}!ObitFInterpolate.c@{Obit\-FInterpolate.c}}
\subsubsection{\setlength{\rightskip}{0pt plus 5cm}{\bf Obit\-FInterpolate}$\ast$ Obit\-FInterpolate\-Copy ({\bf Obit\-FInterpolate} $\ast$ {\em in}, {\bf Obit\-FInterpolate} $\ast$ {\em out}, {\bf Obit\-Err} $\ast$ {\em err})}\label{ObitFInterpolate_8c_a10}


Public: Copy (deep) constructor. 

Copies are made of complex members including disk files; these will be copied applying whatever selection is associated with the input. Parent class members are included but any derived class info is ignored. \begin{Desc}
\item[Parameters:]
\begin{description}
\item[{\em in}]The object to copy \item[{\em out}]An existing object pointer for output or NULL if none exists. \item[{\em err}]Error stack, returns if not empty. \end{description}
\end{Desc}
\begin{Desc}
\item[Returns:]pointer to the new object. \end{Desc}
\index{ObitFInterpolate.c@{Obit\-FInterpolate.c}!ObitFInterpolateGetClass@{ObitFInterpolateGetClass}}
\index{ObitFInterpolateGetClass@{ObitFInterpolateGetClass}!ObitFInterpolate.c@{Obit\-FInterpolate.c}}
\subsubsection{\setlength{\rightskip}{0pt plus 5cm}gconstpointer Obit\-FInterpolate\-Get\-Class (void)}\label{ObitFInterpolate_8c_a9}


Public: Class\-Info pointer. 

\begin{Desc}
\item[Returns:]pointer to the class structure. \end{Desc}
\index{ObitFInterpolate.c@{Obit\-FInterpolate.c}!ObitFInterpolateInit@{ObitFInterpolateInit}}
\index{ObitFInterpolateInit@{ObitFInterpolateInit}!ObitFInterpolate.c@{Obit\-FInterpolate.c}}
\subsubsection{\setlength{\rightskip}{0pt plus 5cm}void Obit\-FInterpolate\-Init (gpointer {\em inn})}\label{ObitFInterpolate_8c_a3}


Private: Initialize newly instantiated object. 

Parent classes portions are (recursively) initialized first \begin{Desc}
\item[Parameters:]
\begin{description}
\item[{\em in}]Pointer to the object to initialize. \end{description}
\end{Desc}
\index{ObitFInterpolate.c@{Obit\-FInterpolate.c}!ObitFInterpolateOffset@{ObitFInterpolateOffset}}
\index{ObitFInterpolateOffset@{ObitFInterpolateOffset}!ObitFInterpolate.c@{Obit\-FInterpolate.c}}
\subsubsection{\setlength{\rightskip}{0pt plus 5cm}{\bf ofloat} Obit\-FInterpolate\-Offset ({\bf Obit\-FInterpolate} $\ast$ {\em in}, {\bf odouble} $\ast$ {\em off}, {\bf Obit\-Err} $\ast$ {\em err})}\label{ObitFInterpolate_8c_a16}


Public: Interpolate Offset in 2D array. 

The object must have an image descriptor to allow determing pixel coordinates. Interpolation between planes is not supported. \begin{Desc}
\item[Parameters:]
\begin{description}
\item[{\em in}]The object to interpolate \item[{\em off}]Coordinate offset in plane \item[{\em err}]Error stack is pixel not inside image. \end{description}
\end{Desc}
\begin{Desc}
\item[Returns:]value, blanked if invalid \end{Desc}
\index{ObitFInterpolate.c@{Obit\-FInterpolate.c}!ObitFInterpolatePixel@{ObitFInterpolatePixel}}
\index{ObitFInterpolatePixel@{ObitFInterpolatePixel}!ObitFInterpolate.c@{Obit\-FInterpolate.c}}
\subsubsection{\setlength{\rightskip}{0pt plus 5cm}{\bf ofloat} Obit\-FInterpolate\-Pixel ({\bf Obit\-FInterpolate} $\ast$ {\em in}, {\bf ofloat} $\ast$ {\em pixel}, {\bf Obit\-Err} $\ast$ {\em err})}\label{ObitFInterpolate_8c_a13}


Public: Interpolate Pixel in 2D array. 

Interpolation between planes is not supported. \begin{Desc}
\item[Parameters:]
\begin{description}
\item[{\em in}]The object to interpolate \item[{\em pixel}]Pixel location (1-rel) in planes and which plane. Should have number of dimensions equal to in. \item[{\em err}]Error stack if pixel not inside image. \end{description}
\end{Desc}
\begin{Desc}
\item[Returns:]value, magic blanked if invalid \end{Desc}
\index{ObitFInterpolate.c@{Obit\-FInterpolate.c}!ObitFInterpolatePosition@{ObitFInterpolatePosition}}
\index{ObitFInterpolatePosition@{ObitFInterpolatePosition}!ObitFInterpolate.c@{Obit\-FInterpolate.c}}
\subsubsection{\setlength{\rightskip}{0pt plus 5cm}{\bf ofloat} Obit\-FInterpolate\-Position ({\bf Obit\-FInterpolate} $\ast$ {\em in}, {\bf odouble} $\ast$ {\em coord}, {\bf Obit\-Err} $\ast$ {\em err})}\label{ObitFInterpolate_8c_a15}


Public: Interpolate Position in 2D array. 

The object must have an image descriptor to allow determing pixel coordinates. Interpolation between planes is not supported. \begin{Desc}
\item[Parameters:]
\begin{description}
\item[{\em in}]The object to interpolate \item[{\em coord}]Coordinate value in plane and which plane. Should have number of dimensions equal to in. \item[{\em err}]Error stack is pixel not inside image. \end{description}
\end{Desc}
\begin{Desc}
\item[Returns:]value, blanked if invalid \end{Desc}
\index{ObitFInterpolate.c@{Obit\-FInterpolate.c}!ObitFInterpolateReplace@{ObitFInterpolateReplace}}
\index{ObitFInterpolateReplace@{ObitFInterpolateReplace}!ObitFInterpolate.c@{Obit\-FInterpolate.c}}
\subsubsection{\setlength{\rightskip}{0pt plus 5cm}void Obit\-FInterpolate\-Replace ({\bf Obit\-FInterpolate} $\ast$ {\em in}, {\bf Obit\-FArray} $\ast$ {\em new\-Array})}\label{ObitFInterpolate_8c_a12}


Public: Replace member {\bf Obit\-FArray}{\rm (p.\,\pageref{structObitFArray})}. 

\begin{Desc}
\item[Parameters:]
\begin{description}
\item[{\em in}]The object to update \item[{\em new\-Array}]The new FArray for in \end{description}
\end{Desc}
