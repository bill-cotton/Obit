\section{Obit\-IOHistory.c File Reference}
\label{ObitIOHistory_8c}\index{ObitIOHistory.c@{ObitIOHistory.c}}
{\bf Obit\-IOHistory}{\rm (p.\,\pageref{structObitIOHistory})} class function definitions. 

{\tt \#include \char`\"{}Obit\-IOHistory.h\char`\"{}}\par
\subsection*{Functions}
\begin{CompactItemize}
\item 
void {\bf Obit\-IOHistory\-Init} (gpointer in)
\begin{CompactList}\small\item\em Private: Initialize newly instantiated object. \item\end{CompactList}\item 
void {\bf Obit\-IOHistory\-Clear} (gpointer in)
\begin{CompactList}\small\item\em Private: Deallocate members. \item\end{CompactList}\item 
{\bf Obit\-IOHistory} $\ast$ {\bf new\-Obit\-IOHistory} (gchar $\ast$name, {\bf Obit\-Info\-List} $\ast$info, {\bf Obit\-Err} $\ast$err)
\begin{CompactList}\small\item\em Public: Constructor. \item\end{CompactList}\item 
gconstpointer {\bf Obit\-IOHistory\-Get\-Class} (void)
\begin{CompactList}\small\item\em Public: Class\-Info pointer. \item\end{CompactList}\item 
gboolean {\bf Obit\-IOHistory\-Same} ({\bf Obit\-IOHistory} $\ast$in, {\bf Obit\-Info\-List} $\ast$in1, {\bf Obit\-Info\-List} $\ast$in2, {\bf Obit\-Err} $\ast$err)
\begin{CompactList}\small\item\em Public: Are underlying structures the same. \item\end{CompactList}\item 
void {\bf Obit\-IOHistory\-Zap} ({\bf Obit\-IOHistory} $\ast$in, {\bf Obit\-Err} $\ast$err)
\begin{CompactList}\small\item\em Public: Delete underlying structures. \item\end{CompactList}\item 
{\bf Obit\-IOHistory} $\ast$ {\bf Obit\-IOHistory\-Copy} ({\bf Obit\-IOHistory} $\ast$in, {\bf Obit\-IOHistory} $\ast$out, {\bf Obit\-Err} $\ast$err)
\begin{CompactList}\small\item\em Public: Copy constructor. \item\end{CompactList}\item 
Obit\-IOCode {\bf Obit\-IOHistory\-Open} ({\bf Obit\-IOHistory} $\ast$in, Obit\-IOAccess access, {\bf Obit\-Info\-List} $\ast$info, {\bf Obit\-Err} $\ast$err)
\begin{CompactList}\small\item\em Public: Open. \item\end{CompactList}\item 
Obit\-IOCode {\bf Obit\-IOHistory\-Close} ({\bf Obit\-IOHistory} $\ast$in, {\bf Obit\-Err} $\ast$err)
\begin{CompactList}\small\item\em Public: Close. \item\end{CompactList}\item 
Obit\-IOCode {\bf Obit\-IOHistory\-Set} ({\bf Obit\-IOHistory} $\ast$in, {\bf Obit\-Info\-List} $\ast$info, {\bf Obit\-Err} $\ast$err)
\begin{CompactList}\small\item\em Public: Init I/O. \item\end{CompactList}\item 
Obit\-IOCode {\bf Obit\-IOHistory\-Read\-Rec} ({\bf Obit\-IOHistory} $\ast$in, {\bf olong} recno, gchar $\ast$hi\-Card, {\bf Obit\-Err} $\ast$err)
\begin{CompactList}\small\item\em Public: Read Record. \item\end{CompactList}\item 
Obit\-IOCode {\bf Obit\-IOHistory\-Write\-Rec} ({\bf Obit\-IOHistory} $\ast$in, {\bf olong} recno, gchar $\ast$hi\-Card, {\bf Obit\-Err} $\ast$err)
\begin{CompactList}\small\item\em Public: Write Record. \item\end{CompactList}\item 
{\bf olong} {\bf Obit\-IOHistory\-Num\-Rec} ({\bf Obit\-IOHistory} $\ast$in)
\begin{CompactList}\small\item\em Public: number of records. \item\end{CompactList}\item 
Obit\-IOCode {\bf Obit\-IOHistory\-Read\-Descriptor} ({\bf Obit\-IOHistory} $\ast$in, {\bf Obit\-Err} $\ast$err)
\begin{CompactList}\small\item\em Public: Read Descriptor. \item\end{CompactList}\item 
Obit\-IOCode {\bf Obit\-IOHistory\-Write\-Descriptor} ({\bf Obit\-IOHistory} $\ast$in, {\bf Obit\-Err} $\ast$err)
\begin{CompactList}\small\item\em Public: Write Descriptor. \item\end{CompactList}\item 
Obit\-IOCode {\bf Obit\-IOHistory\-Flush} ({\bf Obit\-IOHistory} $\ast$in, {\bf Obit\-Err} $\ast$err)
\begin{CompactList}\small\item\em Public: Flush. \item\end{CompactList}\item 
void {\bf Obit\-IOHistory\-Class\-Init} (void)
\begin{CompactList}\small\item\em Public: Class initializer. \item\end{CompactList}\end{CompactItemize}


\subsection{Detailed Description}
{\bf Obit\-IOHistory}{\rm (p.\,\pageref{structObitIOHistory})} class function definitions. 

This is a virtual base class and should never be directly instantiated. Derived classes provide an I/O interface to various underlying disk structures.

\subsection{Function Documentation}
\index{ObitIOHistory.c@{Obit\-IOHistory.c}!newObitIOHistory@{newObitIOHistory}}
\index{newObitIOHistory@{newObitIOHistory}!ObitIOHistory.c@{Obit\-IOHistory.c}}
\subsubsection{\setlength{\rightskip}{0pt plus 5cm}{\bf Obit\-IOHistory}$\ast$ new\-Obit\-IOHistory (gchar $\ast$ {\em name}, {\bf Obit\-Info\-List} $\ast$ {\em info}, {\bf Obit\-Err} $\ast$ {\em err})}\label{ObitIOHistory_8c_a6}


Public: Constructor. 

Initializes class if needed on first call. \begin{Desc}
\item[Parameters:]
\begin{description}
\item[{\em name}]Name [optional] for object \item[{\em info}]Info\-List defining file \item[{\em err}]{\bf Obit\-Err}{\rm (p.\,\pageref{structObitErr})} for reporting errors. \end{description}
\end{Desc}
\begin{Desc}
\item[Returns:]the new object. \end{Desc}
\index{ObitIOHistory.c@{Obit\-IOHistory.c}!ObitIOHistoryClassInit@{ObitIOHistoryClassInit}}
\index{ObitIOHistoryClassInit@{ObitIOHistoryClassInit}!ObitIOHistory.c@{Obit\-IOHistory.c}}
\subsubsection{\setlength{\rightskip}{0pt plus 5cm}void Obit\-IOHistory\-Class\-Init (void)}\label{ObitIOHistory_8c_a20}


Public: Class initializer. 

\index{ObitIOHistory.c@{Obit\-IOHistory.c}!ObitIOHistoryClear@{ObitIOHistoryClear}}
\index{ObitIOHistoryClear@{ObitIOHistoryClear}!ObitIOHistory.c@{Obit\-IOHistory.c}}
\subsubsection{\setlength{\rightskip}{0pt plus 5cm}void Obit\-IOHistory\-Clear (gpointer {\em inn})}\label{ObitIOHistory_8c_a4}


Private: Deallocate members. 

Does (recursive) deallocation of parent class members. \begin{Desc}
\item[Parameters:]
\begin{description}
\item[{\em inn}]Pointer to the object to deallocate. \end{description}
\end{Desc}
\index{ObitIOHistory.c@{Obit\-IOHistory.c}!ObitIOHistoryClose@{ObitIOHistoryClose}}
\index{ObitIOHistoryClose@{ObitIOHistoryClose}!ObitIOHistory.c@{Obit\-IOHistory.c}}
\subsubsection{\setlength{\rightskip}{0pt plus 5cm}Obit\-IOCode Obit\-IOHistory\-Close ({\bf Obit\-IOHistory} $\ast$ {\em in}, {\bf Obit\-Err} $\ast$ {\em err})}\label{ObitIOHistory_8c_a12}


Public: Close. 

\begin{Desc}
\item[Parameters:]
\begin{description}
\item[{\em in}]Pointer to object to be closed. \item[{\em err}]{\bf Obit\-Err}{\rm (p.\,\pageref{structObitErr})} for reporting errors. \end{description}
\end{Desc}
\begin{Desc}
\item[Returns:]error code, OBIT\_\-IO\_\-OK=$>$ OK \end{Desc}
\index{ObitIOHistory.c@{Obit\-IOHistory.c}!ObitIOHistoryCopy@{ObitIOHistoryCopy}}
\index{ObitIOHistoryCopy@{ObitIOHistoryCopy}!ObitIOHistory.c@{Obit\-IOHistory.c}}
\subsubsection{\setlength{\rightskip}{0pt plus 5cm}{\bf Obit\-IOHistory}$\ast$ Obit\-IOHistory\-Copy ({\bf Obit\-IOHistory} $\ast$ {\em in}, {\bf Obit\-IOHistory} $\ast$ {\em out}, {\bf Obit\-Err} $\ast$ {\em err})}\label{ObitIOHistory_8c_a10}


Public: Copy constructor. 

The result will have pointers to the more complex members. Parent class members are included but any derived class info is ignored. \begin{Desc}
\item[Parameters:]
\begin{description}
\item[{\em in}]The object to copy \item[{\em out}]An existing object pointer for output or NULL if none exists. \item[{\em err}]{\bf Obit}{\rm (p.\,\pageref{structObit})} error stack object. \end{description}
\end{Desc}
\begin{Desc}
\item[Returns:]pointer to the new object. \end{Desc}
\index{ObitIOHistory.c@{Obit\-IOHistory.c}!ObitIOHistoryFlush@{ObitIOHistoryFlush}}
\index{ObitIOHistoryFlush@{ObitIOHistoryFlush}!ObitIOHistory.c@{Obit\-IOHistory.c}}
\subsubsection{\setlength{\rightskip}{0pt plus 5cm}Obit\-IOCode Obit\-IOHistory\-Flush ({\bf Obit\-IOHistory} $\ast$ {\em in}, {\bf Obit\-Err} $\ast$ {\em err})}\label{ObitIOHistory_8c_a19}


Public: Flush. 

\begin{Desc}
\item[Parameters:]
\begin{description}
\item[{\em in}]Pointer to object to be accessed. \item[{\em err}]{\bf Obit\-Err}{\rm (p.\,\pageref{structObitErr})} for reporting errors. \end{description}
\end{Desc}
\begin{Desc}
\item[Returns:]return code, OBIT\_\-IO\_\-OK=$>$ OK \end{Desc}
\index{ObitIOHistory.c@{Obit\-IOHistory.c}!ObitIOHistoryGetClass@{ObitIOHistoryGetClass}}
\index{ObitIOHistoryGetClass@{ObitIOHistoryGetClass}!ObitIOHistory.c@{Obit\-IOHistory.c}}
\subsubsection{\setlength{\rightskip}{0pt plus 5cm}gconstpointer Obit\-IOHistory\-Get\-Class (void)}\label{ObitIOHistory_8c_a7}


Public: Class\-Info pointer. 

Initializes class if needed on first call. \begin{Desc}
\item[Returns:]pointer to the class structure. \end{Desc}
\index{ObitIOHistory.c@{Obit\-IOHistory.c}!ObitIOHistoryInit@{ObitIOHistoryInit}}
\index{ObitIOHistoryInit@{ObitIOHistoryInit}!ObitIOHistory.c@{Obit\-IOHistory.c}}
\subsubsection{\setlength{\rightskip}{0pt plus 5cm}void Obit\-IOHistory\-Init (gpointer {\em inn})}\label{ObitIOHistory_8c_a3}


Private: Initialize newly instantiated object. 

Does (recursive) initialization of base class members before this class. \begin{Desc}
\item[Parameters:]
\begin{description}
\item[{\em inn}]Pointer to the object to initialize. \end{description}
\end{Desc}
\index{ObitIOHistory.c@{Obit\-IOHistory.c}!ObitIOHistoryNumRec@{ObitIOHistoryNumRec}}
\index{ObitIOHistoryNumRec@{ObitIOHistoryNumRec}!ObitIOHistory.c@{Obit\-IOHistory.c}}
\subsubsection{\setlength{\rightskip}{0pt plus 5cm}{\bf olong} Obit\-IOHistory\-Num\-Rec ({\bf Obit\-IOHistory} $\ast$ {\em in})}\label{ObitIOHistory_8c_a16}


Public: number of records. 

\begin{Desc}
\item[Parameters:]
\begin{description}
\item[{\em in}]Pointer to open object to be tested \end{description}
\end{Desc}
\begin{Desc}
\item[Returns:]number of records, $<$0 =$>$ problem \end{Desc}
\index{ObitIOHistory.c@{Obit\-IOHistory.c}!ObitIOHistoryOpen@{ObitIOHistoryOpen}}
\index{ObitIOHistoryOpen@{ObitIOHistoryOpen}!ObitIOHistory.c@{Obit\-IOHistory.c}}
\subsubsection{\setlength{\rightskip}{0pt plus 5cm}Obit\-IOCode Obit\-IOHistory\-Open ({\bf Obit\-IOHistory} $\ast$ {\em in}, Obit\-IOAccess {\em access}, {\bf Obit\-Info\-List} $\ast$ {\em info}, {\bf Obit\-Err} $\ast$ {\em err})}\label{ObitIOHistory_8c_a11}


Public: Open. 

The file and selection info member should have been stored in the {\bf Obit\-Info\-List}{\rm (p.\,\pageref{structObitInfoList})} prior to calling. See derived classes for details. \begin{Desc}
\item[Parameters:]
\begin{description}
\item[{\em in}]Pointer to object to be opened. \item[{\em access}]access (OBIT\_\-IO\_\-Read\-Only,OBIT\_\-IO\_\-Read\-Write) \item[{\em info}]{\bf Obit\-Info\-List}{\rm (p.\,\pageref{structObitInfoList})} with instructions for opening \item[{\em err}]{\bf Obit\-Err}{\rm (p.\,\pageref{structObitErr})} for reporting errors. \end{description}
\end{Desc}
\begin{Desc}
\item[Returns:]return code, OBIT\_\-IO\_\-OK=$>$ OK \end{Desc}
\index{ObitIOHistory.c@{Obit\-IOHistory.c}!ObitIOHistoryReadDescriptor@{ObitIOHistoryReadDescriptor}}
\index{ObitIOHistoryReadDescriptor@{ObitIOHistoryReadDescriptor}!ObitIOHistory.c@{Obit\-IOHistory.c}}
\subsubsection{\setlength{\rightskip}{0pt plus 5cm}Obit\-IOCode Obit\-IOHistory\-Read\-Descriptor ({\bf Obit\-IOHistory} $\ast$ {\em in}, {\bf Obit\-Err} $\ast$ {\em err})}\label{ObitIOHistory_8c_a17}


Public: Read Descriptor. 

\begin{Desc}
\item[Parameters:]
\begin{description}
\item[{\em in}]Pointer to object with Obit\-Image\-Descto be read. \item[{\em err}]{\bf Obit\-Err}{\rm (p.\,\pageref{structObitErr})} for reporting errors. \end{description}
\end{Desc}
\begin{Desc}
\item[Returns:]return code, OBIT\_\-IO\_\-OK=$>$ OK \end{Desc}
\index{ObitIOHistory.c@{Obit\-IOHistory.c}!ObitIOHistoryReadRec@{ObitIOHistoryReadRec}}
\index{ObitIOHistoryReadRec@{ObitIOHistoryReadRec}!ObitIOHistory.c@{Obit\-IOHistory.c}}
\subsubsection{\setlength{\rightskip}{0pt plus 5cm}Obit\-IOCode Obit\-IOHistory\-Read\-Rec ({\bf Obit\-IOHistory} $\ast$ {\em in}, {\bf olong} {\em recno}, gchar $\ast$ {\em hi\-Card}, {\bf Obit\-Err} $\ast$ {\em err})}\label{ObitIOHistory_8c_a14}


Public: Read Record. 

\begin{Desc}
\item[Parameters:]
\begin{description}
\item[{\em in}]Pointer to object to be read. \item[{\em recno}]record number (1-rel) -1=$>$ next. \item[{\em hi\-Card}]output history record (70 char) \item[{\em err}]{\bf Obit\-Err}{\rm (p.\,\pageref{structObitErr})} for reporting errors. \end{description}
\end{Desc}
\begin{Desc}
\item[Returns:]return code, OBIT\_\-IO\_\-OK=$>$ OK \end{Desc}
\index{ObitIOHistory.c@{Obit\-IOHistory.c}!ObitIOHistorySame@{ObitIOHistorySame}}
\index{ObitIOHistorySame@{ObitIOHistorySame}!ObitIOHistory.c@{Obit\-IOHistory.c}}
\subsubsection{\setlength{\rightskip}{0pt plus 5cm}gboolean Obit\-IOHistory\-Same ({\bf Obit\-IOHistory} $\ast$ {\em in}, {\bf Obit\-Info\-List} $\ast$ {\em in1}, {\bf Obit\-Info\-List} $\ast$ {\em in2}, {\bf Obit\-Err} $\ast$ {\em err})}\label{ObitIOHistory_8c_a8}


Public: Are underlying structures the same. 

This test is done using values entered into the {\bf Obit\-Info\-List}{\rm (p.\,\pageref{structObitInfoList})} in case the object has not yet been opened. \begin{Desc}
\item[Parameters:]
\begin{description}
\item[{\em in}]{\bf Obit\-IO}{\rm (p.\,\pageref{structObitIO})} for test \item[{\em in1}]{\bf Obit\-Info\-List}{\rm (p.\,\pageref{structObitInfoList})} for first object to be tested \item[{\em in2}]{\bf Obit\-Info\-List}{\rm (p.\,\pageref{structObitInfoList})} for second object to be tested \item[{\em err}]{\bf Obit\-Err}{\rm (p.\,\pageref{structObitErr})} for reporting errors. \end{description}
\end{Desc}
\begin{Desc}
\item[Returns:]TRUE if to objects have the same underlying structures else FALSE \end{Desc}
\index{ObitIOHistory.c@{Obit\-IOHistory.c}!ObitIOHistorySet@{ObitIOHistorySet}}
\index{ObitIOHistorySet@{ObitIOHistorySet}!ObitIOHistory.c@{Obit\-IOHistory.c}}
\subsubsection{\setlength{\rightskip}{0pt plus 5cm}Obit\-IOCode Obit\-IOHistory\-Set ({\bf Obit\-IOHistory} $\ast$ {\em in}, {\bf Obit\-Info\-List} $\ast$ {\em info}, {\bf Obit\-Err} $\ast$ {\em err})}\label{ObitIOHistory_8c_a13}


Public: Init I/O. 

\begin{Desc}
\item[Parameters:]
\begin{description}
\item[{\em in}]Pointer to object to be accessed. \item[{\em info}]{\bf Obit\-Info\-List}{\rm (p.\,\pageref{structObitInfoList})} with instructions \item[{\em err}]{\bf Obit\-Err}{\rm (p.\,\pageref{structObitErr})} for reporting errors. \end{description}
\end{Desc}
\begin{Desc}
\item[Returns:]return code, OBIT\_\-IO\_\-OK=$>$ OK \end{Desc}
\index{ObitIOHistory.c@{Obit\-IOHistory.c}!ObitIOHistoryWriteDescriptor@{ObitIOHistoryWriteDescriptor}}
\index{ObitIOHistoryWriteDescriptor@{ObitIOHistoryWriteDescriptor}!ObitIOHistory.c@{Obit\-IOHistory.c}}
\subsubsection{\setlength{\rightskip}{0pt plus 5cm}Obit\-IOCode Obit\-IOHistory\-Write\-Descriptor ({\bf Obit\-IOHistory} $\ast$ {\em in}, {\bf Obit\-Err} $\ast$ {\em err})}\label{ObitIOHistory_8c_a18}


Public: Write Descriptor. 

\begin{Desc}
\item[Parameters:]
\begin{description}
\item[{\em in}]Pointer to object with {\bf Obit\-Image\-Desc}{\rm (p.\,\pageref{structObitImageDesc})} to be written. \item[{\em err}]{\bf Obit\-Err}{\rm (p.\,\pageref{structObitErr})} for reporting errors. \end{description}
\end{Desc}
\begin{Desc}
\item[Returns:]return code, OBIT\_\-IO\_\-OK=$>$ OK \end{Desc}
\index{ObitIOHistory.c@{Obit\-IOHistory.c}!ObitIOHistoryWriteRec@{ObitIOHistoryWriteRec}}
\index{ObitIOHistoryWriteRec@{ObitIOHistoryWriteRec}!ObitIOHistory.c@{Obit\-IOHistory.c}}
\subsubsection{\setlength{\rightskip}{0pt plus 5cm}Obit\-IOCode Obit\-IOHistory\-Write\-Rec ({\bf Obit\-IOHistory} $\ast$ {\em in}, {\bf olong} {\em recno}, gchar $\ast$ {\em hi\-Card}, {\bf Obit\-Err} $\ast$ {\em err})}\label{ObitIOHistory_8c_a15}


Public: Write Record. 

\begin{Desc}
\item[Parameters:]
\begin{description}
\item[{\em in}]Pointer to object to be written. \item[{\em recno}]Record number (1-rel) -1=$>$ next, overwrites any existing \item[{\em hi\-Card}]input history record (70 char) \item[{\em err}]{\bf Obit\-Err}{\rm (p.\,\pageref{structObitErr})} for reporting errors. \end{description}
\end{Desc}
\begin{Desc}
\item[Returns:]return code, OBIT\_\-IO\_\-OK=$>$ OK \end{Desc}
\index{ObitIOHistory.c@{Obit\-IOHistory.c}!ObitIOHistoryZap@{ObitIOHistoryZap}}
\index{ObitIOHistoryZap@{ObitIOHistoryZap}!ObitIOHistory.c@{Obit\-IOHistory.c}}
\subsubsection{\setlength{\rightskip}{0pt plus 5cm}void Obit\-IOHistory\-Zap ({\bf Obit\-IOHistory} $\ast$ {\em in}, {\bf Obit\-Err} $\ast$ {\em err})}\label{ObitIOHistory_8c_a9}


Public: Delete underlying structures. 

\begin{Desc}
\item[Parameters:]
\begin{description}
\item[{\em in}]Pointer to object to be zapped. \item[{\em err}]{\bf Obit\-Err}{\rm (p.\,\pageref{structObitErr})} for reporting errors. \end{description}
\end{Desc}
