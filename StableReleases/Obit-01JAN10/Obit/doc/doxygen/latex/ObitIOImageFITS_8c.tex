\section{Obit\-IOImage\-FITS.c File Reference}
\label{ObitIOImageFITS_8c}\index{ObitIOImageFITS.c@{ObitIOImageFITS.c}}
{\bf Obit\-IOImage\-FITS}{\rm (p.\,\pageref{structObitIOImageFITS})} class function definitions. 

{\tt \#include $<$errno.h$>$}\par
{\tt \#include \char`\"{}Obit.h\char`\"{}}\par
{\tt \#include \char`\"{}Obit\-IOImage\-FITS.h\char`\"{}}\par
{\tt \#include \char`\"{}Obit\-Image\-Sel.h\char`\"{}}\par
{\tt \#include \char`\"{}Obit\-Table\-List.h\char`\"{}}\par
{\tt \#include \char`\"{}Obit\-File.h\char`\"{}}\par
{\tt \#include \char`\"{}Obit\-FITS.h\char`\"{}}\par
{\tt \#include \char`\"{}Obit\-Mem.h\char`\"{}}\par
\subsection*{Functions}
\begin{CompactItemize}
\item 
void {\bf Obit\-IOImage\-FITSInit} (gpointer in)
\begin{CompactList}\small\item\em Private: Initialize newly instantiated object. \item\end{CompactList}\item 
void {\bf Obit\-IOImage\-FITSClear} (gpointer in)
\begin{CompactList}\small\item\em Private: Deallocate members. \item\end{CompactList}\item 
void {\bf Obit\-IOImage\-AIPSCLEANRead} ({\bf Obit\-IOImage\-FITS} $\ast$in, {\bf olong} $\ast$status)
\begin{CompactList}\small\item\em Private: Read AIPS CLEAN parameters. \item\end{CompactList}\item 
void {\bf Obit\-IOImage\-AIPSCLEANWrite} ({\bf Obit\-IOImage\-FITS} $\ast$in, {\bf olong} $\ast$status)
\begin{CompactList}\small\item\em Private: Write AIPS CLEAN parameters. \item\end{CompactList}\item 
void {\bf Obit\-IOImage\-Keys\-Other\-Read} ({\bf Obit\-IOImage\-FITS} $\ast$in, {\bf olong} $\ast$status, {\bf Obit\-Err} $\ast$err)
\begin{CompactList}\small\item\em Private: Copy other header keywords. \item\end{CompactList}\item 
{\bf Obit\-IOImage\-FITS} $\ast$ {\bf new\-Obit\-IOImage\-FITS} (gchar $\ast$name, {\bf Obit\-Info\-List} $\ast$info, {\bf Obit\-Err} $\ast$err)
\begin{CompactList}\small\item\em Public: Constructor. \item\end{CompactList}\item 
gconstpointer {\bf Obit\-IOImage\-FITSGet\-Class} (void)
\begin{CompactList}\small\item\em Public: Class\-Info pointer. \item\end{CompactList}\item 
gboolean {\bf Obit\-IOImage\-FITSSame} ({\bf Obit\-IO} $\ast$in, {\bf Obit\-Info\-List} $\ast$in1, {\bf Obit\-Info\-List} $\ast$in2, {\bf Obit\-Err} $\ast$err)
\begin{CompactList}\small\item\em Public: Are underlying structures the same. \item\end{CompactList}\item 
void {\bf Obit\-IOImage\-FITSRename} ({\bf Obit\-IO} $\ast$in, {\bf Obit\-Info\-List} $\ast$info, {\bf Obit\-Err} $\ast$err)
\begin{CompactList}\small\item\em Public: Rename underlying structures. \item\end{CompactList}\item 
void {\bf Obit\-IOImage\-FITSZap} ({\bf Obit\-IOImage\-FITS} $\ast$in, {\bf Obit\-Err} $\ast$err)
\begin{CompactList}\small\item\em Public: Delete underlying structures. \item\end{CompactList}\item 
{\bf Obit\-IOImage\-FITS} $\ast$ {\bf Obit\-IOImage\-FITSCopy} ({\bf Obit\-IOImage\-FITS} $\ast$in, {\bf Obit\-IOImage\-FITS} $\ast$out, {\bf Obit\-Err} $\ast$err)
\begin{CompactList}\small\item\em Public: Copy constructor. \item\end{CompactList}\item 
Obit\-IOCode {\bf Obit\-IOImage\-FITSOpen} ({\bf Obit\-IOImage\-FITS} $\ast$in, Obit\-IOAccess access, {\bf Obit\-Info\-List} $\ast$info, {\bf Obit\-Err} $\ast$err)
\begin{CompactList}\small\item\em Public: Open. \item\end{CompactList}\item 
Obit\-IOCode {\bf Obit\-IOImage\-FITSClose} ({\bf Obit\-IOImage\-FITS} $\ast$in, {\bf Obit\-Err} $\ast$err)
\begin{CompactList}\small\item\em Public: Close. \item\end{CompactList}\item 
Obit\-IOCode {\bf Obit\-IOImage\-FITSSet} ({\bf Obit\-IOImage\-FITS} $\ast$in, {\bf Obit\-Info\-List} $\ast$info, {\bf Obit\-Err} $\ast$err)
\begin{CompactList}\small\item\em Public: Init I/O. \item\end{CompactList}\item 
Obit\-IOCode {\bf Obit\-IOImage\-FITSRead} ({\bf Obit\-IOImage\-FITS} $\ast$in, {\bf ofloat} $\ast$data, {\bf Obit\-Err} $\ast$err)
\begin{CompactList}\small\item\em Public: Read. \item\end{CompactList}\item 
Obit\-IOCode {\bf Obit\-IOImage\-FITSWrite} ({\bf Obit\-IOImage\-FITS} $\ast$in, {\bf ofloat} $\ast$data, {\bf Obit\-Err} $\ast$err)
\begin{CompactList}\small\item\em Public: Write. \item\end{CompactList}\item 
Obit\-IOCode {\bf Obit\-IOImage\-FITSRead\-Descriptor} ({\bf Obit\-IOImage\-FITS} $\ast$in, {\bf Obit\-Err} $\ast$err)
\begin{CompactList}\small\item\em Public: Read Descriptor. \item\end{CompactList}\item 
Obit\-IOCode {\bf Obit\-IOImage\-FITSWrite\-Descriptor} ({\bf Obit\-IOImage\-FITS} $\ast$in, {\bf Obit\-Err} $\ast$err)
\begin{CompactList}\small\item\em Public: Write Descriptor. \item\end{CompactList}\item 
Obit\-IOCode {\bf Obit\-IOImage\-FITSFlush} ({\bf Obit\-IOImage\-FITS} $\ast$in, {\bf Obit\-Err} $\ast$err)
\begin{CompactList}\small\item\em Public: Flush. \item\end{CompactList}\item 
void {\bf Obit\-IOImage\-FITSCreate\-Buffer} ({\bf ofloat} $\ast$$\ast$data, {\bf olong} $\ast$size, {\bf Obit\-IOImage\-FITS} $\ast$in, {\bf Obit\-Info\-List} $\ast$info, {\bf Obit\-Err} $\ast$err)
\begin{CompactList}\small\item\em Public: Create buffer. \item\end{CompactList}\item 
{\bf Obit} $\ast$ {\bf new\-Obit\-IOImage\-FITSTable} ({\bf Obit\-IOImage\-FITS} $\ast$in, Obit\-IOAccess access, gchar $\ast$tab\-Type, {\bf olong} $\ast$tab\-Ver, {\bf Obit\-Err} $\ast$err)
\begin{CompactList}\small\item\em Public: Create an associated Table Typed as base class to avoid problems. \item\end{CompactList}\item 
Obit\-IOCode {\bf Obit\-IOImage\-FITSUpdate\-Tables} ({\bf Obit\-IOImage\-FITS} $\ast$in, {\bf Obit\-Info\-List} $\ast$info, {\bf Obit\-Err} $\ast$err)
\begin{CompactList}\small\item\em Public: Update disk resident tables information. \item\end{CompactList}\item 
void {\bf Obit\-IOImage\-FITSUpdate\-Scale} ({\bf Obit\-IOImage\-FITS} $\ast$in, {\bf ofloat} quant, {\bf Obit\-Err} $\ast$err)
\begin{CompactList}\small\item\em Public: Update header BSCALE,BZERO. \item\end{CompactList}\item 
void {\bf Obit\-IOImage\-FITSGet\-File\-Info} ({\bf Obit\-IO} $\ast$in, {\bf Obit\-Info\-List} $\ast$my\-Info, gchar $\ast$prefix, {\bf Obit\-Info\-List} $\ast$out\-List, {\bf Obit\-Err} $\ast$err)
\begin{CompactList}\small\item\em Public: Extract information about underlying file. \item\end{CompactList}\item 
void {\bf Obit\-IOImage\-FITSClass\-Init} (void)
\begin{CompactList}\small\item\em Public: Class initializer. \item\end{CompactList}\end{CompactItemize}


\subsection{Detailed Description}
{\bf Obit\-IOImage\-FITS}{\rm (p.\,\pageref{structObitIOImageFITS})} class function definitions. 

This class is derived from the {\bf Obit\-IO}{\rm (p.\,\pageref{structObitIO})} class.

\subsection{Function Documentation}
\index{ObitIOImageFITS.c@{Obit\-IOImage\-FITS.c}!newObitIOImageFITS@{newObitIOImageFITS}}
\index{newObitIOImageFITS@{newObitIOImageFITS}!ObitIOImageFITS.c@{Obit\-IOImage\-FITS.c}}
\subsubsection{\setlength{\rightskip}{0pt plus 5cm}{\bf Obit\-IOImage\-FITS}$\ast$ new\-Obit\-IOImage\-FITS (gchar $\ast$ {\em name}, {\bf Obit\-Info\-List} $\ast$ {\em info}, {\bf Obit\-Err} $\ast$ {\em err})}\label{ObitIOImageFITS_8c_a19}


Public: Constructor. 

Initializes class on the first call. \begin{Desc}
\item[Parameters:]
\begin{description}
\item[{\em name}]An optional name for the object. \item[{\em info}]if non-NULL it is used to initialize the new object. \item[{\em err}]{\bf Obit\-Err}{\rm (p.\,\pageref{structObitErr})} for error messages. \end{description}
\end{Desc}
\begin{Desc}
\item[Returns:]the new object. \end{Desc}
\index{ObitIOImageFITS.c@{Obit\-IOImage\-FITS.c}!newObitIOImageFITSTable@{newObitIOImageFITSTable}}
\index{newObitIOImageFITSTable@{newObitIOImageFITSTable}!ObitIOImageFITS.c@{Obit\-IOImage\-FITS.c}}
\subsubsection{\setlength{\rightskip}{0pt plus 5cm}{\bf Obit}$\ast$ new\-Obit\-IOImage\-FITSTable ({\bf Obit\-IOImage\-FITS} $\ast$ {\em in}, Obit\-IOAccess {\em access}, gchar $\ast$ {\em tab\-Type}, {\bf olong} $\ast$ {\em tab\-Ver}, {\bf Obit\-Err} $\ast$ {\em err})}\label{ObitIOImageFITS_8c_a34}


Public: Create an associated Table Typed as base class to avoid problems. 

If such an object exists, a reference to it is returned, else a new object is created and entered in the {\bf Obit\-Table\-List}{\rm (p.\,\pageref{structObitTableList})}. Returned object is typed an {\bf Obit}{\rm (p.\,\pageref{structObit})} to prevent circular definitions between the {\bf Obit\-Table}{\rm (p.\,\pageref{structObitTable})} and the {\bf Obit\-IO}{\rm (p.\,\pageref{structObitIO})} classes. \begin{Desc}
\item[Parameters:]
\begin{description}
\item[{\em in}]Pointer to object with associated tables. This MUST have been opened before this call. \item[{\em access}]access (OBIT\_\-IO\_\-Read\-Only,OBIT\_\-IO\_\-Read\-Write, or OBIT\_\-IO\_\-Write\-Only). This is used to determine defaulted version number and a different value may be used for the actual Open. \item[{\em tab\-Type}]The table type (e.g. \char`\"{}AIPS CC\char`\"{}). \item[{\em tab\-Ver}]Desired version number, may be zero in which case the highest extant version is returned for read and the highest+1 for OBIT\_\-IO\_\-Write\-Only. \item[{\em err}]{\bf Obit\-Err}{\rm (p.\,\pageref{structObitErr})} for reporting errors. \end{description}
\end{Desc}
\begin{Desc}
\item[Returns:]pointer to created {\bf Obit\-Table}{\rm (p.\,\pageref{structObitTable})}, NULL on failure. \end{Desc}
\index{ObitIOImageFITS.c@{Obit\-IOImage\-FITS.c}!ObitIOImageAIPSCLEANRead@{ObitIOImageAIPSCLEANRead}}
\index{ObitIOImageAIPSCLEANRead@{ObitIOImageAIPSCLEANRead}!ObitIOImageFITS.c@{Obit\-IOImage\-FITS.c}}
\subsubsection{\setlength{\rightskip}{0pt plus 5cm}void Obit\-IOImage\-AIPSCLEANRead ({\bf Obit\-IOImage\-FITS} $\ast$ {\em in}, {\bf olong} $\ast$ {\em lstatus})}\label{ObitIOImageFITS_8c_a5}


Private: Read AIPS CLEAN parameters. 

Descriptor values niter, beam\-Maj, beam\-Min, beam\-PA \begin{Desc}
\item[Parameters:]
\begin{description}
\item[{\em in}]Pointer to {\bf Obit\-IOImage\-FITS}{\rm (p.\,\pageref{structObitIOImageFITS})}. \item[{\em lstatus}](Output) cfitsio status. \end{description}
\end{Desc}
\begin{Desc}
\item[Returns:]return code, 0=$>$ OK \end{Desc}
\index{ObitIOImageFITS.c@{Obit\-IOImage\-FITS.c}!ObitIOImageAIPSCLEANWrite@{ObitIOImageAIPSCLEANWrite}}
\index{ObitIOImageAIPSCLEANWrite@{ObitIOImageAIPSCLEANWrite}!ObitIOImageFITS.c@{Obit\-IOImage\-FITS.c}}
\subsubsection{\setlength{\rightskip}{0pt plus 5cm}void Obit\-IOImage\-AIPSCLEANWrite ({\bf Obit\-IOImage\-FITS} $\ast$ {\em in}, {\bf olong} $\ast$ {\em lstatus})}\label{ObitIOImageFITS_8c_a6}


Private: Write AIPS CLEAN parameters. 

\begin{Desc}
\item[Parameters:]
\begin{description}
\item[{\em in}]Pointer to {\bf Obit\-IOImage\-FITS}{\rm (p.\,\pageref{structObitIOImageFITS})}. \item[{\em status}](Output) cfitsio status. \end{description}
\end{Desc}
\begin{Desc}
\item[Returns:]return code, 0=$>$ OK \end{Desc}
\index{ObitIOImageFITS.c@{Obit\-IOImage\-FITS.c}!ObitIOImageFITSClassInit@{ObitIOImageFITSClassInit}}
\index{ObitIOImageFITSClassInit@{ObitIOImageFITSClassInit}!ObitIOImageFITS.c@{Obit\-IOImage\-FITS.c}}
\subsubsection{\setlength{\rightskip}{0pt plus 5cm}void Obit\-IOImage\-FITSClass\-Init (void)}\label{ObitIOImageFITS_8c_a38}


Public: Class initializer. 

\index{ObitIOImageFITS.c@{Obit\-IOImage\-FITS.c}!ObitIOImageFITSClear@{ObitIOImageFITSClear}}
\index{ObitIOImageFITSClear@{ObitIOImageFITSClear}!ObitIOImageFITS.c@{Obit\-IOImage\-FITS.c}}
\subsubsection{\setlength{\rightskip}{0pt plus 5cm}void Obit\-IOImage\-FITSClear (gpointer {\em inn})}\label{ObitIOImageFITS_8c_a4}


Private: Deallocate members. 

Does (recursive) deallocation of parent class members. For some reason this wasn't build into the GType class. \begin{Desc}
\item[Parameters:]
\begin{description}
\item[{\em inn}]Pointer to the object to deallocate. \end{description}
\end{Desc}
\index{ObitIOImageFITS.c@{Obit\-IOImage\-FITS.c}!ObitIOImageFITSClose@{ObitIOImageFITSClose}}
\index{ObitIOImageFITSClose@{ObitIOImageFITSClose}!ObitIOImageFITS.c@{Obit\-IOImage\-FITS.c}}
\subsubsection{\setlength{\rightskip}{0pt plus 5cm}Obit\-IOCode Obit\-IOImage\-FITSClose ({\bf Obit\-IOImage\-FITS} $\ast$ {\em in}, {\bf Obit\-Err} $\ast$ {\em err})}\label{ObitIOImageFITS_8c_a26}


Public: Close. 

\begin{Desc}
\item[Parameters:]
\begin{description}
\item[{\em in}]Pointer to object to be closed. \item[{\em err}]{\bf Obit\-Err}{\rm (p.\,\pageref{structObitErr})} for reporting errors. \end{description}
\end{Desc}
\begin{Desc}
\item[Returns:]error code, 0=$>$ OK \end{Desc}
\index{ObitIOImageFITS.c@{Obit\-IOImage\-FITS.c}!ObitIOImageFITSCopy@{ObitIOImageFITSCopy}}
\index{ObitIOImageFITSCopy@{ObitIOImageFITSCopy}!ObitIOImageFITS.c@{Obit\-IOImage\-FITS.c}}
\subsubsection{\setlength{\rightskip}{0pt plus 5cm}{\bf Obit\-IOImage\-FITS}$\ast$ Obit\-IOImage\-FITSCopy ({\bf Obit\-IOImage\-FITS} $\ast$ {\em in}, {\bf Obit\-IOImage\-FITS} $\ast$ {\em out}, {\bf Obit\-Err} $\ast$ {\em err})}\label{ObitIOImageFITS_8c_a24}


Public: Copy constructor. 

The result will have pointers to the more complex members. Parent class members are included but any derived class info is ignored. \begin{Desc}
\item[Parameters:]
\begin{description}
\item[{\em in}]The object to copy \item[{\em out}]An existing object pointer for output or NULL if none exists. \item[{\em err}]{\bf Obit}{\rm (p.\,\pageref{structObit})} error stack object. \end{description}
\end{Desc}
\begin{Desc}
\item[Returns:]pointer to the new object. \end{Desc}
\index{ObitIOImageFITS.c@{Obit\-IOImage\-FITS.c}!ObitIOImageFITSCreateBuffer@{ObitIOImageFITSCreateBuffer}}
\index{ObitIOImageFITSCreateBuffer@{ObitIOImageFITSCreateBuffer}!ObitIOImageFITS.c@{Obit\-IOImage\-FITS.c}}
\subsubsection{\setlength{\rightskip}{0pt plus 5cm}void Obit\-IOImage\-FITSCreate\-Buffer ({\bf ofloat} $\ast$$\ast$ {\em data}, {\bf olong} $\ast$ {\em size}, {\bf Obit\-IOImage\-FITS} $\ast$ {\em in}, {\bf Obit\-Info\-List} $\ast$ {\em info}, {\bf Obit\-Err} $\ast$ {\em err})}\label{ObitIOImageFITS_8c_a33}


Public: Create buffer. 

Not actually used for Images. Should be called after {\bf Obit\-IO}{\rm (p.\,\pageref{structObitIO})} is opened. \begin{Desc}
\item[Parameters:]
\begin{description}
\item[{\em data}](output) pointer to data array \item[{\em size}](output) size of data array in floats. \item[{\em in}]Pointer to object to be accessed. \item[{\em info}]{\bf Obit\-Info\-List}{\rm (p.\,\pageref{structObitInfoList})} with instructions \item[{\em err}]{\bf Obit\-Err}{\rm (p.\,\pageref{structObitErr})} for reporting errors. \end{description}
\end{Desc}
\index{ObitIOImageFITS.c@{Obit\-IOImage\-FITS.c}!ObitIOImageFITSFlush@{ObitIOImageFITSFlush}}
\index{ObitIOImageFITSFlush@{ObitIOImageFITSFlush}!ObitIOImageFITS.c@{Obit\-IOImage\-FITS.c}}
\subsubsection{\setlength{\rightskip}{0pt plus 5cm}Obit\-IOCode Obit\-IOImage\-FITSFlush ({\bf Obit\-IOImage\-FITS} $\ast$ {\em in}, {\bf Obit\-Err} $\ast$ {\em err})}\label{ObitIOImageFITS_8c_a32}


Public: Flush. 

\begin{Desc}
\item[Parameters:]
\begin{description}
\item[{\em in}]Pointer to object to be accessed. \item[{\em err}]{\bf Obit\-Err}{\rm (p.\,\pageref{structObitErr})} for reporting errors. \end{description}
\end{Desc}
\begin{Desc}
\item[Returns:]return code, 0=$>$ OK \end{Desc}
\index{ObitIOImageFITS.c@{Obit\-IOImage\-FITS.c}!ObitIOImageFITSGetClass@{ObitIOImageFITSGetClass}}
\index{ObitIOImageFITSGetClass@{ObitIOImageFITSGetClass}!ObitIOImageFITS.c@{Obit\-IOImage\-FITS.c}}
\subsubsection{\setlength{\rightskip}{0pt plus 5cm}gconstpointer Obit\-IOImage\-FITSGet\-Class (void)}\label{ObitIOImageFITS_8c_a20}


Public: Class\-Info pointer. 

Initializes class if needed on first call. \begin{Desc}
\item[Returns:]pointer to the class structure. \end{Desc}
\index{ObitIOImageFITS.c@{Obit\-IOImage\-FITS.c}!ObitIOImageFITSGetFileInfo@{ObitIOImageFITSGetFileInfo}}
\index{ObitIOImageFITSGetFileInfo@{ObitIOImageFITSGetFileInfo}!ObitIOImageFITS.c@{Obit\-IOImage\-FITS.c}}
\subsubsection{\setlength{\rightskip}{0pt plus 5cm}void Obit\-IOImage\-FITSGet\-File\-Info ({\bf Obit\-IO} $\ast$ {\em in}, {\bf Obit\-Info\-List} $\ast$ {\em my\-Info}, gchar $\ast$ {\em prefix}, {\bf Obit\-Info\-List} $\ast$ {\em out\-List}, {\bf Obit\-Err} $\ast$ {\em err})}\label{ObitIOImageFITS_8c_a37}


Public: Extract information about underlying file. 

\begin{Desc}
\item[Parameters:]
\begin{description}
\item[{\em in}]Object of interest. \item[{\em my\-Info}]Info\-List on basic object with selection \item[{\em prefix}]If Non\-Null, string to be added to beginning of out\-List entry name \item[{\em out\-List}]Info\-List to write entries into\end{description}
\end{Desc}
Following entries for FITS files (\char`\"{}xxx\char`\"{} = prefix): \begin{itemize}
\item xxx\-File\-Name OBIT\_\-string FITS file name \item xxx\-Disk OBIT\_\-oint FITS file disk number \item xxx\-Dir OBIT\_\-string Directory name for xxx\-Disk\end{itemize}
For all File types types: \begin{itemize}
\item xxx\-Data\-Type OBIT\_\-string \char`\"{}UV\char`\"{} = UV data, \char`\"{}MA\char`\"{}=$>$image, \char`\"{}Table\char`\"{}=Table, \char`\"{}OTF\char`\"{}=OTF, etc \item xxx\-File\-Type OBIT\_\-oint File type as Obit\-IOType, OBIT\_\-IO\_\-FITS, OBIT\_\-IO\_\-AIPS\end{itemize}
For xxx\-Data\-Type = \char`\"{}MA\char`\"{} \begin{itemize}
\item xxx\-BLC OBIT\_\-oint[7] (Images only) 1-rel bottom-left corner pixel \item xxx\-TRC OBIT\_\-oint[7] (Images Only) 1-rel top-right corner pixel \begin{Desc}
\item[Parameters:]
\begin{description}
\item[{\em err}]{\bf Obit\-Err}{\rm (p.\,\pageref{structObitErr})} for reporting errors. \end{description}
\end{Desc}
\end{itemize}
\index{ObitIOImageFITS.c@{Obit\-IOImage\-FITS.c}!ObitIOImageFITSInit@{ObitIOImageFITSInit}}
\index{ObitIOImageFITSInit@{ObitIOImageFITSInit}!ObitIOImageFITS.c@{Obit\-IOImage\-FITS.c}}
\subsubsection{\setlength{\rightskip}{0pt plus 5cm}void Obit\-IOImage\-FITSInit (gpointer {\em inn})}\label{ObitIOImageFITS_8c_a3}


Private: Initialize newly instantiated object. 

The GType constructors will call the corresponding routines for each parent class. \begin{Desc}
\item[Parameters:]
\begin{description}
\item[{\em inn}]Pointer to the object to initialize. \end{description}
\end{Desc}
\index{ObitIOImageFITS.c@{Obit\-IOImage\-FITS.c}!ObitIOImageFITSOpen@{ObitIOImageFITSOpen}}
\index{ObitIOImageFITSOpen@{ObitIOImageFITSOpen}!ObitIOImageFITS.c@{Obit\-IOImage\-FITS.c}}
\subsubsection{\setlength{\rightskip}{0pt plus 5cm}Obit\-IOCode Obit\-IOImage\-FITSOpen ({\bf Obit\-IOImage\-FITS} $\ast$ {\em in}, Obit\-IOAccess {\em access}, {\bf Obit\-Info\-List} $\ast$ {\em info}, {\bf Obit\-Err} $\ast$ {\em err})}\label{ObitIOImageFITS_8c_a25}


Public: Open. 

The file etc. info should have been stored in the {\bf Obit\-Info\-List}{\rm (p.\,\pageref{structObitInfoList})}. The image descriptor is read if Read\-Only or Read\-Write and written to disk if opened Write\-Only. For accessing FITS files the following entries in the {\bf Obit\-Info\-List}{\rm (p.\,\pageref{structObitInfoList})} are used: \begin{itemize}
\item \char`\"{}Disk\char`\"{} OBIT\_\-int (1,1,1) FITS \char`\"{}disk\char`\"{} number. \item \char`\"{}File\-Name\char`\"{} OBIT\_\-string (?,1,1) FITS file name. \begin{Desc}
\item[Parameters:]
\begin{description}
\item[{\em in}]Pointer to object to be opened. \item[{\em in}]Pointer to object to be opened. \item[{\em access}]access (OBIT\_\-IO\_\-Read\-Only,OBIT\_\-IO\_\-Read\-Write) \item[{\em info}]{\bf Obit\-Info\-List}{\rm (p.\,\pageref{structObitInfoList})} with instructions for opening \item[{\em err}]{\bf Obit\-Err}{\rm (p.\,\pageref{structObitErr})} for reporting errors. \end{description}
\end{Desc}
\begin{Desc}
\item[Returns:]return code, 0=$>$ OK \end{Desc}
\end{itemize}
\index{ObitIOImageFITS.c@{Obit\-IOImage\-FITS.c}!ObitIOImageFITSRead@{ObitIOImageFITSRead}}
\index{ObitIOImageFITSRead@{ObitIOImageFITSRead}!ObitIOImageFITS.c@{Obit\-IOImage\-FITS.c}}
\subsubsection{\setlength{\rightskip}{0pt plus 5cm}Obit\-IOCode Obit\-IOImage\-FITSRead ({\bf Obit\-IOImage\-FITS} $\ast$ {\em in}, {\bf ofloat} $\ast$ {\em data}, {\bf Obit\-Err} $\ast$ {\em err})}\label{ObitIOImageFITS_8c_a28}


Public: Read. 

Reads row in-$>$my\-Desc-$>$row + 1; plane in-$>$my\-Desc-$>$plane + 1 When OBIT\_\-IO\_\-EOF is returned all data has been read (then is no new data in data) and the I/O has been closed. \begin{Desc}
\item[Parameters:]
\begin{description}
\item[{\em in}]Pointer to object to be read. \item[{\em data}]pointer to buffer to write results. \item[{\em err}]{\bf Obit\-Err}{\rm (p.\,\pageref{structObitErr})} for reporting errors. \end{description}
\end{Desc}
\begin{Desc}
\item[Returns:]return code, 0(OBIT\_\-IO\_\-OK)=$>$ OK, OBIT\_\-IO\_\-EOF =$>$ image finished. \end{Desc}
\index{ObitIOImageFITS.c@{Obit\-IOImage\-FITS.c}!ObitIOImageFITSReadDescriptor@{ObitIOImageFITSReadDescriptor}}
\index{ObitIOImageFITSReadDescriptor@{ObitIOImageFITSReadDescriptor}!ObitIOImageFITS.c@{Obit\-IOImage\-FITS.c}}
\subsubsection{\setlength{\rightskip}{0pt plus 5cm}Obit\-IOCode Obit\-IOImage\-FITSRead\-Descriptor ({\bf Obit\-IOImage\-FITS} $\ast$ {\em in}, {\bf Obit\-Err} $\ast$ {\em err})}\label{ObitIOImageFITS_8c_a30}


Public: Read Descriptor. 

\begin{Desc}
\item[Parameters:]
\begin{description}
\item[{\em in}]Pointer to object with {\bf Obit\-Image\-Desc}{\rm (p.\,\pageref{structObitImageDesc})} to be read. \item[{\em err}]{\bf Obit\-Err}{\rm (p.\,\pageref{structObitErr})} for reporting errors. \end{description}
\end{Desc}
\begin{Desc}
\item[Returns:]return code, 0=$>$ OK \end{Desc}
\index{ObitIOImageFITS.c@{Obit\-IOImage\-FITS.c}!ObitIOImageFITSRename@{ObitIOImageFITSRename}}
\index{ObitIOImageFITSRename@{ObitIOImageFITSRename}!ObitIOImageFITS.c@{Obit\-IOImage\-FITS.c}}
\subsubsection{\setlength{\rightskip}{0pt plus 5cm}void Obit\-IOImage\-FITSRename ({\bf Obit\-IO} $\ast$ {\em in}, {\bf Obit\-Info\-List} $\ast$ {\em info}, {\bf Obit\-Err} $\ast$ {\em err})}\label{ObitIOImageFITS_8c_a22}


Public: Rename underlying structures. 

New name information is given on the info member: \begin{itemize}
\item \char`\"{}new\-File\-Name\char`\"{} OBIT\_\-string (?,1,1) New Name of disk file. \begin{Desc}
\item[Parameters:]
\begin{description}
\item[{\em in}]Pointer to object to be renamed \item[{\em info}]Associated {\bf Obit\-Info\-List}{\rm (p.\,\pageref{structObitInfoList})} \item[{\em err}]{\bf Obit\-Err}{\rm (p.\,\pageref{structObitErr})} for reporting errors. \end{description}
\end{Desc}
\end{itemize}
\index{ObitIOImageFITS.c@{Obit\-IOImage\-FITS.c}!ObitIOImageFITSSame@{ObitIOImageFITSSame}}
\index{ObitIOImageFITSSame@{ObitIOImageFITSSame}!ObitIOImageFITS.c@{Obit\-IOImage\-FITS.c}}
\subsubsection{\setlength{\rightskip}{0pt plus 5cm}gboolean Obit\-IOImage\-FITSSame ({\bf Obit\-IO} $\ast$ {\em in}, {\bf Obit\-Info\-List} $\ast$ {\em in1}, {\bf Obit\-Info\-List} $\ast$ {\em in2}, {\bf Obit\-Err} $\ast$ {\em err})}\label{ObitIOImageFITS_8c_a21}


Public: Are underlying structures the same. 

This test is done using values entered into the {\bf Obit\-Info\-List}{\rm (p.\,\pageref{structObitInfoList})} in case the object has not yet been opened. \begin{Desc}
\item[Parameters:]
\begin{description}
\item[{\em in}]{\bf Obit\-IO}{\rm (p.\,\pageref{structObitIO})} for test \item[{\em in1}]{\bf Obit\-Info\-List}{\rm (p.\,\pageref{structObitInfoList})} for first object to be tested \item[{\em in2}]{\bf Obit\-Info\-List}{\rm (p.\,\pageref{structObitInfoList})} for second object to be tested \item[{\em err}]{\bf Obit\-Err}{\rm (p.\,\pageref{structObitErr})} for reporting errors. \end{description}
\end{Desc}
\begin{Desc}
\item[Returns:]TRUE if to objects have the same underlying structures else FALSE \end{Desc}
\index{ObitIOImageFITS.c@{Obit\-IOImage\-FITS.c}!ObitIOImageFITSSet@{ObitIOImageFITSSet}}
\index{ObitIOImageFITSSet@{ObitIOImageFITSSet}!ObitIOImageFITS.c@{Obit\-IOImage\-FITS.c}}
\subsubsection{\setlength{\rightskip}{0pt plus 5cm}Obit\-IOCode Obit\-IOImage\-FITSSet ({\bf Obit\-IOImage\-FITS} $\ast$ {\em in}, {\bf Obit\-Info\-List} $\ast$ {\em info}, {\bf Obit\-Err} $\ast$ {\em err})}\label{ObitIOImageFITS_8c_a27}


Public: Init I/O. 

\begin{Desc}
\item[Parameters:]
\begin{description}
\item[{\em in}]Pointer to object to be accessed. \item[{\em info}]{\bf Obit\-Info\-List}{\rm (p.\,\pageref{structObitInfoList})} with instructions \item[{\em err}]{\bf Obit\-Err}{\rm (p.\,\pageref{structObitErr})} for reporting errors. \end{description}
\end{Desc}
\begin{Desc}
\item[Returns:]return code, 0=$>$ OK \end{Desc}
\index{ObitIOImageFITS.c@{Obit\-IOImage\-FITS.c}!ObitIOImageFITSUpdateScale@{ObitIOImageFITSUpdateScale}}
\index{ObitIOImageFITSUpdateScale@{ObitIOImageFITSUpdateScale}!ObitIOImageFITS.c@{Obit\-IOImage\-FITS.c}}
\subsubsection{\setlength{\rightskip}{0pt plus 5cm}void Obit\-IOImage\-FITSUpdate\-Scale ({\bf Obit\-IOImage\-FITS} $\ast$ {\em in}, {\bf ofloat} {\em quant}, {\bf Obit\-Err} $\ast$ {\em err})}\label{ObitIOImageFITS_8c_a36}


Public: Update header BSCALE,BZERO. 

\begin{Desc}
\item[Parameters:]
\begin{description}
\item[{\em in}]Pointer to object to be updated. \item[{\em quant}]Quantization desired, 0.0 use pixel value range to set scaling \item[{\em err}]{\bf Obit\-Err}{\rm (p.\,\pageref{structObitErr})} for reporting errors. \end{description}
\end{Desc}
\index{ObitIOImageFITS.c@{Obit\-IOImage\-FITS.c}!ObitIOImageFITSUpdateTables@{ObitIOImageFITSUpdateTables}}
\index{ObitIOImageFITSUpdateTables@{ObitIOImageFITSUpdateTables}!ObitIOImageFITS.c@{Obit\-IOImage\-FITS.c}}
\subsubsection{\setlength{\rightskip}{0pt plus 5cm}Obit\-IOCode Obit\-IOImage\-FITSUpdate\-Tables ({\bf Obit\-IOImage\-FITS} $\ast$ {\em in}, {\bf Obit\-Info\-List} $\ast$ {\em info}, {\bf Obit\-Err} $\ast$ {\em err})}\label{ObitIOImageFITS_8c_a35}


Public: Update disk resident tables information. 

Nothing is needed for FITS files. \begin{Desc}
\item[Parameters:]
\begin{description}
\item[{\em in}]Pointer to object to be updated. \item[{\em info}]{\bf Obit\-Info\-List}{\rm (p.\,\pageref{structObitInfoList})} of parent object (not used here). \item[{\em err}]{\bf Obit\-Err}{\rm (p.\,\pageref{structObitErr})} for reporting errors. \end{description}
\end{Desc}
\begin{Desc}
\item[Returns:]return code, OBIT\_\-IO\_\-OK=$>$ OK \end{Desc}
\index{ObitIOImageFITS.c@{Obit\-IOImage\-FITS.c}!ObitIOImageFITSWrite@{ObitIOImageFITSWrite}}
\index{ObitIOImageFITSWrite@{ObitIOImageFITSWrite}!ObitIOImageFITS.c@{Obit\-IOImage\-FITS.c}}
\subsubsection{\setlength{\rightskip}{0pt plus 5cm}Obit\-IOCode Obit\-IOImage\-FITSWrite ({\bf Obit\-IOImage\-FITS} $\ast$ {\em in}, {\bf ofloat} $\ast$ {\em data}, {\bf Obit\-Err} $\ast$ {\em err})}\label{ObitIOImageFITS_8c_a29}


Public: Write. 

Writes row in-$>$my\-Desc-$>$row + 1; plane in-$>$my\-Desc-$>$plane + 1 Writing partial images is only supported in row at a time mode. When OBIT\_\-IO\_\-EOF is returned the image has been written, data in data is ignored and the I/O is closed. \begin{Desc}
\item[Parameters:]
\begin{description}
\item[{\em in}]Pointer to object to be written. \item[{\em data}]pointer to buffer containing input data. \item[{\em err}]{\bf Obit\-Err}{\rm (p.\,\pageref{structObitErr})} for reporting errors. \end{description}
\end{Desc}
\begin{Desc}
\item[Returns:]return code, 0(OBIT\_\-IO\_\-OK)=$>$ OK OBIT\_\-IO\_\-EOF =$>$ image finished. \end{Desc}
\index{ObitIOImageFITS.c@{Obit\-IOImage\-FITS.c}!ObitIOImageFITSWriteDescriptor@{ObitIOImageFITSWriteDescriptor}}
\index{ObitIOImageFITSWriteDescriptor@{ObitIOImageFITSWriteDescriptor}!ObitIOImageFITS.c@{Obit\-IOImage\-FITS.c}}
\subsubsection{\setlength{\rightskip}{0pt plus 5cm}Obit\-IOCode Obit\-IOImage\-FITSWrite\-Descriptor ({\bf Obit\-IOImage\-FITS} $\ast$ {\em in}, {\bf Obit\-Err} $\ast$ {\em err})}\label{ObitIOImageFITS_8c_a31}


Public: Write Descriptor. 

\begin{Desc}
\item[Parameters:]
\begin{description}
\item[{\em in}]Pointer to object with {\bf Obit\-Image\-Desc}{\rm (p.\,\pageref{structObitImageDesc})} to be written. If info\-List member contains \char`\"{}Quant\char`\"{} entry and it is $>$ 0.0 and an integer output (Bitpix 16, 32) is specified then the output will be quantized at this level. \item[{\em err}]{\bf Obit\-Err}{\rm (p.\,\pageref{structObitErr})} for reporting errors. \end{description}
\end{Desc}
\begin{Desc}
\item[Returns:]return code, 0=$>$ OK \end{Desc}
\index{ObitIOImageFITS.c@{Obit\-IOImage\-FITS.c}!ObitIOImageFITSZap@{ObitIOImageFITSZap}}
\index{ObitIOImageFITSZap@{ObitIOImageFITSZap}!ObitIOImageFITS.c@{Obit\-IOImage\-FITS.c}}
\subsubsection{\setlength{\rightskip}{0pt plus 5cm}void Obit\-IOImage\-FITSZap ({\bf Obit\-IOImage\-FITS} $\ast$ {\em in}, {\bf Obit\-Err} $\ast$ {\em err})}\label{ObitIOImageFITS_8c_a23}


Public: Delete underlying structures. 

Delete the whole FITS file. \begin{Desc}
\item[Parameters:]
\begin{description}
\item[{\em in}]Pointer to object to be zapped. \item[{\em err}]{\bf Obit\-Err}{\rm (p.\,\pageref{structObitErr})} for reporting errors. \end{description}
\end{Desc}
\index{ObitIOImageFITS.c@{Obit\-IOImage\-FITS.c}!ObitIOImageKeysOtherRead@{ObitIOImageKeysOtherRead}}
\index{ObitIOImageKeysOtherRead@{ObitIOImageKeysOtherRead}!ObitIOImageFITS.c@{Obit\-IOImage\-FITS.c}}
\subsubsection{\setlength{\rightskip}{0pt plus 5cm}void Obit\-IOImage\-Keys\-Other\-Read ({\bf Obit\-IOImage\-FITS} $\ast$ {\em in}, {\bf olong} $\ast$ {\em lstatus}, {\bf Obit\-Err} $\ast$ {\em err})}\label{ObitIOImageFITS_8c_a7}


Private: Copy other header keywords. 

\begin{Desc}
\item[Parameters:]
\begin{description}
\item[{\em in}]Pointer to {\bf Obit\-IOImage\-FITS}{\rm (p.\,\pageref{structObitIOImageFITS})}. \item[{\em status}](Output) cfitsio status. \item[{\em err}]{\bf Obit\-Err}{\rm (p.\,\pageref{structObitErr})} stack. \end{description}
\end{Desc}
\begin{Desc}
\item[Returns:]return code, 0=$>$ OK \end{Desc}
