\section{Obit\-AIPSFortran.h File Reference}
\label{ObitAIPSFortran_8h}\index{ObitAIPSFortran.h@{ObitAIPSFortran.h}}
Obit\-AIPSFortran module definition. 

{\tt \#include $<$glib.h$>$}\par
{\tt \#include \char`\"{}Obit.h\char`\"{}}\par
{\tt \#include \char`\"{}Obit\-Err.h\char`\"{}}\par
{\tt \#include \char`\"{}Obit\-AIPS.h\char`\"{}}\par
{\tt \#include \char`\"{}Obit\-AIPSObject.h\char`\"{}}\par
\subsection*{Functions}
\begin{CompactItemize}
\item 
void {\bf obintx\_\-} (const gchar $\ast$pgm\-Name, const {\bf oint} $\ast$len\-Pgm\-Name, const {\bf oint} $\ast$pgm\-Number, const {\bf oint} $\ast$AIPSuser, const {\bf oint} $\ast$number\-AIPSdisk, const gchar $\ast$AIPSdir, const {\bf oint} $\ast$len\-AIPSdir, const {\bf oint} $\ast$number\-FITSdisk, const gchar $\ast$FITSdir, const {\bf oint} $\ast$len\-FITSdir, const {\bf oint} $\ast$F\_\-TRUE, const {\bf oint} $\ast$F\_\-FALSE, {\bf oint} $\ast$ierr)
\begin{CompactList}\small\item\em Public: Init {\bf Obit}{\rm (p.\,\pageref{structObit})} System. \item\end{CompactList}\item 
void {\bf obshtx\_\-} (void)
\begin{CompactList}\small\item\em Public: Shutdown {\bf Obit}{\rm (p.\,\pageref{structObit})} System. \item\end{CompactList}\item 
void {\bf obuvcp\_\-} ({\bf AIPSObj} uvin, {\bf AIPSObj} uvout, {\bf oint} $\ast$ierr)
\begin{CompactList}\small\item\em Public: Copy a uvdata set potentially applying calibration, editing and selection,. \item\end{CompactList}\item 
void {\bf obufwt\_\-} ({\bf AIPSObj} uv, {\bf AIPSObj} image, {\bf oint} $\ast$ierr)
\begin{CompactList}\small\item\em Public: Determine and apply uniform weighting corrections to uv data. \item\end{CompactList}\item 
void {\bf obimuv\_\-} ({\bf AIPSObj} uvdata, {\bf oint} $\ast$ifield, {\bf oint} $\ast$nfield, {\bf AIPSObj} $\ast$image, {\bf AIPSObj} $\ast$beam, {\bf oint} $\ast$do\-Beam, {\bf oint} $\ast$do\-Create, {\bf oint} $\ast$chan, {\bf oint} $\ast$nchan, {\bf oint} $\ast$imchan, {\bf oint} $\ast$ierr)
\begin{CompactList}\small\item\em Public: Make beams and image from uv data. \item\end{CompactList}\item 
void {\bf obiuvi\_\-} ({\bf AIPSObj} uvdata, {\bf oint} $\ast$ifield, {\bf oint} $\ast$nfield, {\bf AIPSObj} $\ast$image, {\bf AIPSObj} $\ast$beam, {\bf oint} $\ast$do\-Beam, {\bf oint} $\ast$do\-Create, {\bf oint} $\ast$chan, {\bf oint} $\ast$nchan, {\bf oint} $\ast$imchan, {\bf oint} $\ast$ierr)
\begin{CompactList}\small\item\em Public: Make beams and image with possible ionospheric corrections. \item\end{CompactList}\end{CompactItemize}


\subsection{Detailed Description}
Obit\-AIPSFortran module definition. 

This is a Utility module with Fortran callable routines for AIPS.

\subsection{Function Documentation}
\index{ObitAIPSFortran.h@{Obit\-AIPSFortran.h}!obimuv_@{obimuv\_\-}}
\index{obimuv_@{obimuv\_\-}!ObitAIPSFortran.h@{Obit\-AIPSFortran.h}}
\subsubsection{\setlength{\rightskip}{0pt plus 5cm}void obimuv\_\- ({\bf AIPSObj} {\em uvdata}, {\bf oint} $\ast$ {\em ifield}, {\bf oint} $\ast$ {\em nfield}, {\bf AIPSObj} $\ast$ {\em image}, {\bf AIPSObj} $\ast$ {\em beam}, {\bf oint} $\ast$ {\em dobeam}, {\bf oint} $\ast$ {\em docreate}, {\bf oint} $\ast$ {\em chan}, {\bf oint} $\ast$ {\em nchan}, {\bf oint} $\ast$ {\em imchan}, {\bf oint} $\ast$ {\em ierr})}\label{ObitAIPSFortran_8h_a4}


Public: Make beams and image from uv data. 

Makes beams or images from a uv data set. If NFIELD is $<$= 0 then a beam is made. If an image is to be made the normalization factor is obtained from the beam, if absent, the beam is remade. The input uvdata is assumed to have been calibrated, selected and had any uniform weighting corrections applied. Note: output IMAGE and BEAM objects should exist (full instantation) prior to call.

Two methods of Fourier transform are available: FFT and DFT. The FFT method supports multiple fields and allows images to be a different size from the beam. The DFT method does a full 3D DFT but supports only a single field and the beam must be the same size as the image. DFT NYI. \begin{Desc}
\item[Parameters:]
\begin{description}
\item[{\em uvdata}]Name of uvdata object Members: \begin{itemize}
\item STOKES C$\ast$4 Desired Stokes parameter (I) \item UVRANGE R(2) UV range in kilo wavelengths (all) \item GUARDBND R(2) Fractional guardband around edge of uv grid (0) \end{itemize}
\item[{\em ifield}]Field to image: 0 =$>$ all nfield (note 1 beam made on ifield = 1 when do\-Beam true (1) and DO3DIM false) \item[{\em nfield}]Number of fields. \item[{\em image}]Array of image AIPS object names MUST be previously instantiated. Members of image(1): \begin{itemize}
\item FTTYPE C$\ast$4 Fourier transform type 'FFT' or 'DFT'. ('FFT') \item IMSIZE I(2,$\ast$) Image size per field (no default) \item CELLSIZE R(2) Cellsize in arcseconds in X and Y (no default) \item CHTYPE C$\ast$4 'LINE', or 'SUM ' for imaging ('SUM') \item SHIFT R(2) Shift in arcsec (DFT imaging) \item RASHIFT R($\ast$) X position shift in arcseconds per field (0) FFT \item DECSHIFT R($\ast$) Y position shift in arcseconds per field (0) FFT \item CENTERX I($\ast$) Center X pixel position per field (std default) \item CENTERY I($\ast$) Center Y pixel position per field (std default) \item CTYPX I X convolving function type (std default) \item XPARM R(10) X convolving function parameters( std default) \item CTYPY I Y convolving function type (std default) \item YPARM R(10) Y convolving function parameters (std default) \item DOZERO L IF true do Zero spacing flux (do if value given) \item ZEROSP R(5) Zero spacing parameters (no zero spacing flux) \item TFLUXG R Total flux to be subtracted from ZEROSP (0.0) \item DOTAPER L If true taper (do if non zero taper given) \item UVTAPER R(2) X and Y taper values (no taper) \end{itemize}
\item[{\em beam}]Array of beam AIPS object names MUST be previously instantiated. Members: \begin{itemize}
\item IMSIZE I(2) Size of beam (no default) \item SUMWTS R Sum of weights used for normalization (make beam) Set when beam gridded. \end{itemize}
\item[{\em dobeam}]if True, make a beam else make an image \item[{\em docreate}]if True, create beams and images underlying AIPS files \item[{\em chan}]First channel in uv data to image \item[{\em nchan}]Number of channels to \char`\"{}average\char`\"{} into the image \item[{\em imchan}]First channel number in output image or beam \item[{\em ierr}]return code, 0=$>$OK \end{description}
\end{Desc}
\index{ObitAIPSFortran.h@{Obit\-AIPSFortran.h}!obintx_@{obintx\_\-}}
\index{obintx_@{obintx\_\-}!ObitAIPSFortran.h@{Obit\-AIPSFortran.h}}
\subsubsection{\setlength{\rightskip}{0pt plus 5cm}void obintx\_\- (const gchar $\ast$ {\em pgm\-Name}, const {\bf oint} $\ast$ {\em len\-Pgm\-Name}, const {\bf oint} $\ast$ {\em pgm\-Number}, const {\bf oint} $\ast$ {\em AIPSuser}, const {\bf oint} $\ast$ {\em number\-AIPSdisk}, const gchar $\ast$ {\em AIPSdir}, const {\bf oint} $\ast$ {\em len\-AIPSdir}, const {\bf oint} $\ast$ {\em number\-FITSdisk}, const gchar $\ast$ {\em FITSdir}, const {\bf oint} $\ast$ {\em len\-FITSdir}, const {\bf oint} $\ast$ {\em F\_\-TRUE}, const {\bf oint} $\ast$ {\em F\_\-FALSE}, {\bf oint} $\ast$ {\em ierr})}\label{ObitAIPSFortran_8h_a0}


Public: Init {\bf Obit}{\rm (p.\,\pageref{structObit})} System. 

Initialize {\bf Obit\-System}{\rm (p.\,\pageref{structObitSystem})} information This should be called ONCE before any {\bf Obit}{\rm (p.\,\pageref{structObit})} routines are called. \begin{itemize}
\item Initialize AIPS disk information if any \item Initialize FITS disk information if any \item Initialize Scratch file list. \begin{Desc}
\item[Parameters:]
\begin{description}
\item[{\em pgm\-Name}]Name of program (max 5 char if AIPS) \item[{\em len\-Pgm\-Name}]Number of characters in pgm\-Name \item[{\em pgm\-Number}]Version number of program (e.g. POPS number). \item[{\em AIPSuser}]AIPS user number if using AIPS files \item[{\em number\-AIPSdisk}]Number of AIPS disks If 0, no AIPS files. \item[{\em AIPSdir}]List of AIPS directory names as long HOLLERITH string \item[{\em len\-AIPSdir}]Number of characters in each entry in AIPSdir \item[{\em number\-FITSdisk}]Number of FITS disks If 0, no FITS files. \item[{\em FITSdir}]List of FITS directory names as long HOLLERITH string \item[{\em len\-FITSdir}]Number of characters in each entry in FITSdir \item[{\em F\_\-TRUE}]Value of Fortran TRUE (used in Fortran interface) \item[{\em F\_\-FALSE}]Value of Fortran FALSE \item[{\em err}]{\bf Obit}{\rm (p.\,\pageref{structObit})} error stack for any error messages. \end{description}
\end{Desc}
\end{itemize}
\index{ObitAIPSFortran.h@{Obit\-AIPSFortran.h}!obiuvi_@{obiuvi\_\-}}
\index{obiuvi_@{obiuvi\_\-}!ObitAIPSFortran.h@{Obit\-AIPSFortran.h}}
\subsubsection{\setlength{\rightskip}{0pt plus 5cm}void obiuvi\_\- ({\bf AIPSObj} {\em uvdata}, {\bf oint} $\ast$ {\em ifield}, {\bf oint} $\ast$ {\em nfield}, {\bf AIPSObj} $\ast$ {\em image}, {\bf AIPSObj} $\ast$ {\em beam}, {\bf oint} $\ast$ {\em dobeam}, {\bf oint} $\ast$ {\em docreate}, {\bf oint} $\ast$ {\em chan}, {\bf oint} $\ast$ {\em nchan}, {\bf oint} $\ast$ {\em imchan}, {\bf oint} $\ast$ {\em ierr})}\label{ObitAIPSFortran_8h_a5}


Public: Make beams and image with possible ionospheric corrections. 

Optionally does Ionospheric calibration for each field using IN table attached to uvdata. Makes beams or images from a uv data set. If nfield is $<$= 0 then a beam is made. If an image is to be made the normalization factor is obtained from the beam, if absent, the beam is remade. The input uvdata is assumed to have been calibrated, selected and had any uniform weighting corrections applied. Note: output image and beam objects should exist (full instantation) prior to call.

Two methods of Fourier transform are available: FFT and DFT. The FFT method supports multiple fields and allows images to be a different size from the beam. The DFT method does a full 3D DFT but supports only a single field and the beam must be the same size as the image. DFT NYI. \begin{Desc}
\item[Parameters:]
\begin{description}
\item[{\em uvdata}]Name of uvdata object Members: \begin{itemize}
\item STOKES C$\ast$4 Desired Stokes parameter (I) \item UVRANGE R(2) UV range in kilo wavelengths (all) \item GUARDBND R(2) Fractional guardband areound edge of uv grid (0) \item DOIONS L If present and true then do ionospheric \item calibration using NI table IONTAB. \item If DOINONS then the following: \item IONTAB C$\ast$32 Name of Io\-Nosphere table \item UVINSCR C$\ast$32 Name of a scratch uv data file. \item NSUBARR I Number of subarrays in data to image \end{itemize}
\item[{\em ifield}]Field to image: 0 =$>$ all nfield (note 1 beam made on ifield = 1 when do\-Beam true (1) and DO3DIM false) \item[{\em nfield}]Number of fields. \item[{\em image}]Array of image AIPS object names Members of image(1): \begin{itemize}
\item FTTYPE C$\ast$4 Fourier transform type 'FFT' or 'DFT'. ('FFT') \item IMSIZE I(2,$\ast$) Image size per field (no default) \item CELLSIZE R(2) Cellsize in arcseconds in X and Y (no default) \item CHTYPE C$\ast$4 'LINE', or 'SUM ' for imaging ('SUM') \item SHIFT R(2) Shift in arcsec (DFT imaging) \item RASHIFT R($\ast$) X position shift in arcseconds per field (0) FFT \item DECSHIFT R($\ast$) Y position shift in arcseconds per field (0) FFT \item CENTERX I($\ast$) Center X pixel position per field (std default) \item CENTERY I($\ast$) Center Y pixel position per field (std default) \item CTYPX I X convolving function type (std default) \item XPARM R(10) X convolving function parameters( std default) \item CTYPY I Y convolving function type (std default) \item YPARM R(10) Y convolving function parameters (std default) \item DOZERO L IF true do Zero spacing flux (do if value given) \item ZEROSP R(5) Zero spacing parameters (no zero spacing flux) \item TFLUXG R Total flux to be subtracted from ZEROSP (0.0) \item DOTAPER L If true taper (do if non zero taper given) \item UVTAPER R(2) X and Y taper values (no taper) \end{itemize}
\item[{\em beam}]Array of beam AIPS object names Members: \begin{itemize}
\item IMSIZE I(2) Size of beam (no default) \item SUMWTS R Sum of weights used for normalization (make beam) Set when beam gridded. \end{itemize}
\item[{\em dobeam}]if True, make a beam else make an image \item[{\em docreate}]if True, create beams and images underlying AIPS files \item[{\em chan}]First channel in uv data to image \item[{\em nchan}]Number of channels to \char`\"{}average\char`\"{} into the image \item[{\em imchan}]First channel number in output image or beam \item[{\em ierr}]return code, 0=$>$OK \end{description}
\end{Desc}
\index{ObitAIPSFortran.h@{Obit\-AIPSFortran.h}!obshtx_@{obshtx\_\-}}
\index{obshtx_@{obshtx\_\-}!ObitAIPSFortran.h@{Obit\-AIPSFortran.h}}
\subsubsection{\setlength{\rightskip}{0pt plus 5cm}void obshtx\_\- (void)}\label{ObitAIPSFortran_8h_a1}


Public: Shutdown {\bf Obit}{\rm (p.\,\pageref{structObit})} System. 

Shutdown {\bf Obit\-System}{\rm (p.\,\pageref{structObitSystem})} information This should be called ONCE after all {\bf Obit}{\rm (p.\,\pageref{structObit})} routines are called. Any remaining scratch objects/files are deleted. \index{ObitAIPSFortran.h@{Obit\-AIPSFortran.h}!obufwt_@{obufwt\_\-}}
\index{obufwt_@{obufwt\_\-}!ObitAIPSFortran.h@{Obit\-AIPSFortran.h}}
\subsubsection{\setlength{\rightskip}{0pt plus 5cm}void obufwt\_\- ({\bf AIPSObj} {\em uv}, {\bf AIPSObj} {\em image}, {\bf oint} $\ast$ {\em ierr})}\label{ObitAIPSFortran_8h_a3}


Public: Determine and apply uniform weighting corrections to uv data. 

Determine and apply uniform weighting corrections to uv data Note: this routine weights by the sum of the weights in a weighting box rather than the number of counts. \begin{Desc}
\item[Parameters:]
\begin{description}
\item[{\em uv}]Name of uvdata, Members \begin{itemize}
\item CHINC I Channel increment (1) \item MAXBLINE R Maximum baseline length \end{itemize}
\item[{\em image}]Name of image defining grid size with members \begin{itemize}
\item CHTYPE C$\ast$4 'LINE' or 'SUM ' ('SUM ') \item UVWTFN C$\ast$2 Uniform weighting option ('UN') \item UVTAPER R(2) U, V taper \item UVBOX I Uniform weighting box (0) \item UVBXFN I box function type (1) 1=pillbox, 2=linear, 3=exponential, 4=Gaussian \item IMSIZE I(2) Number of pixels in X,Y direction. no default \item CELLSIZE R(2) Cellspacing in x,y in arcsec no default. \item NCHAV I Number of channels to be averaged. \item ROBUST R Brigg's weighting factor, -5-$>$Uniform, 5-$>$Natural (0) \end{itemize}
\item[{\em ierr}]return code, 0=$>$OK \end{description}
\end{Desc}
\index{ObitAIPSFortran.h@{Obit\-AIPSFortran.h}!obuvcp_@{obuvcp\_\-}}
\index{obuvcp_@{obuvcp\_\-}!ObitAIPSFortran.h@{Obit\-AIPSFortran.h}}
\subsubsection{\setlength{\rightskip}{0pt plus 5cm}void obuvcp\_\- ({\bf AIPSObj} {\em uvin}, {\bf AIPSObj} {\em uvout}, {\bf oint} $\ast$ {\em ierr})}\label{ObitAIPSFortran_8h_a2}


Public: Copy a uvdata set potentially applying calibration, editing and selection,. 

Copies uvdata from one object to another optionally applying calibration editing and selection/translation. \begin{Desc}
\item[Parameters:]
\begin{description}
\item[{\em uvin}]Name of input uvdata, Members \begin{itemize}
\item UMAX R Maximum acceptable U in wavelengths (default all) \item VMAX R Maximum acceptable V in wavelengths (default all) \item SOURCS C(30)$\ast$16 Names of up to 30 sources, $\ast$=$>$all \item First character of name '-' =$>$ all except \item those specified. \item SELQUA I Qualifier wanted (-1 =$>$ all) \item SELCOD C$\ast$4 Cal code (' ') \item TIMRNG R(8) Start day, hour, min, sec, end day, hour, \item min, sec. 0's =$>$ all \item UVRNG R(2) Minimum and maximum baseline lengths in \item 1000's wavelengths. 0's =$>$ all \item STOKES C$\ast$4 Stokes types wanted. \item 'I','Q','U','V','R','L','IQU','IQUV' \item ' '=$>$ Leave data in same form as in input. \item BCHAN I First channel number selected, 1 rel. to first \item channel in data base. 0 =$>$ all \item ECHAN I Last channel selected. 0=$>$all \item BIF I First IF number selected, 1 rel. to first \item IF in data base. 0 =$>$ all \item EIF I Last IF selected. 0=$>$all \item DOCAL I If $>$0 apply calibration, else not. \item DOPOL I If $>$0 then correct for feed polarization \item based on antenna file info. \item DOACOR L True if autocorrelations wanted (false) \item DOXCOR L True if cross-correlations wanted (true) \item DOWTCL L True if weight calibration wanted. \item DOFQSL L True if FREQSEL random parm present (false) \item FRQSEL I Default FQ table entry to select (-1) \item SELBAN R Bandwidth (Hz) to select (-1.0) \item SELFRQ R Frequency (Hz) to select (-1.0) \item DOBAND I $>$0 if bandpass calibration. (-1) \item DOSMTH L True if smoothing requested. (false) \item SMOOTH R(3) Smoothing parameters (0.0s) \item DXTIME R Integration time (days). Used when applying \item delay corrections to correct for delay error. \item ANTENS I(50) List of antennas selected, 0=$>$all, \item any negative =$>$ all except those specified \item SUBARR I Subarray desired, 0=$>$all \item FGVER I FLAG file version number, if $<$ 0 then \item NO flagging is applied. 0 =$>$ use highest \item numbered table. \item CLUSE I Cal (CL or SN) file version number to apply. \item BLVER I BL Table to apply .le. 0 =$>$ none \item BPVER I BP table to apply .le. 0 =$>$ none \end{itemize}
\item[{\em uvout}]Name of output uvdata \item[{\em ierr}]return code, 0=$>$OK \end{description}
\end{Desc}
