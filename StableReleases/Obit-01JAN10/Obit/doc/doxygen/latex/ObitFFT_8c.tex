\section{Obit\-FFT.c File Reference}
\label{ObitFFT_8c}\index{ObitFFT.c@{ObitFFT.c}}
{\bf Obit\-FFT}{\rm (p.\,\pageref{structObitFFT})} class function definitions. 

{\tt \#include $<$math.h$>$}\par
{\tt \#include \char`\"{}Obit\-FFT.h\char`\"{}}\par
{\tt \#include \char`\"{}Obit\-IOUVFITS.h\char`\"{}}\par
{\tt \#include \char`\"{}Obit\-IOUVAIPS.h\char`\"{}}\par
\subsection*{Functions}
\begin{CompactItemize}
\item 
void {\bf Obit\-FFTInit} (gpointer in)
\begin{CompactList}\small\item\em Private: Initialize newly instantiated object. \item\end{CompactList}\item 
void {\bf Obit\-FFTClear} (gpointer in)
\begin{CompactList}\small\item\em Private: Deallocate members. \item\end{CompactList}\item 
{\bf Obit\-FFT} $\ast$ {\bf new\-Obit\-FFT} (gchar $\ast$name, Obit\-FFTdir dir, Obit\-FFTtype type, {\bf olong} rank, {\bf olong} $\ast$dim)
\begin{CompactList}\small\item\em Public: Constructor. \item\end{CompactList}\item 
gconstpointer {\bf Obit\-FFTGet\-Class} (void)
\begin{CompactList}\small\item\em Public: Class\-Info pointer. \item\end{CompactList}\item 
{\bf olong} {\bf Obit\-FFTSuggest\-Size} ({\bf olong} length)
\begin{CompactList}\small\item\em Public: Suggest efficient size for a transform. \item\end{CompactList}\item 
void {\bf Obit\-FFTR2C} ({\bf Obit\-FFT} $\ast$in, {\bf Obit\-FArray} $\ast$in\-Array, {\bf Obit\-CArray} $\ast$out\-Array)
\begin{CompactList}\small\item\em Public: Real to half Complex. \item\end{CompactList}\item 
void {\bf Obit\-FFTC2R} ({\bf Obit\-FFT} $\ast$in, {\bf Obit\-CArray} $\ast$in\-Array, {\bf Obit\-FArray} $\ast$out\-Array)
\begin{CompactList}\small\item\em Public: Half Complex to Real. \item\end{CompactList}\item 
void {\bf Obit\-FFTC2C} ({\bf Obit\-FFT} $\ast$in, {\bf Obit\-CArray} $\ast$in\-Array, {\bf Obit\-CArray} $\ast$out\-Array)
\begin{CompactList}\small\item\em Public: Full Complex to Complex. \item\end{CompactList}\item 
void {\bf Obit\-FFTClass\-Init} (void)
\begin{CompactList}\small\item\em Public: Class initializer. \item\end{CompactList}\end{CompactItemize}


\subsection{Detailed Description}
{\bf Obit\-FFT}{\rm (p.\,\pageref{structObitFFT})} class function definitions. 

This class is derived from the {\bf Obit}{\rm (p.\,\pageref{structObit})} base class and is based on FFTW.

\subsection{Function Documentation}
\index{ObitFFT.c@{Obit\-FFT.c}!newObitFFT@{newObitFFT}}
\index{newObitFFT@{newObitFFT}!ObitFFT.c@{Obit\-FFT.c}}
\subsubsection{\setlength{\rightskip}{0pt plus 5cm}{\bf Obit\-FFT}$\ast$ new\-Obit\-FFT (gchar $\ast$ {\em name}, Obit\-FFTdir {\em dir}, Obit\-FFTtype {\em type}, {\bf olong} {\em rank}, {\bf olong} $\ast$ {\em dim})}\label{ObitFFT_8c_a6}


Public: Constructor. 

Initializes class if needed on first call. \begin{Desc}
\item[Parameters:]
\begin{description}
\item[{\em name}]An optional name for the object. \item[{\em dir}]The direction of the transform OBIT\_\-FFT\_\-Forward (R2C) or OBIT\_\-FFT\_\-Reverse (C2R). \item[{\em type}]Whether OBIT\_\-FFT\_\-Full\-Complex (full C2C) or OBIT\_\-FFT\_\-Half\-Complex (R2C or C2R). \item[{\em rank}]of matrix range [1,7] \item[{\em dim}]dimensionality of each axis in column major (Fortran) order. If real/half complex is being used, then dim[0] should be the number of reals. \end{description}
\end{Desc}
\begin{Desc}
\item[Returns:]the new object. \end{Desc}
\index{ObitFFT.c@{Obit\-FFT.c}!ObitFFTC2C@{ObitFFTC2C}}
\index{ObitFFTC2C@{ObitFFTC2C}!ObitFFT.c@{Obit\-FFT.c}}
\subsubsection{\setlength{\rightskip}{0pt plus 5cm}void Obit\-FFTC2C ({\bf Obit\-FFT} $\ast$ {\em in}, {\bf Obit\-CArray} $\ast$ {\em in\-Array}, {\bf Obit\-CArray} $\ast$ {\em out\-Array})}\label{ObitFFT_8c_a11}


Public: Full Complex to Complex. 

Must have been created with dir = OBIT\_\-FFT\_\-Reverse have same geometry as constructor call. Transform is in the direction specified in constructor call. \begin{Desc}
\item[Parameters:]
\begin{description}
\item[{\em in}]Object with FFT structures. \item[{\em in\-Array}]Array to be transformed (disturbed on output). \item[{\em out\-Array}]Output array \end{description}
\end{Desc}
\index{ObitFFT.c@{Obit\-FFT.c}!ObitFFTC2R@{ObitFFTC2R}}
\index{ObitFFTC2R@{ObitFFTC2R}!ObitFFT.c@{Obit\-FFT.c}}
\subsubsection{\setlength{\rightskip}{0pt plus 5cm}void Obit\-FFTC2R ({\bf Obit\-FFT} $\ast$ {\em in}, {\bf Obit\-CArray} $\ast$ {\em in\-Array}, {\bf Obit\-FArray} $\ast$ {\em out\-Array})}\label{ObitFFT_8c_a10}


Public: Half Complex to Real. 

Must have been created with dir = OBIT\_\-FFT\_\-Reverse and type = OBIT\_\-FFT\_\-Half\-Complex and have same geometry as constructor call. Note: FFT returned is not normalized. \begin{Desc}
\item[Parameters:]
\begin{description}
\item[{\em in}]Object with FFT structures. \item[{\em in\-Array}]Array to be transformed (disturbed on output). \item[{\em out\-Array}]Output array \end{description}
\end{Desc}
\index{ObitFFT.c@{Obit\-FFT.c}!ObitFFTClassInit@{ObitFFTClassInit}}
\index{ObitFFTClassInit@{ObitFFTClassInit}!ObitFFT.c@{Obit\-FFT.c}}
\subsubsection{\setlength{\rightskip}{0pt plus 5cm}void Obit\-FFTClass\-Init (void)}\label{ObitFFT_8c_a12}


Public: Class initializer. 

\index{ObitFFT.c@{Obit\-FFT.c}!ObitFFTClear@{ObitFFTClear}}
\index{ObitFFTClear@{ObitFFTClear}!ObitFFT.c@{Obit\-FFT.c}}
\subsubsection{\setlength{\rightskip}{0pt plus 5cm}void Obit\-FFTClear (gpointer {\em inn})}\label{ObitFFT_8c_a4}


Private: Deallocate members. 

Does (recursive) deallocation of parent class members. \begin{Desc}
\item[Parameters:]
\begin{description}
\item[{\em inn}]Pointer to the object to deallocate. Actually it should be an Obit\-FFT$\ast$ cast to an Obit$\ast$. \end{description}
\end{Desc}
\index{ObitFFT.c@{Obit\-FFT.c}!ObitFFTGetClass@{ObitFFTGetClass}}
\index{ObitFFTGetClass@{ObitFFTGetClass}!ObitFFT.c@{Obit\-FFT.c}}
\subsubsection{\setlength{\rightskip}{0pt plus 5cm}gconstpointer Obit\-FFTGet\-Class (void)}\label{ObitFFT_8c_a7}


Public: Class\-Info pointer. 

\begin{Desc}
\item[Returns:]pointer to the class structure. \end{Desc}
\index{ObitFFT.c@{Obit\-FFT.c}!ObitFFTInit@{ObitFFTInit}}
\index{ObitFFTInit@{ObitFFTInit}!ObitFFT.c@{Obit\-FFT.c}}
\subsubsection{\setlength{\rightskip}{0pt plus 5cm}void Obit\-FFTInit (gpointer {\em inn})}\label{ObitFFT_8c_a3}


Private: Initialize newly instantiated object. 

Parent classes portions are (recursively) initialized first \begin{Desc}
\item[Parameters:]
\begin{description}
\item[{\em inn}]Pointer to the object to initialize. \end{description}
\end{Desc}
\index{ObitFFT.c@{Obit\-FFT.c}!ObitFFTR2C@{ObitFFTR2C}}
\index{ObitFFTR2C@{ObitFFTR2C}!ObitFFT.c@{Obit\-FFT.c}}
\subsubsection{\setlength{\rightskip}{0pt plus 5cm}void Obit\-FFTR2C ({\bf Obit\-FFT} $\ast$ {\em in}, {\bf Obit\-FArray} $\ast$ {\em in\-Array}, {\bf Obit\-CArray} $\ast$ {\em out\-Array})}\label{ObitFFT_8c_a9}


Public: Real to half Complex. 

Must have been created with dir = OBIT\_\-FFT\_\-Forward and type = OBIT\_\-FFT\_\-Half\-Complex and have same geometry as constructor call. \begin{Desc}
\item[Parameters:]
\begin{description}
\item[{\em in}]Object with FFT structures. \item[{\em in\-Array}]Array to be transformed (undisturbed on output). \item[{\em out\-Array}]Output array \end{description}
\end{Desc}
\index{ObitFFT.c@{Obit\-FFT.c}!ObitFFTSuggestSize@{ObitFFTSuggestSize}}
\index{ObitFFTSuggestSize@{ObitFFTSuggestSize}!ObitFFT.c@{Obit\-FFT.c}}
\subsubsection{\setlength{\rightskip}{0pt plus 5cm}{\bf olong} Obit\-FFTSuggest\-Size ({\bf olong} {\em length})}\label{ObitFFT_8c_a8}


Public: Suggest efficient size for a transform. 

\begin{Desc}
\item[Parameters:]
\begin{description}
\item[{\em length}]number of values to be transformed \end{description}
\end{Desc}
\begin{Desc}
\item[Returns:]a number equal or larger than length that will have an efficient transform. \end{Desc}
