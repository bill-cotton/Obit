\section{Obit\-IOTable\-FITS.h File Reference}
\label{ObitIOTableFITS_8h}\index{ObitIOTableFITS.h@{ObitIOTableFITS.h}}
{\bf Obit\-IOTable\-FITS}{\rm (p.\,\pageref{structObitIOTableFITS})} class definition. 

{\tt \#include \char`\"{}fitsio.h\char`\"{}}\par
{\tt \#include \char`\"{}Obit.h\char`\"{}}\par
{\tt \#include \char`\"{}Obit\-IO.h\char`\"{}}\par
{\tt \#include \char`\"{}Obit\-Table\-Desc.h\char`\"{}}\par
{\tt \#include \char`\"{}Obit\-Table\-Sel.h\char`\"{}}\par
{\tt \#include \char`\"{}Obit\-IOTable\-FITSDef.h\char`\"{}}\par
\subsection*{Classes}
\begin{CompactItemize}
\item 
struct {\bf Obit\-IOTable\-FITS}
\begin{CompactList}\small\item\em Obit\-IOTable\-FITS Class structure. \item\end{CompactList}\item 
struct {\bf Obit\-IOTable\-FITSClass\-Info}
\begin{CompactList}\small\item\em Class\-Info Structure. \item\end{CompactList}\end{CompactItemize}
\subsection*{Defines}
\begin{CompactItemize}
\item 
\#define {\bf Obit\-IOTable\-FITSUnref}(in)\ Obit\-Unref (in)
\begin{CompactList}\small\item\em Macro to unreference (and possibly destroy) an {\bf Obit\-IOTable\-FITS}{\rm (p.\,\pageref{structObitIOTableFITS})} returns a Obit\-IOTable\-Image\-FITS$\ast$ (NULL). \item\end{CompactList}\item 
\#define {\bf Obit\-IOTable\-FITSRef}(in)\ Obit\-Ref (in)
\begin{CompactList}\small\item\em Macro to reference (update reference count) an {\bf Obit\-IOTable\-FITS}{\rm (p.\,\pageref{structObitIOTableFITS})}. \item\end{CompactList}\item 
\#define {\bf Obit\-IOTable\-FITSIs\-A}(in)\ Obit\-Is\-A (in, Obit\-IOTable\-FITSGet\-Class())
\begin{CompactList}\small\item\em Macro to determine if an object is the member of this or a derived class. \item\end{CompactList}\end{CompactItemize}
\subsection*{Functions}
\begin{CompactItemize}
\item 
void {\bf Obit\-IOTable\-FITSClass\-Init} (void)
\begin{CompactList}\small\item\em Public: Class initializer. \item\end{CompactList}\item 
{\bf Obit\-IOTable\-FITS} $\ast$ {\bf new\-Obit\-IOTable\-FITS} (gchar $\ast$name, {\bf Obit\-Info\-List} $\ast$info, {\bf Obit\-Err} $\ast$err)
\begin{CompactList}\small\item\em Public: Constructor. \item\end{CompactList}\item 
gconstpointer {\bf Obit\-IOTable\-FITSGet\-Class} (void)
\begin{CompactList}\small\item\em Public: Class\-Info pointer. \item\end{CompactList}\item 
gboolean {\bf Obit\-IOTable\-FITSSame} ({\bf Obit\-IO} $\ast$in, {\bf Obit\-Info\-List} $\ast$in1, {\bf Obit\-Info\-List} $\ast$in2, {\bf Obit\-Err} $\ast$err)
\begin{CompactList}\small\item\em Public: Are underlying structures the same. \item\end{CompactList}\item 
void {\bf Obit\-IOTable\-FITSZap} ({\bf Obit\-IOTable\-FITS} $\ast$in, {\bf Obit\-Err} $\ast$err)
\begin{CompactList}\small\item\em Public: Delete underlying structures. \item\end{CompactList}\item 
{\bf Obit\-IOTable\-FITS} $\ast$ {\bf Obit\-IOTable\-FITSCopy} ({\bf Obit\-IOTable\-FITS} $\ast$in, {\bf Obit\-IOTable\-FITS} $\ast$out, {\bf Obit\-Err} $\ast$err)
\begin{CompactList}\small\item\em Public: Copy constructor. \item\end{CompactList}\item 
Obit\-IOCode {\bf Obit\-IOTable\-FITSOpen} ({\bf Obit\-IOTable\-FITS} $\ast$in, Obit\-IOAccess access, {\bf Obit\-Info\-List} $\ast$info, {\bf Obit\-Err} $\ast$err)
\begin{CompactList}\small\item\em Public: Open. \item\end{CompactList}\item 
Obit\-IOCode {\bf Obit\-IOTable\-FITSClose} ({\bf Obit\-IOTable\-FITS} $\ast$in, {\bf Obit\-Err} $\ast$err)
\begin{CompactList}\small\item\em Public: Close. \item\end{CompactList}\item 
Obit\-IOCode {\bf Obit\-IOTable\-FITSSet} ({\bf Obit\-IOTable\-FITS} $\ast$in, {\bf Obit\-Info\-List} $\ast$info, {\bf Obit\-Err} $\ast$err)
\begin{CompactList}\small\item\em Public: Init I/O. \item\end{CompactList}\item 
Obit\-IOCode {\bf Obit\-IOTable\-FITSRead} ({\bf Obit\-IOTable\-FITS} $\ast$in, {\bf ofloat} $\ast$data, {\bf Obit\-Err} $\ast$err)
\begin{CompactList}\small\item\em Public: Read. \item\end{CompactList}\item 
Obit\-IOCode {\bf Obit\-IOTable\-FITSRead\-Row} ({\bf Obit\-IOTable\-FITS} $\ast$in, {\bf olong} rowno, {\bf ofloat} $\ast$data, {\bf Obit\-Err} $\ast$err)
\begin{CompactList}\small\item\em Public: Read specifying start row. \item\end{CompactList}\item 
Obit\-IOCode {\bf Obit\-IOTable\-FITSWrite} ({\bf Obit\-IOTable\-FITS} $\ast$in, {\bf ofloat} $\ast$data, {\bf Obit\-Err} $\ast$err)
\begin{CompactList}\small\item\em Public: Write. \item\end{CompactList}\item 
Obit\-IOCode {\bf Obit\-IOTable\-FITSWrite\-Row} ({\bf Obit\-IOTable\-FITS} $\ast$in, {\bf olong} rowno, {\bf ofloat} $\ast$data, {\bf Obit\-Err} $\ast$err)
\begin{CompactList}\small\item\em Public: Write specifying start row. \item\end{CompactList}\item 
Obit\-IOCode {\bf Obit\-IOTable\-FITSFlush} ({\bf Obit\-IOTable\-FITS} $\ast$in, {\bf Obit\-Err} $\ast$err)
\begin{CompactList}\small\item\em Public: Flush. \item\end{CompactList}\item 
Obit\-IOCode {\bf Obit\-IOTable\-FITSRead\-Descriptor} ({\bf Obit\-IOTable\-FITS} $\ast$in, {\bf Obit\-Err} $\ast$err)
\begin{CompactList}\small\item\em Public: Read Descriptor. \item\end{CompactList}\item 
Obit\-IOCode {\bf Obit\-IOTable\-FITSWrite\-Descriptor} ({\bf Obit\-IOTable\-FITS} $\ast$in, {\bf Obit\-Err} $\ast$err)
\begin{CompactList}\small\item\em Public: Write Descriptor. \item\end{CompactList}\item 
void {\bf Obit\-IOTable\-FITSCreate\-Buffer} ({\bf ofloat} $\ast$$\ast$data, {\bf olong} $\ast$size, {\bf Obit\-IOTable\-FITS} $\ast$in, {\bf Obit\-Info\-List} $\ast$info, {\bf Obit\-Err} $\ast$err)
\begin{CompactList}\small\item\em Public: Create buffer. \item\end{CompactList}\item 
void {\bf Obit\-IOTable\-FITSGet\-File\-Info} ({\bf Obit\-IO} $\ast$in, {\bf Obit\-Info\-List} $\ast$my\-Info, gchar $\ast$prefix, {\bf Obit\-Info\-List} $\ast$out\-List, {\bf Obit\-Err} $\ast$err)
\begin{CompactList}\small\item\em Public: Extract information about underlying file. \item\end{CompactList}\end{CompactItemize}


\subsection{Detailed Description}
{\bf Obit\-IOTable\-FITS}{\rm (p.\,\pageref{structObitIOTableFITS})} class definition. 

This class provides an interface to the cfitsio package for FITS images.

This class is derived from the {\bf Obit\-IO}{\rm (p.\,\pageref{structObitIO})} class. Related functions are in the {\bf Obit\-IOTable\-FITSUtil }{\rm (p.\,\pageref{ObitIOTableFITSUtil_8h})} module.\subsection{Usage}\label{ObitIOTableFITS_8h_ObitIOTableFITSUsage}
Instances of this class are for access to FITS image files using the cfitsio package. Instances can be made using the \$new\-Obit\-IOTable\-FITS constructor, or the {\bf Obit\-IOTable\-FITSCopy}{\rm (p.\,\pageref{ObitIOTableFITS_8c_a17})} copy constructor and pointers copied (with reference pointer update) using {\bf Obit\-IORef}{\rm (p.\,\pageref{ObitIO_8h_a1})}. The destructor (when reference count goes to zero) is {\bf Obit\-IOUnref}{\rm (p.\,\pageref{ObitIO_8h_a0})}. This class should seldom need be accessed directly outside of the {\bf Obit\-IO}{\rm (p.\,\pageref{structObitIO})} class. Parameters needed (passed via {\bf Obit\-Info\-List}{\rm (p.\,\pageref{structObitInfoList})}) are: \begin{itemize}
\item \char`\"{}BLC\char`\"{} OBIT\_\-int (?,1,1) the bottom-left corner desired as expressed in 1-rel pixel indices. If absent, the value (1,1,1...) will be assumed. dimension of this array is [IM\_\-MAXDIM]. \item \char`\"{}TRC\char`\"{} OBIT\_\-int (?,1,1) the top-right corner desired as expressed in 1-rel pixel indices. If absent, all pixels are included. dimension of this array is [IM\_\-MAXDIM]. \item \char`\"{}IOBy\char`\"{} OBIT\_\-int (1,1,1) an Obit\-IOSize enum defined in {\bf Obit\-IO.h}{\rm (p.\,\pageref{ObitIO_8h})} giving values OBIT\_\-IO\_\-by\-Row or OBIT\_\-IO\_\-by\-Plane to specify if the data transfers are to be by row or plane at a time. \item \char`\"{}File\-Name\char`\"{} OBIT\_\-string (?,1,1) Name of disk file.\end{itemize}


\subsection{Define Documentation}
\index{ObitIOTableFITS.h@{Obit\-IOTable\-FITS.h}!ObitIOTableFITSIsA@{ObitIOTableFITSIsA}}
\index{ObitIOTableFITSIsA@{ObitIOTableFITSIsA}!ObitIOTableFITS.h@{Obit\-IOTable\-FITS.h}}
\subsubsection{\setlength{\rightskip}{0pt plus 5cm}\#define Obit\-IOTable\-FITSIs\-A(in)\ Obit\-Is\-A (in, Obit\-IOTable\-FITSGet\-Class())}\label{ObitIOTableFITS_8h_a2}


Macro to determine if an object is the member of this or a derived class. 

Returns TRUE if a member, else FALSE in = object to reference \index{ObitIOTableFITS.h@{Obit\-IOTable\-FITS.h}!ObitIOTableFITSRef@{ObitIOTableFITSRef}}
\index{ObitIOTableFITSRef@{ObitIOTableFITSRef}!ObitIOTableFITS.h@{Obit\-IOTable\-FITS.h}}
\subsubsection{\setlength{\rightskip}{0pt plus 5cm}\#define Obit\-IOTable\-FITSRef(in)\ Obit\-Ref (in)}\label{ObitIOTableFITS_8h_a1}


Macro to reference (update reference count) an {\bf Obit\-IOTable\-FITS}{\rm (p.\,\pageref{structObitIOTableFITS})}. 

returns a Obit\-IOTable\-FITS$\ast$. in = object to reference \index{ObitIOTableFITS.h@{Obit\-IOTable\-FITS.h}!ObitIOTableFITSUnref@{ObitIOTableFITSUnref}}
\index{ObitIOTableFITSUnref@{ObitIOTableFITSUnref}!ObitIOTableFITS.h@{Obit\-IOTable\-FITS.h}}
\subsubsection{\setlength{\rightskip}{0pt plus 5cm}\#define Obit\-IOTable\-FITSUnref(in)\ Obit\-Unref (in)}\label{ObitIOTableFITS_8h_a0}


Macro to unreference (and possibly destroy) an {\bf Obit\-IOTable\-FITS}{\rm (p.\,\pageref{structObitIOTableFITS})} returns a Obit\-IOTable\-Image\-FITS$\ast$ (NULL). 

in = object to unreference. 

\subsection{Function Documentation}
\index{ObitIOTableFITS.h@{Obit\-IOTable\-FITS.h}!newObitIOTableFITS@{newObitIOTableFITS}}
\index{newObitIOTableFITS@{newObitIOTableFITS}!ObitIOTableFITS.h@{Obit\-IOTable\-FITS.h}}
\subsubsection{\setlength{\rightskip}{0pt plus 5cm}{\bf Obit\-IOTable\-FITS}$\ast$ new\-Obit\-IOTable\-FITS (gchar $\ast$ {\em name}, {\bf Obit\-Info\-List} $\ast$ {\em info}, {\bf Obit\-Err} $\ast$ {\em err})}\label{ObitIOTableFITS_8h_a4}


Public: Constructor. 

Initializes class on the first call. \begin{Desc}
\item[Parameters:]
\begin{description}
\item[{\em name}]An optional name for the object. \item[{\em info}]if non-NULL it is used to initialize the new object. \item[{\em err}]{\bf Obit\-Err}{\rm (p.\,\pageref{structObitErr})} for error messages. \end{description}
\end{Desc}
\begin{Desc}
\item[Returns:]the new object. \end{Desc}
\index{ObitIOTableFITS.h@{Obit\-IOTable\-FITS.h}!ObitIOTableFITSClassInit@{ObitIOTableFITSClassInit}}
\index{ObitIOTableFITSClassInit@{ObitIOTableFITSClassInit}!ObitIOTableFITS.h@{Obit\-IOTable\-FITS.h}}
\subsubsection{\setlength{\rightskip}{0pt plus 5cm}void Obit\-IOTable\-FITSClass\-Init (void)}\label{ObitIOTableFITS_8h_a3}


Public: Class initializer. 

\index{ObitIOTableFITS.h@{Obit\-IOTable\-FITS.h}!ObitIOTableFITSClose@{ObitIOTableFITSClose}}
\index{ObitIOTableFITSClose@{ObitIOTableFITSClose}!ObitIOTableFITS.h@{Obit\-IOTable\-FITS.h}}
\subsubsection{\setlength{\rightskip}{0pt plus 5cm}Obit\-IOCode Obit\-IOTable\-FITSClose ({\bf Obit\-IOTable\-FITS} $\ast$ {\em in}, {\bf Obit\-Err} $\ast$ {\em err})}\label{ObitIOTableFITS_8h_a10}


Public: Close. 

\begin{Desc}
\item[Parameters:]
\begin{description}
\item[{\em in}]Pointer to object to be closed. \item[{\em err}]{\bf Obit\-Err}{\rm (p.\,\pageref{structObitErr})} for reporting errors. \end{description}
\end{Desc}
\begin{Desc}
\item[Returns:]error code, 0=$>$ OK \end{Desc}
\index{ObitIOTableFITS.h@{Obit\-IOTable\-FITS.h}!ObitIOTableFITSCopy@{ObitIOTableFITSCopy}}
\index{ObitIOTableFITSCopy@{ObitIOTableFITSCopy}!ObitIOTableFITS.h@{Obit\-IOTable\-FITS.h}}
\subsubsection{\setlength{\rightskip}{0pt plus 5cm}{\bf Obit\-IOTable\-FITS}$\ast$ Obit\-IOTable\-FITSCopy ({\bf Obit\-IOTable\-FITS} $\ast$ {\em in}, {\bf Obit\-IOTable\-FITS} $\ast$ {\em out}, {\bf Obit\-Err} $\ast$ {\em err})}\label{ObitIOTableFITS_8h_a8}


Public: Copy constructor. 

The result will have pointers to the more complex members. Parent class members are included but any derived class info is ignored. \begin{Desc}
\item[Parameters:]
\begin{description}
\item[{\em in}]The object to copy \item[{\em out}]An existing object pointer for output or NULL if none exists. \item[{\em err}]{\bf Obit}{\rm (p.\,\pageref{structObit})} error stack object. \end{description}
\end{Desc}
\begin{Desc}
\item[Returns:]pointer to the new object. \end{Desc}
\index{ObitIOTableFITS.h@{Obit\-IOTable\-FITS.h}!ObitIOTableFITSCreateBuffer@{ObitIOTableFITSCreateBuffer}}
\index{ObitIOTableFITSCreateBuffer@{ObitIOTableFITSCreateBuffer}!ObitIOTableFITS.h@{Obit\-IOTable\-FITS.h}}
\subsubsection{\setlength{\rightskip}{0pt plus 5cm}void Obit\-IOTable\-FITSCreate\-Buffer ({\bf ofloat} $\ast$$\ast$ {\em data}, {\bf olong} $\ast$ {\em size}, {\bf Obit\-IOTable\-FITS} $\ast$ {\em in}, {\bf Obit\-Info\-List} $\ast$ {\em info}, {\bf Obit\-Err} $\ast$ {\em err})}\label{ObitIOTableFITS_8h_a19}


Public: Create buffer. 

Should be called after {\bf Obit\-IO}{\rm (p.\,\pageref{structObitIO})} is opened. \begin{Desc}
\item[Parameters:]
\begin{description}
\item[{\em data}](output) pointer to data array \item[{\em size}](output) size of data array in bytes. \item[{\em in}]Pointer to object to be accessed. \item[{\em info}]{\bf Obit\-Info\-List}{\rm (p.\,\pageref{structObitInfoList})} with instructions \item[{\em err}]{\bf Obit\-Err}{\rm (p.\,\pageref{structObitErr})} for reporting errors. \end{description}
\end{Desc}
\index{ObitIOTableFITS.h@{Obit\-IOTable\-FITS.h}!ObitIOTableFITSFlush@{ObitIOTableFITSFlush}}
\index{ObitIOTableFITSFlush@{ObitIOTableFITSFlush}!ObitIOTableFITS.h@{Obit\-IOTable\-FITS.h}}
\subsubsection{\setlength{\rightskip}{0pt plus 5cm}Obit\-IOCode Obit\-IOTable\-FITSFlush ({\bf Obit\-IOTable\-FITS} $\ast$ {\em in}, {\bf Obit\-Err} $\ast$ {\em err})}\label{ObitIOTableFITS_8h_a16}


Public: Flush. 

\begin{Desc}
\item[Parameters:]
\begin{description}
\item[{\em in}]Pointer to object to be accessed. \item[{\em err}]{\bf Obit\-Err}{\rm (p.\,\pageref{structObitErr})} for reporting errors. \end{description}
\end{Desc}
\begin{Desc}
\item[Returns:]return code, 0=$>$ OK \end{Desc}
\index{ObitIOTableFITS.h@{Obit\-IOTable\-FITS.h}!ObitIOTableFITSGetClass@{ObitIOTableFITSGetClass}}
\index{ObitIOTableFITSGetClass@{ObitIOTableFITSGetClass}!ObitIOTableFITS.h@{Obit\-IOTable\-FITS.h}}
\subsubsection{\setlength{\rightskip}{0pt plus 5cm}gconstpointer Obit\-IOTable\-FITSGet\-Class (void)}\label{ObitIOTableFITS_8h_a5}


Public: Class\-Info pointer. 

Initializes class if needed on first call. \begin{Desc}
\item[Returns:]pointer to the class structure. \end{Desc}
\index{ObitIOTableFITS.h@{Obit\-IOTable\-FITS.h}!ObitIOTableFITSGetFileInfo@{ObitIOTableFITSGetFileInfo}}
\index{ObitIOTableFITSGetFileInfo@{ObitIOTableFITSGetFileInfo}!ObitIOTableFITS.h@{Obit\-IOTable\-FITS.h}}
\subsubsection{\setlength{\rightskip}{0pt plus 5cm}void Obit\-IOTable\-FITSGet\-File\-Info ({\bf Obit\-IO} $\ast$ {\em in}, {\bf Obit\-Info\-List} $\ast$ {\em my\-Info}, gchar $\ast$ {\em prefix}, {\bf Obit\-Info\-List} $\ast$ {\em out\-List}, {\bf Obit\-Err} $\ast$ {\em err})}\label{ObitIOTableFITS_8h_a20}


Public: Extract information about underlying file. 

\begin{Desc}
\item[Parameters:]
\begin{description}
\item[{\em in}]Object of interest. \item[{\em my\-Info}]Info\-List on basic object with selection \item[{\em prefix}]If Non\-Null, string to be added to beginning of out\-List entry name \item[{\em out\-List}]Info\-List to write entries into Following entries for FITS files (\char`\"{}xxx\char`\"{} = prefix): \begin{itemize}
\item xxx\-File\-Name OBIT\_\-string FITS file name \item xxx\-Disk OBIT\_\-oint FITS file disk number \item xxx\-Dir OBIT\_\-string Directory name for xxx\-Disk\end{itemize}
For all File types types: \begin{itemize}
\item xxx\-Data\-Type OBIT\_\-string \char`\"{}UV\char`\"{} = UV data, \char`\"{}MA\char`\"{}=$>$image, \char`\"{}Table\char`\"{}=Table, \char`\"{}OTF\char`\"{}=OTF, etc \item xxx\-File\-Type OBIT\_\-oint File type as \char`\"{}FITS\char`\"{}\end{itemize}
For xxx\-Data\-Type = \char`\"{}Table\char`\"{} \begin{itemize}
\item xxx\-Table\-Parent OBIT\_\-string Table parent type (e.g. \char`\"{}MA\char`\"{}, \char`\"{}UV\char`\"{}) \item xxx\-Tab OBIT\_\-string (Tables only) Table type (e.g. \char`\"{}AIPS CC\char`\"{}) \item xxx\-Ver OBIT\_\-oint (Tables Only) Table version number\end{itemize}
\item[{\em err}]{\bf Obit\-Err}{\rm (p.\,\pageref{structObitErr})} for reporting errors. \end{description}
\end{Desc}
\index{ObitIOTableFITS.h@{Obit\-IOTable\-FITS.h}!ObitIOTableFITSOpen@{ObitIOTableFITSOpen}}
\index{ObitIOTableFITSOpen@{ObitIOTableFITSOpen}!ObitIOTableFITS.h@{Obit\-IOTable\-FITS.h}}
\subsubsection{\setlength{\rightskip}{0pt plus 5cm}Obit\-IOCode Obit\-IOTable\-FITSOpen ({\bf Obit\-IOTable\-FITS} $\ast$ {\em in}, Obit\-IOAccess {\em access}, {\bf Obit\-Info\-List} $\ast$ {\em info}, {\bf Obit\-Err} $\ast$ {\em err})}\label{ObitIOTableFITS_8h_a9}


Public: Open. 

The file etc. info should have been stored in the {\bf Obit\-Info\-List}{\rm (p.\,\pageref{structObitInfoList})}. The table descriptor is read if Read\-Only or Read\-Only and written to disk if opened Write\-Only. For accessing FITS files the following entries in the {\bf Obit\-Info\-List}{\rm (p.\,\pageref{structObitInfoList})} are used: \begin{itemize}
\item \char`\"{}Disk\char`\"{} OBIT\_\-int (1,1,1) FITS \char`\"{}disk\char`\"{} number. \item \char`\"{}File\-Name\char`\"{} OBIT\_\-string (?,1,1) FITS file name. \item \char`\"{}Tab\-Name\char`\"{} OBIT\_\-string (?,1,1) Table name (e.g. \char`\"{}AIPS CC\char`\"{}). \item \char`\"{}Ver\char`\"{} OBIT\_\-int (1,1,1) Table version number \begin{Desc}
\item[Parameters:]
\begin{description}
\item[{\em in}]Pointer to object to be opened. \item[{\em access}]access (OBIT\_\-IO\_\-Read\-Only,OBIT\_\-IO\_\-Read\-Write) \item[{\em info}]{\bf Obit\-Info\-List}{\rm (p.\,\pageref{structObitInfoList})} with instructions for opening \item[{\em err}]{\bf Obit\-Err}{\rm (p.\,\pageref{structObitErr})} for reporting errors. \end{description}
\end{Desc}
\begin{Desc}
\item[Returns:]return code, 0=$>$ OK \end{Desc}
\end{itemize}
\index{ObitIOTableFITS.h@{Obit\-IOTable\-FITS.h}!ObitIOTableFITSRead@{ObitIOTableFITSRead}}
\index{ObitIOTableFITSRead@{ObitIOTableFITSRead}!ObitIOTableFITS.h@{Obit\-IOTable\-FITS.h}}
\subsubsection{\setlength{\rightskip}{0pt plus 5cm}Obit\-IOCode Obit\-IOTable\-FITSRead ({\bf Obit\-IOTable\-FITS} $\ast$ {\em in}, {\bf ofloat} $\ast$ {\em data}, {\bf Obit\-Err} $\ast$ {\em err})}\label{ObitIOTableFITS_8h_a12}


Public: Read. 

When OBIT\_\-IO\_\-EOF is returned all data has been read (then is no new data in buffer) and the I/O has been closed. If there are existing rows in the buffer marked as modified (\char`\"{}\_\-status\char`\"{} column value =1) the buffer is rewritten to disk before the new buffer is read. \begin{Desc}
\item[Parameters:]
\begin{description}
\item[{\em in}]Pointer to object to be read. \item[{\em data}]pointer to buffer to receive results. \item[{\em err}]{\bf Obit\-Err}{\rm (p.\,\pageref{structObitErr})} for reporting errors. \end{description}
\end{Desc}
\begin{Desc}
\item[Returns:]return code, 0(OBIT\_\-IO\_\-OK)=$>$ OK, OBIT\_\-IO\_\-EOF =$>$ image finished. \end{Desc}
\index{ObitIOTableFITS.h@{Obit\-IOTable\-FITS.h}!ObitIOTableFITSReadDescriptor@{ObitIOTableFITSReadDescriptor}}
\index{ObitIOTableFITSReadDescriptor@{ObitIOTableFITSReadDescriptor}!ObitIOTableFITS.h@{Obit\-IOTable\-FITS.h}}
\subsubsection{\setlength{\rightskip}{0pt plus 5cm}Obit\-IOCode Obit\-IOTable\-FITSRead\-Descriptor ({\bf Obit\-IOTable\-FITS} $\ast$ {\em in}, {\bf Obit\-Err} $\ast$ {\em err})}\label{ObitIOTableFITS_8h_a17}


Public: Read Descriptor. 

If the table version number is 0, then the highest numbered table of the same name is used. \begin{Desc}
\item[Parameters:]
\begin{description}
\item[{\em in}]Pointer to object with {\bf Obit\-Table\-Desc}{\rm (p.\,\pageref{structObitTableDesc})} to be read. \item[{\em err}]{\bf Obit\-Err}{\rm (p.\,\pageref{structObitErr})} for reporting errors. \end{description}
\end{Desc}
\begin{Desc}
\item[Returns:]return code, 0=$>$ OK \end{Desc}
\index{ObitIOTableFITS.h@{Obit\-IOTable\-FITS.h}!ObitIOTableFITSReadRow@{ObitIOTableFITSReadRow}}
\index{ObitIOTableFITSReadRow@{ObitIOTableFITSReadRow}!ObitIOTableFITS.h@{Obit\-IOTable\-FITS.h}}
\subsubsection{\setlength{\rightskip}{0pt plus 5cm}Obit\-IOCode Obit\-IOTable\-FITSRead\-Row ({\bf Obit\-IOTable\-FITS} $\ast$ {\em in}, {\bf olong} {\em rowno}, {\bf ofloat} $\ast$ {\em data}, {\bf Obit\-Err} $\ast$ {\em err})}\label{ObitIOTableFITS_8h_a13}


Public: Read specifying start row. 

When OBIT\_\-IO\_\-EOF is returned all data has been read (then is no new data in buffer) and the I/O has been closed. If there are existing rows in the buffer marked as modified (\char`\"{}\_\-status\char`\"{} column value =1) the buffer is rewritten to disk before the new buffer is read. \begin{Desc}
\item[Parameters:]
\begin{description}
\item[{\em in}]Pointer to object to be read. \item[{\em rowno}]Starting row number (1-rel) -1=$>$ next. \item[{\em data}]pointer to buffer to receive results. \item[{\em err}]{\bf Obit\-Err}{\rm (p.\,\pageref{structObitErr})} for reporting errors. \end{description}
\end{Desc}
\begin{Desc}
\item[Returns:]return code, 0(OBIT\_\-IO\_\-OK)=$>$ OK, OBIT\_\-IO\_\-EOF =$>$ image finished. \end{Desc}
\index{ObitIOTableFITS.h@{Obit\-IOTable\-FITS.h}!ObitIOTableFITSSame@{ObitIOTableFITSSame}}
\index{ObitIOTableFITSSame@{ObitIOTableFITSSame}!ObitIOTableFITS.h@{Obit\-IOTable\-FITS.h}}
\subsubsection{\setlength{\rightskip}{0pt plus 5cm}gboolean Obit\-IOTable\-FITSSame ({\bf Obit\-IO} $\ast$ {\em in}, {\bf Obit\-Info\-List} $\ast$ {\em in1}, {\bf Obit\-Info\-List} $\ast$ {\em in2}, {\bf Obit\-Err} $\ast$ {\em err})}\label{ObitIOTableFITS_8h_a6}


Public: Are underlying structures the same. 

This test is done using values entered into the {\bf Obit\-Info\-List}{\rm (p.\,\pageref{structObitInfoList})} in case the object has not yet been opened. \begin{Desc}
\item[Parameters:]
\begin{description}
\item[{\em in}]{\bf Obit\-IO}{\rm (p.\,\pageref{structObitIO})} for test \item[{\em in1}]{\bf Obit\-Info\-List}{\rm (p.\,\pageref{structObitInfoList})} for first object to be tested \item[{\em in2}]{\bf Obit\-Info\-List}{\rm (p.\,\pageref{structObitInfoList})} for second object to be tested \item[{\em err}]{\bf Obit\-Err}{\rm (p.\,\pageref{structObitErr})} for reporting errors. \end{description}
\end{Desc}
\begin{Desc}
\item[Returns:]TRUE if to objects have the same underlying structures else FALSE \end{Desc}
\index{ObitIOTableFITS.h@{Obit\-IOTable\-FITS.h}!ObitIOTableFITSSet@{ObitIOTableFITSSet}}
\index{ObitIOTableFITSSet@{ObitIOTableFITSSet}!ObitIOTableFITS.h@{Obit\-IOTable\-FITS.h}}
\subsubsection{\setlength{\rightskip}{0pt plus 5cm}Obit\-IOCode Obit\-IOTable\-FITSSet ({\bf Obit\-IOTable\-FITS} $\ast$ {\em in}, {\bf Obit\-Info\-List} $\ast$ {\em info}, {\bf Obit\-Err} $\ast$ {\em err})}\label{ObitIOTableFITS_8h_a11}


Public: Init I/O. 

\begin{Desc}
\item[Parameters:]
\begin{description}
\item[{\em in}]Pointer to object to be accessed. \item[{\em info}]{\bf Obit\-Info\-List}{\rm (p.\,\pageref{structObitInfoList})} with instructions \item[{\em err}]{\bf Obit\-Err}{\rm (p.\,\pageref{structObitErr})} for reporting errors. \end{description}
\end{Desc}
\begin{Desc}
\item[Returns:]return code, 0=$>$ OK \end{Desc}
\index{ObitIOTableFITS.h@{Obit\-IOTable\-FITS.h}!ObitIOTableFITSWrite@{ObitIOTableFITSWrite}}
\index{ObitIOTableFITSWrite@{ObitIOTableFITSWrite}!ObitIOTableFITS.h@{Obit\-IOTable\-FITS.h}}
\subsubsection{\setlength{\rightskip}{0pt plus 5cm}Obit\-IOCode Obit\-IOTable\-FITSWrite ({\bf Obit\-IOTable\-FITS} $\ast$ {\em in}, {\bf ofloat} $\ast$ {\em data}, {\bf Obit\-Err} $\ast$ {\em err})}\label{ObitIOTableFITS_8h_a14}


Public: Write. 

When OBIT\_\-IO\_\-EOF is returned the data has been written, data in data is ignored and the I/O is closed. \begin{Desc}
\item[Parameters:]
\begin{description}
\item[{\em in}]Pointer to object to be written. \item[{\em data}]pointer to buffer containing input data. \item[{\em err}]{\bf Obit\-Err}{\rm (p.\,\pageref{structObitErr})} for reporting errors. \end{description}
\end{Desc}
\begin{Desc}
\item[Returns:]return code, 0(OBIT\_\-IO\_\-OK)=$>$ OK OBIT\_\-IO\_\-EOF =$>$ image finished. \end{Desc}
\index{ObitIOTableFITS.h@{Obit\-IOTable\-FITS.h}!ObitIOTableFITSWriteDescriptor@{ObitIOTableFITSWriteDescriptor}}
\index{ObitIOTableFITSWriteDescriptor@{ObitIOTableFITSWriteDescriptor}!ObitIOTableFITS.h@{Obit\-IOTable\-FITS.h}}
\subsubsection{\setlength{\rightskip}{0pt plus 5cm}Obit\-IOCode Obit\-IOTable\-FITSWrite\-Descriptor ({\bf Obit\-IOTable\-FITS} $\ast$ {\em in}, {\bf Obit\-Err} $\ast$ {\em err})}\label{ObitIOTableFITS_8h_a18}


Public: Write Descriptor. 

If the table version number is 0, then the highest numbered table of the same name +1 is used. \begin{Desc}
\item[Parameters:]
\begin{description}
\item[{\em in}]Pointer to object with {\bf Obit\-Table\-Desc}{\rm (p.\,\pageref{structObitTableDesc})} to be written. \item[{\em err}]{\bf Obit\-Err}{\rm (p.\,\pageref{structObitErr})} for reporting errors. \end{description}
\end{Desc}
\begin{Desc}
\item[Returns:]return code, 0=$>$ OK \end{Desc}
\index{ObitIOTableFITS.h@{Obit\-IOTable\-FITS.h}!ObitIOTableFITSWriteRow@{ObitIOTableFITSWriteRow}}
\index{ObitIOTableFITSWriteRow@{ObitIOTableFITSWriteRow}!ObitIOTableFITS.h@{Obit\-IOTable\-FITS.h}}
\subsubsection{\setlength{\rightskip}{0pt plus 5cm}Obit\-IOCode Obit\-IOTable\-FITSWrite\-Row ({\bf Obit\-IOTable\-FITS} $\ast$ {\em in}, {\bf olong} {\em rowno}, {\bf ofloat} $\ast$ {\em data}, {\bf Obit\-Err} $\ast$ {\em err})}\label{ObitIOTableFITS_8h_a15}


Public: Write specifying start row. 

When OBIT\_\-IO\_\-EOF is returned the data has been written, data in data is ignored and the I/O is closed. \begin{Desc}
\item[Parameters:]
\begin{description}
\item[{\em in}]Pointer to object to be written. \item[{\em rowno}]Row number (1-rel) to write \item[{\em data}]pointer to buffer containing input data. \item[{\em err}]{\bf Obit\-Err}{\rm (p.\,\pageref{structObitErr})} for reporting errors. \end{description}
\end{Desc}
\begin{Desc}
\item[Returns:]return code, 0(OBIT\_\-IO\_\-OK)=$>$ OK OBIT\_\-IO\_\-EOF =$>$ image finished. \end{Desc}
\index{ObitIOTableFITS.h@{Obit\-IOTable\-FITS.h}!ObitIOTableFITSZap@{ObitIOTableFITSZap}}
\index{ObitIOTableFITSZap@{ObitIOTableFITSZap}!ObitIOTableFITS.h@{Obit\-IOTable\-FITS.h}}
\subsubsection{\setlength{\rightskip}{0pt plus 5cm}void Obit\-IOTable\-FITSZap ({\bf Obit\-IOTable\-FITS} $\ast$ {\em in}, {\bf Obit\-Err} $\ast$ {\em err})}\label{ObitIOTableFITS_8h_a7}


Public: Delete underlying structures. 

\begin{Desc}
\item[Parameters:]
\begin{description}
\item[{\em in}]Pointer to object to be zapped. \item[{\em err}]{\bf Obit\-Err}{\rm (p.\,\pageref{structObitErr})} for reporting errors. \end{description}
\end{Desc}
