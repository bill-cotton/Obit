% $Id: whatisthis.tex 29 2008-09-12 14:53:37Z bill.cotton $ 
%
%  OBIT class definitions
%
%\def\section #1.#2.{\medskip\leftline{\bf #1. #2.}\smallskip}
%\def\bfitem #1#2{{{\bf (#1)}}{\it #2}}
\def\bfi #1#2{{\bf #1:}{ #2}\par}
\def\extname #1#2{ {\bf 1} Extension Name {\it #2}\smallskip}
%\def\tabname #1#2{\bfitem 1{#1 Table Name #2}}
\def\keyword {\leftline{\bf Keywords:}}
%\def\table {\leftline{\bf Table Definition:}}
\documentclass[11pt]{article}
\usepackage{OBITdoc}
\begin{document}
\setcounter{section}{0}
%  Title page

%\centerline{\ttlfont PRELIMINARY}
\vskip 5cm
\centerline{\ttlfont OBIT}
\vskip 1cm
\centerline{\ttlfont Software for the Recently Deceased}
\vskip 3cm
\centerline{\secfont Draft version: 1.0 \today}
\vskip 1cm
\centerline{\secfont W. D. Cotton}
\clearpage

% Table Of Contents
\tableofcontents
\cleardoublepage

\section {Introduction}
OBIT is a software package intended for astronomical, especially
radio--astronomical applications. 
This package uses an object--oriented style with most lower level
classes being defined by their documentation.
This document describes the tables (other than the one for uv data
visibilities) used in the Obit software package.
The documentation uses latex macroes which are translated by a perl
script into the c source code.
OBIT classes are used as front ends to either AIPS disk data or FITS
files using the  cfitsio package
\footnote{The fitsio library is a package of Fortran and c language
callable routines and is available at 
http://heasarc.gsfc.nasa.gov/docs/software/fitsio/fitsio.html.
This package takes care of most of the details of reading and writing
FITS files, including binary tables, and is well documented.}

\clearpage
\section {OBIT Class Definitions}
\begin{itemize}
\item{EXTNAME:}{ A\hfill\break}
\end{itemize}
\subsection{Optional Keywords}
There are a number of keywords that are optional in the binary table
definition but which may be required for a given Optical/IR data table.
\begin{itemize}
\item{:}{ \hfill\break}
\end{itemize}

\clearpage
%%%%%%%%%%%%%%% Obit Class %%%%%%%%%%%%%%%%%%%%%%%%%%%%%%%%%%%%%%%%%%%%%%%%%%%
% Obit base class
%
%\title fooey
\ClassName[{Obit}
{Base class of Obit system.}]
\ClassEnum[{obitObjType}
{Type of underlying data structures}]
\begin{ObitEnum}
\ObitEnumItem[{OBITTYPE\_FITS}{FITS file}]
\ObitEnumItem[{OBITTYPE\_AIPS}{AIPS catalog data}]
\end{ObitEnum}

\tabletitle{Imaging Detector FITS Table (VINCI variant)}
% table name
\tablename{IMAGING\_DETECTOR}
\tableintro{
This table defines the regions on one or more detectors
occupied by various signals.
}
\tableover{
Rectangular regions of the detector array(s) are described in
terms of the x and y coordinates of two opposite corners of the
pixel aray.
There is one row in this table for each region.
Detector pixel arrays are to be stored with the ``x''
dimension varying fastest; pixel numbering starts with 1.
Pixel order should be maintained in the subregions, i.e. increasing
pixel number is the same in both the full detector array and in the
region stored.
}
% Table keyword description
\begin{keywords}
\tablekey[{REVISION}{I}{Revision number of the table definition, now
1.}
{Standard}]
\tablekey[{NDETECT}{I}
{The number of detectors used in the instrument.
}
{Standard}]
\tablekey[{NREGION}{I}
{The number of regions used in the current mode.
}
{Standard}]
\tablekey[{MAX\_COEF}{I}
{The maximum number of higher order polynomial coefficients.
}
{Standard}]
\tablekey[{NUM\_DIM}{I}
{The number of dimensions in the detector array.
}
{Standard}]
\tablekey[{INSTRUME}{A}
{The name of the instrument (e.g. 'vinci').
}
{Standard}]
\tablekey[{MAXTEL}{I}
{The maximum number of telescopes contributing photons to the interferometer.
}
{Standard}]
\tablekey[{DATE\_OBS}{A}
{Start date/time of observations described by this table
'yyyy-mm-dd(Thh:mm:ss)'.}
{Standard}]
\end{keywords}
%
% Table column description
\begin{columns}
\tablecol[{REGION}{ }{I}{(1)}
{Which region number is being described by this row.
}{Standard}] 
\tablecol[{DETECTOR}{ }{I}{(1)}
{Which detector is this region on.
}{Standard}] 
\tablecol[{TELESCOPES}{ }{I}{(MAXTEL)}
{Which telescopes are used.
}{Standard}] 
\tablecol[{CORRELATION}{ }{I}{(1)}
{Correlation type, 0=background (no signal), 1=photometric, 
2=interferometric.
}{Standard}] 
\tablecol[{REGNAME}{ }{A}{(16)}
{A descriptive name for the region, e.g. 'interferometric\_1'.
}{Standard}] 
\tablecol[{CORNER}{pixel}{I}{(2)}
{This defines the corner of this region on the detector as low value
of x, low value of y, inclusive.
}{Standard}] 
\tablecol[{NAXES}{pixel}{I}{(2)}
{This gives the dimension of the data array in the ``x'' and ``y''
dimension. 
}{Standard}] 
\tablecol[{CRVAL}{WCS}{D}{(2)}
{The coordinate reference values at the reference pixels on the ``x'',
and ``y'' axes for this region.  Angles are in degrees and frequencies
in Hz.
}{Standard}] 
\tablecol[{CRPIX}{pixel}{E}{(2)}
{The reference pixel on the ``x'' and ``y'' axes for this region.
Pixels are relative to the region corner, i.e. the low pixel number
corner in the region is pixel (1,1).  These need be neither integers
nor in the actual region.
}{Standard}] 
\tablecol[{CTYPE}{ }{A}{(8,2)}
{The coordinate types for the ``x'' and ``y'' axes of this region.
Standard WCS conventions are used.
}{Standard}] 
\tablecol[{CD}{ }{D}{(2,2)}
{The linear coordinate transformation matrix in order CD\_1\_1,
CD\_1\_2, CD\_2\_1, CD\_2\_2. 
Standard WCS conventions are used.
This transformation is applied after the DMPn/DMCn transformation.
}{Standard}] 
\end{columns}
%
% Table modification history
\begin{history}
\modhistory[{W. D. Cotton}{30/03/2001}{Revision 1: Initial definition}]
\modhistory[{W. D. Cotton}{30/03/2001}{Revision 2: Add CORPOWER,
Optical Train}]
\end{history}
%
%
\clearpage
\begin{references}
\reference{Cotton, W.~D., Tody, D., and Pence, W.~D.\ 1995, \aaps, 113,
159--166.}
\reference{Flatters, C., 1998, AIPS Memo No, 102, NRAO.}
\reference{Wells, D.~C., Greisen, E.~W., and Harten, R.~H.\ 1981, \aaps, 44, 363.}
\end{references} 

\clearpage
\end{document}

