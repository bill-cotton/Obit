% $Id: ObitTalkSoft.tex 2 2008-06-10 15:32:27Z bill.cotton $ 
% ObitTalk Software Documentation
%
\documentclass[11pt]{report}
\usepackage{psfig}
\font\ttlfont=cmbxsl10 at 30pt
\font\secfont=cmbxsl10 at 15pt
\raggedbottom
\topmargin=-.9cm
\topskip=-2.5cm
\textheight=9.0in
\textwidth=6.5in
\oddsidemargin=0cm
\evensidemargin=-1cm
\parindent=0.25in
\topsep=0.5ex
\itemsep=0.5ex

\begin{document}
\setcounter{chapter}{1}
%\setcounter{section}{1}
%  Title page

\topskip 1.5in
\centerline{\ttlfont ObitTalk Software Documentation }
\vskip 3cm
\centerline{\secfont Draft version: 0.0 \today}
\vskip 1cm

% Abstract
\centerline{\secfont Abstract}
This documents describes the ObitTalk interface to AIPS and Obit
software.
ObitTalk is a python package which allows running AIPS and Obit tasks
and direct access to astronomical data and application libraries using
Obit. 
ObitTalk supports both local, distributed and remote (over a network)
data processing. 
\clearpage
\topskip 0in
%  Table of contents
\newpage 
\tableofcontents
%\listoffigures 
%\listoftables 
\newpage

\section {Introduction}
   ObitTalk is derived from the ParselTongue project at JIVE written
by Mark Kettenis and provides a scripting and interactive command line
interface to astronomical data and processing software.  
In particular, AIPS and FITS data structures as used in the AIPS and
Obit software packages  are supported as well as AIPS tasks and Obit
tasks and class libraries and other python enabled software.

   ObitTalk can start tasks and run scripts either locally or on a
remote machine which has an ObitTalkServer process running.  
Some data access is supported through the AIPSUVData, AIPSImage,
FITSUVData and FITSImage classes.  
Currently other interactive python functions only work locally.

   Tasks, scripts and more detailed access to, and manipulation of, data
are available.  
These are described below.
Later sections give the interface details of the major functions.

\section {Obtaining Software}
Obit and related software is available from
http://www.cv.nrao.edu/~bcotton/Obit.html. 
At present there is no system of stable releases so using the
anonymous CVS interface is recommended.
Obit depends heavily on third party software which is described on
this page.
The components of the Obit/ObitTalk package are:
\begin{itemize}
\item Obit\\
Basic obit package and the support for radio interferometry
\item ObitSD\\
Obit ``On The Fly'' (OTF) single dish imaging package.
\item ObitView\\
Image display used by Obit.
\item ObitTalk\\
Scripting and interactive interface to Obit software.
\end{itemize}
These software packages come with installation instructions and config
scripts to build them.

\section {Architectural Design}
ObitTalk is intended to allow local, remote and distributed processing
of astronomical data with both interactive and scripting interfaces.
ObitTalk is python with application specific modules loaded to allow
access to astronomical data and software.

ObitTalk allows use of both external AIPS and Obit tasks as well as
access to data structures and the high level Obit classes and functions.
Tasks, scripts and to a lesser degree data access can be either performed
locally or on a remote computer.
Access to Obit class libraries is currently only available for data
files visible on the computer on which the interactive/scripting
python process is run.
The various aspects of ObitTalk are discussed in the following.

\subsection {Client--Server Organization}
In the modern computing environment it is frequently desirable to have
a user or scripting front end with data and a compute server at a
remote site or another node of a cluster.
ObitTalk implements remote access by a client--server architecture for
much (but not all) of its functionality.
This is implemented by a split between the client and the server
(internally called ``Proxy'') sides.
If the client and server are on separate hosts, they are
independent python processes communicating via xmlrpc.
This protocol allows communication over networks using http protocols
and can, in principle, be quite secure.
If the client and server are on the same host, this is implemented in
a single python process but with the same interface as in the remote
server case.

The location of data is defined in terms of ``disks'', the definition
of which includes a URL for data on remote server machines.
ObitTalk thus knows where the data resides and does any operation and
runs tasks needing it there. 
This mechanism also allows distributed processing over a LAN or cluster.
The remote execution of scripts allows all functionality to be
available on remote as well as local hosts. 

\subsection{Tasks}
Tasks represent a relatively simple interface.
They are external, independent executables which accept parameters and
data files and may modify or produce other data files or
parametric results.
Currently only AIPS and Obit tasks are implemented but others which
follow this model could be.

Obit is intended to be fully interoperable with AIPS and AIPS data
structures are one possible ``native'' data type for Obit.
AIPS and Obit share a common data and calibration model.
This allows AIPS and Obit tasks to both be employed in the same
processing session or script.

All access to tasks is through task objects which contain the relevant
parameters and have methods to start, abort, wait for task execution,
etc. 
Task execution is performed on the server (proxy) side.
Starting tasks consists of spawning an external process and giving it
the task parameters.
These parameters are generally a combination of command line arguments
and parameters in a file which the task process reads.
Messages produced by the process on stdout or stderr are captured and
passed back to the client.
When a task is finished, it may write results back into a disk file
which is then read by the waiting python proxy and returned to the
client side.
For local execution, the proxy is implemented in the same address
space as the client python process.
For remote execution, the proxy is a python server process on the
remote machine.

\subsection{Scripts}
Scripts are implemented in the ObitScript class which is derived from
the Task class and thus shares many properties.
An ObitScript can execute a script which is a character string
containing a sequence of python statements; these scripts have full
access to the Obit/python bindings for data local to the execution host.
An ObitScript can be executed either locally or remotely and either
synchronously or asynchronously.
ObitTalk itself can be given a script as a command line argument and
this feature is used in the implementation of script execution.
The script is wrapped in python code to properly initialize and shut
down Obit and then written to a file.
An external process is than started running ObitTalk on this script file.

\subsection{Access to Obit class functions}
Major Obit classes and functions have bindings to python and are thus
available from python.
All access to Obit c software is through a single dynamic library,
Obit.so.
Separate Obit related packages (Single Dish Obit) will have separate
libraries but all Obit code must be liked into a single dynamic
library.
Since c and python have such very different view of data structures,
the interface is complex.
Python bindings to the major Obit classes and functions are
implemented in the interface.

The python interface is defined in the Obit python subdirectory.
The swig utility is used to interface the c library to python.
The swig interface is defined in the ObitTypeMaps.swig (defines
interface types) and *.inc which defines a python callable interface.
The ObitTypeMaps.swig and *.inc files are concatenated by the Makefile
before running swig.
This generates a single, large shared library module to be imported
into python.
However, this is necessary to get all the classes into the same
address space as Obit is dependent on class structures defining
function pointers.
A more elegant solution may be possible.

Note about swig.  The maintainers of swig have modified the interface 
in more recent versions (after version 1.1?) so the output of swig, 
python/Obit\_wrap.c, is distributed with other source code.  
The make procedure should not attempt to run swig unless you've modified
the interface (*.inc, *.swig).

The python interface defines a python class for each visible Obit
class in a separate *.py file.
The python/Obit classes are thin objects with basically a pointer to
the c structure (the ``me'' member) but this allows mapping the python
memory usage scheme onto Obit's similar but less automatic scheme. 
Thus the c objects created are automatically deleted when the python
reference count drops to zero or the explicit destructor (del) is
invoked.
However, this operation in python is only performed during its
occasional garbage collection; to ensure timely deallocation of large
Obit objects use the explicit Unref function.
All functions are implemented as non-class member routines which take
class arguments but many common functions (e.g. Open, Close, Read) are
also implemented as class member functions.

The convention for the routine names on the c side of the swig
interface is the same as the Obit name but dropping the initial
``Obit''.
This adds another layer of routine call but allows translating the
data types across the python/c divide.
Swig (plus the type maps in swig and .inc files) helps but sometimes 
this is insufficient; having a uniform way of dealing with the
interface improves reliability.
Some of the functions defined in the *.inc files correspond to macro
expansions or straight access to member values in c.

On the python side of the interface, a separate layer of routines
is added although generally the swig defined c routines are available
but while more efficient, direct usage may circumvent the automatic
destruction feature.

\subsubsection{Access}
Functions running in the ObitTalk python session can only ``see'' data
which is locally visible.
There is limited access to data on remote machines but not to other
class  libraries on the remote site.

\subsubsection{OTObit}
In order to simplify the user interface and make it look more like
AIPS/POPS, ObitTalk imports the OTObit module which defines many
convenience functions.
Asynchronous tasking is also implemented in OTObit.
To run a task asynchronously, it is necessary for its messages to
appear in a window other than the one in which the user is typing.
These windows are implemented using the wxPython python module (where
available). 
The functions in OTObit are described more fully in the ObitTalk user
document and in on--line documentation.

\subsection{Access to Data}
Access to data, especially data descriptors is vital even if the bulk
of the processing is performed in tasks.
There are several ways of accessing data depending on whether or not
they {\bf may} be on a remote machine.

\subsubsection{Local access and manipulation}
Bindings to Obit classes and objects through the swig python/c
interface is relatively straightforward as long as all data is locally
visible.
Class objects are created in the c routines and pointers passed to
python.
There are interface routines which allow access to the data directly
and allow access to the Obit class libraries to do major functions,
e.g. imaging, deconvolving, self calibration.
This ``tool--kit'' approach allows task--like functionality in a much
more flexible form.
Since only the high level control functions are in python there is
little efficiency loss over tasks.


\subsubsection{Remote or local access}
Even for processing using tasks on remote machines some data access is
required.
For this ObitTalk provides AIPSData classes for access to
data on remote machines.
In this case, the remote (or local) proxy reads/writes the data and
passes it back and forth over the xmlrpc interface.

\section{Interprocess Communication}
Any time multiple independent processes are used for an operation,
even if they all run in the same machine, some means of communication
or coordination is needed.
In this age of distributed data and computing, communication over
networks is needed.
In ObitTalk and related software, there are two kinds of interprocess
communications.
Python processes which start tasks as independent processes on the same
machine communicate to these processes mainly through disk files.
All other interprocess communications are by means of xmlrpc which
allows these to be over networks.

\subsection{xmlrpc}
Xmlrpc is a relatively simple communication technique using http
protocols in a stateless query-response model.
Thus, any state (e.g., one or more tasks currently active) must reside in
the processes involved and activities such as processing message logs
and waiting for task completion are done by polling from the client
rather than interrupts or messages from the server.
The implementation in Obit and related is a relatively thin layer
which could be replaced by another protocol of similar capabilities.

One of the principle operational difficulties with xmlrpc is the need
for preassigned port numbers; the client needs to know which port the
server is watching.
In theory, port 80 could be used but this requires that a web server
be installed which knows how to talk to the relevant service.

An advantage of xmlrpc is its relative flexibility, since call
arguments and return values are always xml strings, calls do not need
to have predefined arguments and returns.
Another advantage of xmlrpc is that it has widespread implementations;
both python and c interfaces are implemented in Obit and related.

\subsection{Communications with tasks}
In ObitTalk, tasks are started by a python process running in the
target host machine.
Input and output parameters are communicated by means of a locally
visible disk file.
In the case of AIPS tasks, this is a single binary file per host and
each ``POPS'' number has an assigned portion of the (TD) file.
Obit tasks use independent text files for each task instance and
separate files for input and output parameters.
The names of these files are passed as command line arguments when the
task process is started.
Prior to task initiation, the input parameters are written to the
relevant files and the task process started.
When the task finishes, it writes any return parameters and a code
indicating whether the process was satisfied with the execution are
written into the output file.
The python proxy then extracts the output parameters and the
completion code to return to the client.
In the case of a less than completely satisfactory conclusion of a
synchronous execution of the task, a RunTime python exception is
thrown causing any script being executed to abort.

Communication with an executing process is more complex.
AIPS and Obit tasks can interactively use their respective display
processes (XAS and ObitView) to control processing.
The process may also solicit and accept terminal input from the user; in
this case, such communications is over the client--proxy interface.
A feature of AIPS tasks which is not implemented but probably should
be (and added to Obit Tasks) is the ability to pass revised parameters
during task execution (SHOW/TELL in POPS).
In AIPS tasks this is by means of updating parameters in the binary file
and the task occasionally checking for redefined parameters.
This allows users to modify the behavior during run time and in
particular to inform an iterative process that its current state is
sufficient and it should stop.

\subsection{Communications with Image Displays}
Image display processes run independently, usually on the client side
but it is also possible to run them on the server with a X-windows
display on the client.
Communications with display processes can be done over networks.
This allows displays to be run on a user's local (client) machine and
talk to a process on a remote server.
This is relatively straightforward with the AIPS display, XAS, as the
image is passed over the network using a socket connection.
The disadvantage is that the AIPS display has no display controls of
its own and requires another process to send it commands.
AIPS tasks may access the XAS display but ObitTalk has no direct
capability. 

Communication with ObitView is by means of xmlrpc which also allows
network access.
If the ObitView display server is running on a host different from the 
one on which the data resides (not localhost), then the image is
written as a gzipped FITS image and copied to the ObitView server
over the xmlrpc connection where it can load the image.
FITS images may include a URL so may be accessed over a network.
An additional complication is that AIPS data is stored in local data
format (e.g. big vs. little endian) and remote access may not be
useful.
Once an image is loaded into ObitView it has many options for
modifying the display or simple analysis of the image.
Both Obit tasks and ObitTalk have access to ObitView.

\section{Example Usage}
The following sections give examples of the various levels of access
to Obit and AIPS software from python and ObitTalk in particular.
\subsection{ObitTalk Task Example}
An example of creating a task object named im to run
AIPS task IMEAN is: 
\begin{verbatim}
>>> im=AIPSTask("IMEAN")
\end{verbatim}
The parameters of the task can then be set:
\begin{verbatim}
>>> im.inname='07030+51396'; im.inclass='PCUBE'; im.indisk=1; im.inseq=2
>>> im.BLC=AIPSList([10,10]); im.TRC=AIPSList([100,100])
\end{verbatim}
The Inputs can be reviewed:
\begin{verbatim}
>>> im.inputs()
IMEAN:  Task to print the mean, rms and extrema in an image
Adverbs     Values                                   Comments
--------------------------------------------------------------------------------
 dohist      -1.0                                     True (1.0) do histogram plot.
                                                      = 2 => flux on x axis
 userid       0.0                                     User ID.  0=>current user
                                                        32000=>all users
 inname      07030+51396                              Image name (name)
 inclass     PCUBE                                    Image name (class)
 inseq        2.0                                     Image name (seq. #)
 indisk       1.0                                     Disk drive #
 blc         10.0, 10.0, 0.0, 0.0, 0.0, 0.0, 0.0      Bottom left corner of image
                                                      0=>entire image
 trc         100.0, 100.0, 0.0, 0.0, 0.0, 0.0, 0.0    Top right corner of image
                                                      0=>entire image
 nboxes       0.0                                     No. of ranges for histogram.
 pixrange    0.0, 0.0                                 Min and max range for hist.
 functype                                             'LG' => do log10 plot of #
                                                      samples, else linear
 pixavg       0.0                                     Estimate of mean noise value
 pixstd       0.0                                     Estimate of true noise rms
                                                         < 0 => don't do one
                                                         = 0 => 2-passes to get
 docat        1.0                                     Put true RMS in header
 ltype        3.0                                     Type of labeling: 1 border,
                                                      2 no ticks, 3 - 6 standard,
                                                      7 - 10 only tick labels
                                                      <0 -> no date/time
 outfile                                              Name of output log file,
                                                      No output to file if blank
 dotv        -1.0                                     > 0 Do plot on the TV, else
                                                      make a plot file
 grchan       0.0                                     Graphics channel 0 => 1.
\end{verbatim}
and the task run:
\begin{verbatim}
>>> log=im.go()
IMEAN2: Task IMEAN  (release of 31DEC02) begins
IMEAN2: Initial guess for PIXSTD taken from ACTNOISE inheader
IMEAN2: Image= 07030+51396 .PCUBE .   2 1   xywind=    1    1  241  241
IMEAN2: Mean and rms found by fitting peak in histogram:
IMEAN2: Mean=-3.1914E-06 Rms= 2.7893E-04  **** from histogram
IMEAN2: Mean and rms found by including all data:
IMEAN2: Mean= 1.8295E-05 Rms= 5.2815E-04 JY/BEAM  over    174243 pixels
IMEAN2: Flux density =  2.0006E-01 Jy.   beam area =  15.93 pixels
IMEAN2: Minimum=-1.5441E-03 at  164  180    1    1
IMEAN2: Skypos: RA 07 02 04.303  DEC 51 51 23.18
IMEAN2: Skypos: IPOL  1400.000 MHZ
IMEAN2: Maximum= 4.0180E-02 at   93  159    1    1
IMEAN2: Skypos: RA 07 03 36.211  DEC 51 47 11.65
IMEAN2: Skypos: IPOL  1400.000 MHZ
IMEAN2: returns adverbs to AIPS
IMEAN2: Appears to have ended successfully
IMEAN2: smeagle      31DEC02 TST: Cpu=       0.0  Real=       0
\end{verbatim}

\subsection{Obit class from ObitTalk example}
Obit classes  can be accessed from python scripts or interactively.
A number of scripts are available in the python subdirectory 
with names ``script*.py'', most of these can be run in python without ObitTalk.
Obit python functions  should all have documentation strings that can
be accessed interactively. 
The examples shown in this section only work for localy visible data.
The following example from UVImager shows the help facility.
Note: help also works on entire python modules.
\begin{verbatim}
>>> import UVImager
>>>  help(UVImager.UVImage)
Help on function UVImage in module UVImager:

UVImage(err, input={'Channel': 0, 'DoBeam': True, 'DoWeight': False, 'InData': None, 'OutImage
s': None, 'Robust': 0.0, 'TimeRange': [0.0, 0.0, 0.0, 0.0, 0.0, 0.0, 0.0, 0.0], 'UVTaper': [0.
0, 0.0], 'WtBox': 0, 'WtFunc': 1, ...})
    Image a uv data set.
    
    UV Data is weighted and imaged producing an array of images.
    err     = Python Obit Error/message stack
    input   = input parameter dictionary
    
    Input dictionary entries:
    InData   = Input Python OTF to image
    OutImages= Output image mosaic, image objects should be previously defined
    DoBeam   = True if beams are to be made
    DoWeight = If True apply uniform weighting corrections
    Robust   = Briggs robust parameter. (AIPS definition)
    UVTaper  = UV plane taper, sigma in klambda as [u,v]
    WtSize   = Size of weighting grid in cells [same as image nx]
    WtBox    = Size of weighting box in cells [def 1]
    WtFunc   = Weighting convolution function [def. 1]
               1=Pill box, 2=linear, 3=exponential, 4=Gaussian
               if positive, function is of radius, negative in u and v.
    WtPower  = Power to raise weights to.  [def = 1.0]
               Note: a power of 0.0 sets all the output weights to 1 as modified
               by uniform/Tapering weighting.
               Applied in determing weights as well as after.
    Channel  = Channel (1-rel) number to image, 0-> all.
\end{verbatim}

Many of the higher level functions have inputs in the form of a python dictionary
whose current values can be displayed by the class input function, e.g.:
\begin{verbatim}
>>> UVImager.input(UVImager.UVImageInput)
Inputs for  UVImage
   InData     =  None  :  Input UV data
   OutImages  =  None  :  Output image mosaic
   DoBeam     =  True  :  True if beams are to be made
   DoWeight   =  False  :  If True apply uniform weighting corrections to uvdata
   Robust     =  0.0  :  Briggs robust parameter. (AIPS definition)
   UVTaper    =  [0.0, 0.0]  :  UV plane taper, sigma in klambda as [u,v]
   WtSize     =  -1  :  Size of weighting grid in cells [image]
   WtBox      =  0  :  Size of weighting box in cells [def 0]
   WtFunc     =  1  :  Weighting convolution function [def. 1]
   WtPower    =  1.0  :  Power to raise weights to.  [def = 1.0]
   Channel    =  0  :  Channel (1-rel) number to image, 0-> all.
Note: the values displayed for Obit objects are the names you give
them.
\end{verbatim}

   Obit messages and error handling is by means of the Obit type
ObitErr which is an argument to most Obit routines.  This contains
informative as well as error messages; its contents can be displayed
using e.g.:
\begin{verbatim}
>>> OErr.printErr(err)
** Message: information  : Hogbom CLEANed 300 components with 4.025047 Jy
** Message: information  : Reached minimum flux density -0.335160 Jy
** Message: information  : Scaling residuals by 1.079005
** Message: information  : Restoring 300 components
\end{verbatim}
If  the err object indicates an error, OErr.printErr(err), will raise
an exception.

   If scratch files are created (see OTF.ResidCal for an example) then
an Obit shutdown will delete them:
\begin{verbatim}
# Shutdown Obit
OErr.printErr(err)
OSystem.Shutdown(ObitSys)
\end{verbatim}

\subsection{Python script without ObitTalk example}
A python script that is the functional equivalent of the AIPS task
HGEOM (as originally intended, interpolate the pixels of one image
onto the grid defined by another) is shown in the following.
The image is interpolated into a scratch image which is then copied to
the output image as a quantized (1/4 RMS of noise) image , the old
history is copied and new history records written.
Note: This is also implemented as Obit Task HGeom.
This script does many of the initializations done automatically by ObitTalk.
\begin{verbatim}

# python/Obit equivalent of AIPSish HGEOM

import Obit, Image, ImageUtil, OSystem, OErr

# Init Obit
err=OErr.OErr()

# Use default directories
user = 100
pgmNumber = 1
ObitSys=OSystem.OSystem ("HGeom", pgmNumber, user, -1, ["Def"], \
    -1, ["Def"], True, False, err)
#  print any error messages and halt
OErr.printErrMsg(err, "Error with Obit startup")

# Define files (FITS)
# Image to be interpolated onto the grid of tmplFile
inDisk = 1
inFile   = 'input.fits'     
# Image defining the output grid
tmplDisk = 1
tmplFile = 'template.fits'
# output file, the '!' allows overwriting an existing file
outDisk = 1
outFile  = '!HGeomOut.fits'

# Create Python/Obit objects attached to the data files
inImage   = Image.newPFImage("Input image",    inFile,   inDisk,   1, err)
tmplImage = Image.newPFImage("Template image", tmplFile, tmplDisk, 1, err)
outImage  = Image.newPFImage("Output image",   outFile,  outDisk,  0, err)
Image.PClone(tmplImage, outImage, err)   # Same structure etc.
OErr.printErrMsg(err, "Error initializing")

# Generate scratch file from tmplFile
tmpImage  = Image.PScratch(tmplImage, err)
tmpImage.Open(Image.WRITEONLY, err)   # Open
OErr.printErrMsg(err, "Error cloning template")

# Interpolate pixels to temporary file
ImageUtil.PInterpolateImage(inImage, tmpImage, err)
OErr.printErrMsg(err, "Error interpolating")

# Do history to scratch image as table
inHistory  = History.History("history", inImage.List, err)
outHistory = History.History("history", tmpImage.List, err)
History.PCopyHeader(inHistory, outHistory, err)
# Add this programs history
outHistory.Open(History.READWRITE, err)
outHistory.TimeStamp(" Start Obit "+ObitSys.pgmName,err)
outHistory.WriteRec(-1,ObitSys.pgmName+" / input = "+inFile,err)
outHistory.WriteRec(-1,ObitSys.pgmName+" / template = "+tmplFile,err)
outHistory.Close(err)
OErr.printErrMsg(err, "Error with history")

# Copy to quantized integer image with history
print "Write output image"
inHistory  = History.History("history", tmpImage.List, err)
Image.PCopyQuantizeFITS (tmpImage, outImage, err, inHistory=inHistory)

# Say what happened
print "Interpolated",inFile,"to",outFile,"a clone of ",tmplFile

# Shutdown Obit
OErr.printErr(err)  # Print any remaining Obit messages
OSystem.Shutdown(ObitSys)
\end{verbatim}

\section{ObitTalk Implementation}
The following are the internal documentations of major ObitTalk classes.
\subsection{Client side}
The client side software always runs in the machine the user is
running the python process on.

\subsubsection{ObitTask}
The client ObitTask class provides a means to define the parameters,
run, receive messages and receive return values for Obit tasks.
\begin{verbatim}
class ObitTask(AIPSTask.AIPSTask)
This class implements running Obit tasks.

The ObitTask class, derived from the AIPSTask class, handles
client-side task related operations.  Actual task definition
and operations are handled by server-side proxies.
For local operations, the server-side functionality is implemented
in the same address space but remote operation is through an
xmlrpc interface.  Tasks are run as separate processes in all
cases.

Each defined disk has an associated proxy, either local or remote.
A proxy is a module with interface functions,
local proxies are class modules from subdirectory Proxy with the
same name (i.e. AIPSTask) and the server functions are implemented
there.  Remote proxies are specified by a URL and a proxy from the
xmlrpclib module is used.

When an object is created, the task secific parameters and
documentation are retrieved by parsing the task TDF file.
This is performed on the server-side.

Method resolution order:
    ObitTask
    AIPSTask.AIPSTask
    Task.Task
    MinimalMatch.MinimalMatch

Methods defined here:

__init__(self, name)
    Create Obit task object
    
    Creates task object and calls server function to parse TDF
    file to obtain task specific parametrs and documentation.
    Following is a list of class members:
    _default_dict   = Dictionary with default values of parameters
    _input_list     = List of input parameters in order
    _output_list    = List of output parameters in order
    _min_dict       = Parameter minimum values as a List
    _max_dict       = Parameter maximum values as a List
    _hlp_dict       = Parameter descriptions (list of strings)
                      as a dictionary
    _strlen_dict    = String parameter lengths as dictionary
    _help_string    = Task Help documentation as list of strings
    _explain_string = Task Explain documentation as list of strings
    _short_help     = One line description of task
    _message_list   = list of execution messages
    retCode         = Task return code, 0=Finished OK
    debug           = If true save task parameter file
    logFile         = if given, the name of the file in which to
                      write messages
    doWait          = True if synchronous  operation wished
    Current parameter values are given as class members.

abort(self, proxy, tid, sig=9)
    Abort the task specified by PROXY and TID.
    
    Calls abort function for task tid on proxy.
    None return value
    proxy = Proxy giving access to server
    tid   = Task id in pid table of process to be terminated
    sig   = signal to sent to the task

go(self)
    Run the task.
    
    Writes task input parameters, data directories and other
    information in the task parameter file and starts the task
    synchronously returning only when the task terminates.
    Messages are displayed as generated by the task,
    saved in an array returned from the call and, if the task
    member logFile is set, written to this file.

messages(self, proxy=None, tid=None)
    Return messages for the task specified by PROXY and TID.
    
    Returns list of messages and appends them to the object's
    message list.
    proxy = Proxy giving access to server
    tid   = Task id in pid table of process

spawn(self)
    Spawn the task.
    
    Writes task input parameters, data directories and other
    information in the task parameter file and starts the task
    asynchronously returning immediately.  Messages must be
    retrieved calling messages.
    Returns (proxy, tid)

----------------------------------------------------------------------
Data and other attributes defined here:

debug = False

doWait = False

logFile = ''

version = 'OBIT'

----------------------------------------------------------------------
Methods inherited from AIPSTask.AIPSTask:

__call__(self)

__eq__(self, other)
    Check if two task objects are for the same task

__getattr__(self, name)

__setattr__(self, name, value)

copy(self)
    Return a copy of a given task object

defaults(self)
    Set adverbs to their defaults.

feed(self, proxy, tid, banana)
    Feed the task a  BANANA.
    
    Pass a message to a running task's sdtin
    proxy   = Proxy giving access to server
    tid     = Task id in pid table of process
    bananna = text message to pass to task input

finished(self, proxy, tid)
    Determine if task has finished 
    
    Determine whether the task specified by PROXY and TID has
    finished.
    proxy = Proxy giving access to server
    tid   = Task id in pid table of process

inputs(self)
    Display all inputs for this task.

outputs(self)
    Display all outputs for this task.

wait(self, proxy, tid)
    Wait for the task to finish.
    
    proxy = Proxy giving access to server
    tid   = Task id in pid table of process

----------------------------------------------------------------------
Data and other attributes inherited from AIPSTask.AIPSTask:

isbatch = 0

msgkill = 0

userno = 0

----------------------------------------------------------------------
Methods inherited from Task.Task:

explain(self)
    Display help+explain for this task.

help(self)
    Display help for this task.

----------------------------------------------------------------------
Methods inherited from MinimalMatch.MinimalMatch:

__repr__(self)
\end{verbatim}
\subsubsection{AIPSTask}
AIPSTask objects allow means to define the parameters,
run, receive messages and receive return values for AIPS tasks.
\begin{verbatim}
class AIPSTask(Task.Task)
This class implements running AIPS tasks.

The AIPSTask class, handles client-side task related operations.
Actual task definition and operations are handled by server-side
proxies. For local operations, the server-side functionality is
implemented in the same address space but remote operation is
through an xmlrpc interface.  Tasks are run as separate processes
in all cases.

Each defined disk has an associated proxy, either local or remote.
A proxy is a module with interface functions,
local proxies are class modules from subdirectory Proxy with the
same name (i.e. ObitTask) and the server functions are implemented
there.  Remote proxies are specified by a URL and a proxy from the
xmlrpclib module is used.

When an object is created, the task secific parameters and
documentation are retrieved by parsing the task Help file.and the
POPSDAT.HLP file for parameter definitions.  This is performed on
the server-side.

Method resolution order:
    AIPSTask
    Task.Task
    MinimalMatch.MinimalMatch

Methods defined here:

__call__(self)

__eq__(self, other)
    Check if two task objects are for the same task

__getattr__(self, name)

__init__(self, name, **kwds)
    Create AIPS task object
    
    Creates task object and calls server function to parse the
    task help and POPSDAT.HLP files to obtain task specific
    parametrs and documentation.
    Following is a list of class members:
    _default_dict   = Dictionary with default values of parameters
    _input_list     = List of input parameters in order
    _output_list    = List of output parameters in order
    _min_dict       = Parameter minimum values as a List
    _max_dict       = Parameter maximum values as a List
    _hlp_dict       = Parameter descriptions (list of strings)
                      as a dictionary
    _strlen_dict    = String parameter lengths as dictionary
    _help_string    = Task Help documentation as list of strings
    _explain_string = Task Explain documentation as list of strings
    _short_help     = One line description of task
    _message_list   = list of execution messages
    Current parameter values are given as class members.

__setattr__(self, name, value)

abort(self, proxy, tid, sig=15)
    Abort the task specified by PROXY and TID.
    
    Calls abort function for task tid on proxy.
    None return value
    proxy = Proxy giving access to server
    tid   = Task id in pid table of process to be terminated
    sig   = signal to sent to the task

copy(self)
    Return a copy of a given task object

defaults(self)
    Set adverbs to their defaults.

feed(self, proxy, tid, banana)
    Feed the task a  BANANA.
    
    Pass a message to a running task's sdtin
    proxy   = Proxy giving access to server
    tid     = Task id in pid table of process
    bananna = text message to pass to task input

finished(self, proxy, tid)
    Determine if task has finished 
    
    Determine whether the task specified by PROXY and TID has
    finished.
    proxy = Proxy giving access to server
    tid   = Task id in pid table of process

go(self)
    Run the task.
    
    Writes task input parameters in the task parameter file and
    starts the task synchronously returning only when the task
    terminates. Messages are displayed as generated by the task,
    saved in an array returned from the call and, if the task
    member logFile is set, written to this file.

inputs(self)
    Display all inputs for this task.

messages(self, proxy=None, tid=None)
    Return task messages
    
    Returns list of messages and appends them to the object's
    message list.        
    proxy = Proxy giving access to server
    tid   = Task id in pid table of process

outputs(self)
    Display all outputs for this task.

spawn(self)
    Spawn the task.
     
    Writes task input parameters, task parameter file and starts
    the task asynchronously returning immediately.  Messages must be
    retrieved calling messages.
    Returns (proxy, tid)

wait(self, proxy, tid)
    Wait for the task to finish.
    
    proxy = Proxy giving access to server
    tid   = Task id in pid table of process

----------------------------------------------------------------------
Data and other attributes defined here:

doWait = False

isbatch = 0

logFile = ''

msgkill = 0

userno = 0

version = 'OLD'

----------------------------------------------------------------------
Methods inherited from Task.Task:

explain(self)
    Display help+explain for this task.

help(self)
    Display help for this task.

----------------------------------------------------------------------
Methods inherited from MinimalMatch.MinimalMatch:

__repr__(self)
\end{verbatim}

\subsubsection{ObitScript}
This is the client interface to local or remote script execution.
\begin{verbatim}
DESCRIPTION
    This module provides the ObitScript class.
    This class allows running Obit/python scripts either
    locally or remotely
    
    ObitScripts are derived from Task and share most of execution properties.
    In particular, ObitScripts can be executed either locally or remotely.
    In this context a script is a character string containing a sequence of
    ObitTalk or other python commands and may be included when the script
    object is created or attached later.
    An example:
    script="import OSystem
    print 'Welcome user',OSystem.PGetAIPSuser()
    "

CLASSES
    ObitScriptMessageLog
    Task.Task(MinimalMatch.MinimalMatch)
        ObitScript
    
  class ObitScript(Task.Task)
     This class implements running Obit/python Script
     
     The ObitScript class, handles client-side script related operations.
     Actual script operations are handled by server-side proxies.
     For local operations, the server-side functionality is
     implemented in the same address space but remote operation is
     through an xmlrpc interface.  
     
     An ObitScript has an associated proxy, either local or remote.
     A proxy is a module with interface functions,
     local proxies are class modules from subdirectory Proxy with the
     same name (i.e. ObitScript) and the server functions are implemented
     there.  Remote proxies are specified by a URL and a proxy from the
     xmlrpclib module is used.
     
     Method resolution order:
         ObitScript
         Task.Task
         MinimalMatch.MinimalMatch
     
     Methods defined here:
     
     __call__(self)
     
     __getattr__(self, name)
     
     __init__(self, name, **kwds)
         Create ObitScript task object
         
         Creates Script Object.
         name  = name of script object
         Optional Keywords:
             script   = Script to execute as string
             URL      = URL on which the script is to be executed
                        Default = None = local execution
             AIPSDirs = List of AIPS directories on URL
                        Default = current AIPS directories on url
             FITSDirs = List of FITS directories on URL
                        Default = current FITS directories on url
             AIPSUser = AIPS user number for AIPS data files
                        Default is current
             version  = AIPS version string, Default = current
         Following is a list of class members:
             url      = URL of execution server, None=Local
             proxy    = Proxy for URL
             script   = Script as text string
             userno   = AIPS user number
             AIPSDirs = List of AIPS directories on URL
             FITSDirs = List of FITS directories on URL
             AIPSUser = AIPS user number for AIPS data files
             version  = AIPS version string
             _message_list = messages from Script execution
     
     __setattr__(self, name, value)
     
     abort(self, proxy, tid, sig=15)
         Abort the script specified by PROXY and TID.
         
         Calls abort function for task tid on proxy.
         None return value
         proxy = Proxy giving access to server
         tid   = Task id in pid table of process to be terminated
         sig   = signal to sent to the task
     
     explain(self)
         List script
     
     feed(self, proxy, tid, banana)
         Feed the script a  BANANA.
         
         Pass a message to a running script's sdtin
         proxy   = Proxy giving access to server
         tid     = Script task id in pid table of process
         bananna = text message to pass to script input
     
     finished(self, proxy, tid)
         Determine if script has finished 
         
         Determine whether the script specified by PROXY and TID has
         finished.
         proxy = Proxy giving access to server
         tid   = Task id in pid table of process
     
     go(self)
         Execute the script.
         
         Writes task input parameters in the task parameter file and
         starts the task synchronously returning only when the task
         terminates. Messages are displayed as generated by the task,
         saved in an array returned from the call and, if the task
         member logFile is set, written to this file.
     
     help(self)
         List script.
     
     inputs(self)
         List script
     
     messages(self, proxy=None, tid=None)
         Return task messages
         
         Returns list of messages and appends them to the object's
         message list.        
         proxy = Proxy giving access to server
         tid   = Task id in pid table of process
     
     outputs(self)
         Not defined.
     
     spawn(self)
         Spawn the script.
          
         Starts script asynchronously returning immediately
         Messages must be retrieved calling messages.
         Returns (proxy, tid)
     
     wait(self, proxy, tid)
         Wait for the script to finish.
         
         proxy = Proxy giving access to server
         tid   = Task id in pid table of process
     
     ----------------------------------------------------------------------
     Data and other attributes defined here:
     
     AIPSDirs = []
     
     FITSDirs = []
     
     debug = False
     
     doWait = False
     
     isbatch = 32000
     
     logFile = ''
     
     msgkill = 0
     
     proxy = <module 'LocalProxy' from '/export/users/bcotton/share/obittal...
     
     script = ''
     
     url = None
     
     userno = 0
     
     version = 'TST'
     
     ----------------------------------------------------------------------
     Methods inherited from MinimalMatch.MinimalMatch:
     
     __repr__(self)
    
  class ObitScriptMessageLog
     Methods defined here:
     
     __init__(self)
     
     zap(self)
         Zap message log.
     
     ----------------------------------------------------------------------
     Data and other attributes defined here:
     
     userno = -1
\end{verbatim}

\subsubsection{AIPSData}
This is the client interface to local or remote AIPS data.
\begin{verbatim}
Help on module AIPSData:

NAME
    AIPSData

FILE
    /export/users/bcotton/share/obittalk/python/AIPSData.py

DESCRIPTION
    This module provides the AIPSImage and AIPSUVData classes.  These
    classes implement most of the data oriented verb-like functionality
    from classic AIPS.

CLASSES
    AIPSCat
    _AIPSData
        AIPSImage
        AIPSUVData
    
    class AIPSCat
    Methods defined here:
    
    __init__(self, disk)
    
    __repr__(self)
    
    class AIPSImage(_AIPSData)
    This class describes an AIPS image.
    
    Methods inherited from _AIPSData:
    
    __getattr__(self, name)
    
    __init__(self, name, klass, disk, seq)
    
    __repr__(self)
    
    __str__(self)
    
    exists(self)
        Check whether this image or data set exists.
        
        Returns True if the image or data set exists, False otherwise.
    
    getrow_table(self, type, version, rowno)
        Get a row from an extension table.
        
        Returns row ROWNO from version VERSION of extension table TYPE
        as a dictionary.
    
    header(self)
        Get the header for this image or data set.
        
        Returns the header as a dictionary.
    
    header_table(self, type, version)
        Get the header of an extension table.
        
        Returns the header of version VERSION of the extension table
        TYPE.
    
    table(self, type, version)
    
    table_highver(self, type)
        Get the highest version of an extension table.
        
        Returns the highest available version number of the extension
        table TYPE.
    
    tables(self)
        Get the list of extension tables.
    
    verify(self)
        Verify whether this image or data set can be accessed.
    
    zap(self)
        Destroy this image or data set.
    
    zap_table(self, type, version)
        Destroy an extension table.
        
        Deletes version VERSION of the extension table TYPE.  If
        VERSION is 0, delete the highest version of table TYPE.  If
        VERSION is -1, delete all versions of table TYPE.
    
    ----------------------------------------------------------------------
    Properties inherited from _AIPSData:
    
    disk
        Disk where this data set is stored.
    
        lambdaself
    
    klass
        Class of this data set.
    
        lambdaself
    
    name
        Name of this data set.
    
        lambdaself
    
    seq
        Sequence number of this data set.
    
        lambdaself
    
    userno
        User number used to access this data set.
    
        lambdaself
    
    class AIPSUVData(_AIPSData)
    This class describes an AIPS UV data set.
    
    Methods inherited from _AIPSData:
    
    __getattr__(self, name)
    
    __init__(self, name, klass, disk, seq)
    
    __repr__(self)
    
    __str__(self)
    
    exists(self)
        Check whether this image or data set exists.
        
        Returns True if the image or data set exists, False otherwise.
    
    getrow_table(self, type, version, rowno)
        Get a row from an extension table.
        
        Returns row ROWNO from version VERSION of extension table TYPE
        as a dictionary.
    
    header(self)
        Get the header for this image or data set.
        
        Returns the header as a dictionary.
    
    header_table(self, type, version)
        Get the header of an extension table.
        
        Returns the header of version VERSION of the extension table
        TYPE.
    
    table(self, type, version)
    
    table_highver(self, type)
        Get the highest version of an extension table.
        
        Returns the highest available version number of the extension
        table TYPE.
    
    tables(self)
        Get the list of extension tables.
    
    verify(self)
        Verify whether this image or data set can be accessed.
    
    zap(self)
        Destroy this image or data set.
    
    zap_table(self, type, version)
        Destroy an extension table.
        
        Deletes version VERSION of the extension table TYPE.  If
        VERSION is 0, delete the highest version of table TYPE.  If
        VERSION is -1, delete all versions of table TYPE.
    
    ----------------------------------------------------------------------
    Properties inherited from _AIPSData:
    
    disk
        Disk where this data set is stored.
    
        lambdaself
    
    klass
        Class of this data set.
    
        lambdaself
    
    name
        Name of this data set.
    
        lambdaself
    
    seq
        Sequence number of this data set.
    
        lambdaself
    
    userno
        User number used to access this data set.
    
        lambdaself

\end{verbatim}
\subsubsection{FITSData}
This is the client interface to local FITS data.
There appears to be no server side equivalent.
\begin{verbatim}
Help on module FITSData:

NAME
    FITSData

FILE
    /export/users/bcotton/share/obittalk/python/FITSData.py

DESCRIPTION
    This module provides the FITSImage and FITSUVData classes.  These
    classes implement most of the data oriented verb-like functionality
    from classic FITS.

CLASSES
    _FITSData
        FITSImage
        FITSUVData
    
    class FITSImage(_FITSData)
    This class describes an FITS image.
    
    Methods inherited from _FITSData:
    
    __getattr__(self, filename)
    
    __init__(self, name, disk)
    
    __repr__(self)
    
    __str__(self)
    
    exists(self)
        Check whether this image or data set exists.
        
        Returns True if the image or data set exists, False otherwise.
    
    getrow_table(self, type, version, rowno)
        Get a row from an extension table.
        
        Returns row ROWNO from version VERSION of extension table TYPE
        as a dictionary.
    
    header(self)
        Get the header for this image or data set.
        
        Returns the header as a dictionary.
    
    header_table(self, type, version)
        Get the header of an extension table.
        
        Returns the header of version VERSION of the extension table
        TYPE.
    
    table(self, type, version)
    
    table_highver(self, type)
        Get the highest version of an extension table.
        
        Returns the highest available version number of the extension
        table TYPE.
    
    tables(self)
        Get the list of extension tables.
    
    verify(self)
        Verify whether this image or data set can be accessed.
    
    zap(self)
        Destroy this image or data set.
    
    zap_table(self, type, version)
        Destroy an extension table.
        
        Deletes version VERSION of the extension table TYPE.  If
        VERSION is 0, delete the highest version of table TYPE.  If
        VERSION is -1, delete all versions of table TYPE.
    
    ----------------------------------------------------------------------
    Properties inherited from _FITSData:
    
    disk
        Disk where this data set is stored.
    
        lambdaself
    
    filename
        Filename of this data set.
    
        lambdaself
    
    class FITSUVData(_FITSData)
    This class describes an FITS UV data set.
    
    Methods inherited from _FITSData:
    
    __getattr__(self, filename)
    
    __init__(self, name, disk)
    
    __repr__(self)
    
    __str__(self)
    
    exists(self)
        Check whether this image or data set exists.
        
        Returns True if the image or data set exists, False otherwise.
    
    getrow_table(self, type, version, rowno)
        Get a row from an extension table.
        
        Returns row ROWNO from version VERSION of extension table TYPE
        as a dictionary.
    
    header(self)
        Get the header for this image or data set.
        
        Returns the header as a dictionary.
    
    header_table(self, type, version)
        Get the header of an extension table.
        
        Returns the header of version VERSION of the extension table
        TYPE.
    
    table(self, type, version)
    
    table_highver(self, type)
        Get the highest version of an extension table.
        
        Returns the highest available version number of the extension
        table TYPE.
    
    tables(self)
        Get the list of extension tables.
    
    verify(self)
        Verify whether this image or data set can be accessed.
    
    zap(self)
        Destroy this image or data set.
    
    zap_table(self, type, version)
        Destroy an extension table.
        
        Deletes version VERSION of the extension table TYPE.  If
        VERSION is 0, delete the highest version of table TYPE.  If
        VERSION is -1, delete all versions of table TYPE.
    
    ----------------------------------------------------------------------
    Properties inherited from _FITSData:
    
    disk
        Disk where this data set is stored.
    
        lambdaself
    
    filename
        Filename of this data set.
    
        lambdaself
\end{verbatim}

\subsection{Server side (proxy)}
The following are implemented in the proxy (remote or local).

\subsubsection{ObitTask}
The server side ObitTask class performs the functions of parsing the
Task definition (TDF) file for the parameters and documentation as
well as operations involving the task execution.
\begin{verbatim}
Help on module Proxy.ObitTask in Proxy:

NAME
    Proxy.ObitTask - Obit tasking interface

FILE
    /export/users/bcotton/share/obittalk/python/Proxy/ObitTask.py

DESCRIPTION
    This module contains classes useful for an Obit tasking interface to python.
    An ObitTask object contains input parameters for a given Obit program.
       The parameters for a given task are defined in a Task Definition File
    (TDF) which gives the order, names, types, ranges and dimensionalities.
    A TDF is patterened after AIPS HELP files.
       The Task Definition File can be derived from the AIPS Help file with the
    addition of:
     - A line before the beginning of each parameter definition of the form:
     **PARAM** [type] [dim] **DEF** [default]
         where [type] is float or str (string) and [dim] is the 
         dimensionality as a blank separated list of integers, e.g.
         **PARAM** str 12 5       (5 strings of 12 characters)
         default (optional) is the default value
     HINT: No matter what POPS thinks, all strings are multiples of 4 characters
     For non AIPS usage dbl (double), int (integer=long), boo (boolean
     "T" or "F") are defined.

CLASSES
    Proxy.Task.Task
        ObitTask
    
    class ObitTask(Proxy.Task.Task)
    Server-side Obit task interface
    
    Methods defined here:
    
    __init__(self)
    
    messages(self, tid)
        Return task's messages.
        
        Return a list of messages each as a tuple (1, message)
        tid   = Task id in pid table of process
    
    params(self, name, version)
        Return parameter set for version VERSION of task NAME.
    
    spawn(self, name, version, userno, msgkill, isbatch, input_dict)
        Start the task.
        
        Writes task input parameters, data directories and other
        information in the task parameter file and starts the task
        asynchronously returning immediately.  Messages must be
        retrieved calling messages.
        If the debug parameter is True then a "Dbg" copy of the task
        input file is created which will not br automatically destroyed.
        name        task name
        version     version of task
        userno      AIPS user number
        msgkill     AIPS msgkill level, not used in Obit tasks
        isbatch     True if this is a batch process , not used in Obit tasks
        input_dict  Input parameters as dictionary
        Returns task id
    
    wait(self, tid)
        Wait for the task to finish.
        
        Waits for Obit task to finishes, reads the task output
        parameter file and deleted tas input and output parameter
        file. 
        Returns output parameters in adictionary.
        tid   = Task id in pid table of process
    
    ----------------------------------------------------------------------
    Methods inherited from Proxy.Task.Task:
    
    abort(self, tid, sig=9)
        Abort the task
        
        Calls abort function for task tid
        None return value
        tid   = Task id in pid table of process to be terminated
        sig   = signal to sent to the task
    
    feed(self, tid, banana)
        Feed the task a  BANANA.
        
        Pass a message to a running task's sdtin
        tid     = Task id in pid table of process
        bananna = text message to pass to task input
    
    finished(self, tid)
        Check whether the task has finished.
        
        tid   = Task id in pid table of process
\end{verbatim}
\subsubsection{AIPSTask}
The Server side AIPSTask class performs the functions of parsing the
Task help file for the parameters and the POPSDAT.HLP file to get
their definitions and documentation as well as operations involving
the task execution. 
\begin{verbatim}
Help on module Proxy.AIPSTask in Proxy:

NAME
    Proxy.AIPSTask

FILE
    /export/users/bcotton/share/obittalk/python/Proxy/AIPSTask.py

DESCRIPTION
    This module provides the bits and pieces to implement an AIPSTask
    proxy object.

CLASSES
    AIPSMessageLog
    Proxy.Task.Task
        AIPSTask
    
    class AIPSMessageLog
    Methods defined here:
    
    __init__(self)
    
    zap(self, userno)
        Zap message log.
    
    class AIPSTask(Proxy.Task.Task)
    Server-side AIPS task interface
    
    Methods defined here:
    
    __init__(self)
    
    abort(self, tid, sig=15)
        Abort the task specified by PROXY and TID.
        
        Calls abort function for task tid on proxy.
        None return value
        proxy = Proxy giving access to server
        tid   = Task id in pid table of process to be terminated
        sig   = signal to sent to the task
                AIPS seems to ignore SIGINT, so use SIGTERM instead.
    
    messages(self, tid)
        Return task's messages.
        
        Return a list of messages each as a tuple (1, message)
        tid   = Task id in pid table of process
    
    params(self, name, version)
        Return parameter set for version VERSION of task NAME.
    
    spawn(self, name, version, userno, msgkill, isbatch, input_dict)
        Start the task.
        
        Writes task input parameters in theTD file and starts the task
        asynchronously returning immediately.
        Messages must be  retrieved calling messages.
        Attempts to use singel hardcoded AIPS TV
        name        task name
        version     version of task
        userno      AIPS user number
        msgkill     AIPS msgkill level,
        isbatch     True if this is a batch process
        input_dict  Input parameters as dictionary
        Returns task id
    
    wait(self, tid)
        Wait for the task to finish.
        
        When task returns, the output parameters are parser from the
        TD file.
        Returns output parameters in adictionary.
        tid   = Task id in pid table of process
    
    ----------------------------------------------------------------------
    Methods inherited from Proxy.Task.Task:
    
    feed(self, tid, banana)
        Feed the task a  BANANA.
        
        Pass a message to a running task's sdtin
        tid     = Task id in pid table of process
        bananna = text message to pass to task input
    
    finished(self, tid)
        Check whether the task has finished.
        
        tid   = Task id in pid table of process

\end{verbatim}

\subsubsection{ObitScript}
This is the server interface to local or remote script execution.
\begin{verbatim}
DESCRIPTION
    This module provides the bits and pieces to implement an ObitScript
    proxy object.

CLASSES
    Proxy.Task.Task
        ObitScript
    
    class ObitScript(Proxy.Task.Task)
      Server-side ObitScript script interface
      
      Methods defined here:
      
      __init__(self)
      
      abort(self, tid, sig=15)
          Abort the script specified by PROXY and TID.
          
          Calls abort function for script tid on proxy.
          No return value
          proxy = Proxy giving access to server
          tid   = Script id in pid table of process to be terminated
          sig   = signal to sent to the script
                  ObitScript seems to ignore SIGINT, so use SIGTERM instead.
      
      messages(self, tid)
          Return script's messages.
          
          Return a list of messages each as a tuple (1, message)
          tid   = Script id in pid table of process
      
      spawn(self, name, version, userno, msgkill, isbatch, input_dict)
          Start the script.in an externak ObitTalk
          
          Writes script test into temporary file in /tmp and executes ObitTalk
          asynchronously returning immediately.
          Messages must be  retrieved calling messages.
          name        script name
          version     version of any AIPS tasks
          userno      AIPS user number
          msgkill     AIPStask msgkill level,
          isbatch     True if this is a batch process
          input_dict  Input info as dictionary
          Returns script id
      
      wait(self, tid)
          Wait for the script to finish.
          
          Waits until script is finished
          tid   = Script id in pid table of process
      
      ----------------------------------------------------------------------
      Methods inherited from Proxy.Task.Task:
      
      feed(self, tid, banana)
          Feed the task a  BANANA.
          
          Pass a message to a running task's sdtin
          tid     = Task id in pid table of process
          bananna = text message to pass to task input
      
      finished(self, tid)
          Check whether the task has finished.
          
          tid   = Task id in pid table of process
\end{verbatim}

\subsubsection{AIPSData}
Remote data access functions live in the proxy AIPSData class.
\begin{verbatim}
Help on module Proxy.AIPSData in Proxy:

NAME
    Proxy.AIPSData

FILE
    /export/users/bcotton/share/obittalk/python/Proxy/AIPSData.py

DESCRIPTION
    This module provides the bits and pieces to implement AIPSImage and
    AIPSUVData objects.

CLASSES
    AIPSCat
    AIPSData
        AIPSImage
        AIPSUVData
    
    class AIPSCat
    Methods defined here:
    
    __init__(self)
    
    cat(self, disk, userno)
    
    class AIPSData
    Methods defined here:
    
    __init__(self)
    
    exists(self, desc)
    
    getrow_table(self, desc, type, version, rowno)
    
    header(self, desc)
    
    header_table(self, desc, type, version)
    
    table_highver(self, desc, type)
    
    tables(self, desc)
    
    verify(self, desc)
    
    zap(self, desc)
    
    zap_table(self, desc, type, version)
    
    class AIPSImage(AIPSData)
    Methods inherited from AIPSData:
    
    __init__(self)
    
    exists(self, desc)
    
    getrow_table(self, desc, type, version, rowno)
    
    header(self, desc)
    
    header_table(self, desc, type, version)
    
    table_highver(self, desc, type)
    
    tables(self, desc)
    
    verify(self, desc)
    
    zap(self, desc)
    
    zap_table(self, desc, type, version)
    
    class AIPSUVData(AIPSData)
    Methods inherited from AIPSData:
    
    __init__(self)
    
    exists(self, desc)
    
    getrow_table(self, desc, type, version, rowno)
    
    header(self, desc)
    
    header_table(self, desc, type, version)
    
    table_highver(self, desc, type)
    
    tables(self, desc)
    
    verify(self, desc)
    
    zap(self, desc)
    
    zap_table(self, desc, type, version)

\end{verbatim}

\end{document}

% for lists
\begin{enumerate}
\item  \hfil\break
\end{enumerate}

% for figures
\begin{figure}
\centerline{\psfig{figure=graphic.eps,height=4in,angle=-90}}
\caption{ 

}
\label{graphic}
\end{figure}

% Example figures
\begin{figure}
\centerline{
\psfig{figure=ZerkFig1.eps,height=3.5in}
\psfig{figure=ZerkFig2.eps,height=3.5in}
}
\centerline{
\psfig{figure=ZerkFig3.eps,height=3.5in}
\psfig{figure=ZerkFig4.eps,height=3.5in}
}
\caption{ 
Top: Fitted model ionospheric OPD screen rendered as a plane in 3-D
viewed from different angles. 
The surface representing the OPD screen is shown with its projection
onto the bottom of the box.\hfill\break
Bottom: As above but without the linear gradients to emphasize the
curvature. 
The radius shown is 10$^\circ$.
Models computed and plotted by
http://wyant.opt-sci.arizona.edu/zernikes/zernikes.htm. 
}
\label{IonModel}
\end{figure}


% Example table
\begin{table}[t]
\caption{Observing dates}
\vskip 0.1in
\begin{center}
\begin{tabular}{|l|c|c|c|c|l|}   \hline
\hline
Date & Start IAT & End IAT \\
\hline
 12 October 1998 & 21 00 & 21 15\\
 22 January 2000 & 17 00 & 29 00$^1$\\
 23 January 2000 & 17 00 & 29 00$^2$\\
\hline
\end{tabular}
\end{center}
\hfill\break
Notes:\hfill\break
$^1$ Times beyond 24 are the next day\hfill\break
$^2$ Not including 20 30 to 23 00\hfill\break
\label{Observations}
\end{table}






Single dish/OTF imaging classes and utilities.
These require the ObitSD python directory in the PYTHONPATH.
\begin{itemize}
\item {\bf CCBUtil}        GBT CCB utility package
\item {\bf CleanOTF}       Single dish (Hogbom) CLEAN
\item {\bf GBTDCROTF}      Convert GBD DCR data to OTF format
\item {\bf OTFDesc}        OTF Descriptor
\item {\bf OTFGetAtmCor}   OTF Atmospheric correction utilities
\item {\bf OTFGetSoln}     OTF calibration solution utilities
\item {\bf OTF}            OTF ("On the Fly") data
\item {\bf OTFSoln2Cal}    Utilities to convert OTF solutiion to calibration tables
\item {\bf OTFUtil}        OTF Utilities
\end{itemize}

