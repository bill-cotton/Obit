% $Id$ 
%
%  OBIT class definitions
%
%\def\section #1.#2.{\medskip\leftline{\bf #1. #2.}\smallskip}
%\def\bfitem #1#2{{{\bf (#1)}}{\it #2}}
\def\bfi #1#2{{\bf #1:}{ #2}\par}
\def\extname #1#2{ {\bf 1} Extension Name {\it #2}\smallskip}
%\def\tabname #1#2{\bfitem 1{#1 Table Name #2}}
\def\keyword {\leftline{\bf Keywords:}}
%\def\table {\leftline{\bf Table Definition:}}
\documentclass[11pt]{article}
\usepackage{OBITdoc}
\begin{document}
\setcounter{section}{0}
%  Title page

%\centerline{\ttlfont PRELIMINARY}
\vskip 5cm
\centerline{\ttlfont OBIT }
\vskip 1cm
\centerline{\ttlfont Merx mollis mortibus nuper}
\vskip 3cm
\centerline{\secfont Version: 1.2 \today}
\vskip 1cm
\centerline{\secfont W. D. Cotton}
\clearpage

% Table Of Contents
\tableofcontents
\cleardoublepage

\section {Introduction}
OBIT is a software library package intended for astronomical,
especially radio--astronomical applications. 
This document describes the high level view of the software system and
is the fundamental definition of the tables  used in the Obit software
package.  

Obit is an object--oriented set of class and utility libraries
allowing access to multiple disk--resident data formats.
In particular, access to either AIPS disk data or FITS files.
Documentation of the Obit classes uses the doxygen documentation
system and can be accessed starting at doc/doxygen/html/index.html.

\section {Design Objectives}
The principle motivation for Obit is to provide a software environment
where I can develop algorithms and develop targeted production
software for particular projects, especially wide area sky surveys.
AIPS has gotten too unwieldy for new development.
However, there are a number of technical problems for which this can be
considered exploratory.

\subsection{Multiple Persistent ``Native'' Data Forms}
With proper layering and inheritence strategies, multiple external data
forms can be used as ``Native'' data, i.e. do not need to be
translated to/from another persistent data system.
This requires that there be a common data (including calibration)
model and that the data models can all be mapped onto that of the
software (and vice versa).
For the interferometric and image applications in Obit, the current
implementation supports both AIPS and FITS files as internal formats.
At the applications level, the only time the distinction is made is
when the object is associated with a particular external AIPS entry or
FITS file.
Derived IO classes translate the disk resident structures to Obit's
memory resident forms and back again.

\subsection{Flexible Class System in c}
Medium to large scale software packages are much more tractable using
object-oriented methodology.
In practice this means a class system supporting inheritance and
polymorphism.
The obvious candidates for this are C++ and Java.
In practice, C++ software has serious robustness problems and it
appears beyond my ability to write (actually to debug) C++ code to a
sufficient standard.
Java is a much better designed language but has a performance penalty
of a factor of several.

   The fallback position is a ``roll my own'' class system in ANSCI
c.
This seems to be a reasonable compromize between robustness and
performance.
However, the c family has issues, especially with memory usage which
have been only partially solved in Obit.

Each Obit class has a global structure which contains class information,
these include function pointers and a pointer to the parent class
structure.

In practice, the class and inheritence system is implemented using
include files and the c preprocessor.
Each class definition header recursively includes its parents object
definition, making all data members effectively ``public''.
Member functions of the class can be either those of the parent
class, set to those specific to the derived class, set to NULL, or new
class function added.
Private member functions are allowed by declaring them ``static''
inside the class implementation file.
Class member functions can be invoked either explicitly or by
referencing the function from the class structure.

This implementation has the drawback that a object must be explicitly
included as arguments in calls to its function, this makes the
relationship less obvious.
The convention has been adopted to make the ``self'' object the first
argument in the call sequence.

\subsection{Robust Object Managment}
The c language family has no garbage collection meaning that
applications must do their own memory managment.
A further complication is that the c family memory managment tools are
very unforgiving and errors are frequently very difficult to locate.
E.g. a second ``free'' of a block of memory causes no immediate
problem but at some later, and usually unrelated point, the program
will abort.  Part of the solution is to adopt the glib memory
managment routines - at least g\_free does not abort given a null
pointer.

   Many objects contain substantial resources, memory, open files etc.
which would be expensive and risky to frequently copy.
The usual solution to this problem is to allow multiple pointers to
the same object; thus an object can serve as a member of another
object(s) as well as stand on its own.
This aggravates a further problem that each object must be destroyed
and its resources released EXACTLY once.
The standard solution to this problem is ``reference counting'', each
object keeps a count of the current number of references.

   This solution has yet further consequences, a great deal of
discipline is required to keep the reference count correct.
The Obit solution to this problem is a pair of member functions for
each class to reference and unreference an object.
The reference function increments the objects reference count and
returns a pointer to the object.
The unreference function decrements the reference count, destroying
the object if it hits zero and returns a NULL pointer.
There is no explicit object destructor.

A further complication is that objects may be passed as a parent
class.  
To allow type checking, each class has an ``IsA'' function to
determine if an object is of the correct type and is valid.
This is usually wrapped into a glib assertion that can be turned off
in production (debugged) software.

\subsection{Parallelization}
The use of Fortran named commons in AIPS makes high level
parallelization difficult in that environment.
The use of multiple objects with independent resources reduces the
difficulty.
In addition, most classes have an ObitThread member to provide
multi--threaded operation, locking and synchronism mechanisms.
cfitsio, which is used for I/O to FITS files, is not thread--safe
which limits parallel, multi--threaded usage to in--memory
applications.

Multi-threading is implemented in the most CPU intensive operation by
splitting the work on a single I/O buffer amoung various threads.
This means fairly small work packets so the threading is done using 
gthreads thread pools.
This is accessed via ObitThreadIterator.

\subsection{Portability}
The glib library was intended to reduce portability problems and many
of its features are adopted.

\subsection{Name Space Control}
The name space is controlled by including the class name at the
beginning of all public function names.
All begin with ``Obit''.

\subsection{Passing Control Parameters}
One of the great causes of instability in software is the changing of
call sequences through which control parameters are passed when these
are changed.
This problem is dealt with in Obit by means of the ObitInfoList which
is basically an associative array (labeled, arbitrary pieces of
information). 
Objects for which control parameters may be relevant have an
ObitInfoList member.
High level routines put control information into this container rather
than passing it through call sequences. 
Lower level routines can then take the appropriate response to a
missing entry, either assume a default value or treat the absence as
an error condition. 

\subsection{Error handling}
Trapping error conditions and giving a traceback to their cause is
combined with informative and warning messages in the ObitErr class.
An ObitErr has an error indication word and a message stack; each
message has an associated error level.
Utility macroes are provided such that when a routine detects a
condition that it cannot handle, it set an error flag, registers an
error message giving the software locations and returns.  
The calling routine can, deal with the problem and remove the error
condition and messages, add additional diagnostic information and
return, or note its location in the error log and return.
Routines with an ObitErr argument should check that no prior error
condition exists and return if one does.
At the appropriate level, the contents of the message log can be
displayed and any error condition cleared.
At present, the output is to stdout or a specified log file, but more
formal logging, such as AIPS logging, is easily implemented.
The error object also contains a ``prtLv'' (print level) member that
can be used to control the amount of diagnostic messages.
The prtLv and logging to a file are enables using the ObitErrInit
function.


\section{Obit Software System}
The Obit system maintains its basic information in global structures.
These are initialized by a call to ObitSystemStartup and released by a
call to ObitSystemShutdown.

\subsection{General Software Issues}
Obit is written in ANSI c of the glib dialect to assist in
portability.  
Thus most data types are declared with the ``g'' prefix (e.g. gfloat =
float). 
Also, g\_malloc and g\_free are used to allocate and free memory.

Each class and structure in Obit has an ``IsA'' function to test that
the pointer passed is to a valid instance of that structure or class.

The glib library provides assertion macroes which are used to test for
programming errors but the tests can be turned off in a production
compilation.
The assertion  g\_assert should be used with the class IsA test
functions to assure that the inputs to each routine are valid.
If the assertion fails, the program aborts and dumps core.

\subsubsection{Obit Initialization/Shutdown}
Persistent data forms are kept in a number of directories referenced
by ``disk'' number.
There are separate lists for AIPS and FITS data and these are
initialized when Obit is initialized using the ObitSystemStartup
function.
If the relevant directories are given in standard Unix environment
variables, they can be picked up automatically.
This is illustrated in the following code fragment:
\begin{verbatim}
  ObitSystem *mySystem;
  ObitErr *err;
  gint pgmNumber = 1, user=100, nAIPS = 0, nFITS=0;

  /* Initialize Obit */
  err = newObitErr();  /* Create error/message structure*/
  mySystem = ObitSystemStartup ("test", pgmNumber, user, nAIPS, NULL,
                                nFITS, NULL, (oint)TRUE, (oint)FALSE, err);
  ObitErrLog(err); /* show any error messages on err */
\end{verbatim}
Here the nAIPS and nFITS give the number of explicit AIPS of FITS
directories. 
The NULL following these arguments in the call sequence indicate that
the standard defaults are to be used.
If the directory name array passed is NULL, startup will search for
standard environment names. 
These are \$DA01, \$DA02 ... for AIPS and \$FITS, \$FITS01 ... for
FITS.
If given explicitly, the directories can be either absolute or
relative.
If the directory number for FITS files is 0, the name is assumed to be
an adequate relative or absolute pathname.
Also specified in the ObitSystemStartup call are the name of the
program (here ``test'') and a program number, pgmNumber, which is the
POPS number for AIPS usage.

   Obit classes also need initialization .
This is done automatically for a class and all its parents whenever
an Obit object is created.

Obit programs may generate scratch files which are recorded by the
ObitSystem class.
If these are not explicitly deleted (using the relevant Unref
function) then they will be deleted by the ObitSystem shutdown:
\begin{verbatim}
    /* Shutdown Obit */
    mySystem = ObitSystemShutdown (mySystem);
\end{verbatim}

\subsubsection{Memory}
Use and misuse of memory is one of the biggest source of bugs in the c
family of languages.
Obit uses the glib memory allocation routines to avoid some of the
spectacular failures of the c malloc/free routines.
As an additional layer of security and as an aid in finding memory
leaks, Obit contains an ObitMem class for allocation and maintaining
blocks of memory.
This class is used extensively in Obit and maintains a list of
optionally named blocks of memory allocated.
To be deallocated, the pointer must be in the currently active list.
This list can be printed using ObitMemPrint.
The class can also be queried if a given pointer is inside an
allocated block.

   Many memory related problems (leaks, array overruns, NULL
allocations...) can also be diagnosed using the memwatch package (as
modified for glib/Obit) from Johan Lindh (http://www.linkdata.se/)
This option is invoked using the -DMEMWATCH compile option and
compiling and linking all Obit code.  Very useful diagnostics are
written into the file memwatch.log in the current working directory.
This option overrides the ObitMem class usage.
Don't forget to recompile/link without this option when through
debugging as it incurs substantial overhead.
It is important that all software linked together is all compiled
with or without the -DMEMWATCH option.
The Unix utility valgrind is very good (if very slow) at finding
memory related problems.

\subsubsection{Messages and error handling}
Messages and error handling is by means of the ObitErr class.
An ObitErr has an error condition flag and a stack of messages each
with a category, i.e. error, informative...
Low level routines enter messages on the stack which is then
explicitly logged and cleared using the ObitErrLog function.
If an error is encountered, this condition is entered into the ObitErr
structure and can be dealt with as appropriate by the calling
routine.
On entry, routines should check if an error condition exists and if so
return.
Tracebacks which are useful for debugging are obtained using the
Obit\_traceback macros.

Message logging can be customized by an application specific logging
handler.
Currently all messages are written to stdout or a specified log file.

The most useful ways of entering information into an ObitErr are
through macroes defined in the class:
\begin{itemize}
\item Obit\_log\_error(err,errCode,format...)\\
log error message.
\item Obit\_return\_if\_fail(expr,err,format...) \\
Macro to evaluate expression, log in err and return on failure.
\item Obit\_retval\_if\_fail(expr,err,out,format...) \\
Macro to evaluate expression, log in err and return a value 
on failure.
\item Obit\_traceback\_msg(err,me,name)\\
Macro for traceback when an error in a called routine is encountered.
Writes traceback info and returns (no return value).
Gives location (file, routine, line) where message generated.
\item Obit\_traceback\_val(err,me,name,out)\\
Macro for traceback logging with a return value
Writes traceback info and returns, passing a value.
Gives location (file, routine, line) where message generated.
\item  Obit\_cfitsio(err)\_error\\
Macro to dump cfitsio error stack to an ObitErr.
\end{itemize}
The traceback macroes should be used to check for errors in routines
called which cannot be handled in the current level and need to be
passed to higher levels.

\subsection{Obit Classes}
Many, but not all, of the structures in Obit are inherited from the
basal Obit virtual class.
Obit includes template versions (ObitTEMPLATE*) files to assist in
generating new classes; these template files contain the basic parts
needed for a new class.
The Obit class system is described below.

\subsubsection{Organization}
The class name of the class being defined in a file begins with the
name of the class and include the following files for class xxx.
\begin{enumerate}
\item include/xxx.h\\
Class interface definition.  Defines structures, macroes and public
functions.
\item include/xxxDef.h\\
File included in include/xxx.h which defines the object structure.
It first includes the corresponding file from its parent class.
\item include/xxxClassDef.h\\
File included in include/xxx.h which defines the class structure.
It first includes the corresponding file from its parent class.
\item src/xxx.c\\
File containing the source code for the class; including class initialization.
\end{enumerate}
Derived classes begin with the name of the parent class such that the
inheritance is given in the class name.

Obit objects are connected to their respective classes by a pointer to
a global class structure which contains a pointer to the parent class
structure and function pointers for the class.
Class member functions may be invoked either explicitly or via the
class function pointers.
Class objects, and all parent classes are initialized upon the
instantiation of the first object of that class.

Obit objects have explicit creators but no explicit destructors.
Instead, a referencing/defererencing system is used to destroy objects
when the last reference to them is removed.
A reference (pointer) is passed from the ObitRef (or class specific)
function which increments the objects reference count.
A reference is deleted using the ObitUnref (or class specific)
function which decrements the object reference count, destroying it if
it goes to zero and returns a null pointer.
{\bf The integrity of Obit depends on strict adherence to this
convention.}

   There are also ``secret'' references to objects needed to avoid a
cyclical set of references.
These references should not be used to manage the object's reference
count, i.e. should not be passed to Ref or Unref functions.
Local, convenience, copies of object pointers should also be treated
this way.

The Obit virtual class object contains the following members:
\begin{enumerate}
\item gint32 ObitId;\\
 Recognition bit pattern to identify the start of an Obit object.
\item gpointer ClassInfo;\\
ClassInfo pointer for class with name, base and function pointers.
\item gint ReferenceCount;\\
 Reference count for object (numbers of pointers attaching).
\item gchar *name;\\
 Name of object [OPTIONAL], this is used in error and informative messages.
\end{enumerate}

The Obit virtual class structure contains the following members:
\begin{enumerate}
\item gboolean initialized;\\
Have I been initialized?
\item gboolean hasScratch;\\
 Are disk resident "scratch" objects of this class possible?
\item gchar* ClassName;\\
Name of class ("Obit")
\item gconstpointer ParentClass;\\
Pointer to parent class ClassInfo, Null if none.
\item ObitClassInitFP ObitClassInit;\\
 Function pointer to Class initializer
\item newObitFP newObit;\\
Function pointer to newObit (default constructor).
Initializes class and parent if needed.
\item ObitCopyFP ObitCopy;\\
 Function pointer to shallow copy constructor.
\item ObitCloneFP ObitClone;\\
Function pointer to deep copy constructor
\item ObitRefFP ObitRef;\\
Function pointer to Object Reference
\item ObitUnrefFP ObitUnref;\\
Function pointer to Object Unreference
\item ObitIsAFP ObitIsA;\\
Function pointer to test if a class member.
\item ObitClearFP ObitClear;\\
Private Function pointer to deallocation function.
\item ObitInitFP ObitInit;\\
Private Function pointer to object initializer.
\end{enumerate}

\subsubsection{Inheritance}
The object and class strucures are defined in the *Def.h and
*ClassDef.h files as described above.
Inheritance is by means of including the class object and class
structure definition files (*Def.h and *ClassDef.h) at the beginning
of the corresponding class files.
This recursive include assures the full inherited structure is present
in each object and class.

   A template class, ObitTEMPLATE, is provided to simplify creating
new classes.
To create a new class copy the ObitTEMPLATE*.h and ObitTEMPLATE*.h to
new files where TEMPLATE is replaced by the actual class name.
Then, inside each of these files make the same substitutions.
If the parent class is not the basic Obit class, then make the
following changes:
\begin{enumerate}
\item Obit*.c;\\
Near the top of the file change the valus of ObitParentGetClass to
the Obit?GetClass function of the parent class.
\item Obit*.h;\\
Add an include to the Obit*.h file of the parent class.
\item Obit*Def.h;\\
Change include of ObitDef.h to the corresponding parent file,
e.g. ObitTEMPLATEDef.h
\item Obit*ClassDef.h;\\
Change include of ObitClassDef.h to the corresponding parent file,
e.g. ObitTEMPLATEClassDef.h
\end{enumerate}

\subsubsection{Constructors}
Obit objects are explicitly created by a constructor.
The default constructor has the name new[ObitClassname] and accepts an
optional name for the object.
This name is used in error and informative messages and should reflect
its usage.
Derived classes may have other constructors.

\subsubsection{Reference/destructors}
Obit classes have no explicit destructors.
Instead, this is handled by the Obit reference scheme.
Pointers to an object are assigned by means of the ObitRef (or class
specific) method which increments the object reference count.
The ObitUnref (or class specific) method removes a reference,
decrementing the reference count and destroying the object when the
count goes to zero.

\subsubsection{ObitInfoList}
Most Obit classes have an ObitInfoList member which is a container for
labeled arrays of information.
Most control information is passed to and stored by the object in this
member.

\subsection{Obit Interface to Persistent Data}
Obit supports multiple disk-resident forms; currently AIPS and FITS
images and UV data as binary tables.
Interfaces to external data are via derivations from the ObitIO
class.
Explicit derived classes are needed for each data type (UV Image,
table...) and for each form (AIPS, FITS).
Except for associating an object with an explicit external source (data
file), most of Obit should not be aware of the source of the data
outside of the explicit IO classes.
The cfitsio package is used for basic FITS I/O.
\footnote{The fitsio library is a package of Fortran and c language
callable routines and is available at 
http://heasarc.gsfc.nasa.gov/docs/software/fitsio/fitsio.html.
This package takes care of most of the details of reading and writing
FITS files, including binary tables, and is well documented.}

Directories containing AIPS or FITS data files are registered with
Obit in the ObitSystemStartup call.
These directories are then referenced by a (1--rel) index.

\subsubsection{Images}
Images are accessed a plane (or row) at a time and are internally
kept as a 1-D array of float.
An ObitImage may have associated tables.
A descriptor object (class ObitImageDesc) is used to describe the data.
An ObitIO object is used to access the external data.

\subsubsection{UV data}
UV data is accessed as blocks of ``rows'' of visibility data which are
internally kept as 1-D arrays of float for efficient access.
The structure of UV data assumed is that of AIPS; ``Random'',
descriptive data followed by a rectangular data array.
``Compressed'' data is supported.
UV data in FITS format use binary tables for the visibility data but
these tables are handled differently from normal tables for increased
performance.
An ObitUV may have associated tables.
A descriptor object (class ObitUVDesc) is used to describe the data.
An ObitIO object is used to access the external data.

\subsubsection{Tables}
ObitTables are always associated with Image, UV data or other data.
A ``row'' in a table corresponds to a row in a FITS binary table and
consists of an array of arrays of data of the same native type.
Access is a row at a time and row data are converted to/from explicit
class structures.
A descriptor object (class ObitTableDesc) is used to describe the data.
An ObitIO object is used to access the external data.

\subsection{Threading}
Multi--threaded executation in Obit is by means of the ObitThread
class which uses the gthreads implementation of glib.
Multi--threading is currently only supported for in--memory operations as
cfitsio is not thread-safe.
This class provides pools of threads to reduce the overhead of
starting/stopping many threads doing the same operation.
Multiple copies of a routine can be executed in parallel using
ObitThreadIterator; the function to be run in multiple copies should
have a signature of type ObitThreadFunc, e.g.:
\begin{verbatim}
gpointer ThreadFunction (gpointer arg);
\end{verbatim}
A simple demo program is the following:
\begin{verbatim}
/* Demo threading concepts */
#include "Obit.h"
#include "ObitThread.h"
/* Function argument structure */
typedef struct {
  ObitThread *thread;  /* Threading control object 
                          MUST be the same as passed to
                          ObitThreadIterator */
  olong      ithread;  /* Thread number */
  olong      ndata;    /* Number of entries in data */
  ofloat     *data;    /* data array */
} FuncArg;

/**
 * Demo routine to run in parallel threads.
 * Fills data values in array arg->data
 * Arguments are given in the FuncArg structure passed as arg
 * \param arg  Pointer to FuncArg argument with elements:
 * \li thread  Thread control object
 * \li ithread thread number
 * \li ndata   number of entries in data
 * \li data    data array
 * \return NULL
 */
gpointer fn(gpointer arg) {
  FuncArg  *farg = (FuncArg*)arg;
  olong j;
  
  /* Say I'm alive */
  fprintf (stdout," Running thread %d\n", farg->ithread);

  /* Fill data array */
  for (j=0; j<farg->ndata; j++) farg->data[j] = 0.01*j*farg->ithread;
 
  /* Indicate completion */
  ObitThreadPoolDone (farg->thread, (gpointer)&farg->ithread);
  
 return NULL;
} /* end fn */

/* Main program to demo threading */
int main ( int argc, char **argv )
{
  ObitThread *th=NULL;
  olong i, nThreads = 5;
  ObitThreadFunc func=(ObitThreadFunc)fn;
  gpointer *fargs[5];
  FuncArg args[5];
  
  /* Enable threading on 5 processors/cores */
  th = newObitThread();
  ObitThreadAllowThreads(th, nThreads);

  /* Initialize thread function argments */
  for (i=0; i<nThreads; i++) {
    args[i].thread  = th;
    args[i].ithread = i;
    args[i].ndata   = 1000;
    args[i].data    = g_malloc0(1000*sizeof(ofloat));
    fargs[i] = (gpointer)&args[i];
  }

  /* Execute - run 5 parallel copies of routine fn,
     will return on completion of all threads */
  ObitThreadIterator (th, nThreads, func, fargs);
  
  /* Shut down the thread pool */
  ObitThreadPoolFree (th);

  return 0;
} /* end main */
\end{verbatim}

Synchronization of threads uses glib asynchronous message queues.
Each copy of a threaded routine MUST call ObitThreadPoolDone to
indicate it is completed (although there is a 1 minute timeout).

The ObitThread class  provides mutexes for absolute 
locking of associated resources as well as RWLocks to allow multiple
read accesses but only a single write (and no concurrent reads).
Usage of threading needs OBIT\_THREADS\_ENABLED defined at compile time and
the output of  pkg-config --libs gthread-2.0 added to the libraries.

Multi-threading in a task using a user--specified parameter
(``nThreads'') given in the ObitInfoList myInput, is initialized by a
call to ObitThreadInit (myInput).
Alternatively, ObitThreadAllowThreads can be used as in the example above.

\subsection{Documentation}
Class documentation uses the doxygen system which uses specially coded
comments in the code to generate documentation.
The class documentation in html format starts at doc/doxygen/html/index.html.

\subsection{History}
Storing processing history is done principally via the ObitHistory class.
Both FITS (as Tables) and AIPS cataloged data are supported.
This class is derived from the Obit class.
The AIPS conventions for history records are to give the program name followed
by parameters in keyword=value form and to preceed any non parsable text
by the FITS comment header comment delimiter '/'.

History tables are accessed as tables although the AIPS implementation 
is a pre-table version.  
History records are blocked into 70 character  fixed strings althought
AIPS internally uses 72. 

History records are stored in a system dependent fashion.
AIPS history records are stored in an AIPS HI* file with 72 characters 
per record (Obit only uses 70).
For FITS files, history records are normally kept in A History binary table
but can be read or written to the more traditional HISTORY keywords using
ObitHistoryCopyHeader, or ObitHistoryCopy2Header for entire collections
or ObitFileFITS functions ObitFileFITSReadHistory and ObitFileFITSWriteHistory
for individual records.
Access to the history component of an object (e.g. Image, UV data) can
be obtained using the ObitInfoList containing the information defining the 
underlying file.
This uses routine newObitHistoryValue.
Then history lines can be read or written one at a time using ObitHistoryOpen, 
ObitHistoryReadRec, ObitHistoryWriteRec, ObitHistoryTimeStamp, ObitHistoryClose.
The contents of entire history files may be copied  using ObitHistoryCopy, or
ObitHistoryCopyHeader to copy HISTORY records from a FITS header or
ObitHistoryCopy2Header to copy a history file to a FITS header.

\section{Interferometry Software}
   Much Obit software  is designed to support radio interferometers.
The ObitUV class represents an interface to radio interferometric
data.
The following sections describe some of the major classes for
interferometry.
Many of the algorithms used are adopted from AIPS.

\subsection{UV Data Modeling}
Lists of models or images can be Fourier transformed by the
ObitSkyModel class and either divided into or subtracted from (added
to) an ObitUV.
Both DFT and Gridded model calculations are included for non spatially
variant SkyModels and only DFT for these.
Specialized subclasses of ObitSkyModel include:
\begin{itemize}
\item {\bf ObitSkyModelVM} \\
Base class of spatially variant sky models.
\item {\bf ObitSkyModelVMBeam} \\
Spatial variation given by tabulated beam shapes.
\item {\bf ObitSkyModelVMIon} \\
Spatial variation given by ionospheric model.
\item {\bf ObitSkyModelVMSquint} \\
Spatial variation given by calculation of (E)VLA beam squint.
\item {\bf ObitSkyModelMF } \\
Tabulated frequency dependent sky model.
\end{itemize}

\subsection{Imaging}
   The basic interface to the imaging software is the ObitUVImager
class. 
Given an ObitUV with control parameters, this class can produce an
ObitImageMosaic, an array of images that tile a region of the sky.
Optionally, the array can be flattened into a single image.
Specialized subclasses of ObitUVImager include:
\begin{itemize}
\item {\bf ObitUVImagerIon} \\
Imaging for spatial variation given by ionospheric model.
\item {\bf ObitUVImagerSquint} \\
Imaging for spatial variation given by calculation of (E)VLA beam squint.
\item {\bf ObitUVImagerMF} \\
Imaging for Tabulated frequency dependent sky model.
\item {\bf ObitUVImagerWB} \\
Sault-Wieringa imaging.
\end{itemize}

\subsection{Deconvolution}
   There is a heirarchy of classes derived from the ObitDCon
(Deconvolution) virtual base class.
All current implementations are variants of the CLEAN algorithm.
Both Clark and SDI/Greisen CLEANS are implemented.
This heirarchy is currently (Mar. 2010) only partially implemented but
the high level design is shown in the following:

\begin{verbatim}

 Deconvolution Class Hierarchy
 =============================

                          ObitDcon
                              |
                          ObitDconClean
                              |
                        ----------------
                        |              |
              ObitDconCleanImage  ObitDConCleanVis
               (APCLN like)          (MX like)
                                         |
                           -------------------------
                           |                       |
                 ObitDConCleanVisMF        
                 (MultiFrequency)             

\end{verbatim}

These classes are described in the following:
\begin{itemize}
\item ObitDCon\\
Virtual base class of all deconvolution methods.
\begin{verbatim}
Members:
  ObitImageMosaic

Functions:
  Create, Copy, Clean, Deconvolve
  Set Image Mosaic
\end{verbatim}
\item ObitDConClean\\
Virtual base class of CLEAN although much functionality used by
derived classes is defined here.
\begin{verbatim}
Major Members:
   ObitDConCleanWindowList
   ObitDConCleanBmHist
   ObitDConCleanPxHist
   ObitDConCleanPixelList
   BeamPatch as FArray
   CLEAN control parameters (as arrays per field):
      Gain, Factor, Flux, Niter

Functions: (Called from Deconvolve)
   ObitDConCleanRestore     Restore subtracted components
   ObitDConCleanXRestore    Restores components from one field
                            appearing in another
   ObitDConCleanFlatten     Flatten multiple facets to one 
   ObitDConCleanSelect      Select components to be subtracted
                            Load PixelList, BeamPatch, minor CLEANing
   ObitDConCleanSub         Subtract components, produce new residual
                            image(s),
   ObitDConCleanPixelStats  Prepare for minor cycle:
      ObitDConCleanPixelHist   Get image pixel histogram (one or more images)
      ObitDConCleanBeamHist    Get beam histogram
      ObitDConCleanDecide      Set Patch, min flux for next clean
\end{verbatim}
\item ObitDConCleanImage\\
APCLN-like image only CLEAN.
This implements a variation on the basic BGC CLEAN.
This implementation is not limited to interferometry but can work on
any image for which a stationary convolution (``Dirty Beam'') can be
derived.
\begin{verbatim}
Major Members:
   Transfer image (FFT of beam) as ObitCArray
   FFT object(s)

Functions:
   derived versions of:
   ObitDConCleanSub         Subtract components, produce new residual
                            Use convolution.  Uses Fourier Convolution
			    Theorem to convolve components with the
			    dirty beam to subtract from the previous
			    residual image.
                            image(s)
\end{verbatim}
\item ObitDConCleanVis\\
MX-like Cotton-Schwab CLEAN where components are subtracted from the
visibility data which is reimaged to get the next round of residual
images. 
SDI/Greisen variant also supported.
Supports AIPS  2D or 3D  type.
\begin{verbatim}
Major Members:
   ObitUVImager

Functions:
   ObitDConCleanVisPickNext   Selects next field to be cleaned

   derived versions of:
   ObitDConCleanDeconvolve  Control deconvolution algorithm.
   ObitDConCleanSub         Subtract components, produce new residual
                            Uses ObitSkyModel and ObitImager

\end{verbatim}
\item ObitDConCleanVisMR\\
Multi-resolution CLEAN.
NYI
\end{itemize}
 
A number of related classes are described in the following:
\begin{itemize}
\item ObitDConCleanWindow\\
List of Windows for each field that describe the regions on which the
CLEAN algorithm is to work.
This list is expandable and hides the details of the window description
by returning an image mask of the selected values.
Initial implementation is for restangular and round windows.
\item ObitDConCleanBmHist\\
This class manages the beam histogram used in the B. G. Clark CLEAN
algorithm which is used in Obit classes.
\item ObitDConCleanPxHist\\
This class manages the image pixel histogram used in the B. G. Clark
CLEAN algorithm which is used in Obit classes.
\item ObitDConCleanPxList\\
This class manages lists of selected residuals from one or more images
and implements the basic BGC CLEAN algorithm.
Subclass ObitDConCleanPxListMF used for tabuilated spectrum wideband imaging.
\end{itemize}

\section{Building Obit}
Obit comes with a configure script to construct the Makefiles to build
Obit.
An installation script, InstallObit.sh, is included in the
distribution which attempts to do a full installation.
The following describes the installation of Obit in case
InstallObit.sh does not work. 
Note: there are a number of third party packages that Obit requires
and these must be installed prior to Obit.
See the README file for more current details.
The basic installation is thus:
\begin{verbatim}
 % gtar xzvf Obit1.0.tgz
 % cd Obit
 % ./configure
 % make
\end{verbatim}

Several of the external packages are not critical and Obit will build
without them.
In particular, these are plplot, cfitsio, GSL, and FFTW3.
However, if one of these is not installed or not found, then an attempt
to use them will cause the program to abort.
pgplot is used only in debugging and is not required for production use.
The configure script searches in the standard places for external
libraries and if they are in nonstandard places it will need some
help.
This help is provided in the form of arguments to the configure script:
\begin{verbatim}
  --with-x                use the X Window System (pgplot)
  --with-pgplot=DIR       search for PGPLOT in DIR
                          The libraries and header files are expected
                          here
  --with-cfitsio=DIR      search for CFITSIO in DIR/include and DIR/lib
  --with-cfitsio-includes=DIR
                          search for CFITSIO includes in DIR
  --with-fftw=DIR         search for FFTW in DIR/include and DIR/lib
  --with-fftw-includes=DIR
                          search for FFTW includes in DIR
  --with-gsl=DIR          search for GSL in DIR/include and DIR/lib
  --with-gsl-includes=DIR search for GSL includes in DIR
  --with-xmlrpc=DIR       search for XMLRPC in DIR/include and DIR/lib
  --with-xmlrpc-includes=DIR
                          search for XMLRPC includes in DIR
  --with-www=DIR          search for WWW in DIR/include and DIR/lib
  --with-www-includes=DIR search for WWW includes in DIR
  --with-zlib=DIR         search for ZLIB in DIR/include and DIR/lib
  --with-zlib-includes=DIR
                          search for ZLIB includes in DIR
\end{verbatim}

The following gives an example of use when libraries are installed in
\$HOME/opt/:
\begin{verbatim}
 % configure --with-pgplot=$HOME/opt/pgplot \
--with-cfitsio=$HOME/opt/cfitsio \
--with-fftw=$HOME/opt/fftw
\end{verbatim}

When using Obit from python, specify the PYTHONPATH environment
variable as ``Obit/python'' where for Obit substitute the base
directory of the Obit package. 


\subsection{Changing the Structure of Obit}
Changes in the structure of the package may require modifications to
the configure script.
This requires a number of stages:
\begin{itemize}
\item  aclocal \\
This program generates an aclocal.m4 file as needed by autoconf.
This is based on the configure.ac script and the files in the
subdirectory m4 (needs -I m4 option)
This utility is available from http://ftp.gnu.org.
This needs the following files
\begin{itemize}
\item  configure.ac \\
This files defines which libraries and features are needed and which
Makefiles should be constructed
\item  m4/cfitsio.m4\\
This file contains macroes used in finding the cfitsio headers and
library. 
\item   m4/fftw.m4\\
This file contains macroes used in finding the FFTW headers and
library. 
\item   m4/pgplot.m4\\
This file contains macroes used in finding the pgplot headers and
library. 
\end{itemize}
\item  autoconf \\
This utility is available from http://ftp.gnu.org.
This program generates the configure script and needs the following
\begin{itemize}
\item  configure.ac \\
This files defines which libraries and features are needed and which
Makefiles should be constructed
\item  aclocal.m4 \\
Generated by aclocal and contains the m4 macroes needed by autoconf
\end{itemize}
\item  configure \\
This program (the configure script) generates the Makefiles specified
in configure.ac based on the Makefile.in template files. 
This operation is largely by string substitution of @xxx@ type
symbols.
The values substituted are mostly determined by the m4 macroes in the
*.m4 files.
Configure uses the following:
\begin{itemize}
\item  missing \\
Common stub for a few missing GNU programs while installing.
\item  config.guess \\
Attempt to guess a canonical system name.
\item  config.sub\\
Configuration validation subroutine script.
\item  install-sh \\
install - install a program, script, or datafile.
\item   mkinstalldirs\\ ???
Doesn't seem to be needed.
\item   py-compile\\ ??
Doesn't seem to be needed.
\end{itemize}
\end{itemize}

\section{External software}
Obit uses the following external packages:
\begin{itemize}
\item  glib \\
The gnu library for large scale software in ansci c.
Available at http://www.gtk.org/.
\item  doxygen \\
System for generating human readable software documentation from
comments in the software.
Available at http://www.stack.nl/\~\ dimitri/doxygen/ .
\item  cfitsio \\
Basic FITS I/O package.
Available at http://heasarc.gsfc.nasa.gov/docs/software/fitsio/fitsio.html.
\item  FFTW3 \\
Fast Fourier Transform.
If configure does not find the FFTW3 library, FFTs are done using the GSL library
if that was found.
Available at http://www.fftw.org/.
FFTW2 can also be used but it MUST be installed with the configure
option --enable-float to be used by Obit.
\item  GSL \\
GNU Scientific Library.
If configure does not find the GSL library, the functions using it are stubbed
such that the software will compile and link but any attempt to use a function
relying on gsl routines will fail.
Available at http://www.gnu.org/software/gsl/.
\item  Python \\
Scripting language to which Obit has a binding.
The binding interface is built using swig.
(Note: use version 1.1? of swig. Although the output of swig is 
distributed  so you don't need swig unless you want to modify the
python interface. (http://www.swig.org/))
Python is available from http://www.python.org/download/.
Python 2.3 or later is needed, 2.5 currently preferred.
\item  xmlrpc \\
Interprocess communications is via xmlrpc using the xmlrpc-c and libwww 
libraries.\\
xmlrpc-c-1.2 (http://download.sourceforge.net/xmlrpc-c/xmlrpc-c-\%\{ver\}.tar.gz),\\
w3c-libwww-5.4.0 
(http://www.w3.org/Library/Distribution/w3c-libwww-\%\{ver\}.tar.gz)
libcurl works much better than libwww.
\item  zlib \\
Probably everywhere you want to be.
(if not, http://www.gzip.org/zlib/)
\end{itemize}

\section{User interfaces}
Currently only AIPS programs, user written c programs and python.

\subsection{AIPS interface}
A Fortran callable interface for Obit is defined in utility classes
ObitAIPSFortran and ObitAIPSObject.
Obit is only accessable from AIPS (POPS) in the form of tasks.
For an example, see the OBTST task in subdirectory AIPS.

\subsection{Python Interface}
The intent of the Python interface is to allow scripting use of Obit.
As such, the details of most data structures are not directly
accessable from python but are manipulated at the file level by
calling c routines.
However, in order to allow proper scripting, some access to data
members, especially the various descriptor classes is necessary.
The contents of the various descriptors (e.g. ImageDesc, UVDesc) can
be read and nonstructural values modified using an interface with a
python dictionary object as an intermediary.
Very rudimentary access is also provided for image pixel values and
row data in tables.
Images are accessed through FArray (arrays of floats) and many high
level functions are available on FArrays.

The python interface is defined in the python subdirectory.
The swig utility is used to interface the c library to python.
The swig interface is defined in the ObitTypeMaps.swig (defines
interface types) and *.inc which defines a python callable interface.
The ObitTypeMaps.swig and *.inc files are concatenated by the Makefile
before running swig.
This generates a single, large shared library module to be imported
into python.
However, this is necessary to get all the classes into the same
address space as Obit is dependent on class structures defining
function pointers.
A more elegant solution may be possible.

Note about swig.  The maintainers of swig have modified the interface 
in more recent versions (after version 1.1?) so the output of swig, 
python/Obit\_wrap.c, is distributed with other source code.  The Make 
procedure will not attempt to run swig unless you've mofidied the interface
(*.inc, *.swig).

The python interface defines a python class for each visible Obit
class in a separate *.py file.
The python/Obit classes are thin objects with basically a pointer to
the c structure (the ``me'' member) but this allows mapping the python
memory usage scheme onto Obit's similar but less automatic scheme. 
Thus the c objects created are automatically deleted when the python
reference count drops to zero or the explicit destructor (del) is
invoked.
However, this operation in python is only performed during its
occasional garbage collection; to ensure timely deallocation of large
Obit objects use the explicit Unref function.
All functions are implemented as nonclass member routines which take
class arguments but many common functions (e.g. Open, Close, Read) are
also implemented as class member functions.
Class .py files are internally documented.

The convention for the routine names on the c side of the swig
interface is the same as the Obit name but droping the initial
``Obit''.
This adds another layer of routine call but allows translating the
data types across the python/c divide.
Swig (plus the type maps in swig and .inc files) helps but sometimes 
this is insufficient; having a uniform way of dealing with the
interface helps.
Some of the functions defined in the *.inc files correspond to macro
expansions or straight access to member values in c.

   On the python side of the interface, a separate layer of routines
is added although generally the swig defined c routines are avaliable
and while more efficient, this may circumvent the automatic
destruction feature.

   Several of the python modules define functions that combine
multiple basic Obit function into higher level functional units
(closer to the equivalent of c programs).
The more complicated of the higher level routines (e.g. UVImager.py
member UVImageInput = imaging component of AIPS IMAGR) have input 
structures containing defaults and which can be viewed using the class
input function; this is similar in functionality to the AIPS
``INPUT''.
Routines defined in the *.py files should contain the necessary
information in the python doc strings.

\subsection{Obit usage from Python}
Obit can be accessed from python scripts or interactively.
A number of scripts are available in the python subdirectory 
with names ``script*.py''.
Either interactively or in a script, Obit needs to be initialized
before usage:
\begin{verbatim}
import OTF, OErr, OSystem, Image
err=OErr.OErr()
userid = 100
nAIPS = -1
nFITS = 1
ObitSys=OSystem.OSystem  ("Python", 1, userid, nAIPS, ["Def"], 
nFITS, ["./"], 1, 0, err)
OErr.printErrMsg(err, "Error with Obit startup")
\end{verbatim}
where nAIPS are the number of AIPS directories specified, nFITS are
the number of FITS directories specified.
If either of these values are -1 (and ``directory'' = ``Def'') then the
startup will search for standard environment names.
These are \$DA01, \$DA02 ... for AIPS and \$FITS,\$FITS01 ... for
FITS.
If given explicitly, the directories can be either absolute or
relative.
If the directory number for FITS files is 0, the name is assumed to be
an adequate relative or absolute pathname.
In the example above, all standard AIPS directories will be used by
default and the current directory for FITS files.

Functions in all *.py files should have documentation strings that can
be accessed interactively as in the following example from UVImager.
Note: help also works on entire python modules.
\begin{verbatim}
>>>  help(UVImager.UVImage)
Help on function UVImage in module UVImager:

UVImage(err, input={'Channel': 0, 'DoBeam': True, 'DoWeight': False, 'InData': None, 'OutImage
s': None, 'Robust': 0.0, 'TimeRange': [0.0, 0.0, 0.0, 0.0, 0.0, 0.0, 0.0, 0.0], 'UVTaper': [0.
0, 0.0], 'WtBox': 0, 'WtFunc': 1, ...})
    Image a uv data set.
    
    UV Data is weighted and imaged producing an array of images.
    err     = Python Obit Error/message stack
    input   = input parameter dictionary
    
    Input dictionary entries:
    InData   = Input Python OTF to image
    OutImages= Output image mosaic, image objects should be previously defined
    DoBeam   = True if beams are to be made
    DoWeight = If True apply uniform weighting corrections
    Robust   = Briggs robust parameter. (AIPS definition)
    UVTaper  = UV plane taper, sigma in klambda as [u,v]
    WtSize   = Size of weighting grid in cells [same as image nx]
    WtBox    = Size of weighting box in cells [def 1]
    WtFunc   = Weighting convolution function [def. 1]
               1=Pill box, 2=linear, 3=exponential, 4=Gaussian
               if positive, function is of radius, negative in u and v.
    WtPower  = Power to raise weights to.  [def = 1.0]
               Note: a power of 0.0 sets all the output weights to 1 as modified
               by uniform/Tapering weighting.
               Applied in determing weights as well as after.
    Channel  = Channel (1-rel) number to image, 0-> all.
\end{verbatim}

Many of the higher level functions have inputs in the form of a python dictionary
whose current values can be displayed by the class input function, e.g.:
\begin{verbatim}
>>> UVImager.input(UVImager.UVImageInput)
Inputs for  UVImage
   InData  =  None  :  Input UV data
   OutImages  =  None  :  Output image mosaic
   DoBeam  =  True  :  True if beams are to be made
   DoWeight  =  False  :  If True apply uniform weighting corrections to uvdata
   Robust  =  0.0  :  Briggs robust parameter. (AIPS definition)
   UVTaper  =  [0.0, 0.0]  :  UV plane taper, sigma in klambda as [u,v]
   WtSize  =  -1  :  Size of weighting grid in cells [image]
   WtBox  =  0  :  Size of weighting box in cells [def 0]
   WtFunc  =  1  :  Weighting convolution function [def. 1]
   WtPower  =  1.0  :  Power to raise weights to.  [def = 1.0]
   Channel  =  0  :  Channel (1-rel) number to image, 0-> all.
Note: the values displayed for Obit objects are the names you give
them.
\end{verbatim}

   Obit messages and error handling is by means of the Obit type
ObitErr which is an argument to most Obit routines.  This contains
informative as well as error messages; its contents can be displayed
using e.g.:
\begin{verbatim}
>>> OErr.printErr(err)
** Message: information  : Hogbom CLEANed 300 components with 4.025047 Jy
** Message: information  : Reached minimum flux density -0.335160 Jy
** Message: information  : Scaling residuals by 1.079005
** Message: information  : Restoring 300 components
\end{verbatim}
If  the err object indicates an error, OErr.printErr(err), will raise
an exception.

   If scratch files are created (see OTF.ResidCal for an example) then
an Obit shutdown will delete them:
\begin{verbatim}
# Shutdown Obit
OErr.printErr(err)
OSystem.Shutdown(ObitSys)
\end{verbatim}

Note: AIPS files as well as FITS files can be accessed from python.
However, OTF files have no equivalent in AIPS.
An obscure problem of debugging Obit software running in python is
getting the debuger to stop python when it knows about the Obit c
code.
The function Obit.Bomb() causes an abort (segmentation violation)
which allows you to set breakpoints and restart.

\section{Python script example}
A python script that is the functional equivalent of the AIPS task
HGEOM (as originally intended, interpolate the pixels of one image
onto the grid defined by another) is shown in the following.
The image is interpolated into a scratch image which is then copied to
the output image as a quantized (1/4 RMS of noise) image , the old
history is copied and new history records written.
\begin{verbatim}

# python/Obit equivalent of AIPSish HGEOM

import Obit, Image, ImageUtil, OSystem, OErr

# Init Obit
err=OErr.OErr()

# Use default directories
user = 100
pgmNumber = 1
ObitSys=OSystem.OSystem ("HGeom", pgmNumber, user, -1, ["Def"], 
-1, ["Def"], 1, 0, err)
#  print any error messages and halt
OErr.printErrMsg(err, "Error with Obit startup")

# Define files (FITS)
# Image to be interpolated onto the grid of tmplFile
inDisk = 1
inFile   = 'input.fits'     
# Image defining the output grid
tmplDisk = 1
tmplFile = 'template.fits'
# output file, the '!' allows overwriting an existing file
outDisk = 1
outFile  = '!HGeomOut.fits'

# Create Python/Obit objects attached to the data files
inImage   = Image.newPFImage("Input image",    inFile,   inDisk,   1, err)
tmplImage = Image.newPFImage("Template image", tmplFile, tmplDisk, 1, err)
outImage  = Image.newPFImage("Output image",   outFile,  outDisk,  0, err)
Image.PClone(tmplImage, outImage, err)   # Same structure etc.
OErr.printErrMsg(err, "Error initializing")

# Generate scratch file from tmplFile
tmpImage  = Image.PScratch(tmplImage, err)
tmpImage.Open(Image.WRITEONLY, err)   # Open
OErr.printErrMsg(err, "Error cloning template")

# Interpolate pixels to temporary file
ImageUtil.PInterpolateImage(inImage, tmpImage, err)
OErr.printErrMsg(err, "Error interpolating")

# Do history to scratch image as table
inHistory  = History.History("history", inImage.List, err)
outHistory = History.History("history", tmpImage.List, err)
History.PCopyHeader(inHistory, outHistory, err)
# Add this programs history
outHistory.Open(History.READWRITE, err)
outHistory.TimeStamp(" Start Obit "+ObitSys.pgmName,err)
outHistory.WriteRec(-1,ObitSys.pgmName+" / input = "+inFile,err)
outHistory.WriteRec(-1,ObitSys.pgmName+" / template = "+tmplFile,err)
outHistory.Close(err)
OErr.printErrMsg(err, "Error with history")

# Copy to quantized integer image with history
print "Write output image"
inHistory  = History.History("history", tmpImage.List, err)
Image.PCopyQuantizeFITS (tmpImage, outImage, err, inHistory=inHistory)

# Say what happened
print "Interpolated",inFile,"to",outFile,"a clone of ",tmplFile

# Shutdown Obit
OErr.printErr(err)  # Print any remaining Obit messages
OSystem.Shutdown(ObitSys)
\end{verbatim}

\section {C Language Example}
The template Obit program test/Template.c is both a template to use
for writing new Obit c programs and as an example which computes the
mean and RMS of a window in an image.
This program reads its input from a structured text file and writes
its output into a similar file.
In addition, many parameters can be passed on the command line.
The program and its inputs are described in the test/Template.doc
(ascii, NOT MSWord) file.
The following example is the main routine in this program which shows
the major features of an Obit program.
\begin{verbatim}

/* Program globals */
gchar *pgmName = "TEMPLATE";       /* Program name */
gchar *infile  = "TEMPLATE.inp";   /* File with program inputs */
gchar *outfile = "TEMPLATE.out";   /* File to contain program outputs */
gint  pgmNumber;       /* Program number (like POPS no.) */
gint  AIPSuser;        /* AIPS user number number (like POPS no.) */
gint  nAIPS=0;         /* Number of AIPS directories */
gchar **AIPSdirs=NULL; /* List of AIPS data directories */
gint  nFITS=0;         /* Number of FITS directories */
gchar **FITSdirs=NULL; /* List of FITS data directories */

int main ( int argc, char **argv )
/*----------------------------------------------------------------------- */
/*   Template Obit program - compute mean and RMS of an image             */
/*----------------------------------------------------------------------- */
{
  oint ierr = 0;
  ObitInfoList *myInput = NULL, *myOutput = NULL;
  ObitSystem   *mySystem= NULL;
  ObitImage    *inImage= NULL;
  ObitInfoType type;
  ObitErr      *err= NULL;
  gint32       dim[MAXINFOELEMDIM] = {1,1,1,1,1};
  gint         blc[IM_MAXDIM] = {1,1,1,1,1,1,1};
  gint         trc[IM_MAXDIM] = {0,0,0,0,0,0,0};
  gint         i, Aseq, disk, cno;
  gchar        *strTemp, inFile[128];
  gchar        Aname[13], Aclass[7], *Atype = "MA";
  gfloat       mean, rms;

   /* Startup - parse command line */
  err = newObitErr();
  myInput = TEMPLATEIn (argc, argv, err);
  if (err->error) ierr = 1;
  ObitErrLog(err); /* show any error messages on err */
  if (ierr!=0) return ierr;

  /* Initialize Obit */
  mySystem = ObitSystemStartup (pgmName, pgmNumber, AIPSuser, nAIPS, AIPSdirs, 
                                nFITS, FITSdirs, (oint)TRUE, (oint)FALSE, err);
  if (err->error) ierr = 1;
  ObitErrLog(err); /* show any error messages on err */
  if (ierr!=0) return ierr;

  /* Create basic input Image Object */
  inImage = newObitImage("input Image");
  
  /* Get input parameters from myInput */
  ObitInfoListGet(myInput, "blc", &type, dim, blc, err); /* BLC */
  ObitInfoListGet(myInput, "trc", &type, dim, blc, err); /* TRC */

  /* File type  - could be either AIPS or FITS */
  ObitInfoListGetP (myInput, "Type", &type, dim, (gpointer)&strTemp);
  if (!strncmp (strTemp, "AIPS", 4)) { /* AIPS input */
    /* input AIPS disk */
    ObitInfoListGet(myInput, "inDisk", &type, dim, &disk, err);
    /* input AIPS name */
    ObitInfoListGet(myInput, "inName", &type, dim, Aname, err);
    /* input AIPS class */
    ObitInfoListGet(myInput, "inClass", &type, dim, Aclass, err);
    /* input AIPS sequence */
    ObitInfoListGet(myInput, "inSeq", &type, dim, &Aseq, err);

    /* Find catalog number */
    cno = ObitAIPSDirFindCNO(disk, AIPSuser, Aname, Aclass, Atype, Aseq, err);
    
    /* define image */
    ObitImageSetAIPS (inImage, OBIT_IO_byPlane, disk, cno, AIPSuser, blc, trc, err);
    
  } else if (!strncmp (strTemp, "FITS", 4)) {  /* FITS input */
    /* input FITS file name */
    for (i=0; i<128; i++) inFile[i] = 0;
    ObitInfoListGet(myInput, "inFile", &type, dim, inFile, err);
    
    /* input FITS disk */
    ObitInfoListGet(myInput, "inDisk", &type, dim, &disk, err);

    /* define image */
    ObitImageSetFITS (inImage, OBIT_IO_byPlane, disk, inFile, blc, trc, err);
    
  } else { /* Unknown type - barf and bail */
    Obit_log_error(err, OBIT_Error, "%s: Unknown Image type %s", 
                   pgmName, strTemp);
    if (err->error) ierr = 1;
    ObitErrLog(err); /* show error messages */
    return ierr;
  }

  /* error check */
  if (err->error) ierr = 1;
  ObitErrLog(err); /* show any error messages on err */
  if (ierr!=0) return ierr;

  /* Open and read Image, image on member image, an ObitFArray */
  ObitImageOpen (inImage, OBIT_IO_ReadOnly, err);
  ObitImageRead (inImage, NULL, err);
  if (err->error) ierr = 1;
  ObitErrLog(err); /* show error messages */
  if (ierr!=0) return ierr;

  /* Get statistics from inImage FArray */
  mean = ObitFArrayMean(inImage->image);
  rms  = ObitFArrayRMS(inImage->image);

  ObitImageClose (inImage, err); /* Close */
  if (err->error) ierr = 1;
  ObitErrLog(err); /* show error messages */
  if (ierr!=0) return ierr;

  /* Tell results */
   Obit_log_error(err, OBIT_InfoErr, 
		  "%s: mean %f RMS %f", pgmName, mean, rms);

  /* Set up output */
  myOutput = defaultOutputs(err);
  dim[0] = 1; dim[1] = 1;
  ObitInfoListPut (myOutput, "mean", OBIT_float, dim, &mean, err);
  ObitInfoListPut (myOutput, "rms", OBIT_float, dim, &rms, err);
  ObitReturnDump (outfile, myOutput, err);

  /* show any messages and errors */
  if (err->error) ierr = 1;
  ObitErrLog(err);
  if (ierr!=0) return ierr;
  
  /* cleanup */
  myInput   = ObitInfoListUnref(myInput);    /* delete input list */
  myOutput  = ObitInfoListUnref(myOutput);   /* delete output list */
  inImage   = ObitUnref(inImage);
  
  /* Shutdown Obit */
  mySystem = ObitSystemShutdown (mySystem);
  
  return ierr;
} /* end of main */

\end{verbatim}

\section{Obit Tables}
This document uses latex macroes which are translated by a perl
script (ObitTables.pl) into the c source code.

\subsection{LaTeX Macros}
Tables used in Obit are defined in this document by using LaTeX
macroes to formally define the table.
These macroes are:
\begin{itemize}
\item tabletitle\{Title of table, e.g. ``Antenna table for uv data''\}
\item tablename\{Name of table, e.g. ``AN''\}
\item tableintro\{Short description of class\}
\item tableover\{Overview of usage of class\}
\item tablekey[\{name\}\{type code\}\{ software name\} \{default value\}
\{(range of indices)\} \{description\}]\\
Defines Table keyword, entries with a default value are not required to
be present.
\item tablecol[\{name\}\{units\}\{type code\} \{(dimensionality)\} \{software
name\} \{description\}]\\
Defines Table column.
If the description contains the string ``[OPTIONAL]'' its presence
will not be required; otherwise it will be checked.
\end{itemize}


%%%%%%%%%%%%%%% Obit Tables definition %%%%%%%%%%%%%%%%%%%%%%%%%%%%%%%%%%%%%%%%%%%%%%%%%%%
\clearpage
%%%%%%%%%%%%%%% ObitTableHistory Class %%%%%%%%%%%%%%%%%%%%%%%%%%%%%%%%%%%%%%%%%%%%%%%%%%%
% 
%
\ClassName[{ObitTableHistory}]
ObitTableHistory Class
\tabletitle{Processing History for non-AIPS data}
% table name
\tablename{History}
\tableintro[
{This class contains tabular data and allows access.
This file is used in NON-AIPS applications to store the processing
history of the associated data.
Entries consist of 70 character strings which should be self labeling.}
]
\tableover{
In memory tables are stored in a fashion similar to how they are 
stored on disk - in large blocks in memory rather than structures.
The details of the storage in the buffer are kept in the 
ObitTableDesc.
}
% Table keyword description
No Keywords in table.
%\begin{keywords}
%\end{keywords}
%
% Table column description
\begin{columns}
\tablecol[{"ENTRY  "}{"        " }{A}{(70)}{entry}
{Processing History entry}
] 
\end{columns}
%
% Table modification history
\begin{history}
\modhistory[{W. D. Cotton}{04/06/2004}{Revision 1: FOB}]
\end{history}
\clearpage
%
%%%%%%%%%%%%%%% ObitTableAN Class %%%%%%%%%%%%%%%%%%%%%%%%%%%%%%%%%%%%%%%%%%%%%%%%%%%
% 
%
\ClassName[{ObitTableAN}]
ObitTableAN Class
\tabletitle{Antenna table for uv data}
% table name
\tablename{AN}
\tableintro[
{This class contains tabular data and allows access.
"AIPS AN" contains information about the locations and characteristics
of antennas in a UV data set.
Also time information and the state of the Earth's orientation.
Each subarray in a uv data set will have it's own AN table in version
numbers the same order as the subarray.
Polarization calibration information is included. }
]
\tableover{
In memory tables are stored in a fashion similar to how they are 
stored on disk - in large blocks in memory rather than structures.
Due to the word alignment requirements of some machines, they are 
stored by order of the decreasing element size: 
double, float long, int, short, char rather than the logical order.
The details of the storage in the buffer are kept in the 
ObitTableDesc.
}
% Table keyword description
\begin{keywords}
\tablekey[{"REVISION"}{J}{revision}{10}{}
{Revision number of the table definition.}
]
% Must be position 2 to index
\tablekey[{"NO\_IF  "}{J}{numIF}{0}{()}
{The number of IFs}
]
\tablekey[{"ARRAYX  "}{D}{ArrayX}{0.0}{}
{Array center X coord. (meters, earth center)}
]
\tablekey[{"ARRAYY  "}{D}{ArrayY}{0.0}{}
{Array center Y coord. (meters, earth center)}
]
\tablekey[{"ARRAYZ  "}{D}{ArrayZ}{0.0}{}
{Array center Z coord. (meters, earth center)}
]
\tablekey[{"GSTIA0 "}{D}{GSTiat0}{0.0}{}
{GST at time=0 (degrees) on the reference date}
]
\tablekey[{"DEGPDY  "}{D}{DegDay}{360.0}{}
{Earth rotation rate (deg/IAT day)}
]
\tablekey[{"FREQ    "}{D}{Freq}{1.0}{}
{Obs. Reference Frequency for subarray(Hz)}
]
\tablekey[{"RDATE   "}{A}{RefDate}{"YYYYMMDD"}{}
{Reference date as "YYYYMMDD"}
]
\tablekey[{"POLARX  "}{E}{PolarX}{0.0}{}
{Polar position X (meters) on ref. date}
]
\tablekey[{"POLARY  "}{E}{PolarY}{0.0}{}
{Polar position Y (meters) on ref. date}
]
\tablekey[{"UT1UTC  "}{E}{ut1Utc}{0.0}{}
{UT1-UTC  (time sec.)  }
]
\tablekey[{"DATUTC  "}{E}{dataUtc}{0.0}{}
{data time-UTC  (time sec.)}
]
\tablekey[{"TIMSYS"}{A}{TimeSys}{"IAT"}{}
{Time system, 'IAT' or 'UTC'}
]
\tablekey[{"ARRNAM  "}{A}{ArrName}{"unknown"}{}
{Array name}
]
\tablekey[{"XYZHAND"}{A}{XYZHand}{"RIGHT"}{}
{Handedness of coordinate system'}
]
\tablekey[{"FRAME"}{A}{FRAME}{"UNKNOWN "}{}
{Reference frame of coordinate system'}
]
\tablekey[{"NUMORB  "}{J}{numOrb}{0}{()}
{Number of orbital parameters}
]
\tablekey[{"NOPCAL "}{J}{numPCal}{4}{()}
{Number of polarization calibration constants per IF}
]
\tablekey[{"FREQID  "}{J}{FreqID}{0}{}
{Denotes the FQ ID for which the AN poln. parms have been modified.}
]
\tablekey[{"IATUTC  "}{E}{iatUtc}{0.0}{}
{IAT - UTC (sec).}
]
\tablekey[{"POLTYPE "}{A}{polType}{"        "}{}
{Polarization parameterazation type, 'APPR', 'RAPPR', 'ORI-ELP'}
]
\tablekey[{"P\_REFANT"}{J}{P\_Refant}{0}{}
{Polarization reference antenna}
]
\tablekey[{"P\_DIFF01"}{E}{P\_Diff01}{0.0}{}
{Right-Left Phase difference in radians IF 1}
]
\tablekey[{"P\_DIFF02"}{E}{P\_Diff02}{0.0}{}
{Right-Left Phase difference in radians IF 2}
]
\tablekey[{"P\_DIFF03"}{E}{P\_Diff03}{0.0}{}
{Right-Left Phase difference in radians IF 3}
]
\tablekey[{"P\_DIFF04"}{E}{P\_Diff04}{0.0}{}
{Right-Left Phase difference in radians IF 4}
]
\tablekey[{"P\_DIFF05"}{E}{P\_Diff05}{0.0}{}
{Right-Left Phase difference in radians IF 5}
]
\tablekey[{"P\_DIFF06"}{E}{P\_Diff06}{0.0}{}
{Right-Left Phase difference in radians IF 6}
]
\tablekey[{"P\_DIFF07"}{E}{P\_Diff07}{0.0}{}
{Right-Left Phase difference in radians IF 7}
]
\tablekey[{"P\_DIFF08"}{E}{P\_Diff08}{0.0}{}
{Right-Left Phase difference in radians IF 8}
]
\end{keywords}
%
% Table column description
\begin{columns}
\tablecol[{"ANNAME  "}{"        " }{A}{(8)}{AntName}
{Station name}
] 
\tablecol[{"STABXYZ "}{"METERS " }{D}{(3)}{StaXYZ}
{X,Y,Z offset from array center}
] 
\tablecol[{"ORBPARM "}{"       " }{D}{(numOrb)}{OrbParm}
{Orbital parameters.}
] 
\tablecol[{"NOSTA   "}{"       " }{J}{(1)}{noSta}
{Station number, used as an index in other tables, uv data}
] 
\tablecol[{"MNTSTA  "}{"       " }{J}{(1)}{mntSta}
{Mount type, 0=altaz, 1=equatorial, 2=orbiting}
] 
\tablecol[{"STAXOF  "}{"METERS " }{E}{(1)}{staXof}
{Axis offset}
] 
\tablecol[{"DIAMETER"}{"METERS " }{E}{(1)}{diameter}
{[OPTIONAL] Diameter of dish}
] 
\tablecol[{"BEAMFWHM"}{"DEG/M  " }{E}{(numIF)}{BeamFWHM}
{[OPTIONAL] FWHM (degrees/meter) of single dish each IF, scales as wavelength}
] 
\tablecol[{"POLTYA  "}{"       " }{A}{(4)}{polTypeA}
{Feed A feed poln. type 'R','L','X','Y', actually only one valid character.}
] 
\tablecol[{"POLAA   "}{"DEGREES " }{E}{(1)}{PolAngA}
{Feed A feed position angle}
] 
\tablecol[{"POLCALA"}{"       " }{E}{(numPCal,numIF)}{PolCalA}
{Feed A poln. cal parameter. }
] 
\tablecol[{"POLTYB "}{"       " }{A}{(4)}{polTypeB}
{Feed B feed poln. type 'R','L','X','Y'}
] 
\tablecol[{"POLAB  "}{"DEGREES " }{E}{(1)}{PolAngB}
{Feed B feed position angle}
] 
\tablecol[{"POLCALB"}{"       " }{E}{(numPCal,numIF)}{PolCalB}
{Feed B poln. cal parameter}
] 
\end{columns}
%
% Table modification history
\begin{history}
\modhistory[{W. D. Cotton}{02/03/2003}{Revision 1: Copied from AIPS}]
\modhistory[{W. D. Cotton}{28/07/2010}{Add keywords NO\_IF, XYZHAND,
FRAME,  columns DIAMETER, BEAMFWHM, redefine numPCal}]
\end{history}
\clearpage
%%%%%%%%%%%%%%% ObitTableAT Class %%%%%%%%%%%%%%%%%%%%%%%%%%%%%%%%%%%%%%%%%%%%%%%%%%%
% 
%
\ClassName[{ObitTableAT}]
ObitTableAT Class
\tabletitle{ Antenna polarization table for uv data}
% table name
\tablename{AT}
\tableintro[
{This class contains tabular data and allows access.
"AIPS AT" contains Antenna polarization and something else.
An ObitTableAT is the front end to a persistent disk resident structure.
Both FITS and AIPS cataloged data are supported.
This class is derived from the ObitTable class. }
]
\tableover{
In memory tables are stored in a fashion similar to how they are 
stored on disk - in large blocks in memory rather than structures.
Due to the word alignment requirements of some machines, they are 
stored by order of the decreasing element size: 
double, float long, int, short, char rather than the logical order.
The details of the storage in the buffer are kept in the 
ObitTableDesc.
}
% Table keyword description
\begin{keywords}
\tablekey[{"TABREV"}{J}{revision}{3}{}
{Revision number of the table definition.}
]
\tablekey[{"OBSCODE"}{A}{obscode}{"AA000"}{}
{Observation code}
]
\tablekey[{"RDATE   "}{A}{RefDate}{"YYYY-MM-DD"}{}
{Reference date as "YYYYMMDD"}
]
\tablekey[{"NO\_STKD "}{J}{numStkd}{}{}
{The number of Stokes(?)}
]
\tablekey[{"STK\_1"}{J}{stk1}{}{}
{First Stokes(?)}
]
\tablekey[{"NO\_BAND "}{J}{numBand}{}{()}
{The number of Bands(?).}
]
\tablekey[{"NO\_CHAN"}{J}{numChan}{}{}
{The number of spectral channels.}
]
\tablekey[{"REF\_FREQ"}{D}{refFreq}{}{}
{Reference Frequency.}
]
\tablekey[{"CHAN\_BW "}{E}{chanBW}{}{}
{Channel bandwidth.}
]
\tablekey[{"REF\_PIXL "}{E}{refPixl}{}{}
{Reference Pixel..}
]
\tablekey[{"NOPCAL"}{J}{noPCal}{}{}
{Number of polarization calibration parameters.}
]
\tablekey[{"POLTYPE"}{A}{polType}{"APPROX  "}{}
{The Polarization calibration type.}
]
\end{keywords}
%
% Table column description
\begin{columns}
\tablecol[{"TIME    "}{"DAYS   " }{D}{(1)}{Time}
{The center time.}
] 
\tablecol[{"TIME\_INTERVAL "}{"DAYS   " }{E}{(1)}{TimeI}
{Time interval of record }
] 
\tablecol[{"ANNAME  "}{"        " }{A}{(8)}{AntName}
{Station name}
] 
\tablecol[{"ANTENNA\_NO"}{"        " }{J}{(1)}{antennaNo}
{Antenna number}
] 
\tablecol[{"ARRAY      "}{"        " }{J}{(1)}{Array}
{Array number}
] 
\tablekey[{"FREQID  "}{J}{FreqID}{0}{}
{The frequency ID.}
]
\tablecol[{"NO\_LEVELS"}{"        " }{J}{(1)}{numLevels}
{Number of levels of something(?)}
] 
\tablecol[{"POLTYA  "}{"       " }{A}{(1)}{polTypeA}
{Feed A feed poln. type 'R','L','X','Y', actually only one valid character.}
] 
\tablecol[{"POLAA   "}{"DEGREES " }{E}{(numBand)}{PolAngA}
{Feed A feed position angle}
] 
\tablecol[{"POLCALA"}{"       " }{E}{(numBand)}{PolCalA}
{Feed A poln. cal parameter. }
] 
\tablecol[{"POLTYB "}{"       " }{A}{(1)}{polTypeB}
{Feed B feed poln. type 'R','L','X','Y'}
] 
\tablecol[{"POLAB  "}{"DEGREES " }{E}{(numBand)}{PolAngB}
{Feed B feed position angle}
] 
\tablecol[{"POLCALB"}{"       " }{E}{(numBand)}{PolCalB}
{Feed B poln. cal parameter}
] 
\end{columns}
%
% Table modification history
\begin{history}
\modhistory[{W. D. Cotton}{02/22/2006}{Revision 1: Copied from AIPS}]
\end{history}
%
\clearpage
%%%%%%%%%%%%%%% ObitTableBL Class %%%%%%%%%%%%%%%%%%%%%%%%%%%%%%%%%%%%%%%%%%%%%%%%%%%
% 
%
\ClassName[{ObitTableBL}]
ObitTableBL Class
\tabletitle{Baseline dependent calibration for uv data}
% table name
\tablename{BL}
\tableintro[
{This class contains tabular data and allows access.
"AIPS BL" contains baseline dependent additive and multiplicative
terms for the correction of UV data.
Each table row contains is for a single baseline and has a correction
for each IF. 
An ObitTableBL is the front end to a persistent disk resident structure.
Both FITS and AIPS cataloged data are supported.
This class is derived from the ObitTable class. }
]
\tableover{
In memory tables are stored in a fashion similar to how they are 
stored on disk - in large blocks in memory rather than structures.
Due to the word alignment requirements of some machines, they are 
stored by order of the decreasing element size: 
double, float long, int, short, char rather than the logical order.
The details of the storage in the buffer are kept in the 
ObitTableDesc.
}
% Table keyword description
\begin{keywords}
\tablekey[{"NO\_ANT "}{J}{numAnt}{}{}
{The number of antennas.
}
]
% Must be position 2 to index
\tablekey[{"NO\_POL "}{J}{numPol}{1}{(1,2)}
{The number of polarizations.
}
]
\tablekey[{"NO\_IF  "}{J}{numIF}{}{()}
{The number of IFs.
}
]
\end{keywords}
%
% Table column description
\begin{columns}
\tablecol[{"TIME    "}{"DAYS   " }{E}{(1)}{Time}
{The center time.}
] 
\tablecol[{"SOURCE ID  "}{"        " }{J}{(1)}{SourID}
{Source ID number}
] 
\tablecol[{"SUBARRAY   "}{"        " }{J}{(1)}{SubA}
{Subarray number}
] 
\tablecol[{"ANTENNA1   "}{"        " }{J}{(1)}{ant1}
{First antenna number of baseline}
] 
\tablecol[{"ANTENNA2   "}{"        " }{J}{(1)}{ant2}
{Second antenna number of baseline}
] 
\tablecol[{"FREQ ID    "}{"        " }{J}{(1)}{FreqID}
{Freqid number}
] 
\tablecol[{"REAL M\#NO\_POL"}{"        " }{E}{(numIF)}{RealM}
{Real (Multiplicative correction Poln \# NO\_POL )}
] 
\tablecol[{"IMAG M\#NO\_POL"}{"        " }{E}{(numIF)}{ImagM}
{Imaginary (Multiplicative correction Poln \# NO\_POL )}
] 
\tablecol[{"REAL A\#NO\_POL"}{"        " }{E}{(numIF)}{RealA}
{Real (Additive correction Poln \# NO\_POL )}
] 
\tablecol[{"IMAG A\#NO\_POL"}{"        " }{E}{(numIF)}{ImagA}
{Imaginary (Additive correction Poln \# NO\_POL )}
] 
\end{columns}
%
% Table modification history
\begin{history}
\modhistory[{W. D. Cotton}{03/03/2003}{Revision 1: Copied from AIPS}]
\end{history}
%

\clearpage
%%%%%%%%%%%%%%% ObitTableBP Class %%%%%%%%%%%%%%%%%%%%%%%%%%%%%%%%%%%%%%%%%%%%%%%%%%%
% 
%
\ClassName[{ObitTableBP}]
ObitTableBP Class
\tabletitle{UV data BandPass calibration table}
% table name
\tablename{BP}
\tableintro[
{This class contains tabular data and allows access.
"AIPS BP" contains bandpass calibration informatioin for UV data.
An ObitTableBP is the front end to a persistent disk resident structure.
Both FITS and AIPS cataloged data are supported.
This class is derived from the ObitTable class. }
]
\tableover{
In memory tables are stored in a fashion similar to how they are 
stored on disk - in large blocks in memory rather than structures.
Due to the word alignment requirements of some machines, they are 
stored by order of the decreasing element size: 
double, float long, int, short, char rather than the logical order.
The details of the storage in the buffer are kept in the 
ObitTableDesc.
}
% Table keyword description
\begin{keywords}
\tablekey[{"NO\_ANT "}{J}{numAnt}{}{}
{The number of antennas}
]
% Must be position 2 to index
\tablekey[{"NO\_POL "}{J}{numPol}{1}{(1,2)}
{The number of antennas}
]
\tablekey[{"NO\_IF  "}{J}{numIF}{1}{()}
{The number of IFs}
]
\tablekey[{"NO\_CHAN"}{J}{numChan}{1}{()}
{Number of frequency channels}
]
\tablekey[{"STRT\_CHN"}{J}{startChan}{1}{}
{Start channel number}
]
\tablekey[{"NO\_SHFTS"}{J}{numShifts}{1}{}
{If numShifts = 1 BP entries are from cross-power data, if 2 are from total power, if 3 are a mixture, anything else then type is unknown and will assume cross-power}
]
\tablekey[{"LOW\_SHFT"}{J}{lowShift}{1}{}
{Most negative shift}
]
\tablekey[{"SHFT\_INC"}{J}{shiftInc}{1}{}
{Shift increment}
]
\tablekey[{"BP\_TYPE"}{A}{BPType}{}{}
{BP type: ' ' => standard BP table, CHEBSHEV' => Chebyshev polynomial coeff.}
]
\end{keywords}
%
% Table column description
\begin{columns}
\tablecol[{"TIME    "}{"DAYS   " }{D}{(1)}{Time}
{The center time.}
] 
\tablecol[{"INTERVAL "}{"DAYS   " }{E}{(1)}{TimeI}
{Time interval of record }
] 
\tablecol[{"SOURCE ID "}{"        " }{J}{(1)}{SourID}
{Source ID number}
] 
\tablecol[{"SUBARRAY "}{"       " }{J}{(1)}{SubA}
{Subarray number}
] 
\tablecol[{"ANTENNA "}{"       " }{J}{(1)}{antNo}
{Antenna number}
] 
\tablecol[{"BANDWIDTH "}{"HZ     " }{E}{(1)}{BW}
{andwidth of an individual channel}
] 
\tablecol[{"CHN\_SHIFT "}{"       " }{D}{(numIF)}{ChanShift}
{Frequency shift for each IF}
] 
\tablecol[{"FREQ ID "}{"       " }{J}{(1)}{FreqID}
{Freq. id number}
] 
\tablecol[{"REFANT \#NO\_POL"}{"       " }{J}{(1)}{RefAnt}
{Reference Antenna}
] 
\tablecol[{"WEIGHT \#NO\_POL"}{"       " }{E}{(numIF)}{Weight}
{Weights for complex bandpass}
] 
\tablecol[{"REAL \#NO\_POL"}{"       " }{E}{(numChan,numIF)}{Real}
{Real (channel gain Poln \# NO\_POL )}
] 
\tablecol[{"IMAG \#NO\_POL"}{"       " }{E}{(numChan,numIF)}{Imag}
{Imaginary (channel gain Poln \# NO\_POL)}
] 
\end{columns}
%
% Table modification history
\begin{history}
\modhistory[{W. D. Cotton}{03/03/2003}{Revision 1: Copied from AIPS}]
\end{history}
%
%
\clearpage
%%%%%%%%%%%%%%% ObitTableCC Class %%%%%%%%%%%%%%%%%%%%%%%%%%%%%%%%%%%%%%%%%%%%%%%%%%%
%
\ClassName[{ObitTableCC}]
\tabletitle{Clean Components table.}

% table name
\tablename{CC}
\tableintro[
{This class contains tabular data and allows access.
"AIPS CC" contains sky model components obtained either by CLEANing or
fitting.
Both FITS and AIPS cataloged data are supported.
This class is derived from the ObitTable class. }
]
\tableover{
In memory tables are stored in a fashion similar to how they are 
stored on disk - in large blocks in memory rather than structures.
The details of the storage in the buffer are kept in the 
ObitTableDesc.
For non--point components, there are extra parameters:
\begin{enumerate}
\item param[0] = major axis size (deg)
\item param[1] = minor axis size (deg)
\item param[2] = position angle (deg)
\item param[3] = type
\begin{itemize}
\item 0 = point,
\item 1 = Gaussian on sky,
\item 2 = Convolved Gaussian,
\item 3 = Uniform optically thin sphere
\item type + 10 $=>$ following parameters are spectral coefficients
of powers of $ln(\nu/\nu_0)$
\item type + 20 $=>$ following parameters are tabulated spectra
\end{itemize}
\item param[4] spectral index (if type$>=$10, $<=19$)
\item param[5] spectral curvature (if type$>=$10, $<=19$)
\item param[6] spectral curvature square (if type$>=$10, $<=19$)
\item param[3+i] spectral channel i (1-rel) (if type$>=$20, $<=29$)
\end{enumerate}
}
% Table keyword description
\begin{keywords}
\tablekey[{"NO\_PARMS"}{J}{noParms}{0}{(0,5)}
{Number of parameters for non point components, usually 0 or 4+}
]
\end{keywords}
%
% Table column description
\begin{columns}
\tablecol[{"FLUX    "}{"JY     " }{E}{(1)}{Flux}
{Component flux density}
] 
\tablecol[{"DELTAX  "}{"DEGREE  "}{E}{(1)}{DeltaX}
{Component X position.}
] 
\tablecol[{"DELTAY  "}{"DEGREE  "}{E}{(1)}{DeltaY}
{Component Y position.}
] 
\tablecol[{"DELTAZ  "}{"DEGREE  "}{E}{(1)}{DeltaZ}
{[OPTIONAL]Component Z position.}
] 
\tablecol[{"PARMS   "}{"        "}{E}{(noParms)}{parms}
{[OPTIONAL] Component parameters;}
] 
\end{columns}
%
% Table modification history
\begin{history}
\modhistory[{W. D. Cotton}{15/11/2004}{Revision 1: Copied from AIPS}]
\end{history}

\clearpage
%%%%%%%%%%%%%%% ObitTableCL Class %%%%%%%%%%%%%%%%%%%%%%%%%%%%%%%%%%%%%%%%%%%%%%%%%%%
%
\ClassName[{ObitTableCL}]
\tabletitle{CaLibration table for uv data.}

% table name
\tablename{CL}
\tableintro[
{This class contains tabular data and allows access.
"AIPS CL" contains amplitude/phase/delay and rate calibration
information to be applied to a multi source UV data set.
An ObitTableCL is the front end to a persistent disk resident structure.
Both FITS and AIPS cataloged data are supported.
This class is derived from the ObitTable class. }
]
\tableover{
In memory tables are stored in a fashion similar to how they are 
stored on disk - in large blocks in memory rather than structures.
Due to the word alignment requirements of some machines, they are 
stored by order of the decreasing element size: 
double, float long, int, short, char rather than the logical order.
The details of the storage in the buffer are kept in the 
ObitTableDesc.
}
% Table keyword description
\begin{keywords}
\tablekey[{"REVISION"}{J}{revision}{10}{}
{Revision number of the table definition}
]
% Must be position 2 to index
\tablekey[{"NO\_POL  "}{J}{numPol}{1}{(1,2)}
{The number of polarizations}
]
\tablekey[{"NO\_IF  "}{J}{numIF}{1}{()}
{The number of IFs}
]
\tablekey[{"NO\_ANT  "}{J}{numAnt}{1}{}
{The number of antennas in table}
]
\tablekey[{"NO\_TERM"}{J}{numTerm}{0}{()}
{The number of terms in model polynomial}
]
\tablekey[{"MGMOD   "}{D}{mGMod}{1.0}{}
{The Mean Gain modulus 
}
]
\end{keywords}
%
% Table column description
\begin{columns}
\tablecol[{"TIME    "}{"DAYS   " }{D}{(1)}{Time}
{The center time of the solution}
] 
\tablecol[{"TIME INTERVAL"}{"DAYS    "}{E}{(1)}{TimeI}
{Solution interval.}
] 
\tablecol[{"SOURCE ID"}{"        "}{J}{(1)}{SourID}
{Source Identifier number }
] 
\tablecol[{"ANTENNA NO."}{"       " }{J}{(1)}{antNo}
{Antenna number }
] 
\tablecol[{"SUBARRAY"}{"        "}{J}{(1)}{SubA}
{Subarray number.}
] 
\tablecol[{"FREQ ID"}{"        "}{J}{(1)}{FreqID}
{Frequency ID}
] 
\tablecol[{"I.FAR.ROT"}{"RAD/M**2"}{E}{(1)}{IFR}
{Ionospheric Faraday Rotation }
] 
\tablecol[{"GEODELAY "}{"SECONDS "}{D}{(numTerm)}{GeoDelay}
{Geometric delay polynomial series at TIME}
] 
\tablecol[{"DOPPOFF  "}{"SEC/SEC "}{E}{(numIF)}{DopplerOff}
{Doppler offset for each IF}
] 
\tablecol[{"ATMOS  "}{"SECONDS "}{E}{(1)}{atmos}
{Atmospheric delay }
] 
\tablecol[{"DATMOS  "}{"SEC/SEC "}{E}{(1)}{Datmos}
{Time derivative of ATMOS}
] 
\tablecol[{"MBDELAY\#NO\_POL"}{"SECONDS "}{E}{(1)}{MBDelay}
{Multiband delay poln \# NO\_POL  }
] 
\tablecol[{"CLOCK \#NO\_POL"}{"SECONDS "}{E}{(1)}{clock}
{"Clock" epoch error }
] 
\tablecol[{"DCLOCK \#NO\_POL"}{"SEC/SEC"}{E}{(1)}{Dclock}
{Time derivative of CLOCK }
] 
\tablecol[{"DISP \#NO\_POL"}{"SECONDS "}{E}{(1)}{dispers}
{Dispersive delay (sec at wavelength = 1m)for Poln \# NO\_POL}
] 
\tablecol[{"DDISP \#NO\_POL"}{"SEC/SEC"}{E}{(1)}{Ddispers}
{Time derivative of DISPfor Poln \# NO\_POL}
] 
\tablecol[{"REAL\#NO\_POL"}{"       " }{E}{(numIF)}{Real}
{Real (gain Poln \# NO\_POL )}
] 
\tablecol[{"IMAG\#NO\_POL"}{"        "}{E}{(numIF)}{Imag}
{Imaginary (gain Poln \# NO\_POL)}
] 
\tablecol[{"RATE \#NO\_POL"}{"SEC/SEC "}{E}{(numIF)}{Rate}
{Residual fringe rate  Poln \# NO\_POL}
] 
\tablecol[{"DELAY \#NO\_POL"}{"SECONDS "}{E}{(numIF)}{Delay}
{ Residual group delay Poln \# NO\_POL}
] 
\tablecol[{"WEIGHT \#NO\_POL"}{"        "}{E}{(numIF)}{Weight}
{Weight of soln. Poln \# NO\_POL}
] 
\tablecol[{"REFANT \#NO\_POL"}{"        "}{J}{(numIF)}{RefAnt}
{Reference antenna Poln \# NO\_POL }
] 
\end{columns}
%
% Table modification history
\begin{history}
\modhistory[{W. D. Cotton}{28/02/2003}{Revision 1: Copied from AIPS}]
\end{history}

\clearpage
%%%%%%%%%%%%%%% ObitTableCQ Class %%%%%%%%%%%%%%%%%%%%%%%%%%%%%%%%%%%%%%%%%%%%%%%%%%%
% 
%
\ClassName[{ObitTableCQ}]
ObitTableCQ Class
\tabletitle{VLBA Correlator parameter frequency tablefor uv data }
% table name
\tablename{CQ}
\tableintro[
{This class contains tabular data and allows access.
"AIPS CQ" table contains VLBA-like correlation parameters which are
used for making a number of instrumental corrections.
An ObitTableCQ is the front end to a persistent disk resident structure.
Both FITS and AIPS cataloged data are supported.
This class is derived from the ObitTable class. }
]
\tableover{
In memory tables are stored in a fashion similar to how they are 
stored on disk - in large blocks in memory rather than structures.
Due to the word alignment requirements of some machines, they are 
stored by order of the decreasing element size: 
double, float long, int, short, char rather than the logical order.
The details of the storage in the buffer are kept in the 
ObitTableDesc.
}
% Table keyword description
\begin{keywords}
\tablekey[{"TABREV"}{J}{revision}{1}{}
{Revision number of the table definition}
]
\tablekey[{"NO\_IF"}{J}{numIF}{}{()}
{The number of IFs}
]
\end{keywords}
%
% Table column description
\begin{columns}
\tablecol[{"FRQSEL "}{"        " }{J}{(1)}{FrqSel}
{Frequency ID \{IFQDCQ in AIPSish\}}
] 
\tablecol[{"SUBARRAY "}{"        " }{J}{(1)}{SubA}
{Subarray number \{ISUBCQ\}}
] 
\tablecol[{"FFT\_SIZE "}{"        " }{J}{(numIF)}{FFTSize}
{Size of FFT in correlator \{NFFTCQ\}}
] 
\tablecol[{"NO\_CHAN "}{"        " }{J}{(numIF)}{numChan}
{No. of channels in correlator\{NCHCQ\}}
] 
\tablecol[{"SPEC\_AVG "}{"        " }{J}{(numIF)}{SpecAvg}
{Spectral averaging factor\{NSAVCQ\}}
] 
\tablecol[{"EDGE\_FRQ "}{"HZ      " }{D}{(numIF)}{EdgeFreq}
{Edge frequency \{DFRQCQ\}}
] 
\tablecol[{"CHAN\_BW "}{"HZ      " }{D}{(numIF)}{ChanBW}
{Channel bandwidth \{DCBWCQ\}}
] 
\tablecol[{"TAPER\_FN "}{"        " }{A}{(8,numIF)}{TaperFn}
{Taper function \{LTAPCQ\}}
] 
\tablecol[{"OVR\_SAMP "}{"        " }{J}{(numIF)}{OverSamp}
{Oversampling factor \{NOVSCQ\}}
] 
\tablecol[{"ZERO\_PAD "}{"        " }{J}{(numIF)}{ZeroPad}
{Zero-padding factor \{NZPDCQ\}}
] 
\tablecol[{"FILTER "}{"        " }{J}{(numIF)}{Filter}
{Filter type \{IFLTCQ\}}
] 
\tablecol[{"TIME\_AVG "}{"SECONDS " }{E}{(numIF)}{TimeAvg}
{Time averaging interval \{TAVGCQ\}}
] 
\tablecol[{"NO\_BITS "}{"        " }{J}{(numIF)}{numBits}
{Quantization (no. of bits per recorded sample)\{NBITCQ\}}
] 
\tablecol[{"FFT\_OVLP "}{"        " }{J}{(numIF)}{FFTOverlap}
{FFT overlap factor \{IOVLCQ\}}
] 
\end{columns}
%
% Table modification history
\begin{history}
\modhistory[{W. D. Cotton}{04/03/2003}{Revision 1: Copied from AIPS}]
\end{history}
%
%
\clearpage
%%%%%%%%%%%%%%% ObitTableCP Class %%%%%%%%%%%%%%%%%%%%%%%%%%%%%%%%%%%%%%%%%%%%%%%%%%%
% 
%
\ClassName[{ObitTableCP}]
ObitTableCP Class
\tabletitle{AIPS Source polarization spectrum table}
% table name
\tablename{CP}
\tableintro[
{This class contains tabular data and allows access.
"AIPS CP" table contains source polarization information
An ObitTableCP is the front end to a persistent disk resident structure.
Both FITS and AIPS cataloged data are supported.
This class is derived from the ObitTable class. }
]
\tableover{
In memory tables are stored in a fashion similar to how they are 
stored on disk - in large blocks in memory rather than structures.
Due to the word alignment requirements of some machines, they are 
stored by order of the decreasing element size: 
double, float long, int, short, char rather than the logical order.
The details of the storage in the buffer are kept in the 
ObitTableDesc.
}
% Table keyword description
\begin{keywords}
\tablekey[{"TABREV"}{J}{revision}{1}{}
{Revision number of the table definition}
]
\tablekey[{"NO\_IF"}{J}{numIF}{}{()}
{The number of IFs}
]
\tablekey[{"NO\_CHAN"}{J}{numChan}{}{()}
{Number of frequency channels}
]
\tablekey[{"FREQID  "}{J}{FreqID}{0}{}
{The Frequency ID for which the source parameters are relevant.}
]
\end{keywords}
%
% Table column description
\begin{columns}
\tablecol[{"SOURCE ID"}{"        "}{J}{(1)}{SourID}
{Source Identifier number }
] 
\tablecol[{"I       "}{"Jy      " }{E}{(numChan,numIF)}{IPol}
{Stokes I flux density  per channel/IF, channel varing fastest.}
] 
\tablecol[{"Q       "}{"Jy      " }{E}{(numChan,numIF)}{QPol}
{Stokes Q flux density  per channel/IF, channel varing fastest.}
] 
\tablecol[{"U       "}{"Jy      " }{E}{(numChan,numIF)}{UPol}
{Stokes U flux density  per channel/IF, channel varing fastest.}
] 
\tablecol[{"V       "}{"Jy      " }{E}{(numChan,numIF)}{VPol}
{Stokes V flux density  per channel/IF, channel varing fastest.}
] 
\end{columns}
%
% Table modification history
\begin{history}
\modhistory[{W. D. Cotton}{08/16/2010}{Revision 1: Copied from AIPS}]
\end{history}
%
%
\clearpage
%%%%%%%%%%%%%%% ObitTableCT Class %%%%%%%%%%%%%%%%%%%%%%%%%%%%%%%%%%%%%%%%%%%%%%%%%%%
% 
%
\ClassName[{ObitTableCT}]
ObitTableCT Class
\tabletitle{Earth orientation parameter table for uv data}
% table name
\tablename{CT}
\tableintro[
{This class contains tabular data and allows access.
"AIPS CT" contains Earth orientation model parameters.
An ObitTableCT is the front end to a persistent disk resident structure.
Both FITS and AIPS cataloged data are supported.
This class is derived from the ObitTable class. }
]
\tableover{
In memory tables are stored in a fashion similar to how they are 
stored on disk - in large blocks in memory rather than structures.
Due to the word alignment requirements of some machines, they are 
stored by order of the decreasing element size: 
double, float long, int, short, char rather than the logical order.
The details of the storage in the buffer are kept in the 
ObitTableDesc.
}
% Table keyword description
\begin{keywords}
\tablekey[{"TABREV"}{J}{revision}{2}{}
{Revision number of the table definition.}
]
\tablekey[{"OBSCODE"}{A}{obscode}{"AA000"}{}
{Observation code}
]
\tablekey[{"RDATE   "}{A}{RefDate}{"YYYYMMDD"}{}
{Reference date as "YYYYMMDD"}
]
\tablekey[{"NO\_STKD "}{J}{numStkd}{}{}
{The number of Stokes(?)}
]
\tablekey[{"STK\_1"}{J}{stk1}{}{}
{First Stokes(?)}
]
\tablekey[{"NO\_BAND "}{J}{numBand}{}{()}
{The number of Bands(?).}
]
\tablekey[{"NO\_CHAN"}{J}{numChan}{}{}
{The number of spectral channels.}
]
\tablekey[{"REF\_FREQ"}{D}{refFreq}{}{}
{Reference Frequency.}
]
\tablekey[{"CHAN\_BW "}{E}{chanBW}{}{}
{Channel bandwidth.}
]
\tablekey[{"REF\_PIXL "}{E}{refPixl}{}{}
{Reference Pixel..}
]
\tablekey[{"C\_SRVR "}{A}{CSrvr}{}{}
{Who knows?}
]
\tablekey[{"C\_VERSN"}{A}{CVersn}{}{}
{Who knows?}
]
\tablekey[{"A\_VERSN"}{A}{AVersn}{}{}
{Who knows?}
]
\tablekey[{"I\_VERSN"}{A}{IVersn}{}{}
{Who knows?}
]
\tablekey[{"E\_VERSN"}{A}{EVersn}{}{}
{Who knows?}
]
\tablekey[{"ACCELGRV"}{D}{accelgrv}{}{}
{Acceleration due to gravity m/s/s}
]
\tablekey[{"E-FLAT"}{D}{Eflat}{}{}
{Earth Flattening???}
]
\tablekey[{"EARTHRAD"}{D}{earthrad}{}{}
{Earth radius}
]
\tablekey[{"MMSEMS"}{D}{mmsems}{}{}
{???}
]
\tablekey[{"EPHEPOC"}{J}{ephepoc}{}{}
{Ephemeris epoch???}
]
\tablekey[{"TIDELAG"}{D}{tidelag}{}{}
{TIDELAG}
]
\tablekey[{"GAUSS"}{D}{gauss}{}{}
{GAUSS}
]
\tablekey[{"GMMOON"}{D}{gmmoon}{}{}
{GMMOON }
]
\tablekey[{"GMSUN"}{D}{gmsun}{}{}
{GMSUN}
]
\tablekey[{"LOVE\_H"}{D}{loveH}{}{}
{LOVE\_H}
]
\tablekey[{"LOVE\_L"}{D}{loveL}{}{}
{LOVE\_L }
]
\tablekey[{"PRE\_DATA"}{D}{preData}{}{}
{PRE\_DATA}
]
\tablekey[{"REL\_DATA"}{D}{relData}{}{}
{REL\_DATA}
]
\tablekey[{"TIDALUT1"}{J}{tidalut1}{}{}
{TIDALUT1}
]
\tablekey[{"TSECAU"}{D}{tsecau}{}{}
{TSECAU }
]
\tablekey[{"U-GRV-CN "}{J}{UGrvCn}{}{}
{U-GRV-CN}
]
\tablekey[{"VLIGHT  "}{J}{vlight}{}{}
{Speed of light m/s}
]
\end{keywords}
%
% Table column description
\begin{columns}
\tablecol[{"TIME    "}{"DAYS   " }{D}{(1)}{Time}
{The center time.}
] 
\tablecol[{"UT1-UTC"}{"SECONDS " }{D}{(1)}{ut1utc}
{UT1-UTC}
] 
\tablecol[{"IAT-UTC"}{"SECONDS " }{D}{(1)}{iatutc}
{IAT-UTC}
] 
\tablecol[{"A1-UTC"}{"SECONDS " }{D}{(1)}{a1utc}
{A1-UTC}
] 
\tablecol[{"UT1 TYPE"}{"        " }{A}{(1)}{ut1Type}
{UT1 TYPE}
] 
\tablecol[{"WOBXY "}{"MILLIARC" }{D}{(2)}{wobXY}
{Earth pole wobble}
] 
\tablecol[{"WOB TYPE"}{"        " }{A}{(1)}{wobType}
{WOB TYPES}
] 
\tablecol[{"DPSI"}{"RAD     " }{D}{(1)}{dpsi}
{DPSI}
] 
\tablecol[{"DDPSI"}{"RAD/SEC " }{D}{(1)}{ddpsi}
{DDPSI }
] 
\tablecol[{"DEPS"}{"RAD     " }{D}{(1)}{deps}
{DEPS}
] 
\tablecol[{"DDEPS "}{"RAD/SEC " }{D}{(1)}{ddeps}
{DDEPS}
] 
\end{columns}
%
% Table modification history
\begin{history}
\modhistory[{W. D. Cotton}{02/22/2006}{Revision 1: Copied from AIPS}]
\end{history}
%

\clearpage
%%%%%%%%%%%%%%% ObitTableFG Class %%%%%%%%%%%%%%%%%%%%%%%%%%%%%%%%%%%%%%%%%%%%%%%%%%%
% 
%
\ClassName[{ObitTableFG}]
ObitTableFG Class
\tabletitle{Flag table for uv data documentation}
% table name
\tablename{FG}
\tableintro[
{This class contains tabular data and allows access.
"AIPS FG" contains descriptions of data to be ignored
An ObitTableFG is the front end to a persistent disk resident structure.
Both FITS and AIPS cataloged data are supported.
This class is derived from the ObitTable class. }
]
\tableover{
In memory tables are stored in a fashion similar to how they are 
stored on disk - in large blocks in memory rather than structures.
Due to the word alignment requirements of some machines, they are 
stored by order of the decreasing element size: 
double, float long, int, short, char rather than the logical order.
The details of the storage in the buffer are kept in the 
ObitTableDesc.
}
% Table keyword description
No Keywords in table.
%\begin{keywords}
%\end{keywords}
%
% Table column description
\begin{columns}
\tablecol[{"SOURCE  "}{"       " }{J}{(1)}{SourID}
{Source ID as defined in the SOURCE table}
] 
\tablecol[{"SUBARRAY "}{"       " }{J}{(1)}{SubA}
{Subarray number}
] 
\tablecol[{"FREQ ID "}{"       " }{J}{(1)}{freqID}
{Frequency ID number}
] 
\tablecol[{"ANTS   "}{"       " }{J}{(2)}{ants}
{first and secong antenna  numbers for a baseline, 0=$>$all}
] 
\tablecol[{"TIME RANGE "}{"DAYS    " }{E}{(2)}{TimeRange}
{Start and end time of data to be flagged }
] 
\tablecol[{"IFS     "}{"       " }{J}{(2)}{ifs}
{First and last IF numbers to flag}
] 
\tablecol[{"CHANS   "}{"       " }{J}{(2)}{chans}
{First and last channel numbers to flag}
] 
\tablecol[{"PFLAGS  "}{"       " }{X}{(4)}{pFlags}
{Polarization flags, same order as in data, T=$>$flagged}
] 
\tablecol[{"REASON  "}{"       " }{A}{(24)}{reason}
{Reason for flagging}
] 
\end{columns}
%
% Table modification history
\begin{history}
\modhistory[{W. D. Cotton}{02/03/2003}{Revision 1: Copied from AIPS}]
\end{history}
%
\clearpage
%%%%%%%%%%%%%%% ObitTableFQ Class %%%%%%%%%%%%%%%%%%%%%%%%%%%%%%%%%%%%%%%%%%%%%%%%%%%
% Obit base class
%
%\title fooey
\ClassName[{ObitTableFQ}]
%\ClassEnum[{obitObjType}
%{Type of underlying data structures}]
%\begin{ObitEnum}
%\ObitEnumItem[{OBITTYPE\_FITS}{FITS file}]
%\ObitEnumItem[{OBITTYPE\_AIPS}{AIPS catalog data}]
%\end{ObitEnum}

\tabletitle{UV data FreQuency Table}
% table name
\tablename{FQ}
\tableintro[
{This class contains tabular data and allows access.
"AIPS FQ" contains frequency related information for ``IF'' in uv
data.
An ``IF'' is a construct that allows sets of arbitrarily spaced frequencies.
An ObitTableFQ is the front end to a persistent disk resident structure.
Both FITS and AIPS cataloged data are supported.
This class is derived from the ObitTable class.}
]
\tableover{
In memory tables are stored in a fashion similar to how they are 
stored on disk - in large blocks in memory rather than structures.
Due to the word alignment requirements of some machines, they are 
stored by order of the decreasing element size: 
double, float long, int, short, char rather than the logical order.
The details of the storage in the buffer are kept in the 
ObitTableDesc.
}
% Table keyword description
\begin{keywords}
\tablekey[{"NO\_IF  "}{J}{numIF}{}{()}
{The number of IFs, used to dimension table column entries}
]
\end{keywords}
%
% Table column description
\begin{columns}
\tablecol[{"FRQSEL  "}{"       " }{J}{(1)}{fqid}
{Frequency ID number for row, this is a random parameter in the uv data}
] 
\tablecol[{"IF FREQ "}{"HZ     " }{D}{(numIF)}{freqOff}
{Offset from reference frequency for each IF}
] 
\tablecol[{"CH WIDTH"}{"HZ     " }{E}{(numIF)}{chWidth}
{Bandwidth of an individual channel, now always written and read as a signed value}
] 
\tablecol[{"TOTAL BANDWIDTH "}{"HZ     " }{E}{(numIF)}{totBW}
{Total bandwidth of the IF, now written and read as an unsigned value}
] 
\tablecol[{"SIDEBAND "}{"       " }{J}{(numIF)}{sideBand}
{Sideband of the IF (-1 =$>$ lower, +1 =$>$ upper), now always written and read as +1}
] 
\tablecol[{"RXCODE  "}{"       " }{A}{(numIF,8)}{RxCode}
{[OPTIONAL] Band code}
\end{columns}
%
% Table modification history
\begin{history}
\modhistory[{W. D. Cotton}{02/03/2002}{Revision 1: Copied from AIPS}]
\modhistory[{W. D. Cotton}{05/16/2012}{Added RXCODE}]
\end{history}
%
\clearpage
%%%%%%%%%%%%%%% ObitTableGC Class %%%%%%%%%%%%%%%%%%%%%%%%%%%%%%%%%%%%%%%%%%%%%%%%%%%
% 
%
\ClassName[{ObitTableGC}]
ObitTableGC Class
\tabletitle{Gain curve table for uv data}
% table name
\tablename{GC}
\tableintro[
{This class contains tabular data and allows access.
"AIPS GC" contains Antenna gain curve information
An ObitTableGC is the front end to a persistent disk resident structure.
Both FITS and AIPS cataloged data are supported.
This class is derived from the ObitTable class. }
]
\tableover{
In memory tables are stored in a fashion similar to how they are 
stored on disk - in large blocks in memory rather than structures.
Due to the word alignment requirements of some machines, they are 
stored by order of the decreasing element size: 
double, float long, int, short, char rather than the logical order.
The details of the storage in the buffer are kept in the 
ObitTableDesc.
}
% Table keyword description
\begin{keywords}
\tablekey[{"NO\_BAND "}{J}{numBand}{}{()}
{The number of Bands(?).}
]
% Must be position 2 to index
\tablekey[{"NO\_POL"}{J}{numPol}{1}{(1,2)}
{The number of polarizations.}
]
\tablekey[{"NO\_TABS"}{J}{numTabs}{1}{()}
{The number of ??.}
]
\tablekey[{"TABREV"}{J}{revision}{3}{}
{Revision number of the table definition.}
]
\end{keywords}
%
% Table column description
\begin{columns}
\tablecol[{"ANTENNA\_NO"}{"        " }{J}{(1)}{antennaNo}
{Antenna number}
]
\tablecol[{"SUBARRAY "}{"        " }{J}{(1)}{SubArray}
{Subarray number}
] 
\tablecol[{"FREQ ID  "}{"        " }{J}{(1)}{FreqID}
{Frequency id of scan}
] 
\tablecol[{"TYPE\_\#NO\_POL"}{"        " }{J}{(numBand)}{Type}
{TYPE\_\#NO\_POL}
] 
\tablecol[{"NTERM\_\#NO\_POL"}{"        " }{J}{(numBand)}{NTerm}
{NTERM\_\#NO\_POL}
] 
\tablecol[{"X\_TYP\_\#NO\_POL"}{"        " }{J}{(numBand)}{XTyp}
{X\_TYP\_\#NO\_POL}
] 
\tablecol[{"Y\_TYP\_\#NO\_POL"}{"        " }{J}{(numBand)}{YTyp}
{Y\_TYP\_\#NO\_POL}
] 
\tablecol[{"X\_VAL\_\#NO\_POL"}{"        " }{E}{(numBand)}{XVal}
{X\_TYP\_\#NO\_POL}
] 
\tablecol[{"Y\_VAL\_\#NO\_POL"}{"        " }{E}{(numTabs,numBand)}{YVal}
{Y\_TYP\_\#NO\_POL}
] 
\tablecol[{"GAIN\_\#NO\_POL"}{"        " }{E}{(numTabs,numBand)}{gain}
{GAIN\_\#NO\_POL}
] 
\tablecol[{"SENS\_\#NO\_POL"}{"K/JY    " }{E}{(numBand)}{sens}
{X\_TYP\_\#NO\_POL}
] 
\end{columns}
%
% Table modification history
\begin{history}
\modhistory[{W. D. Cotton}{02/22/2006}{Revision 1: Copied from AIPS}]
\end{history}
%

\clearpage
%%%%%%%%%%%%%%% ObitTableIM Class %%%%%%%%%%%%%%%%%%%%%%%%%%%%%%%%%%%%%%%%%%%%%%%%%%%
% 
%
\ClassName[{ObitTableIM}]
ObitTableIM Class
\tabletitle{IM table for uv data}
% table name
\tablename{IM}
\tableintro[
{This class contains tabular data and allows access.
"AIPS IM" contains the correlator model.
An ObitTableIM is the front end to a persistent disk resident structure.
Both FITS and AIPS cataloged data are supported.
This class is derived from the ObitTable class. }
]
\tableover{
In memory tables are stored in a fashion similar to how they are 
stored on disk - in large blocks in memory rather than structures.
Due to the word alignment requirements of some machines, they are 
stored by order of the decreasing element size: 
double, float long, int, short, char rather than the logical order.
The details of the storage in the buffer are kept in the 
ObitTableDesc.
}
% Table keyword description
\begin{keywords}
\tablekey[{"RDATE   "}{A}{RefDate}{"YYYY-MM-DD"}{}
{Reference date as "YYYYMMDD"}
]
% Must be position 2 to index
\tablekey[{"NO\_POL "}{J}{numPol}{1}{(1,2)}
{The number of polarizations.}
]
\tablekey[{"OBSCODE"}{A}{obscode}{"AA000"}{}
{Observation code}
]
\tablekey[{"NO\_STKD "}{J}{numStkd}{}{}
{The number of Stokes(?)}
]
\tablekey[{"NO\_BAND "}{J}{numBand}{}{()}
{The number of Bands(?).}
]
\tablekey[{"NO\_CHAN"}{J}{numChan}{}{}
{The number of spectral channels.}
]
\tablekey[{"REF\_FREQ"}{D}{refFreq}{}{}
{Reference Frequency.}
]
\tablekey[{"CHAN\_BW "}{E}{chanBW}{}{}
{Channel bandwidth.}
]
\tablekey[{"REF\_PIXL "}{E}{refPixl}{}{}
{Reference frequency Pixel.}
]
\tablekey[{"STK\_1   "}{E}{stk1}{-1.0}{}
{[OPTIONAL]Reference frequency Pixel.}
]
\tablekey[{"NPOLY "}{J}{npoly}{}{()}
{The number of polynomial terms.}
]
\tablekey[{"TABREV"}{J}{tabrev}{2}{}
{Revision number of the table definition.}
]
\tablekey[{"REVISION"}{E}{revision}{2.0}{}
{Revision number of the table definition.}
]
\end{keywords}
%
% Table column description
\begin{columns}
\tablecol[{"TIME    "}{"DAYS   " }{D}{(1)}{Time}
{The center time.}
] 
\tablecol[{"TIME\_INTERVAL "}{"DAYS   " }{E}{(1)}{TimeI}
{Time interval of record }
] 
\tablecol[{"SOURCE\_ID  "}{"        " }{J}{(1)}{SourID}
{Source ID number}
] 
\tablecol[{"ANTENNA\_NO"}{"       " }{J}{(1)}{antennaNo}
{Antenna number}
] 
\tablecol[{"ARRAY      "}{"        " }{J}{(1)}{Array}
{Array number}
] 
\tablecol[{"FREQID"}{"        "}{J}{(1)}{FreqID}
{Frequency ID}
] 
\tablecol[{"I.FAR.ROT"}{"RAD/M**2"}{E}{(1)}{IFR}
{Ionospheric Faraday Rotation }
] 
\tablecol[{"FREQ.VAR"}{"HZ      "}{E}{(numBand)}{FreqVar}
{FREQ.VAR (?) }
] 
\tablecol[{"PDELAY\_\#NO\_POL "}{"SECONDS "}{D}{(npoly,numBand)}{PDelay}
{Phase delay }
] 
\tablecol[{"GDELAY\_\#NO\_POL "}{"SECONDS "}{D}{(npoly)}{GDelay}
{Group delay }
] 
\tablecol[{"PRATE\_\#NO\_POL "}{"SECONDS "}{D}{(npoly,numBand)}{PRate}
{Phase delay rate}
] 
\tablecol[{"GRATE\_\#NO\_POL "}{"SECONDS "}{D}{(npoly)}{GRate}
{Group delay rate}
] 
\tablecol[{"DISP\_\#NO\_POL "}{"SECONDS "}{E}{()}{Disp}
{Dispersion}
] 
\tablecol[{"DDISP\_\#NO\_POL "}{"SEC/SEC  "}{E}{()}{DRate}
{Dispersion rate}
] 
\end{columns}
%
% Table modification history
\begin{history}
\modhistory[{W. D. Cotton}{02/22/2006}{Revision 1: Copied from AIPS}]
\modhistory[{W. D. Cotton}{10/15/2014}{Revision 2: Add STK\_1}]
\end{history}
%
\clearpage
%%%%%%%%%%%%%%% ObitTableNI Class %%%%%%%%%%%%%%%%%%%%%%%%%%%%%%%%%%%%%%%%%%%%%%%%%%%
% 
%
\ClassName[{ObitTableNI}]
ObitTableIN Class
\tabletitle{Ionospheric calibration table class for uv data}
% table name
\tablename{NI}
\tableintro[
{This class contains tabular data and allows access.
"AIPS NI" contains the results of ionospheric modeling fits.
The ionospheric phase screen over the array is modeled in terms of a
Zernike polynomial.
An ObitTableNI is the front end to a persistent disk resident structure.
Both FITS and AIPS cataloged data are supported.
This was formerly the IN table but an AIPS bug does nasty things to
``IN'' tables.
This class is derived from the ObitTable class. }
]
\tableover{
In memory tables are stored in a fashion similar to how they are 
stored on disk - in large blocks in memory rather than structures.
Due to the word alignment requirements of some machines, they are 
stored by order of the decreasing element size: 
double, float long, int, short, char rather than the logical order.
The details of the storage in the buffer are kept in the 
ObitTableDesc.
}
% Table keyword description
\begin{keywords}
\tablekey[{"REVISION"}{J}{revision}{1}{}
{Revision number of the table definition.}
]
\tablekey[{"NUM\_COEF"}{J}{numCoef}{5}{()}
{Number of Zernike coefficients}
]
\tablekey[{"H\_ION   "}{E}{heightIon}{1.0e10}{}
{Height of the ionospheric phase screen (km?)}
]
\tablekey[{"REF\_FREQ"}{D}{refFreq}{}{}
{Reference frequency for the phase screen model}
]
\end{keywords}
%
% Table column description
\begin{columns}
\tablecol[{"TIME    "}{"DAYS   " }{D}{(1)}{Time}
{The center time}
] 
\tablecol[{"TIME INTERVAL"}{"DAYS   " }{E}{(1)}{TimeI}
{ime interval of the solution}
] 
\tablecol[{"ANTENNA NO."}{"        " }{J}{(1)}{antNo}
{Antenna number, 0=$>$ all}
] 
\tablecol[{"SOURCE ID"}{"        " }{J}{(1)}{SourId}
{Source number, 0=$>$ all}
] 
\tablecol[{"SUBARRAY"}{"        " }{J}{(1)}{SubA}
{Subarray number, 0=$>$ all}
] 
\tablecol[{"WEIGHT "}{"        " }{E}{(1)}{weight}
{Weight}
] 
\tablecol[{"COEF   "}{"        " }{E}{(numCoef)}{coef}
{Zernike model coeffients}
] 
\end{columns}
%
% Table modification history
\begin{history}
\modhistory[{W. D. Cotton}{28/02/2003}{Revision 1: Copied from AIPS}]
\end{history}
%
\clearpage
%%%%%%%%%%%%%%% ObitTableMC Class %%%%%%%%%%%%%%%%%%%%%%%%%%%%%%%%%%%%%%%%%%%%%%%%%%%
% 
%
\ClassName[{ObitTableMC}]
ObitTableMC Class
\tabletitle{ MC table for uv data}
% table name
\tablename{MC}
\tableintro[
{This class contains tabular data and allows access.
"AIPS MC" contains VLBA Correlator model calculations.
An ObitTableMC is the front end to a persistent disk resident structure.
Both FITS and AIPS cataloged data are supported.
This class is derived from the ObitTable class. }
]
\tableover{
In memory tables are stored in a fashion similar to how they are 
stored on disk - in large blocks in memory rather than structures.
Due to the word alignment requirements of some machines, they are 
stored by order of the decreasing element size: 
double, float long, int, short, char rather than the logical order.
The details of the storage in the buffer are kept in the 
ObitTableDesc.
}
% Table keyword description
\begin{keywords}
\tablekey[{"OBSCODE"}{A}{obscode}{"AA000"}{}
{Observation code}
]
% Must be position 2 to index
\tablekey[{"NO\_POL"}{J}{numPol}{1}{(1,2)}
{The number of polarizations.}
]
\tablekey[{"RDATE   "}{A}{RefDate}{"YYYYMMDD"}{}
{Reference date as "YYYYMMDD"}
]
\tablekey[{"NO\_STKD "}{J}{numStkd}{}{}
{The number of Stokes(?)}
]
\tablekey[{"STK\_1"}{J}{stk1}{}{}
{First Stokes(?)}
]
\tablekey[{"NO\_BAND "}{J}{numBand}{}{()}
{The number of Bands(?).}
]
\tablekey[{"NO\_CHAN"}{J}{numChan}{}{}
{The number of spectral channels.}
]
\tablekey[{"REF\_FREQ"}{D}{refFreq}{}{}
{Reference Frequency.}
]
\tablekey[{"CHAN\_BW "}{E}{chanBW}{}{}
{Channel bandwidth.}
]
\tablekey[{"REF\_PIXL "}{E}{refPixl}{}{}
{Reference Pixel..}
]
\tablekey[{"FFT\_SIZE"}{J}{FFTSize}{}{}
{FFT size.}
]
\tablekey[{"OVERSAMP"}{J}{oversamp}{}{}
{Oversampling factor}
]
\tablekey[{"ZERO\_PAD"}{J}{zeroPad}{}{}
{Zero padding factor}
]
\tablekey[{"TAPER\_FN"}{A}{taperFn}{}{}
{Tapering function}
]
\tablekey[{"TABREV"}{J}{revision}{1}{}
{Revision number of the table definition.}
]
\end{keywords}
%
% Table column description
\begin{columns}
\tablecol[{"TIME    "}{"DAYS   " }{D}{(1)}{Time}
{The center time.}
] 
\tablecol[{"SOURCE\_ID  "}{"        " }{J}{(1)}{SourID}
{Source ID number}
] 
\tablecol[{"ANTENNA\_NO"}{"        " }{J}{(1)}{antennaNo}
{Antenna number}
]
\tablecol[{"ARRAY      "}{"        " }{J}{(1)}{Array}
{Array number}
] 
\tablecol[{"FREQID"}{"        "}{J}{(1)}{FreqID}
{Frequency ID}
] 
\tablecol[{"ATMOS  "}{"SECONDS "}{D}{(1)}{atmos}
{Atmospheric delay }
] 
\tablecol[{"DATMOS  "}{"SEC/SEC "}{D}{(1)}{Datmos}
{Time derivative of ATMOS}
] 
\tablecol[{"GDELAY"}{"SECONDS "}{D}{(1)}{GDelay}
{Group delay }
] 
\tablecol[{"GRATE "}{"SECONDS "}{D}{(1)}{GRate}
{Group delay rate}
] 
\tablecol[{"CLOCK\_\#NO\_POL"}{"SECONDS "}{D}{(1)}{clock}
{"Clock" epoch error }
] 
\tablecol[{"DCLOCK\_\#NO\_POL"}{"SEC/SEC"}{D}{(1)}{Dclock}
{Time derivative of CLOCK }
] 
\tablecol[{"LO\_OFFSET\_\#NO\_POL"}{"HZ      "}{E}{(1)}{LOOffset}
{LO Offset }
] 
\tablecol[{"DLO\_OFFSET\_\#NO\_POL"}{"Hz/SEC  "}{E}{(1)}{DLOOffset}
{Time derivative of LO offset }
] 
\tablecol[{"DISP \#NO\_POL"}{"SECONDS "}{E}{(1)}{disp}
{Dispersive delay (sec at wavelength = 1m)for Poln \# NO\_POL}
] 
\tablecol[{"DDISP \#NO\_POL"}{"SEC/SEC"}{E}{(1)}{Ddisp}
{Time derivative of DISPfor Poln \# NO\_POL}
] 
\end{columns}
%
% Table modification history
\begin{history}
\modhistory[{W. D. Cotton}{02/22/2006}{Revision 1: Copied from AIPS}]
\end{history}
%
\clearpage
%%%%%%%%%%%%%%% ObitTableMF Class %%%%%%%%%%%%%%%%%%%%%%%%%%%%%%%%%%%%%%%%%%%%%%%%%%%
% 
%
\ClassName[{ObitTableMF}]
ObitTableMF Class
\tabletitle{AIPS Model Fit Table}
% table name
\tablename{MF}
\tableintro[
{This class contains tabular data and allows access.
"AIPS MF" the results of a model fit to an image or uv data of the
type produced by SAD/VSAD.
An ObitTableMF is the front end to a persistent disk resident structure.
Both FITS and AIPS cataloged data are supported.
This class is derived from the ObitTable class. }
]
\tableover{
In memory tables are stored in a fashion similar to how they are 
stored on disk - in large blocks in memory rather than structures.
Due to the word alignment requirements of some machines, they are 
stored by order of the decreasing element size: 
double, float long, int, short, char rather than the logical order.
The details of the storage in the buffer are kept in the 
ObitTableDesc.
}
% Table keyword description
\begin{keywords}
\tablekey[{"DEPTH1"}{E}{depth1}{1}{}
{Dimensions 3 in image}
]
\tablekey[{"DEPTH2"}{E}{depth2}{1}{}
{Dimensions 4 in image}
]
\tablekey[{"DEPTH3"}{E}{depth3}{1}{}
{Dimensions 5 in image}1
]
\tablekey[{"DEPTH4"}{E}{depth4}{1}{}
{Dimensions 6 in image}
]
\tablekey[{"DEPTH5"}{E}{depth5}{1}{}
{Dimensions 7 in image}
]
\end{keywords}
%
% Table column description
\begin{columns}
\tablecol[{"PLANE "}{"        "}{E}{(1)}{plane}
{Plane number}
] 
\tablecol[{"PEAK INT"}{"JY/BEAM "}{E}{(1)}{Peak}
{Peak Ipol}
] 
\tablecol[{"I FLUX "}{"JY      "}{E}{(1)}{IFlux}
{Integrated Ipol flux}
] 
\tablecol[{"DELTAX "}{"DEGREE  "}{E}{(1)}{DeltaX}
{X offset of center}
] 
\tablecol[{"DELTAY "}{"DEGREE  "}{E}{(1)}{DeltaY}
{Y offset of center}
] 
\tablecol[{"MAJOR AX"}{"DEGREE  "}{E}{(1)}{MajorAx}
{Fitted major axis size}
] 
\tablecol[{"MINOR AX"}{"DEGREE  "}{E}{(1)}{MinorAx}
{Fitted minor axis size}
] 
\tablecol[{"POSANGLE"}{"DEGREE  "}{E}{(1)}{PosAngle}
{Fitted PA}
] 
\tablecol[{"Q FLUX  "}{"JY      "}{E}{(1)}{QFlux}
{Integrated Q flux density}
] 
\tablecol[{"U FLUX "}{"JY      "}{E}{(1)}{UFlux}
{Integrated U flux density}
] 
\tablecol[{"V FLUX  "}{"JY      "}{E}{(1)}{VFlux}
{Integrated Y flux density}
] 
\tablecol[{"ERR PEAK"}{"JY/BEAM "}{E}{(1)}{errPeak}
{rror in Peak Ipol}
] 
\tablecol[{"ERR FLUX"}{"JY      "}{E}{(1)}{errIFlux}
{rror in Integrated Ipol flux}
] 
\tablecol[{"ERR DLTX"}{"DEGREE  "}{E}{(1)}{errDeltaX}
{Error in X offset of center}
] 
\tablecol[{"ERR DLTY"}{"DEGREE  "}{E}{(1)}{errDeltaY}
{Error in Y offset of center}
] 
\tablecol[{"ERR MAJA"}{"DEGREE  "}{E}{(1)}{errMajorAx}
{Error in Fitted major axis size}
] 
\tablecol[{"ERR MINA"}{"DEGREE  "}{E}{(1)}{errMinorAx}
{Error in Fitted minor axis size}
] 
\tablecol[{"ERR PA "}{"DEGREE  " }{E}{(1)}{errPosAngle}
{Error in Fitted PA}
] 
\tablecol[{"ERR QFLX"}{"JY      " }{E}{(1)}{errQFlux}
{Error in Integrated Q flux density}
] 
\tablecol[{"ERR UFLX"}{"JY      " }{E}{(1)}{errUFlux}
{Error in Integrated U flux density}
] 
\tablecol[{"ERR VFLX"}{"JY      " }{E}{(1)}{errVFlux}
{Error in Integrated Y flux density}
] 
\tablecol[{"TYPE MOD"}{"        "}{E}{(1)}{TypeMod}
{Model type 1 = Gaussian}
] 
\tablecol[{"D0 MAJOR"}{"DEGREE  "}{E}{(1)}{D0Major}
{Deconvolved best major axis}
] 
\tablecol[{"D0 MINOR"}{"DEGREE  "}{E}{(1)}{D0Minor}
{Deconvolved best minor axis}
] 
\tablecol[{"D0 POSANG"}{"DEGREE  "}{E}{(1)}{D0PosAngle}
{Deconvolved best PA}
] 
\tablecol[{"D- MAJOR"}{"DEGREE  "}{E}{(1)}{DmMajor}
{Deconvolved least major axis}
] 
\tablecol[{"D- MINOR"}{"DEGREE  "}{E}{(1)}{DmMinor}
{econvolved least minor axis}
] 
\tablecol[{"D- POSAN"}{"DEGREE  "}{E}{(1)}{DmPosAngle}
{Deconvolved least PA}
] 
\tablecol[{"D+ MAJOR"}{"DEGREE  "}{E}{(1)}{DpMajor}
{Deconvolved most major axis}
] 
\tablecol[{"D+ MINOR"}{"DEGREE  "}{E}{(1)}{DpMinor}
{Deconvolved most minor axis}
] 
\tablecol[{"D+ POSAN"}{"DEGREE  "}{E}{(1)}{DpPosAngle}
{Deconvolved most PA}
] 
\tablecol[{"RES RMS "}{"JY/BEAM "}{E}{(1)}{ResRMS}
{RMS of Ipol residual}
] 
\tablecol[{"RES PEAK"}{"JY/BEAM "}{E}{(1)}{ResPeak}
{Peak in Ipol residual}
] 
\tablecol[{"RES FLUX"}{"JY      "}{E}{(1)}{ResFlux}
{Integrated Ipol in residual}
] 
\tablecol[{"CENTER X"}{"PIXEL   "}{E}{(1)}{PixelCenterX}
{Center x position in pixels}
] 
\tablecol[{"CENTER Y"}{"PIXEL  "}{E}{(1)}{PixelCenterY}
{Center y position in pixels}
] 
\tablecol[{"MAJ AXIS"}{"PIXEL  "}{E}{(1)}{PixelMajorAxis}
{Fitted major axis in pixels}
] 
\tablecol[{"MIN AXIS"}{"PIXEL  "}{E}{(1)}{PixelMinorAxis}
{Fitted minor axis in pixels}
] 
\tablecol[{"PIXEL PA"}{"DEGREE  "}{E}{(1)}{PixelPosAngle}
{Fitted PA(?)}
] 
\end{columns}
%
% Table modification history
\begin{history}
\modhistory[{W. D. Cotton}{03/03/2003}{Revision 1: Copied from AIPS}]
\end{history}
%
%
\clearpage
%%%%%%%%%%%%%%% ObitTableNX Class %%%%%%%%%%%%%%%%%%%%%%%%%%%%%%%%%%%%%%%%%%%%%%%%%%%
% 
%
\ClassName[{ObitTableNX}]
ObitTableNX Class
\tabletitle{iNdeX table for uv data}
% table name
\tablename{NX}
\tableintro[
{This class contains tabular data and allows access.
"AIPS NX" contains an index for a uv data file giving the times,
source and visibility range for a sequence of scans.
A scan is a set of observations in the same mode and on the same source.
An ObitTableNX is the front end to a persistent disk resident structure.
This class is derived from the ObitTable class. }
]
\tableover{
In memory tables are stored in a fashion similar to how they are 
stored on disk - in large blocks in memory rather than structures.
Due to the word alignment requirements of some machines, they are 
stored by order of the decreasing element size: 
double, float long, int, short, char rather than the logical order.
The details of the storage in the buffer are kept in the 
ObitTableDesc.
}
% Table keyword description
No Keywords in table.
%\begin{keywords}
%\end{keywords}
%
% Table column description
\begin{columns}
\tablecol[{"TIME    "}{"DAYS   " }{E}{(1)}{Time}
{The center time of the sscan.}
] 
\tablecol[{"TIME INTERVAL "}{"DAYS   " }{E}{(1)}{TimeI}
{Duration of scan}
] 
\tablecol[{"SOURCE ID "}{"       " }{J}{(1)}{SourID}
{Source ID as defined in then SOURCE table}
] 
\tablecol[{"SUBARRAY "}{"        " }{J}{(1)}{SubA}
{Subarray number}
] 
\tablecol[{"START VIS "}{"        "}{J}{(1)}{StartVis}
{First visibility number (1-rel) in scan}
] 
\tablecol[{"END VIS  "}{"        " }{J}{(1)}{EndVis}
{Last visibility number (1-rel) in scan}
] 
\tablecol[{"FREQ ID  "}{"        " }{J}{(1)}{FreqID}
{Frequency id of scan}
] 
\end{columns}
%
% Table modification history
\begin{history}
\modhistory[{W. D. Cotton}{02/03/2003}{Revision 1: Copied from AIPS}]
\end{history}
%
\clearpage
%%%%%%%%%%%%%%% ObitTableOB Class %%%%%%%%%%%%%%%%%%%%%%%%%%%%%%%%%%%%%%%%%%%%%%%%%%%
% 
%
\ClassName[{ObitTableOB}]
ObitTableOB Class
\tabletitle{ Orbital antenna table for uv data}
% table name
\tablename{OB}
\tableintro[
{This class contains tabular data and allows access.
"AIPS OB" contains Orbiting VLBI station parameters.
An ObitTableOB is the front end to a persistent disk resident structure.
Both FITS and AIPS cataloged data are supported.
This class is derived from the ObitTable class. }
]
\tableover{
In memory tables are stored in a fashion similar to how they are 
stored on disk - in large blocks in memory rather than structures.
Due to the word alignment requirements of some machines, they are 
stored by order of the decreasing element size: 
double, float long, int, short, char rather than the logical order.
The details of the storage in the buffer are kept in the 
ObitTableDesc.
}
% Table keyword description
\begin{keywords}
\tablekey[{"TABREV"}{J}{revision}{2}{}
{Revision number of the table definition.}
]
\end{keywords}
%
% Table column description
\begin{columns}
\tablecol[{"ANTENNA\_NO"}{"       " }{J}{(1)}{antennaNo}
{Antenna number}
] 
\tablecol[{"SUBARRAY   "}{"        " }{J}{(1)}{SubA}
{Subarray number}
] 
\tablecol[{"TIME    "}{"DAYS   " }{D}{(1)}{Time}
{The center time.}
] 
\tablecol[{"ORBXYZ    "}{"METERS  " }{D}{(3)}{orbXYZ}
{Obital position }
] 
\tablecol[{"VELXYZ    "}{"M/SEC   " }{D}{(3)}{velXYZ}
{Orbital velocity}
] 
\tablecol[{"SUN\_ANGLE "}{"DEGREES  " }{E}{(3)}{sunAngle}
{Relative position of the Sun}
] 
\tablecol[{"ECLIPSE "}{"DAYS    " }{E}{(4)}{eclipse }
{ECLIPSE }
] 
\tablecol[{"ORIENTATION "}{"DEGREES  " }{E}{(1)}{orientation}
{ORIENTATION}
] 
\end{columns}
%
% Table modification history
\begin{history}
\modhistory[{W. D. Cotton}{02/22/2006}{Revision 1: Copied from AIPS}]
\end{history}
%

\clearpage
%%%%%%%%%%%%%%% ObitTableOF Class %%%%%%%%%%%%%%%%%%%%%%%%%%%%%%%%%%%%%%%%%%%%%%%%%%%
% 
%
\ClassName[{ObitTableOF}]
ObitTableOF Class
\tabletitle{OF table for uv data}
% table name
\tablename{OF}
\tableintro[
{This class contains tabular data and allows access.
"AIPS OF" contains some sort of VLA operational information.
An ObitTableOF is the front end to a persistent disk resident structure.
This class is derived from the ObitTable class. }
]
\tableover{
In memory tables are stored in a fashion similar to how they are 
stored on disk - in large blocks in memory rather than structures.
Due to the word alignment requirements of some machines, they are 
stored by order of the decreasing element size: 
double, float long, int, short, char rather than the logical order.
The details of the storage in the buffer are kept in the 
ObitTableDesc.
}
% Table keyword description
\begin{keywords}
\tablekey[{"REVISION"}{J}{revision}{10}{}
{Revision number of the table definition}
]
\tablekey[{"OBSCODE"}{A}{obscode}{"AA000"}{}
{Observation code}
]
\tablekey[{"RDATE   "}{A}{RefDate}{"YYYYMMDD"}{}
{Reference date as "YYYYMMDD"}
]
\end{keywords}
%
% Table column description
\begin{columns}
\tablecol[{"TIME    "}{"DAYS   " }{E}{(1)}{Time}
{The center time of the sscan.}
] 
\tablecol[{"TIME INTERVAL "}{"DAYS   " }{E}{(1)}{TimeI}
{Duration of scan}
] 
\tablecol[{"SOURCE ID "}{"       " }{J}{(1)}{SourID}
{Source ID as defined in then SOURCE table}
] 
\tablecol[{"ANTENNA NO."}{"        " }{J}{(1)}{antNo}
{Antenna number}
]
\tablecol[{"SUBARRAY "}{"        " }{J}{(1)}{SubA}
{Subarray number}
] 
\tablecol[{"FREQ ID  "}{"        " }{J}{(1)}{FreqID}
{Frequency id of scan}
] 
\tablecol[{"REFN PTG FLAG VA "}{"        "}{L}{(1)}{refPnt}
{Reference pointing(?) flag}
] 
\tablecol[{"SHADOWED FLAG VA "}{"        "}{L}{(1)}{shadow}
{Antenna shadowed(?) flag}
] 
\end{columns}
%
% Table modification history
\begin{history}
\modhistory[{W. D. Cotton}{02/03/2003}{Revision 1: Copied from AIPS}]
\end{history}
%
\clearpage
%%%%%%%%%%%%%%% ObitTablePC Class %%%%%%%%%%%%%%%%%%%%%%%%%%%%%%%%%%%%%%%%%%%%%%%%%%%
% 
%
\ClassName[{ObitTablePC}]
ObitTablePC Class
\tabletitle{Pulsed cal. table for uv data}
% table name
\tablename{PC}
\tableintro[
{This class contains tabular data and allows access.
"AIPS PC" contains pulsed phase cal. info
An ObitTablePC is the front end to a persistent disk resident structure.
Both FITS and AIPS cataloged data are supported.
This class is derived from the ObitTable class. }
]
\tableover{
In memory tables are stored in a fashion similar to how they are 
stored on disk - in large blocks in memory rather than structures.
Due to the word alignment requirements of some machines, they are 
stored by order of the decreasing element size: 
double, float long, int, short, char rather than the logical order.
The details of the storage in the buffer are kept in the 
ObitTableDesc.
}
% Table keyword description
\begin{keywords}
\tablekey[{"TABREV"}{J}{revision}{3}{}
{Revision number of the table definition.}
]
% Must be position 2 to index
\tablekey[{"NO\_POL"}{J}{numPol}{1}{(1,2)}
{The number of polarizations.}
]
\tablekey[{"NO\_BAND "}{J}{numBand}{1}{()}
{The number of Bands(?).}
]
\tablekey[{"NO\_TONES"}{J}{numTones}{1}{()}
{The number of Tones.}
]
\end{keywords}
%
% Table column description
\begin{columns}
\tablecol[{"TIME    "}{"DAYS   " }{D}{(1)}{Time}
{The center time.}
] 
\tablecol[{"SOURCE\_ID  "}{"        " }{J}{(1)}{SourID}
{Source ID number}
] 
\tablecol[{"ANTENNA\_NO"}{"        " }{J}{(1)}{antennaNo}
{Antenna number}
]
\tablecol[{"ARRAY      "}{"        " }{J}{(1)}{Array}
{Array number}
] 
\tablecol[{"FREQID"}{"        "}{J}{(1)}{FreqID}
{Frequency ID}
] 
\tablecol[{"CABLE\_CAL"}{"SECONDS "}{D}{(1)}{CableCal}
{CABLE\_CAL }
] 
\tablecol[{"STATE\_\#NO\_POL "}{"PERCENT "}{E}{(4,numBand)}{State}
{State counts(?)}
] 
\tablecol[{"PC\_FREQ\_\#NO\_POL "}{"HZ      "}{D}{(numTones,numBand)}{PCFreq}
{Frequencies of the phase tones}
] 
\tablecol[{"PC\_REAL\_\#NO\_POL "}{"        "}{E}{(numTones,numBand)}{PCReal}
{Real part of tone phase}
] 
\tablecol[{"PC\_IMAG\_\#NO\_POL "}{"        "}{E}{(numTones,numBand)}{PCImag}
{Imaginary part of tone phase}
] 
\tablecol[{"PC\_RATE\_\#NO\_POL "}{"SEC/SEC "}{E}{(numTones,numBand)}{PCRate}
{Tone rate phase}
] 
\end{columns}
%
% Table modification history
\begin{history}
\modhistory[{W. D. Cotton}{02/22/2006}{Revision 1: Copied from AIPS}]
\end{history}

\clearpage
%
%%%%%%%%%%%%%%% ObitTablePD Class %%%%%%%%%%%%%%%%%%%%%%%%%%%%%%%%%%%%%%%%%%%%%%%%%%%
% 
%
\ClassName[{ObitTablePD}]
{AIPS Antenna polarization spectrum table for UV data.}
%\ClassEnum[{obitObjType}
%{Type of underlying data structures}]
%\begin{ObitEnum}
%\ObitEnumItem[{OBITTYPE\_FITS}{FITS file}]
%\ObitEnumItem[{OBITTYPE\_AIPS}{AIPS catalog data}]
%\end{ObitEnum}

\tabletitle{AIPS Antenna polarization spectrum table for UV data}
% table name
\tablename{PD}
\tableintro[
{This class contains tabular data and allows access.
"AIPS PD" tables contain instrumental polarization parameters for each channel
An ObitTablePD is the front end to a persistent disk resident structure.
Both FITS and AIPS cataloged data are supported.
This class is derived from the ObitTable class. }
]
\tableover{
In memory tables are stored in a fashion similar to how they are 
stored on disk - in large blocks in memory rather than structures.
Due to the word alignment requirements of some machines, they are 
stored by order of the decreasing element size: 
double, float long, int, short, char rather than the logical order.
The details of the storage in the buffer are kept in the 
ObitTableDesc.
}
% Table keyword description
\begin{keywords}
\tablekey[{"REVISION"}{J}{revision}{1}{}
{Revision number of the table definition.}
]
% Must be position 2 to index
\tablekey[{"NO\_POL  "}{J}{numPol}{2}{(1,2)}
{The number of orthogonal polarizations.}
]
\tablekey[{"NO\_IF  "}{J}{numIF}{1}{()}
{The number of IFs}
]
\tablekey[{"NO\_CHAN"}{J}{numChan}{}{()}
{Number of frequency channels}
]
\tablekey[{"NO\_ANT  "}{J}{numAnt}{1}{}
{The number of antennas in table.}
]
\tablekey[{"POLTYPE "}{A}{polType}{"        "}{}
{Polarization parameterazation type, 'ORI-ELP', 'APPROX', 'VLBI',
'X-Y LIN }
]
\end{keywords}
%
% Table column description
\begin{columns}
\tablecol[{"ANTENNA"}{"       " }{J}{(1)}{antNo}
{Antenna number }
] 
\tablecol[{"SUBARRAY"}{"        "}{J}{(1)}{SubA}
{Subarray number.}
] 
\tablecol[{"FREQ ID"}{"        "}{J}{(1)}{FreqID}
{Frequency ID}
] 
\tablecol[{"REFANT"}{"        "}{J}{(1)}{RefAnt}
{Reference antenna Poln \# NO\_POL }
] 
\tablecol[{"P\_DIFF"}{"DEGREE  "}{E}{(numChan,numIF)}{RLPhase}
{ Right-left phase difference per channel/IF, channel varing fastest.}
] 
\tablecol[{"REAL \#NO\_POL"}{"       " }{E}{(numChan,numIF)}{Real}
{First parameter, real or ellipticity per channel/IF for poln  \# NO\_POL, 
channel varing fastest.}
] 
\tablecol[{"IMAG \#NO\_POL"}{"        "}{E}{(numChan,numIF)}{Imag}
{Second parameter, imaginary or orientation per channel/IF for poln  \# NO\_POL, 
channel varing fastest.}
] 
\end{columns}
%
% Table modification history
\begin{history}
\modhistory[{W. D. Cotton}{08/16/2010}{Revision 1: Copied from AIPS}]
\end{history}
%
\clearpage
%%%%%%%%%%%%%%% ObitTablePO Class %%%%%%%%%%%%%%%%%%%%%%%%%%%%%%%%%%%%%%%%%%%%%%%%%%%
% 
%
\ClassName[{ObitTablePO}]
ObitTablePO Class
\tabletitle{Planetary position table for uv data}
% table name
\tablename{PO}
\tableintro[
{This class contains tabular data and allows access.
"AIPS PO" contains position information for solar system objects.
An ObitTablePC is the front end to a persistent disk resident structure.
Both FITS and AIPS cataloged data are supported.
This class is derived from the ObitTable class. }
]
\tableover{
In memory tables are stored in a fashion similar to how they are 
stored on disk - in large blocks in memory rather than structures.
Due to the word alignment requirements of some machines, they are 
stored by order of the decreasing element size: 
double, float long, int, short, char rather than the logical order.
The details of the storage in the buffer are kept in the 
ObitTableDesc.
}
% Table keyword description
\begin{keywords}
\tablekey[{"TABREV"}{J}{revision}{1}{}
{Revision number of the table definition.}
]
\tablekey[{"RDATE   "}{A}{RefDate}{"YYYYMMDD"}{}
{Reference date as "YYYYMMDD"}
]
\end{keywords}
%
% Table column description
\begin{columns}
\tablecol[{"TIME    "}{"DAYS   " }{D}{(1)}{Time}
{The center time.}
] 
\tablecol[{"SOURCE\_ID  "}{"        " }{J}{(1)}{SourID}
{Source ID number}
] 
\tablecol[{"SOURCE RA"}{"DEGREES " }{D}{(1)}{RA}
{Right ascension at mean EPOCH (actually equinox) }
] 
\tablecol[{"SOURCE DEC"}{"DEGREES " }{D}{(1)}{Dec}
{Declination at mean EPOCH (actually equinox) }
] 
\tablecol[{"SOURCE DISTANCE"}{"AU      " }{D}{(1)}{Dist}
{Declination at mean EPOCH (actually equinox) }
] 
\end{columns}
%
% Table modification history
\begin{history}
\modhistory[{W. D. Cotton}{02/25/2015}{Revision 1: Copied from AIPS}]
\end{history}

\clearpage
%%%%%%%%%%%%%%% ObitTablePO Class %%%%%%%%%%%%%%%%%%%%%%%%%%%%%%%%%%%%%%%%%%%%%%%%%%%
% 
%
\ClassName[{ObitTablePS}]
ObitTablePS Class
\tabletitle{Processing Summary}
% table name
\tablename{PS}
\tableintro[
{This class contains tabular data and allows access.
"AIPS PS" contains the summary of the processing of a source.
An ObitTablePS is the front end to a persistent disk resident structure.
This class is derived from the ObitTable class. }
]
\tableover{
In memory tables are stored in a fashion similar to how they are 
stored on disk - in large blocks in memory rather than structures.
Due to the word alignment requirements of some machines, they are 
stored by order of the decreasing element size: 
double, float long, int, short, char rather than the logical order.
The details of the storage in the buffer are kept in the 
ObitTableDesc.
}
% Table keyword description
\begin{keywords}
\tablekey[{"REVISION"}{J}{revision}{1}{}
]
\end{keywords}
%
% Table column description
\begin{columns}
\tablecol[{"FIELD NAME    "}{"       " }{A}{(16)}{FieldName}
{Field name (source)}
] 
\tablecol[{"OBSDATE "}{"      " }{A}{(8)}{obsdate}
{Observing ref. date.(YYYYMMDD)}
] 
\tablecol[{"RA POINT "}{"DEG    " }{D}{(1)}{RAPoint}
{Pointing RA at epoch 2000 }
] 
\tablecol[{"DEC POINT"}{"DEG    " }{D}{(1)}{DecPoint}
{Pointing Dec at epoch 2000}
] 
\tablecol[{"STATUS   "}{"        " }{A}{(8)}{Status}
{Processing status}
] 
\tablecol[{"IQU      "}{"        "}{L}{(3)}{IQU}
{IQU Flags True if processed}
] 
\tablecol[{"SPECIAL  "}{"        " }{L}{(1)}{Special}
{Special processing applied}
] 
\tablecol[{"NO. FIELDS"}{"        " }{J}{(1)}{NoFields}
{Number of sub fields}
] 
\tablecol[{"CC LIMIT  "}{"        " }{L}{(3)}{CCLimit}
{Flags to using all possible CC.}
] 
\tablecol[{"PERCENT  "}{"        " }{E}{(3)}{PerCent}
{Percent of visibility data I,Q,U}
]
\tablecol[{"PIX MAX  "}{"        " }{E}{(3)}{PixMax}
{Pixel max (Jy) I,Q,U}
] 
\tablecol[{"PIX MIN  "}{"        " }{E}{(3)}{PixMin}
{Pixel min (Jy) I,Q,U}
] 
\tablecol[{"QUALITY  "}{"        " }{E}{(3)}{Quality}
{Quality measure, RMS I,Q,U}
] 
\tablecol[{"COMMENTS "}{"        " }{A}{(80)}{Comment}
{Comments}
] 
\end{columns}
%
% Table modification history
\begin{history}
\modhistory[{W. D. Cotton}{03/15/2006}{Revision 1: Copied from AIPS}]
\end{history}
%
\clearpage

%%%%%%%%%%%%%%% ObitTableSN Class %%%%%%%%%%%%%%%%%%%%%%%%%%%%%%%%%%%%%%%%%%%%%%%%%%%
% 
%
\ClassName[{ObitTableSN}]
{SolutioN table for UV data.}
%\ClassEnum[{obitObjType}
%{Type of underlying data structures}]
%\begin{ObitEnum}
%\ObitEnumItem[{OBITTYPE\_FITS}{FITS file}]
%\ObitEnumItem[{OBITTYPE\_AIPS}{AIPS catalog data}]
%\end{ObitEnum}

\tabletitle{SolutioN table for UV data}
% table name
\tablename{SN}
\tableintro[
{This class contains tabular data and allows access.
"AIPS SN" amplitude/phase/delay and rate calibration information derived
from a calibration solution and can be used either for "self calibration"
or correction a multisource CaLibration table.
An ObitTableSN is the front end to a persistent disk resident structure.
Both FITS and AIPS cataloged data are supported.
This class is derived from the ObitTable class. }
]
\tableover{
In memory tables are stored in a fashion similar to how they are 
stored on disk - in large blocks in memory rather than structures.
Due to the word alignment requirements of some machines, they are 
stored by order of the decreasing element size: 
double, float long, int, short, char rather than the logical order.
The details of the storage in the buffer are kept in the 
ObitTableDesc.
}
% Table keyword description
\begin{keywords}
\tablekey[{"REVISION"}{J}{revision}{10}{}
{Revision number of the table definition.
}
]
% Must be position 2 to index
\tablekey[{"NO\_POL  "}{J}{numPol}{1}{(1,2)}
{The number of polarizations.
}
]
\tablekey[{"NO\_IF  "}{J}{numIF}{}{()}
{The number of IFs
}
]
\tablekey[{"NO\_ANT  "}{J}{numAnt}{1}{}
{The number of antennas in table.
}
]
\tablekey[{"NO\_NODES"}{J}{numNodes}{0}{}
{The number of interpolation nodes.
}
]
\tablekey[{"MGMOD   "}{D}{mGMod}{1.0}{}
{The Mean Gain modulus 
}
]
\tablekey[{"APPLIED"}{L}{isApplied}{FALSE}{}
{True if table has been applied to a CL table.}
]
\end{keywords}
%
% Table column description
\begin{columns}
\tablecol[{"TIME    "}{"DAYS   " }{D}{(1)}{Time}
{The center time of the solution}
] 
\tablecol[{"TIME INTERVAL"}{"DAYS    "}{E}{(1)}{TimeI}
{Solution interval.}
] 
\tablecol[{"SOURCE ID"}{"        "}{J}{(1)}{SourID}
{Source Identifier number }
] 
\tablecol[{"ANTENNA NO."}{"       " }{J}{(1)}{antNo}
{Antenna number }
] 
\tablecol[{"SUBARRAY"}{"        "}{J}{(1)}{SubA}
{Subarray number.}
] 
\tablecol[{"FREQ ID"}{"        "}{J}{(1)}{FreqID}
{Frequency ID}
] 
\tablecol[{"I.FAR.ROT"}{"RAD/M**2"}{E}{(1)}{IFR}
{Ionospheric Faraday Rotation }
] 
\tablecol[{"NODE NO."}{"        "}{J}{(1)}{NodeNo}
{Node number }
] 
\tablecol[{"MBDELAY\#NO\_POL"}{"SECONDS "}{E}{(1)}{MBDelay}
{Multiband delay poln \# NO\_POL  }
] 
\tablecol[{"REAL\#NO\_POL"}{"       " }{E}{(numIF)}{Real}
{Real (gain Poln \# NO\_POL )}
] 
\tablecol[{"IMAG\#NO\_POL"}{"        "}{E}{(numIF)}{Imag}
{Imaginary (gain Poln \# NO\_POL)}
] 
\tablecol[{"DELAY \#NO\_POL"}{"SECONDS "}{E}{(numIF)}{Delay}
{ Residual group delay Poln \# NO\_POL}
] 
\tablecol[{"RATE \#NO\_POL"}{"SEC/SEC "}{E}{(numIF)}{Rate}
{Residual fringe rate  Poln \# NO\_POL}
] 
\tablecol[{"WEIGHT \#NO\_POL"}{"        "}{E}{(numIF)}{Weight}
{Weight of soln. Poln \# NO\_POL}
] 
\tablecol[{"REFANT \#NO\_POL"}{"        "}{J}{(numIF)}{RefAnt}
{Reference antenna Poln \# NO\_POL }
] 
\end{columns}
%
% Table modification history
\begin{history}
\modhistory[{W. D. Cotton}{28/02/2002}{Revision 1: Copied from AIPS}]
\end{history}
%
\clearpage
%%%%%%%%%%%%%%% ObitTableSU Class %%%%%%%%%%%%%%%%%%%%%%%%%%%%%%%%%%%%%%%%%%%%%%%%%%%
% 
%
\ClassName[{ObitTableSU}]
ObitTableSU Class
\tabletitle{SoUrce table for uv data}
% table name
\tablename{SU}
\tableintro[
{This class contains tabular data and allows access.
"AIPS SU" contains information about astronomical sources.
An ObitTableSU is the front end to a persistent disk resident structure.
Both FITS and AIPS cataloged data are supported.
This class is derived from the ObitTable class. }
]
\tableover{
In memory tables are stored in a fashion similar to how they are 
stored on disk - in large blocks in memory rather than structures.
Due to the word alignment requirements of some machines, they are 
stored by order of the decreasing element size: 
double, float long, int, short, char rather than the logical order.
The details of the storage in the buffer are kept in the 
ObitTableDesc.
}
% Table keyword description
\begin{keywords}
\tablekey[{"NO\_IF  "}{J}{numIF}{}{()}
{The number of IFs}
]
\tablekey[{"VELTYP  "}{A}{velType}{"LSR"}{}
{Velocity type}
]
\tablekey[{"VELDEF  "}{A}{velDef}{"RADIO"}{}
{Velocity definition 'RADIO' or 'OPTICAL'}
]
\tablekey[{"FREQID  "}{J}{FreqID}{0}{}
{The Frequency ID for which the source parameters are relevant.}
]
\end{keywords}
%
% Table column description
\begin{columns}
\tablecol[{"ID. NO. "}{"       " }{J}{(1)}{SourID}
{Source ID}
] 
\tablecol[{"SOURCE  "}{"       " }{A}{(16)}{Source}
{Source name }
] 
\tablecol[{"QUAL    "}{"       " }{J}{(1)}{Qual}
{Source qualifier}
] 
\tablecol[{"CALCODE "}{"       " }{A}{(4)}{CalCode}
{Calibrator code}
] 
\tablecol[{"IFLUX   "}{"JY     " }{E}{(numIF)}{IFlux}
{Total Stokes I flux density per IF}
] 
\tablecol[{"QFLUX   "}{"JY     " }{E}{(numIF)}{QFlux}
{Total Stokes Q flux density per IF}
] 
\tablecol[{"UFLUX   "}{"JY     " }{E}{(numIF)}{UFlux}
{Total Stokes U flux density per IF}
] 
\tablecol[{"VFLUX   "}{"JY     " }{E}{(numIF)}{VFlux}
{Total Stokes V flux densityper IF }
] 
\tablecol[{"FREQOFF "}{"HZ     " }{D}{(numIF)}{FreqOff}
{Frequency offset (Hz) from IF nominal per IF}
] 
\tablecol[{"BANDWIDTH"}{"HZ     " }{D}{(1)}{Bandwidth}
{Bandwidth}
] 
\tablecol[{"RAEPO   "}{"DEGREES " }{D}{(1)}{RAMean}
{Right ascension at mean EPOCH (actually equinox) }
] 
\tablecol[{"DECEPO  "}{"DEGREES " }{D}{(1)}{DecMean}
{Declination at mean EPOCH (actually equinox) }
] 
\tablecol[{"EPOCH   "}{"YEARS   " }{D}{(1)}{Epoch}
{Mean Epoch (really equinox) for position in yr. since year 0.0}
] 
\tablecol[{"RAAPP   "}{"DEGREES " }{D}{(1)}{RAApp}
{Apparent Right ascension }
] 
\tablecol[{"DECAPP  "}{"DEGREES " }{D}{(1)}{DecApp}
{Apparent Declination}
] 
\tablecol[{"RAOBS   "}{"DEGREES " }{D}{(1)}{RAObs}
{[OPTIONAL] Observed Right ascension }
] 
\tablecol[{"DECOBS  "}{"DEGREES " }{D}{(1)}{DecObs}
{[OPTIONAL] Observed Declination}
] 
\tablecol[{"LSRVEL "}{"M/SEC    " }{D}{(numIF)}{LSRVel}
{LSR velocity per IF }
] 
\tablecol[{"RESTFREQ"}{"HZ     " }{D}{(numIF)}{RestFreq}
{Line rest frequency per IF }
] 
\tablecol[{"PMRA   "}{"DEG/DAY " }{D}{(1)}{PMRa}
{Proper motion (deg/day) in RA}
] 
\tablecol[{"PMDEC  "}{"DEG/DAY " }{D}{(1)}{PMDec}
{Proper motion (deg/day) in declination}
] 
\end{columns}
%
% Table modification history
\begin{history}
\modhistory[{W. D. Cotton}{09/03/2003}{Revision 1: Copied from AIPS}]
\modhistory[{W. D. Cotton}{08/29/2012}{Revision 2: Add RAOBS, DECOBS}]
\end{history}
%
%
\clearpage
%%%%%%%%%%%%%%% ObitTableTY Class %%%%%%%%%%%%%%%%%%%%%%%%%%%%%%%%%%%%%%%%%%%%%%%%%%%
% 
%
\ClassName[{ObitTableTY}]
ObitTableTY Class
\tabletitle{ Temperature table for uv data}
% table name
\tablename{TY}
\tableintro[
{This class contains tabular data and allows access.
"AIPS TY" contains Antenna and system temperature information.
An ObitTableTY is the front end to a persistent disk resident structure.
Both FITS and AIPS cataloged data are supported.
This class is derived from the ObitTable class. }
]
\tableover{
In memory tables are stored in a fashion similar to how they are 
stored on disk - in large blocks in memory rather than structures.
Due to the word alignment requirements of some machines, they are 
stored by order of the decreasing element size: 
double, float long, int, short, char rather than the logical order.
The details of the storage in the buffer are kept in the 
ObitTableDesc.
}
% Table keyword description
\begin{keywords}
\tablekey[{"NO\_IF"}{J}{numIF}{}{()}
{The number of IFs}
]
% Must be position 2 to index
\tablekey[{"NO\_POL "}{J}{numPol}{1}{(1,2)}
{The number of polarizations.}
]
\tablekey[{"REVISION"}{J}{revision}{10}{}
{Revision number of the table definition.}
]
\end{keywords}
%
% Table column description
\begin{columns}
\tablecol[{"TIME    "}{"DAYS   " }{E}{(1)}{Time}
{The center time.}
] 
\tablecol[{"TIME INTERVAL "}{"DAYS   " }{E}{(1)}{TimeI}
{Time interval of record }
] 
\tablecol[{"SOURCE ID  "}{"        " }{J}{(1)}{SourID}
{Source ID number}
] 
\tablecol[{"ANTENNA NO."}{"        " }{J}{(1)}{antennaNo}
{Antenna number}
]
\tablecol[{"SUBARRAY   "}{"        " }{J}{(1)}{SubA}
{Subarray number}
] 
\tablecol[{"FREQ ID  "}{"        " }{J}{(1)}{FreqID}
{Frequency id of scan}
] 
\tablecol[{"TSYS \#NO\_POL"}{"KELVINS " }{E}{(numIF)}{Tsys}
{Real (gain Poln \# NO\_POL )}
] 
\tablecol[{"TANT \#NO\_POL"}{"KELVINS "}{E}{(numIF)}{Tant}
{Imaginary (gain Poln \# NO\_POL)}
] 
\end{columns}
%
% Table modification history
\begin{history}
\modhistory[{W. D. Cotton}{02/22/2006}{Revision 1: Copied from AIPS}]
\end{history}
%
\clearpage
%%%%%%%%%%%%%%% ObitTableVL Class %%%%%%%%%%%%%%%%%%%%%%%%%%%%%%%%%%%%%%%%%%%%%%%%%%%
% 
\ClassName[{ObitTableVL}]
ObitTableVL Class 
\tabletitle{NVSS full format catalog file}
% table name
\tablename{VL}
\tableintro[
{This class contains tabular data and allows access.
"AIPS VL" contains a catalog of sources in the format produced for the
NVSS survey.
The table contains an index in the INDEX?? keywords that give the
first table row number (1-rel) for a given hour of RA.
An ObitTableVL is the front end to a persistent disk resident structure.
Both FITS and AIPS cataloged data are supported.
This class is derived from the ObitTable class. }
]
\tableover{
In memory tables are stored in a fashion similar to how they are 
stored on disk - in large blocks in memory rather than structures.
Due to the word alignment requirements of some machines, they are 
stored by order of the decreasing element size: 
double, float long, int, short, char rather than the logical order.
The details of the storage in the buffer are kept in the 
ObitTableDesc.
}
% Table keyword description
\begin{keywords}
\tablekey[{"REVISION"}{J}{revision}{1}{}
{Revision number of the table definition}
]
\tablekey[{"BM\_MAJOR"}{E}{BeamMajor}{0.0125}{}
{Restoring beam major axis in deg.}
]
\tablekey[{"BM\_MINOR"}{E}{BeamMinor}{0.0125}{}
{Restoring beam minor axis in deg.}
]
\tablekey[{"BM\_PA"}{E}{BeamPA}{0.0}{}
{Restoring beam position angle of major axis in deg.}
]
\tablekey[{"SORTORT"}{J}{SortOrder}{0}{}
{Column number for sort (neg -> descending)}
]
\tablekey[{"NUM\_INDE"}{J}{numIndexed}{1}{}
{Number of rows in table when indexed}
]
\tablekey[{"INDEX00"}{J}{index00}{1}{}
{First entry for RA=00 h}
]
\tablekey[{"INDEX01"}{J}{index01}{1}{}
{First entry for RA=01 h}
]
\tablekey[{"INDEX02"}{J}{index02}{1}{}
{First entry for RA=02 h}
]
\tablekey[{"INDEX03"}{J}{index03}{1}{}
{First entry for RA=03 h}
]
\tablekey[{"INDEX04"}{J}{index04}{1}{}
{First entry for RA=04 h}
]
\tablekey[{"INDEX05"}{J}{index05}{1}{}
{First entry for RA=05 h}
]
\tablekey[{"INDEX06"}{J}{index06}{1}{}
{First entry for RA=06 h}
]
\tablekey[{"INDEX07"}{J}{index07}{1}{}
{First entry for RA=07 h}
]
\tablekey[{"INDEX08"}{J}{index08}{1}{}
{First entry for RA=08 h}
]
\tablekey[{"INDEX09"}{J}{index09}{1}{}
{First entry for RA=09 h}
]
\tablekey[{"INDEX10"}{J}{index10}{1}{}
{First entry for RA=10 h}
]
\tablekey[{"INDEX11"}{J}{index11}{1}{}
{First entry for RA=11 h}
]
\tablekey[{"INDEX12"}{J}{index12}{1}{}
{First entry for RA=12 h}
]
\tablekey[{"INDEX13"}{J}{index13}{1}{}
{First entry for RA=13 h}
]
\tablekey[{"INDEX14"}{J}{index14}{1}{}
{First entry for RA=14 h}
]
\tablekey[{"INDEX15"}{J}{index15}{1}{}
{First entry for RA=15 h}
]
\tablekey[{"INDEX16"}{J}{index16}{1}{}
{First entry for RA=16 h}
]
\tablekey[{"INDEX17"}{J}{index17}{1}{}
{First entry for RA=17 h}
]
\tablekey[{"INDEX18"}{J}{index18}{1}{}
{First entry for RA=18 h}
]
\tablekey[{"INDEX19"}{J}{index19}{1}{}
{First entry for RA=19 h}
]
\tablekey[{"INDEX20"}{J}{index20}{1}{}
{First entry for RA=20 h}
]
\tablekey[{"INDEX21"}{J}{index21}{1}{}
{First entry for RA=21 h}
]
\tablekey[{"INDEX22"}{J}{index22}{1}{}
{First entry for RA=22 h}
]
\tablekey[{"INDEX23"}{J}{index23}{1}{}
{First entry for RA=23 h}
]
\end{keywords}
%
% Table column description
\begin{columns}
%] 
\tablecol[{"RA(2000)"}{"DEGREE  " }{D}{(1)}{Ra2000}
{RA (J2000)}
] 
\tablecol[{"DEC(2000)"}{"DEGREE  " }{D}{(1)}{Dec2000}
{Dec (J2000)}
] 
\tablecol[{"PEAK INT  "}{"JY/BEAM " }{E}{(1)}{PeakInt}
{Peak Ipol }
] 
\tablecol[{"MAJOR AX"}{"DEGREE  " }{E}{(1)}{MajorAxis}
{Fitted major axis size}
] 
\tablecol[{"MINOR AX"}{"DEGREE  " }{E}{(1)}{MinorAxis}
{Fitted minor axis siz}
] 
\tablecol[{"POSANGLE"}{"DEGREE  " }{E}{(1)}{PosAngle}
{Fitted PA}
] 
\tablecol[{"Q CENTER"}{"JY/BEAM " }{E}{(1)}{QCenter}
{Center Q flux density}
] 
\tablecol[{"U CENTER"}{"JY/BEAM " }{E}{(1)}{UCenter}
{enter U flux density}
] 
\tablecol[{"P FLUX"}{"JY      " }{E}{(1)}{PFlux}
{Integrated polarized flux density}
] 
\tablecol[{"I RMS "}{"JY/BEAM " }{E}{(1)}{IRMS}
{Ipol RMS uncertainty }
] 
\tablecol[{"POL RMS "}{"JY/BEAM " }{E}{(1)}{PolRMS}
{RMS (sigma) in Qpol and Upol}
] 
\tablecol[{"RES RMS"}{"JY/BEAM " }{E}{(1)}{ResRMS}
{RMS of Ipol residual}
] 
\tablecol[{"RES PEAK"}{"JY/BEAM " }{E}{(1)}{ResPeak}
{Peak in Ipol residual}
] 
\tablecol[{"RES FLUX"}{"JY      " }{E}{(1)}{ResFlux}
{Integrated Ipol residual}
] 
\tablecol[{"CENTER X"}{"PIXEL   " }{E}{(1)}{CenterX}
{Center x position in pixels in FIELD}
] 
\tablecol[{"CENTER Y"}{"PIXEL   " }{E}{(1)}{CenterY}
{Center y position in pixels in FIELD}
] 
\tablecol[{"FIELD  "}{"       " }{A}{(8)}{Field}
{Name of survey field}
] 
\tablecol[{"JD PROCESSED"}{"DAYS    " }{J}{(1)}{JDProcess}
{Julian date on which entry was derived from image.}
] 
\end{columns}
%
% Table modification history
\begin{history}
\modhistory[{W. D. Cotton}{02/03/2003}{Revision 1: Copied from AIPS}]
\end{history}
%
\clearpage
%%%%%%%%%%%%%%% ObitTableVZ Class %%%%%%%%%%%%%%%%%%%%%%%%%%%%%%%%%%%%%%%%%%%%%%%%%%%
% 
%
\ClassName[{ObitTableVZ}]
ObitTableVZ Class
\tabletitle{NVSS short format catalog file}
% table name
\tablename{VZ}
\tableintro[
{This class contains tabular data and allows access.
"AIPS VZ" contains a catalog of sources.
An ObitTableVZ is the front end to a persistent disk resident structure.
Both FITS and AIPS cataloged data are supported.
This class is derived from the ObitTable class. }
]
\tableover{
In memory tables are stored in a fashion similar to how they are 
stored on disk - in large blocks in memory rather than structures.
Due to the word alignment requirements of some machines, they are 
stored by order of the decreasing element size: 
double, float long, int, short, char rather than the logical order.
The details of the storage in the buffer are kept in the 
ObitTableDesc.
}
% Table keyword description
\begin{keywords}
\tablekey[{"REVISION"}{J}{revision}{1}{}
{Revision number of the table definition}
]
\tablekey[{"REF\_FREQ"}{D}{refFreq}{1.4e9}{}
{The Reference frequency}
]
\end{keywords}
%
% Table column description
\begin{columns}
\tablecol[{"RA(2000)"}{"DEGREE  " }{D}{(1)}{Ra2000}
{RA (J2000)}
] 
\tablecol[{"DEC(2000)"}{"DEGREE  " }{D}{(1)}{Dec2000}
{Dec (J2000)}
] 
\tablecol[{"PEAK INT  "}{"JY/BEAM " }{E}{(1)}{PeakInt}
{Peak Ipol }
] 
\tablecol[{"QUAL  "}{"        " }{J}{(1)}{Quality}
{Quality (crowding) measure, 0=> best, higher worse}
] 
\end{columns}
%
% Table modification history
\begin{history}
\modhistory[{W. D. Cotton}{02/03/2003}{Revision 1: Copied from AIPS}]
\end{history}
%
%
\clearpage
%%%%%%%%%%%%%%% ObitTableWX Class %%%%%%%%%%%%%%%%%%%%%%%%%%%%%%%%%%%%%%%%%%%%%%%%%%%
% Obit base class
%
\ClassName[{ObitTableWX}]
{Weather table for UV data.}

\tabletitle{Weather table for UV data}
% table name
\tablename{WX}
\tableintro[
{This class contains tabular data and allows access.
"AIPS WX" tables contain weather information.
An ObitTableWX is the front end to a persistent disk resident structure.
Both FITS and AIPS cataloged data are supported.
This class is derived from the ObitTable class. }
]
\tableover{
In memory tables are stored in a fashion similar to how they are 
stored on disk - in large blocks in memory rather than structures.
Due to the word alignment requirements of some machines, they are 
stored by order of the decreasing element size: 
double, float long, int, short, char rather than the logical order.
The details of the storage in the buffer are kept in the 
ObitTableDesc.
}
% Table keyword description
\begin{keywords}
\tablekey[{"TABREV"}{J}{revision}{3}{}
{Revision number of the table definition.}
]
\tablekey[{"RDATE   "}{A}{RefDate}{"YYYYMMDD"}{}
{Reference date as "YYYYMMDD"}
]
\end{keywords}
%
% Table column description
\begin{columns}
\tablecol[{"TIME    "}{"DAYS   " }{D}{(1)}{Time}
{The center time of the solution}
] 
\tablecol[{"TIME INTERVAL"}{"DAYS    "}{E}{(1)}{TimeI}
{Solution interval.}
] 
\tablecol[{"ANTENNA NO."}{"       " }{J}{(1)}{antNo}
{Antenna number }
] 
\tablecol[{"SUBARRAY"}{"        "}{J}{(1)}{SubA}
{Subarray number.}
] 
\tablecol[{"TEMPERATURE"}{"CENTIGRA"}{E}{(1)}{temperature}
{Temperature}
] 
\tablecol[{"PRESSURE"}{"MILLIBAR"}{E}{(1)}{pressure}
{Pressure}
] 
\tablecol[{"DEWPOINT"}{"CENTIGRA"}{E}{(1)}{dewpoint}
{Dew Point}
] 
\tablecol[{"WIND\_VELOCITY"}{"M/SEC  "}{E}{(1)}{windVelocity}
{Wind velocity}
] 
\tablecol[{"WIND\_DIRECTION"}{"DEGREES "}{E}{(1)}{windDirection}
{Wind direction (azimuth)}
] 
\tablecol[{"WVR\_H2O"}{"        "}{E}{(1)}{wvrH2O}
{Water vapor radiometer}
] 
\tablecol[{"ONOS\_ELECTRON"}{"        "}{E}{(1)}{onosElectron}
{Ionospheric electron something}
] 
\end{columns}
%
% Table modification history
\begin{history}
\modhistory[{W. D. Cotton}{22/02/2006}{Revision 1: Copied from AIPS}]
\end{history}
%
\clearpage

%%%%%%%%%%%%%%% ObitTableIDI_ANTENNA Class %%%%%%%%%%%%%%%%%%%%%%%%%%%%%%%%%%%%%%%%%%%
% 
%
\ClassName[{ObitTableIDI\_ANTENNA}]
ObitTableAN Class
\tabletitle{Antenna table for IDI uv data}
% table name
\tablename{IDI\_ANTENNA}
\tableintro[
{This class contains tabular data and allows access.
This table is part of the IDI uv data format.
"IDI\_ANTENNA" contains information about the characteristics
of antennas in a UV data set.
Only FITS cataloged data are supported.
Polarization calibration information is included. }
]
\tableover{
In memory tables are stored in a fashion similar to how they are 
stored on disk - in large blocks in memory rather than structures.
Due to the word alignment requirements of some machines, they are 
stored by order of the decreasing element size: 
double, float long, int, short, char rather than the logical order.
The details of the storage in the buffer are kept in the 
ObitTableDesc.
}
% Table keyword description
\begin{keywords}
\tablekey[{"TABREV  "}{J}{tabrev}{1}{}
{Table format revision number}
]
\tablekey[{"NO\_STKD "}{J}{no\_stkd}{1}{}
{Number of Stokes parameters}
]
\tablekey[{"STK\_1   "}{J}{stk\_1}{-1}{}
{First Stokes parameter}
]
\tablekey[{"NO\_BAND "}{J}{no\_band}{1}{()}
{Number of frequency bands }
]
\tablekey[{"NO\_CHAN "}{J}{no\_chan}{1}{}
{Number of frequency channels}
]
\tablekey[{"REF\_FREQ"}{D}{ref\_freq}{1.0}{}
{Reference frequency (Hz)}
]
\tablekey[{"CHAN\_BW "}{D}{chan\_bw}{1.0}{}
{Channel bandwidth (Hz)}
]
\tablekey[{"REF\_PIXL"}{J}{ref\_pixl}{1.0}{}
{Reference frequency bin}
]
\tablekey[{"OBSCODE "}{A}{obscode}{"    "}{}
{Observation project code}
]
\tablekey[{"RDATE   "}{A}{RefDate}{"YYYY-MM-DD"}{}
{Reference date as "YYYY-MM-DD"}
]
\tablekey[{"NOPCAL "}{J}{numPCal}{4}{()}
{Number of polarization calibration constants}
]
\tablekey[{"POLTYPE "}{A}{polType}{"        "}{}
{Polarization parameterazation type, 'APPR', 'RAPPR', 'ORI-ELP'}
]
\end{keywords}
%
% Table column description
\begin{columns}
\tablecol[{"TIME    "}{"DAY    " }{D}{(1)}{time}
{Time.}
] 
\tablecol[{"TIME\_INTERVAL"}{"DAY    " }{E}{(1)}{time\_interval}
{Time interval over which antenna characteristics valid.}
] 
\tablecol[{"ANNAME  "}{"        " }{A}{(8)}{AntName}
{Station name}
] 
\tablecol[{"ANTENNA\_NO"}{"       " }{J}{(1)}{antenna\_no}
{Station number, used as an index in other tables, uv data}
] 
\tablecol[{"ARRAY"}{"       " }{J}{(1)}{array}
{Subarray number}
] 
\tablecol[{"FREQID"}{"       " }{J}{(1)}{freqid}
{Frequency group id}
] 
\tablecol[{"NO\_LEVELS"}{"       " }{J}{(1)}{no\_levels}
{Quantization of data, no. levels.}
] 
\tablecol[{"POLTYA  "}{"       " }{A}{(1)}{polTypeA}
{Feed A feed poln. type 'R','L','X','Y', actually only one valid character.}
] 
\tablecol[{"POLAA   "}{"DEGREES " }{E}{(no\_band)}{PolAngA}
{Feed A feed position angle}
] 
\tablecol[{"POLCALA"}{"       " }{E}{(numPCal,no\_band)}{PolCalA}
{Feed A poln. cal parameter. }
] 
\tablecol[{"POLTYB "}{"       " }{A}{(1)}{polTypeB}
{Feed B feed poln. type 'R','L','X','Y'}
] 
\tablecol[{"POLAB  "}{"DEGREES " }{E}{(no\_band)}{PolAngB}
{Feed B feed position angle}
] 
\tablecol[{"POLCALB"}{"       " }{E}{(numPCal,no\_band)}{PolCalB}
{Feed B poln. cal parameter}
] 
\tablecol[{"BEAMFWHM"}{"DEGREES " }{E}{(no\_band)}{BeamFWHM}
{[OPTIONAL] Beam Full Width Half Maximum}
] 
\end{columns}
%
% Table modification history
\begin{history}
\modhistory[{W. D. Cotton}{28/04/2007}{Revision 1: Initial definition}]
\modhistory[{W. D. Cotton}{02/09/2009}{Revision 1: Update definition}]
\end{history}
\clearpage

%
%%%%%%%%%%%%%%% ObitTableIDI_ARRAY_GEOMETRY Class %%%%%%%%%%%%%%%%%%%%%%%%%%%%%%%%%%%%%%%%%%%%%%%%%%%
% 
%
\ClassName[{ObitTableIDI\_ARRAY\_GEOMETRY}]
ObitTableIDI\_ARRAY\_GEOMETRY Class
\tabletitle{IDI Array geometry table for uv data}
% table name
\tablename{IDI\_ARRAY\_GEOMETRY}
\tableintro[
{This class contains tabular data and allows access.
This table is part of the IDI uv data format.
IDI\_ARRAY\_GEOMETRY contains information about the locations 
of antennas in a UV data set.
Also time information and the state of the Earth's orientation.
Only FITS cataloged data are supported.
}
\tableover{
In memory tables are stored in a fashion similar to how they are 
stored on disk - in large blocks in memory rather than structures.
Due to the word alignment requirements of some machines, they are 
stored by order of the decreasing element size: 
double, float long, int, short, char rather than the logical order.
The details of the storage in the buffer are kept in the 
ObitTableDesc.
}
% Table keyword description
\begin{keywords}
\tablekey[{"TABREV  "}{J}{tabrev}{1}{}
{Table format revision number}
]
\tablekey[{"NO\_STKD "}{J}{no\_stkd}{1}{}
{Number of Stokes parameters}
]
\tablekey[{"STK\_1   "}{J}{stk\_1}{-1}{}
{First Stokes parameter}
]
\tablekey[{"NO\_BAND "}{J}{no\_band}{1}{()}
{Number of frequency bands }
]
\tablekey[{"NO\_CHAN "}{J}{no\_chan}{1}{}
{Number of frequency channels}
]
\tablekey[{"REF\_FREQ"}{D}{ref\_freq}{1.0}{}
{Reference frequency (Hz)}
]
\tablekey[{"CHAN\_BW "}{D}{chan\_bw}{1.0}{}
{Channel bandwidth (Hz)}
]
\tablekey[{"REF\_PIXL"}{J}{ref\_pixl}{1.0}{}
{Reference frequency bin}
]
\tablekey[{"OBSCODE "}{A}{obscode}{"    "}{}
{Observation project code}
]
\tablekey[{"ARRNAM  "}{A}{ArrName}{}{}
{Array name}
]
\tablekey[{"RDATE   "}{A}{RefDate}{"YYYY-MM-DD"}{}
{Reference date as "YYYY-MM-DD"}
]
\tablekey[{"FRAME   "}{A}{frame}{"GEOCENTRIC"}{}
{Coordinate system}
]
\tablekey[{"NUMORB  "}{J}{numOrb}{0}{()}
{Number of orbital parameters}
]
\tablekey[{"FREQ    "}{D}{Freq}{1.0}{}
{Obs. Reference Frequency for subarray(Hz)}
]
\tablekey[{"TIMSYS"}{A}{TimeSys}{"IAT"}{}
{Time system, 'IAT' or 'UTC'}
]
\tablekey[{"GSTIA0 "}{D}{GSTiat0}{0.0}{}
{GST at time=0 (degrees) on the reference date}
]
\tablekey[{"DEGPDY  "}{D}{DegDay}{360.0}{}
{Earth rotation rate (deg/IAT day)}
]
\tablekey[{"UT1UTC  "}{E}{ut1Utc}{}{}
{UT1-UTC  (time sec.)  }
]
\tablekey[{"IATUTC  "}{E}{iatUtc}{0.0}{}
{data time-UTC  (time sec.)}
]
\tablekey[{"POLARX  "}{E}{PolarX}{0.0}{}
{Polar position X (meters) on ref. date}
]
\tablekey[{"POLARY  "}{E}{PolarY}{0.0}{}
{Polar position Y (meters) on ref. date}
]
\tablekey[{"ARRAYX  "}{D}{ArrayX}{0.0}{}
{Array center X coord. (meters, earth center)}
]
\tablekey[{"ARRAYY  "}{D}{ArrayY}{0.0}{}
{Array center Y coord. (meters, earth center)}
]
\tablekey[{"ARRAYZ  "}{D}{ArrayZ}{0.0}{}
{Array center Z coord. (meters, earth center)}
]
\end{keywords}
%
% Table column description
\begin{columns}
\tablecol[{"ANNAME  "}{"        " }{A}{(8)}{AntName}
{Station name}
] 
\tablecol[{"STABXYZ "}{"METERS " }{D}{(3)}{StaXYZ}
{X,Y,Z offset from array center}
] 
\tablecol[{"DERXYZ  "}{"METERS " }{E}{(3)}{derXYZ}
{look this up}
] 
\tablecol[{"ORBPARM "}{"       " }{D}{(numOrb)}{OrbParm}
{Orbital parameters.}
] 
\tablecol[{"NOSTA   "}{"       " }{I}{(1)}{noSta}
{Station number, used as an index in other tables, uv data}
] 
\tablecol[{"MNTSTA  "}{"       " }{J}{(1)}{mntSta}
{Mount type, 0=altaz, 1=equatorial, 2=orbiting, 3=dipole}
] 
\tablecol[{"STAXOF  "}{"METERS " }{E}{(3)}{staXof}
{Axis offset (in 3D???)}
] 
\tablecol[{"DIAMETER"}{"METERS " }{E}{(1)}{diameter}
{[OPTIONAL] Antenna diameter}
] 
\end{columns}
%
% Table modification history
\begin{history}
\modhistory[{W. D. Cotton}{28/04/2007}{Revision 1: Initial definition}]
\modhistory[{W. D. Cotton}{02/09/2009}{Revision 1: Update definition}]
\end{history}
\clearpage
%

%%%%%%%%%%%%%%% ObitTableIDI\_BANDPASS Class %%%%%%%%%%%%%%%%%%%%%%%%%%%%%%%%%%%%%%%%%%%%%%%%%%%
% 
%
\ClassName[{ObitTableIDI\_BANDPASS}]

\tabletitle{IDI UV data System Temperature Table}
% table name
\tablename{IDI\_BANDPASS}
\tableintro[
{This class contains tabular data and allows access.
This table is part of the IDI uv data format.
"IDI\_BANDPASS" contains bandpass calibration data.
Only FITS cataloged data are supported.
This class is derived from the ObitTable class.}
]
\tableover{
In memory tables are stored in a fashion similar to how they are 
stored on disk - in large blocks in memory rather than structures.
Due to the word alignment requirements of some machines, they are 
stored by order of the decreasing element size: 
double, float long, int, short, char rather than the logical order.
The details of the storage in the buffer are kept in the 
ObitTableDesc.
}
% Table keyword description
\begin{keywords}
\tablekey[{"TABREV  "}{J}{tabrev}{1}{}
{Table format revision number}
]
% Must be position 2 to index
\tablekey[{"NO\_POL  "}{J}{numPol}{1}{(1,2)}
{The number of polarizations}
]
\tablekey[{"NO\_BAND "}{J}{no\_band}{1}{()}
{[OPTIONAL] Number of frequency bands }
]
\tablekey[{"NO\_CHAN "}{J}{no\_chan}{1}{}
{[OPTIONAL] Number of frequency channels}
]
\tablekey[{"NO\_ANT  "}{J}{numAnt}{2}{()}
{The maximum antenna number}
]
\tablekey[{"NO\_BACH "}{J}{numBach}{1}{()}
{The number of spectral channels tabulated}
]
\tablekey[{"STRT\_CHN"}{J}{strtChn}{1}{()}
{Data channel number (1-rel) of first channel in the table}
]
\tablekey[{"NO\_STKD "}{J}{no\_stkd}{1}{}
{Number of Stokes parameters}
]
\tablekey[{"STK\_1   "}{J}{stk\_1}{-1}{}
{First Stokes parameter}
]
\tablekey[{"REF\_FREQ"}{D}{ref\_freq}{1.0}{}
{Reference frequency (Hz)}
]
\tablekey[{"CHAN\_BW "}{D}{chan\_bw}{1.0}{}
{Channel bandwidth (Hz)}
]
\tablekey[{"REF\_PIXL"}{J}{ref\_pixl}{1.0}{}
{Reference frequency bin}
]
\tablekey[{"OBSCODE "}{A}{obscode}{"    "}{}
{Observation project code}
]
\end{keywords}
%
% Table column description
\begin{columns}
\tablecol[{"TIME    "}{"DAYS   " }{D}{(1)}{Time}
{The center time}
] 
\tablecol[{"TIME\_INTERVAL"}{"DAYS    "}{E}{(1)}{TimeI}
{Time interval.}
] 
\tablecol[{"SOURCE\_ID"}{"        "}{J}{(1)}{SourID}
{Source Identifier number }
] 
\tablecol[{"ANTENNA\_NO"}{"       " }{J}{(1)}{antNo}
{Antenna number }
] 
\tablecol[{"ARRAY"}{"        "}{J}{(1)}{Array}
{Array number.}
] 
\tablecol[{"FREQID  "}{"       " }{J}{(1)}{fqid}
{Frequency ID number for row}
] 
\tablecol[{"BANDWIDTH"}{"HZ     " }{E}{(no\_band)}{bandwidth}
{Channel bandwidth per band}
] 
\tablecol[{"BAND\_FREQ"}{"HZ     " }{D}{(no\_band)}{band\_freq}
{Frequency per band}
] 
\tablecol[{"REFANT\_\#NO\_POL""}{"       " }{J}{(no\_band)}{refant}
{Frequency per band}
] 
\tablecol[{"BREAL\_\#NO\_POL"}{"        "}{E}{(numBach,no\_band)}{breal}
{Real correction for poln \# NO\_POL}
] 
\tablecol[{"BIMAG\_\#NO\_POL"}{"        "}{E}{(numBach,no\_band)}{bimag}
{Imaginary correction for poln \# NO\_POL}
] 
\end{columns}
%
% Table modification history
\begin{history}
\modhistory[{W. D. Cotton}{02/09/2009}{Revision 1: Initial definition}]
\end{history}
%
\clearpage

%%%%%%%%%%%%%%% ObitTableIDI\_CALIBRATION Class %%%%%%%%%%%%%%%%%%%%%%%%%%%%%%%%%%%%%%%%%%%%%%%%%%%
% 
%
\ClassName[{ObitTableIDI\_CALIBRATION}]

\tabletitle{IDI UV data Calibration Table}
% table name
\tablename{IDI\_CALIBRATION}
\tableintro[
{This class contains tabular data and allows access.
This table is part of the IDI uv data format.
"IDI\_CALIBRATION" contains calibration correction for uv data.
Only FITS cataloged data are supported.
This class is derived from the ObitTable class.}
]
\tableover{
In memory tables are stored in a fashion similar to how they are 
stored on disk - in large blocks in memory rather than structures.
Due to the word alignment requirements of some machines, they are 
stored by order of the decreasing element size: 
double, float long, int, short, char rather than the logical order.
The details of the storage in the buffer are kept in the 
ObitTableDesc.
}
% Table keyword description
\begin{keywords}
\tablekey[{"TABREV  "}{J}{tabrev}{1}{}
{Table format revision number}
]
% Must be position 2 to index
\tablekey[{"NO\_POL  "}{J}{numPol}{1}{(1,2)}
{The number of polarizations}
]
\tablekey[{"NO\_BAND "}{J}{no\_band}{1}{()}
{[OPTIONAL] Number of frequency bands }
]
\tablekey[{"NO\_ANT  "}{J}{numAnt}{}{()}
{The maximum antenna number}
]
\tablekey[{"NO\_STKD "}{J}{no\_stkd}{1}{}
{Number of Stokes parameters}
]
\tablekey[{"STK\_1   "}{J}{stk\_1}{-1}{}
{First Stokes parameter}
]
\tablekey[{"NO\_CHAN "}{J}{no\_chan}{1}{}
{Number of frequency channels}
]
\tablekey[{"REF\_FREQ"}{D}{ref\_freq}{1.0}{}
{Reference frequency (Hz)}
]
\tablekey[{"CHAN\_BW "}{D}{chan\_bw}{1.0}{}
{Channel bandwidth (Hz)}
]
\tablekey[{"REF\_PIXL"}{J}{ref\_pixl}{1.0}{}
{Reference frequency bin}
]
\tablekey[{"OBSCODE "}{A}{obscode}{"    "}{}
{Observation project code}
]
\end{keywords}
%
% Table column description
\begin{columns}
\tablecol[{"TIME    "}{"DAYS   " }{D}{(1)}{Time}
{The center time}
] 
\tablecol[{"TIME\_INTERVAL"}{"DAYS    "}{E}{(1)}{TimeI}
{Time interval.}
] 
\tablecol[{"SOURCE\_ID"}{"        "}{J}{(1)}{SourID}
{Source Identifier number }
] 
\tablecol[{"ANTENNA\_NO"}{"       " }{J}{(1)}{antNo}
{Antenna number }
] 
\tablecol[{"ARRAY"}{"        "}{J}{(1)}{Array}
{Array number.}
] 
\tablecol[{"FREQID  "}{"       " }{J}{(1)}{fqid}
{Frequency ID number for row, this is a random parameter in the uv data}
] 
\tablecol[{"TSYS\_\#NO\_POL"}{"K       "}{E}{(no\_band)}{TSys}
{System temperature for poln \# NO\_POL}
] 
\tablecol[{"TANT\_\#NO\_POL"}{"K       "}{E}{(no\_band)}{TAnt}
{Antenna temperature for poln \# NO\_POL}
] 
\tablecol[{"SENSITIVITY\_\#NO\_POL"}{"K/JY    "}{E}{(no\_band)}{sensitivity}
{Sensitivity for poln \# NO\_POL}
] 
\tablecol[{"PHASE\_\#NO\_POL"}{"RAD     "}{E}{(no\_band)}{phase}
{Phase for poln \# NO\_POL}
] 
\tablecol[{"RATE\_\#NO\_POL"}{"SEC/SEC "}{E}{(no\_band)}{rate}
{Rate of change of delay for poln \# NO\_POL}
] 
\tablecol[{"DELAY\_\#NO\_POL"}{"SEC     "}{E}{(no\_band)}{delay}
{Delay for poln \# NO\_POL}
] 
\tablecol[{"REAL\_\#NO\_POL"}{"        "}{E}{(no\_band)}{real}
{Real part of gain for poln \# NO\_POL}
] 
\tablecol[{"IMAG\_\#NO\_POL"}{"        "}{E}{(no\_band)}{imag}
{Imaginary part of gain for poln \# NO\_POL}
] 
\tablecol[{"WEIGHT\_\#NO\_POL"}{"        "}{E}{(no\_band)}{weight}
{Weight for poln \# NO\_POL}
] 
\tablecol[{"REFANT\_\#NO\_POL"}{"        "}{J}{(no\_band)}{refant}
{Reference antenna for poln \# NO\_POL}
] 
\end{columns}
%
% Table modification history
\begin{history}
\modhistory[{W. D. Cotton}{02/09/2009}{Revision 1: Initial definition}]
\end{history}
%
\clearpage

%%%%%%%%%%%%%%% ObitTableIDI\_FLAG Class %%%%%%%%%%%%%%%%%%%%%%%%%%%%%%%%%%%%%%%%%%%%%%%%%%%
% 
%
\ClassName[{ObitTableIDI\_FLAG}]

\tabletitle{IDI UV data Flag Table}
% table name
\tablename{IDI\_FLAG}
\tableintro[
{This class contains tabular data and allows access.
This table is part of the IDI uv data format.
"IDI\_FLAG" contains descriptions of data to be ignored in uv data.
An ObitTableIDI\_FLAG is the front end to a persistent disk resident structure.
Only FITS cataloged data are supported.
This class is derived from the ObitTable class.}
]
\tableover{
In memory tables are stored in a fashion similar to how they are 
stored on disk - in large blocks in memory rather than structures.
Due to the word alignment requirements of some machines, they are 
stored by order of the decreasing element size: 
double, float long, int, short, char rather than the logical order.
The details of the storage in the buffer are kept in the 
ObitTableDesc.
}
% Table keyword description
\begin{keywords}
\tablekey[{"TABREV  "}{J}{tabrev}{1}{}
{Table format revision number}
]
\tablekey[{"NO\_BAND "}{J}{no\_band}{1}{()}
{[OPTIONAL] Number of frequency bands }
]
\tablekey[{"NO\_STKD "}{J}{no\_stkd}{1}{}
{Number of Stokes parameters}
]
\tablekey[{"STK\_1   "}{J}{stk\_1}{-1}{}
{First Stokes parameter}
]
\tablekey[{"NO\_CHAN "}{J}{no\_chan}{1}{}
{Number of frequency channels}
]
\tablekey[{"REF\_FREQ"}{D}{ref\_freq}{1.0}{}
{Reference frequency (Hz)}
]
\tablekey[{"CHAN\_BW "}{D}{chan\_bw}{1.0}{}
{Channel bandwidth (Hz)}
]
\tablekey[{"REF\_PIXL"}{J}{ref\_pixl}{1.0}{}
{Reference frequency bin}
]
\tablekey[{"OBSCODE "}{A}{obscode}{"    "}{}
{Observation project code}
]
\end{keywords}
%
% Table column description
\begin{columns}
\tablecol[{"SOURCE\_ID"}{"        "}{J}{(1)}{SourID}
{Source Identifier number }
] 
\tablecol[{"ARRAY"}{"        "}{J}{(1)}{Array}
{Array number.}
] 
\tablecol[{"ANTS"}{"       " }{J}{(2)}{ants}
{Antenna numbers of baseline, 0 means any}
] 
\tablecol[{"FREQID  "}{"       " }{J}{(1)}{fqid}
{Frequency ID number for row, this is a random parameter in the uv data}
] 
\tablecol[{"TIMERANG"}{"DAY    " }{E}{(2)}{timerange}
{Beginning and end times}
] 
\tablecol[{"BANDS"}{"       " }{J}{(no\_band)}{bands}
{Band flag, not 0 means corresponding band flagged}
] 
\tablecol[{"CHANS"}{"       " }{J}{(2)}{chans}
{Lowest and highest (1-rel) spectral channel numbers}
] 
\tablecol[{"PFLAGS"}{"       " }{J}{(4)}{pflags}
{Polarization flags, non 0 means corresponding Stokes is flagged}
] 
\tablecol[{"REASON"}{"       " }{A}{(40)}{reason}
{Reason for flagging}
] 
\tablecol[{"SEVERITY"}{"       " }{J}{(4)}{severity}
{Severity, -1:no level, 0:known bad, 1:probably bad, 2:may be bad }
] 
\end{columns}
%
% Table modification history
\begin{history}
\modhistory[{W. D. Cotton}{02/09/2009}{Revision 1: Initial definition}]
\end{history}
%
\clearpage

%%%%%%%%%%%%%%% ObitTableIDI\_FREQUENCY Class %%%%%%%%%%%%%%%%%%%%%%%%%%%%%%%%%%%%%%%%%%%%%%%%%%%
% 
%
\ClassName[{ObitTableIDI\_FREQUENCY}]

\tabletitle{IDI UV data Frequency Table}
% table name
\tablename{IDI\_FREQUENCY}
\tableintro[
{This class contains tabular data and allows access.
This table is part of the IDI uv data format.
"IDI\_FREQUENCY" contains frequency related information for ``Bands'' in uv
data.
A ``Band'' is a construct that allows sets of arbitrarily spaced frequencies.
An ObitTableIDI\_FREQUENCY is the front end to a persistent disk resident structure.
Only FITS cataloged data are supported.
This class is derived from the ObitTable class.}
]
\tableover{
In memory tables are stored in a fashion similar to how they are 
stored on disk - in large blocks in memory rather than structures.
Due to the word alignment requirements of some machines, they are 
stored by order of the decreasing element size: 
double, float long, int, short, char rather than the logical order.
The details of the storage in the buffer are kept in the 
ObitTableDesc.
}
% Table keyword description
\begin{keywords}
\tablekey[{"TABREV  "}{J}{tabrev}{1}{}
{Table format revision number}
]
\tablekey[{"NO\_STKD "}{J}{no\_stkd}{1}{}
{Number of Stokes parameters}
]
\tablekey[{"STK\_1   "}{J}{stk\_1}{-1}{}
{First Stokes parameter}
]
\tablekey[{"NO\_BAND "}{J}{no\_band}{1}{()}
{Number of frequency bands }
]
\tablekey[{"NO\_CHAN "}{J}{no\_chan}{1}{}
{Number of frequency channels}
]
\tablekey[{"REF\_FREQ"}{D}{ref\_freq}{1.0}{}
{Reference frequency (Hz)}
]
\tablekey[{"CHAN\_BW "}{D}{chan\_bw}{1.0}{}
{Channel bandwidth (Hz)}
]
\tablekey[{"REF\_PIXL"}{J}{ref\_pixl}{1.0}{}
{Reference frequency bin}
]
\tablekey[{"OBSCODE "}{A}{obscode}{"    "}{}
{Observation project code}
]
\tablekey[{"RDATE   "}{A}{RefDate}{"YYYY-MM-DD"}{}
{Reference date as "YYYY-MM-DD"}
]
\end{keywords}
%
% Table column description
\begin{columns}
\tablecol[{"FREQID  "}{"       " }{J}{(1)}{fqid}
{Frequency ID number for row, this is a random parameter in the uv data}
] 
\tablecol[{"BANDFREQ"}{"HZ     " }{D}{(no\_band)}{bandfreq}
{Offset from reference frequency for each Band}
] 
\tablecol[{"CH\_WIDTH"}{"HZ     " }{E}{(no\_band)}{chWidth}
{Bandwidth of an individual channel, now always written and read as a signed value}
] 
\tablecol[{"TOTAL\_BANDWIDTH "}{"HZ     " }{E}{(no\_band)}{totBW}
{Total bandwidth of the IF, now written and read as an unsigned value}
] 
\tablecol[{"SIDEBAND "}{"       " }{J}{(no\_band)}{sideBand}
{Sideband of the IF (-1 =$>$ lower, +1 =$>$ upper), now always written and read as +1}
] 
%\tablecol[{"BB\_CHAN "}{"       " }{J}{(no\_band)}{bb\_chan}
%{Look this one up}
%] 
\end{columns}
%
% Table modification history
\begin{history}
\modhistory[{W. D. Cotton}{28/04/2007}{Revision 1: Initial definition}]
\modhistory[{W. D. Cotton}{02/09/2009}{Revision 1: Update definition}]
\end{history}
%
\clearpage

%%%%%%%%%%%%%%% ObitTableIDI\_GAIN\_CURVE Class %%%%%%%%%%%%%%%%%%%%%%%%%%%%%%%%%%%%%%%%%%%%%%%%%%%
% 
%
\ClassName[{ObitTableIDI\_GAIN\_CURVE}]

\tabletitle{IDI UV data Gain Curve Table}
% table name
\tablename{IDI\_GAIN\_CURVE}
\tableintro[
{This class contains tabular data and allows access.
This table is part of the IDI uv data format.
"IDI\_GAIN\_CURVE" contains antenna gain as a function of pointing position.
An ObitTableIDI\_GAIN\_CURVE is the front end to a persistent disk resident structure.
Only FITS cataloged data are supported.
This class is derived from the ObitTable class.}
]
\tableover{
In memory tables are stored in a fashion similar to how they are 
stored on disk - in large blocks in memory rather than structures.
Due to the word alignment requirements of some machines, they are 
stored by order of the decreasing element size: 
double, float long, int, short, char rather than the logical order.
The details of the storage in the buffer are kept in the 
ObitTableDesc.
}
% Table keyword description
\begin{keywords}
\tablekey[{"TABREV  "}{J}{tabrev}{1}{}
{Table format revision number}
]
% Must be position 2 to index
\tablekey[{"NO\_POL  "}{J}{numPol}{}{(1,2)}
{[OPTIONAL] The number of polarizations}
]
\tablekey[{"NO\_BAND "}{J}{no\_band}{1}{()}
{[OPTIONAL] Number of frequency bands }
]
\tablekey[{"NO\_TABS"}{J}{numTabs}{}{()}
{The number of tabulated terms}
]
\tablekey[{"NO\_CHAN "}{J}{no\_chan}{1}{}
{Number of frequency channels}
]
\tablekey[{"NO\_STKD "}{J}{no\_stkd}{1}{}
{Number of Stokes parameters}
]
\tablekey[{"STK\_1   "}{J}{stk\_1}{-1}{}
{First Stokes parameter}
]
\tablekey[{"REF\_FREQ"}{D}{ref\_freq}{1.0}{}
{Reference frequency (Hz)}
]
\tablekey[{"CHAN\_BW "}{D}{chan\_bw}{1.0}{}
{Channel bandwidth (Hz)}
]
\tablekey[{"REF\_PIXL"}{J}{ref\_pixl}{1.0}{}
{Reference frequency bin}
]
\tablekey[{"OBSCODE "}{A}{obscode}{"    "}{}
{Observation project code}
]
\end{keywords}
%
% Table column description
\begin{columns}
\tablecol[{"ANTENNA\_NO"}{"       " }{J}{(1)}{antNo}
{Antenna number }
] 
\tablecol[{"ARRAY"}{"        "}{J}{(1)}{Array}
{Array number.}
] 
\tablecol[{"FREQID  "}{"       " }{J}{(1)}{fqid}
{Frequency ID number for row, this is a random parameter in the uv data}
] 
\tablecol[{"TYPE\_\#NO\_POL"}{"       " }{J}{(no\_band)}{type}
{Gain curve type for poln \# NO\_POL}
] 
\tablecol[{"NTERM\_\#NO\_POL"}{"       " }{J}{(no\_band)}{nterm}
{Number of terms for poln \# NO\_POL}
] 
\tablecol[{"X\_TYP\_\#NO\_POL"}{"       " }{J}{(no\_band)}{x\_typ}
{X value types for poln \# NO\_POL}
] 
\tablecol[{"Y\_TYP\_\#NO\_POL"}{"       " }{J}{(no\_band)}{y\_typ}
{Y value types for poln \# NO\_POL}
] 
\tablecol[{"X\_VAL\_\#NO\_POL"}{"       " }{E}{(no\_band)}{x\_val}
{X values for poln \# NO\_POL}
] 
\tablecol[{"Y\_VAL\_\#NO\_POL"}{"       " }{E}{(numTabs,no\_band)}{y\_val}
{Y values for poln \# NO\_POL}
] 
\tablecol[{"GAIN\_\#NO\_POL"}{"       " }{E}{(numTabs,no\_band)}{gain}
{for poln \# NO\_POL}
] 
\tablecol[{"SENS\_\#NO\_POL"}{"       " }{E}{(no\_band)}{sens}
{for poln \# NO\_POL}
] 
\end{columns}
%
% Table modification history
\begin{history}
\modhistory[{W. D. Cotton}{02/09/2009}{Revision 1: Initial definition}]
\end{history}
%
\clearpage

%%%%%%%%%%%%%%% ObitTableIDI\_INTERFEROMETER_MODEL Class %%%%%%%%%%%%%%%%%%%%%%%%%%%%%%%%%%%%%%%%%%%
% 
%
\ClassName[{ObitTableIDI\_INTERFEROMETER\_MODEL}]

\tabletitle{IDI UV data Interferometer Model Table}
% table name
\tablename{IDI\_INTERFEROMETER\_MODEL}
\tableintro[
{This class contains tabular data and allows access.
This table is part of the IDI uv data format.
"IDI\_INTERFEROMETER\_MODEL" contains the correlator model used for the data.
Only FITS cataloged data are supported.
This class is derived from the ObitTable class.}
]
\tableover{
In memory tables are stored in a fashion similar to how they are 
stored on disk - in large blocks in memory rather than structures.
Due to the word alignment requirements of some machines, they are 
stored by order of the decreasing element size: 
double, float long, int, short, char rather than the logical order.
The details of the storage in the buffer are kept in the 
ObitTableDesc.
}
% Table keyword description
\begin{keywords}
\tablekey[{"TABREV  "}{J}{tabrev}{1}{}
{Table format revision number}
]
% Must be position 2 to index
\tablekey[{"NO\_POL  "}{J}{numPol}{1}{(1,2)}
{[OPTIONAL] The number of polarizations}
]
\tablekey[{"NPOLY "}{J}{npoly}{1}{()}
{Number of polynomial parameters}
]
\tablekey[{"NO\_STKD "}{J}{no\_stkd}{1}{}
{Number of Stokes parameters}
]
\tablekey[{"STK\_1   "}{J}{stk\_1}{-1}{}
{First Stokes parameter}
]
\tablekey[{"NO\_BAND "}{J}{no\_band}{1}{()}
{[OPTIONAL] Number of frequency bands }
]
\tablekey[{"NO\_CHAN "}{J}{no\_chan}{1}{}
{Number of frequency channels}
]
\tablekey[{"REF\_FREQ"}{D}{ref\_freq}{1.0}{}
{Reference frequency (Hz)}
]
\tablekey[{"CHAN\_BW "}{D}{chan\_bw}{1.0}{}
{Channel bandwidth (Hz)}
]
\tablekey[{"REF\_PIXL"}{D}{ref\_pixl}{1.0}{}
{Reference frequency bin}
]
\tablekey[{"OBSCODE "}{A}{obscode}{"    "}{}
{Observation project code}
]
\end{keywords}
%
% Table column description
\begin{columns}
\tablecol[{"TIME    "}{"DAYS   " }{D}{(1)}{Time}
{The center time.}
] 
\tablecol[{"TIME\_INTERVAL "}{"DAYS   " }{E}{(1)}{TimeI}
{Time interval of record }
] 
\tablecol[{"SOURCE\_ID  "}{"        " }{J}{(1)}{SourID}
{Source ID number}
] 
\tablecol[{"ANTENNA\_NO"}{"       " }{J}{(1)}{antennaNo}
{Antenna number}
] 
\tablecol[{"ARRAY      "}{"        " }{J}{(1)}{Array}
{Array number}
] 
\tablecol[{"FREQID"}{"        "}{J}{(1)}{FreqID}
{Frequency ID}
] 
\tablecol[{"I.FAR.ROT"}{"RAD/M**2"}{E}{(1)}{IFR}
{Ionospheric Faraday Rotation }
] 
\tablecol[{"FREQ.VAR"}{"HZ      "}{E}{(no\_band)}{FreqVar}
{FREQ.VAR (?) }
] 
\tablecol[{"PDELAY\_\#NO\_POL "}{"SECONDS "}{D}{(npoly,no\_band)}{PDelay}
{Phase delay }
] 
\tablecol[{"GDELAY\_\#NO\_POL "}{"SECONDS "}{D}{(npoly,no\_band)}{GDelay}
{Group delay }
] 
\tablecol[{"PRATE\_\#NO\_POL "}{"SECONDS "}{D}{(npoly,no\_band)}{PRate}
{Phase delay rate}
] 
\tablecol[{"GRATE\_\#NO\_POL "}{"SECONDS "}{D}{(npoly,no\_band)}{GRate}
{Group delay rate}
] 
\tablecol[{"DISP\_\#NO\_POL "}{"SECONDS "}{E}{()}{Disp}
{Dispersion}
] 
\tablecol[{"DDISP\_\#NO\_POL "}{"SEC/SEC  "}{E}{()}{DRate}
{Dispersion rate}
] 
\end{columns}
%
% Table modification history
\begin{history}
\modhistory[{W. D. Cotton}{02/09/2009}{Revision 1: Initial definition}]
\end{history}
%
\clearpage

%%%%%%%%%%%%%%% ObitTableIDI\_PHASE-CAL Class %%%%%%%%%%%%%%%%%%%%%%%%%%%%%%%%%%%%%%%%%%%
% 
%
\ClassName[{ObitTableIDI\_PHASE-CAL}]

\tabletitle{IDI UV data Interferometer Model Table}
% table name
\tablename{IDI\_PHASE-CAL}
\tableintro[
{This class contains tabular data and allows access.
This table is part of the IDI uv data format.
"IDI\_PHASE-CAL" contains instrumental phase calibration data
Only FITS cataloged data are supported.
This class is derived from the ObitTable class.}
]
\tableover{
In memory tables are stored in a fashion similar to how they are 
stored on disk - in large blocks in memory rather than structures.
Due to the word alignment requirements of some machines, they are 
stored by order of the decreasing element size: 
double, float long, int, short, char rather than the logical order.
The details of the storage in the buffer are kept in the 
ObitTableDesc.
}
% Table keyword description
\begin{keywords}
\tablekey[{"TABREV  "}{J}{tabrev}{1}{}
{Table format revision number}
]
% Must be position 2 to index
\tablekey[{"NO\_POL  "}{J}{numPol}{1}{(1,2)}
{The number of polarizations}
]
\tablekey[{"NO\_BAND "}{J}{no\_band}{1}{()}
{[OPTIONAL] Number of frequency bands }
]
\tablekey[{"NO\_TABS "}{J}{numTones}{1}{()}
{Number of tones used}
]
\tablekey[{"NO\_CHAN "}{J}{no\_chan}{1}{}
{Number of frequency channels}
]
\tablekey[{"NO\_STKD "}{J}{no\_stkd}{1}{}
{Number of Stokes parameters}
]
\tablekey[{"STK\_1   "}{J}{stk\_1}{-1}{}
{First Stokes parameter}
]
\tablekey[{"REF\_FREQ"}{D}{ref\_freq}{1.0}{}
{Reference frequency (Hz)}
]
\tablekey[{"CHAN\_BW "}{D}{chan\_bw}{1.0}{}
{Channel bandwidth (Hz)}
]
\tablekey[{"REF\_PIXL"}{J}{ref\_pixl}{1.0}{}
{Reference frequency bin}
]
\tablekey[{"OBSCODE "}{A}{obscode}{"    "}{}
{Observation project code}
]
\end{keywords}
%
% Table column description
\begin{columns}
\tablecol[{"TIME    "}{"DAYS   " }{D}{(1)}{Time}
{The center time.}
] 
\tablecol[{"TIME\_INTERVAL"}{"DAYS   " }{E}{(1)}{TimeI}
{The spanned time.}
] 
\tablecol[{"SOURCE\_ID  "}{"        " }{J}{(1)}{SourID}
{Source ID number}
] 
\tablecol[{"ANTENNA\_NO"}{"        " }{J}{(1)}{antennaNo}
{Antenna number}
]
\tablecol[{"ARRAY      "}{"        " }{J}{(1)}{Array}
{Array number}
] 
\tablecol[{"FREQID"}{"        "}{J}{(1)}{FreqID}
{Frequency ID}
] 
\tablecol[{"CABLE\_CAL"}{"SECONDS "}{D}{(1)}{CableCal}
{CABLE\_CAL }
] 
\tablecol[{"STATE\_\#NO\_POL "}{"PERCENT "}{E}{(4,no\_band)}{State}
{State counts(?)}
] 
\tablecol[{"PC\_FREQ\_\#NO\_POL "}{"HZ      "}{D}{(numTones,no\_band)}{PCFreq}
{Frequencies of the phase tones}
] 
\tablecol[{"PC\_REAL\_\#NO\_POL "}{"        "}{E}{(numTones,no\_band)}{PCReal}
{Real part of tone phase}
] 
\tablecol[{"PC\_IMAG\_\#NO\_POL "}{"        "}{E}{(numTones,no\_band)}{PCImag}
{Imaginary part of tone phase}
] 
\tablecol[{"PC\_RATE\_\#NO\_POL "}{"SEC/SEC "}{E}{(numTones,no\_band)}{PCRate}
{Tone rate phase}
] 
\end{columns}
%
% Table modification history
\begin{history}
\modhistory[{W. D. Cotton}{02/09/2009}{Revision 1: Initial definition}]
\end{history}
%
\clearpage

%%%%%%%%%%%%%%% ObitTableIDI_SOURCE Class %%%%%%%%%%%%%%%%%%%%%%%%%%%%%%%%%%%%%%%%%%%%%%%%%%%
% 
%
\ClassName[{ObitTableIDI\_SOURCE}]
ObitTableIDI\_SOURCE Class
\tabletitle{IDI Source table for UV data}
% table name
\tablename{IDI\_SOURCE}
\tableintro[
{This class contains tabular data and allows access.
This table is part of the IDI uv data format.
"IDI\_SOURCE" contains information about astronomical sources.
An ObitTableIDI\_SOURCE is the front end to a persistent disk resident structure.
Only FITS cataloged data are supported.
This class is derived from the ObitTable class. }
]
\tableover{
In memory tables are stored in a fashion similar to how they are 
stored on disk - in large blocks in memory rather than structures.
Due to the word alignment requirements of some machines, they are 
stored by order of the decreasing element size: 
double, float long, int, short, char rather than the logical order.
The details of the storage in the buffer are kept in the 
ObitTableDesc.
}
% Table keyword description
\begin{keywords}
\tablekey[{"TABREV  "}{J}{tabrev}{1}{}
{Table format revision number}
]
\tablekey[{"NO\_BAND "}{J}{no\_band}{1}{()}
{[OPTIONAL] Number of frequency bands }
]
\tablekey[{"NO\_CHAN "}{J}{no\_chan}{1}{}
{Number of frequency channels}
]
\tablekey[{"NO\_STKD "}{J}{no\_stkd}{1}{}
{Number of Stokes parameters}
]
\tablekey[{"STK\_1   "}{J}{stk\_1}{-1}{}
{First Stokes parameter}
]
\tablekey[{"REF\_FREQ"}{D}{ref\_freq}{1.0}{}
{Reference frequency (Hz)}
]
\tablekey[{"CHAN\_BW "}{D}{chan\_bw}{1.0}{}
{Channel bandwidth (Hz)}
]
\tablekey[{"REF\_PIXL"}{J}{ref\_pixl}{1.0}{}
{Reference frequency bin}
]
\tablekey[{"OBSCODE "}{A}{obscode}{"    "}{}
{Observation project code}
]
\end{keywords}
%
% Table column description
\begin{columns}
\tablecol[{"SOURCE\_ID"}{"       " }{J}{(1)}{SourID}
{[OPTIONAL] Source ID}
] 
\tablecol[{"ID\_NO."}{"       " }{J}{(1)}{VLBAID}
{[OPTIONAL] Non conforming VLBA version of Source ID}
] 
\tablecol[{"SOURCE  "}{"       " }{A}{(16)}{Source}
{Source name }
] 
\tablecol[{"QUAL    "}{"       " }{J}{(1)}{Qual}
{Source qualifier}
] 
\tablecol[{"CALCODE "}{"       " }{A}{(4)}{CalCode}
{Calibrator code}
] 
\tablecol[{"FREQID  "}{"       " }{J}{(1)}{FreqID}
{Frequency group  ID}
] 
\tablecol[{"IFLUX   "}{"JY     " }{E}{(no\_band)}{IFlux}
{Total Stokes I flux density per band}
] 
\tablecol[{"QFLUX   "}{"JY     " }{E}{(no\_band)}{QFlux}
{Total Stokes Q flux density per band}
] 
\tablecol[{"UFLUX   "}{"JY     " }{E}{(no\_band)}{UFlux}
{Total Stokes U flux density per band}
] 
\tablecol[{"VFLUX   "}{"JY     " }{E}{(no\_band)}{VFlux}
{Total Stokes V flux density per band }
] 
\tablecol[{"ALPHA   "}{"       " }{E}{(no\_band)}{alpha}
{Spectral index per IF }
] 
\tablecol[{"FREQOFF "}{"HZ     " }{E}{(no\_band)}{FreqOff}
{Frequency offset (Hz) from band nominal per band}
] 
\tablecol[{"RAEPO   "}{"DEGREES " }{D}{(1)}{RAMean}
{Right ascension at mean EPOCH (actually equinox) }
] 
\tablecol[{"DECEPO  "}{"DEGREES " }{D}{(1)}{DecMean}
{Declination at mean EPOCH (actually equinox) }
] 
\tablecol[{"EQUINOX  "}{"       " }{A}{(8)}{Equinox}
{[OPTIONAL] Mean Epoch (really equinox) for position in yr. since year 0.0}
] 
\tablecol[{ "EPOCH   "}{"       " }{A}{(8)}{VLBAEquinox}
{[OPTIONAL] VLBA screwup Mean Epoch (really equinox) for position in yr. since year 0.0}
] 
\tablecol[{"RAAPP   "}{"DEGREES " }{D}{(1)}{RAApp}
{Apparent Right ascension }
] 
\tablecol[{"DECAPP  "}{"DEGREES " }{D}{(1)}{DecApp}
{Apparent Declination}
] 
\tablecol[{"SYSVEL "}{"M/SEC    " }{D}{(no\_band)}{SysVel}
{Systemic velocity per Band (IF) }
] 
\tablecol[{"VELTYP  "}{"        " }{A}{(8)}{VelTyp}
{Velocity type}
] 
\tablecol[{"VELDEF  "}{"        " }{A}{(8)}{VelDef}
{Velocity definition 'RADIO' or 'OPTICAL'}
] 
\tablecol[{"RESTFREQ"}{"HZ     " }{D}{(no\_band)}{RestFreq}
{Line rest frequency per Band (IF) }
] 
\tablecol[{"PMRA   "}{"DEG/DAY " }{D}{(1)}{PMRa}
{Proper motion (deg/day) in RA}
] 
\tablecol[{"PMDEC  "}{"DEG/DAY " }{D}{(1)}{PMDec}
{Proper motion (deg/day) in declination}
] 
\tablecol[{"PARALLAX"}{"ARCSEC  " }{E}{(1)}{parallax}
{Parallax}
] 
\tablecol[{"EPOCH"}{"YEAR    " }{D}{(1)}{Epoch}
{[OPTIONAL]Epoch of position}
] 
\end{columns}
%
% Table modification history
\begin{history}
\modhistory[{W. D. Cotton}{28/04/2007}{Revision 1: Initial definition}]
\modhistory[{W. D. Cotton}{02/09/2009}{Revision 1: Update definition}]
\end{history}
%
%
\clearpage

%%%%%%%%%%%%%%% ObitTableIDI\_SYSTEM_TEMPERATURE Class %%%%%%%%%%%%%%%%%%%%%%%%%%%%%%%%%%%%%%%%%%%
% 
%
\ClassName[{ObitTableIDI\_SYSTEM\_TEMPERATURE}]

\tabletitle{IDI UV data System Temperature Table}
% table name
\tablename{IDI\_SYSTEM\_TEMPERATURE}
\tableintro[
{This class contains tabular data and allows access.
This table is part of the IDI uv data format.
"IDI\_SYSTEM\_TEMPERATURE" contains measured system and/or antenna temperatures
Only FITS cataloged data are supported.
This class is derived from the ObitTable class.}
]
\tableover{
In memory tables are stored in a fashion similar to how they are 
stored on disk - in large blocks in memory rather than structures.
Due to the word alignment requirements of some machines, they are 
stored by order of the decreasing element size: 
double, float long, int, short, char rather than the logical order.
The details of the storage in the buffer are kept in the 
ObitTableDesc.
}
% Table keyword description
\begin{keywords}
\tablekey[{"TABREV  "}{J}{tabrev}{1}{}
{Table format revision number}
]
% Must be position 2 to index
\tablekey[{"NO\_POL  "}{J}{numPol}{}{(1,2)}
{[OPTIONAL] The number of polarizations}
]
\tablekey[{"NO\_BAND "}{J}{no\_band}{1}{()}
{[OPTIONAL] Number of frequency bands }
]
\tablekey[{"NO\_CHAN "}{J}{no\_chan}{1}{}
{Number of frequency channels}
]
\tablekey[{"NO\_STKD "}{J}{no\_stkd}{1}{}
{Number of Stokes parameters}
]
\tablekey[{"STK\_1   "}{J}{stk\_1}{-1}{}
{First Stokes parameter}
]
\tablekey[{"REF\_FREQ"}{D}{ref\_freq}{1.0}{}
{Reference frequency (Hz)}
]
\tablekey[{"CHAN\_BW "}{D}{chan\_bw}{1.0}{}
{Channel bandwidth (Hz)}
]
\tablekey[{"REF\_PIXL"}{J}{ref\_pixl}{1.0}{}
{Reference frequency bin}
]
\tablekey[{"OBSCODE "}{A}{obscode}{"    "}{}
{Observation project code}
]
\end{keywords}
%
% Table column description
\begin{columns}
\tablecol[{"TIME    "}{"DAYS   " }{D}{(1)}{Time}
{The center time}
] 
\tablecol[{"TIME\_INTERVAL"}{"DAYS    "}{E}{(1)}{TimeI}
{Time interval.}
] 
\tablecol[{"SOURCE\_ID"}{"        "}{J}{(1)}{SourID}
{Source Identifier number }
] 
\tablecol[{"ANTENNA\_NO"}{"       " }{J}{(1)}{antNo}
{Antenna number }
] 
\tablecol[{"ARRAY"}{"        "}{J}{(1)}{Array}
{Array number.}
] 
\tablecol[{"FREQID  "}{"       " }{J}{(1)}{fqid}
{Frequency ID number for row, this is a random parameter in the uv data}
] 
\tablecol[{"TSYS\_\#NO\_POL"}{"K       "}{E}{(no\_band)}{TSys}
{System temperature for poln \# NO\_POL}
] 
\tablecol[{"TANT\_\#NO\_POL"}{"K       "}{E}{(no\_band)}{TAnt}
{Antenna temperature for poln \# NO\_POL}
] 
\end{columns}
%
% Table modification history
\begin{history}
\modhistory[{W. D. Cotton}{02/09/2009}{Revision 1: Initial definition}]
\end{history}
%
\clearpage

%%%%%%%%%%%%%%% IDI_UV_DATA Class %%%%%%%%%%%%%%%%%%%%%%%%%%%%%%%%%%%%%%%%%%%%%%%%%%%
% 
%
\ClassName[{ObitTableIDI\_UV\_DATA}]
ObitTableIDI\_UV\_DATA Class
\tabletitle{IDI UV data}
% table name
\tablename{IDI\_UV\_DATA}
\tableintro[
{This class contains tabular data and allows access.
This table is part of the IDI uv data format.
An IDI\_UV\_DATA table radio interferometer observational data.}
]
\tableover{
In memory tables are stored in a fashion similar to how they are 
stored on disk - in large blocks in memory rather than structures.
Due to the word alignment requirements of some machines, they are 
stored by order of the decreasing element size: 
double, float long, int, short, char rather than the logical order.
The details of the storage in the buffer are kept in the 
ObitTableDesc.
}
% Table keyword description
\begin{keywords}
\tablekey[{"TABREV  "}{J}{tabrev}{1}{}
{Table format revision number}
]
\tablekey[{"NO\_STKD "}{J}{no\_stkd}{1}{}
{Number of Stokes parameters}
]
\tablekey[{"STK\_1   "}{J}{stk\_1}{-1}{}
{First Stokes parameter}
]
\tablekey[{"NO\_BAND "}{J}{no\_band}{1}{()}
{Number of frequency bands (IF)}
]
\tablekey[{"NO\_CHAN "}{J}{no\_chan}{1}{}
{Number of frequency channels}
]
\tablekey[{"REF\_FREQ"}{D}{ref\_freq}{1.0}{}
{Reference frequency (Hz)}
]
\tablekey[{"CHAN\_BW "}{D}{chan\_bw}{1.0}{}
{Channel bandwidth (Hz)}
]
\tablekey[{"REF\_PIXL"}{J}{ref\_pixl}{1.0}{}
{Reference frequency bin}
]
\tablekey[{"OBSCODE "}{A}{obscode}{"    "}{}
{Observation project code}
]
\tablekey[{"RDATE   "}{A}{RefDate}{"YYYY-MM-DD"}{}
{Reference date as "YYYY-MM-DD"}
]
\tablekey[{"NMATRIX "}{J}{nmatrix}{1}{}
{Number of UV data matricies}
]
\tablekey[{"MAXIS   "}{J}{maxis}{1}{}
{Number of UV data matrix axes}
]
\tablekey[{"EQUINOX   "}{A}{Equinox}{"J2000   "}{}
{[OPTIONAL] System and mean equinox of coordinates}
]
\tablekey[{"WEIGHTYP "}{A}{WeighTyp}{"NORMAL"}{}
{[OPTIONAL] Type of data weights}
]
\tablekey[{"TMATX01"}{L}{tmatx01}{FALSE}{}
{[OPTIONAL] If TRUE col. 1 has the visibility matrix}
]
\tablekey[{"TMATX02"}{L}{tmatx02}{FALSE}{}
{[OPTIONAL] If TRUE col. 2 has the visibility matrix}
]
\tablekey[{"TMATX03"}{L}{tmatx03}{FALSE}{}
{[OPTIONAL] If TRUE col. 3 has the visibility matrix}
]
\tablekey[{"TMATX04"}{L}{tmatx04}{FALSE}{}
{[OPTIONAL] If TRUE col. 4 has the visibility matrix}
]
\tablekey[{"TMATX05"}{L}{tmatx05}{FALSE}{}
{[OPTIONAL] If TRUE col. 5 has the visibility matrix}
]
\tablekey[{"TMATX06"}{L}{tmatx06}{FALSE}{}
{[OPTIONAL] If TRUE col. 6 has the visibility matrix}
]
\tablekey[{"TMATX07"}{L}{tmatx07}{FALSE}{}
{[OPTIONAL] If TRUE col. 7 has the visibility matrix}
]
\tablekey[{"TMATX08"}{L}{tmatx08}{FALSE}{}
{[OPTIONAL] If TRUE col. 8 has the visibility matrix}
]
\tablekey[{"TMATX09"}{L}{tmatx09}{FALSE}{}
{[OPTIONAL] If TRUE col. 9 has the visibility matrix}
]
\tablekey[{"TMATX10"}{L}{tmatx10}{FALSE}{}
{[OPTIONAL] If TRUE col. 10 has the visibility matrix}
]
\tablekey[{"TMATX11"}{L}{tmatx11}{FALSE}{}
{[OPTIONAL] If TRUE col. 11 has the visibility matrix}
]
\tablekey[{"TMATX12"}{L}{tmatx12}{FALSE}{}
{[OPTIONAL] If TRUE col. 12 has the visibility matrix}
]
\tablekey[{"TMATX13"}{L}{tmatx13}{FALSE}{}
{[OPTIONAL] If TRUE col. 13 has the visibility matrix}
]
\tablekey[{"TMATX14"}{L}{tmatx14}{FALSE}{}
{[OPTIONAL] If TRUE col. 14 has the visibility matrix}
]
\tablekey[{"TMATX15"}{L}{tmatx15}{FALSE}{}
{[OPTIONAL] If TRUE col. 15 has the visibility matrix}
]
\tablekey[{"MAXIS1"}{J}{maxis1}{1}{()}
{[OPTIONAL] Number of pixels on axis 1 }
]
\tablekey[{"CTYPE1"}{A}{ctype1}{"    "}{}
{[OPTIONAL] Type of data axis 1 }
]
\tablekey[{"CDELT1"}{D}{cdelt1}{0.0}{}
{[OPTIONAL] Increment in coordinate between pixels on axis 1 }
]
\tablekey[{"CRPIX1"}{E}{crpix1}{1.0}{}
{[OPTIONAL] Reference pixel on axis 1  }
]
\tablekey[{"CRVAL1"}{D}{crval1}{0.0}{}
{[OPTIONAL] Reference coordinate on axis 1 }
]
\tablekey[{"MAXIS2"}{J}{maxis2}{1}{()}
{[OPTIONAL] Number of pixels on axis 2 }
]
\tablekey[{"CTYPE2"}{A}{ctype2}{"    "}{}
{[OPTIONAL] Type of data axis 2 }
]
\tablekey[{"CDELT2"}{D}{cdelt2}{0.0}{}
{[OPTIONAL] Increment in coordinate between pixels on axis 2 }
]
\tablekey[{"CRPIX2"}{E}{crpix2}{1.0}{}
{[OPTIONAL] Reference pixel on axis 2  }
]
\tablekey[{"CRVAL2"}{D}{crval2}{0.0}{}
{[OPTIONAL] Reference coordinate on axis 2 }
]
\tablekey[{"MAXIS3"}{J}{maxis3}{1}{()}
{[OPTIONAL] Number of pixels on axis 3 }
]
\tablekey[{"CTYPE3"}{A}{ctype3}{"    "}{}
{[OPTIONAL] Type of data axis 3 }
]
\tablekey[{"CDELT3"}{D}{cdelt3}{0.0}{}
{[OPTIONAL] Increment in coordinate between pixels on axis 3 }
]
\tablekey[{"CRPIX3"}{E}{crpix3}{1.0}{}
{[OPTIONAL] Reference pixel on axis 3  }
]
\tablekey[{"CRVAL3"}{D}{crval3}{0.0}{}
{[OPTIONAL] Reference coordinate on axis 3 }
]
\tablekey[{"MAXIS4"}{J}{maxis4}{1}{()}
{[OPTIONAL] Number of pixels on axis 4 }
]
\tablekey[{"CTYPE4"}{A}{ctype4}{"    "}{}
{[OPTIONAL] Type of data axis 4 }
]
\tablekey[{"CDELT4"}{D}{cdelt4}{0.0}{}
{[OPTIONAL] Increment in coordinate between pixels on axis 4 }
]
\tablekey[{"CRPIX4"}{E}{crpix4}{1.0}{}
{[OPTIONAL] Reference pixel on axis 4  }
]
\tablekey[{"CRVAL4"}{D}{crval4}{0.0}{}
{[OPTIONAL] Reference coordinate on axis 4 }
]
\tablekey[{"MAXIS5"}{J}{maxis5}{1}{()}
{[OPTIONAL] Number of pixels on axis 5 }
]
\tablekey[{"CTYPE5"}{A}{ctype5}{"    "}{}
{[OPTIONAL] Type of data axis 5 }
]
\tablekey[{"CDELT5"}{D}{cdelt5}{0.0}{}
{[OPTIONAL] Increment in coordinate between pixels on axis 5 }
]
\tablekey[{"CRPIX5"}{E}{crpix5}{1.0}{}
{[OPTIONAL] Reference pixel on axis 5  }
]
\tablekey[{"CRVAL5"}{D}{crval5}{0.0}{}
{[OPTIONAL] Reference coordinate on axis 5 }
]
\tablekey[{"MAXIS6"}{J}{maxis6}{1}{}
{[OPTIONAL] Number of pixels on axis 6 }
]
\tablekey[{"CTYPE6"}{A}{ctype6}{"    "}{}
{[OPTIONAL] Type of data axis 6 }
]
\tablekey[{"CDELT6"}{D}{cdelt6}{0.0}{}
{[OPTIONAL] Increment in coordinate between pixels on axis 6 }
]
\tablekey[{"CRPIX6"}{E}{crpix6}{1.0}{}
{[OPTIONAL] Reference pixel on axis 6  }
]
\tablekey[{"CRVAL6"}{D}{crval6}{0.0}{}
{[OPTIONAL] Reference coordinate on axis 6 }
]
\tablekey[{"MAXIS7"}{J}{maxis7}{1}{}
{[OPTIONAL] Number of pixels on axis 7 }
]
\tablekey[{"CTYPE7"}{A}{ctype7}{"    "}{}
{[OPTIONAL] Type of data axis 7 }
]
\tablekey[{"CDELT7"}{D}{cdelt7}{0.0}{}
{[OPTIONAL] Increment in coordinate between pixels on axis 7 }
]
\tablekey[{"CRPIX7"}{E}{crpix7}{1.0}{}
{[OPTIONAL] Reference pixel on axis 7  }
]
\tablekey[{"CRVAL7"}{D}{crval7}{0.0}{}
{[OPTIONAL] Reference coordinate on axis 7 }
]
\tablekey[{"DATE-OBS"}{A}{dateObs}{"0000-00-00"}{}
{[OPTIONAL] Observing date as YYYY-MM-DD}
]
\tablekey[{"TELESCOP"}{A}{teles}{}{}
{[OPTIONAL] Telescope used}
]
\tablekey[{"OBSERVER"}{A}{observer}{}{}
{[OPTIONAL] Observer making the observations}
]
\tablekey[{"VIS\_SCAL"}{E}{visScale}{1.0}{}
{[OPTIONAL] Scaling factor for visibilities}
]
\tablekey[{"SORT    "}{A}{sort}{"  "}{}
{[OPTIONAL] Sort order of data}
]
\end{keywords}
%
% Table column description
\begin{columns}
\tablecol[{"UU      "}{"SECONDS " }{E}{(1)}{uu}
{[OPTIONAL] U component of baseline. Unspecified projection}
] 
\tablecol[{"VV      "}{"SECONDS " }{E}{(1)}{vv}
{[OPTIONAL] V component of baseline. Unspecified projection}
] 
\tablecol[{"WW      "}{"SECONDS " }{E}{(1)}{ww}
{[OPTIONAL] W component of baseline. Unspecified projection}
] 
\tablecol[{"UU---SIN"}{"SECONDS " }{E}{(1)}{uusin}
{[OPTIONAL] U component of baseline. Sine projection}
] 
\tablecol[{"VV---SIN"}{"SECONDS " }{E}{(1)}{vvsin}
{[OPTIONAL] V component of baseline. Sine projection}
] 
\tablecol[{"WW---SIN"}{"SECONDS " }{E}{(1)}{wwsin}
{[OPTIONAL] W component of baseline. Sine projection}
] 
\tablecol[{"UU---NCP"}{"SECONDS " }{E}{(1)}{uuncp}
{[OPTIONAL] U component of baseline. NCP projection}
] 
\tablecol[{"VV---NCP"}{"SECONDS " }{E}{(1)}{vvncp}
{[OPTIONAL] V component of baseline. NCP projection}
] 
\tablecol[{"WW---NCP"}{"SECONDS " }{E}{(1)}{wwncp}
{[OPTIONAL] W component of baseline. NCP projection}
] 
\tablecol[{"DATE    "}{"DAYS    " }{D}{(1)}{date}
{Julian Date}
] 
\tablecol[{"TIME    "}{"DAYS   " }{D}{(1)}{Time}
{The center time.}
] 
\tablecol[{"BASELINE"}{"        " }{J}{(1)}{Baseline}
{Interferometer baseline as ant 1*256+ant1 (1-rel).}
] 
\tablecol[{"ARRAY   "}{"        " }{J}{(1)}{Array}
{[OPTIONAL] Array number (1-rel)}
] 
\tablecol[{"FILTER  "}{"        " }{J}{(1)}{Filter}
{[OPTIONAL]  VLBA Filter number.}
] 
\tablecol[{"SOURCE\_ID"}{"        " }{J}{(1)}{Source}
{[OPTIONAL] Source ID, index into SOURCE table.}
] 
\tablecol[{"SOURCE"}{"        " }{J}{(1)}{VLBASource}
{[OPTIONAL] Hack for nonstandard VLBA usage Source ID, index into SOURCE table.}
] 
\tablecol[{"FREQID  "}{"        " }{J}{(1)}{FreqID}
{[OPTIONAL] Frequency group ID.}
] 
\tablecol[{"INTTIM"}{"SECONDS " }{E}{(1)}{IntTim}
{[OPTIONAL] The integration time.}
] 
\tablecol[{"WEIGHT"}{"       " }{E}{(maxis2,no\_band)}{Weight}
{Weight per IF/poln.}
] 
\tablecol[{"GATEID  "}{"       " }{J}{(1)}{GateID}
{[OPTIONAL] VLBA Pulsar gate ID.}
] 
\tablecol[{"FLUX    "}{"        " }{E}{(maxis1,maxis2,maxis3,maxis4,maxis5)}{Flux}
{Visibility matrix}
] 
\end{columns}
%
% Table modification history
\begin{history}
\modhistory[{W. D. Cotton}{28/04/2007}{Revision 1: Initial definition}]
\modhistory[{W. D. Cotton}{02/09/2009}{Revision 1: Update definition}]
\end{history}
%

\clearpage

%%%%%%%%%%%%%%% ObitTableIDI\_WEATHER Class %%%%%%%%%%%%%%%%%%%%%%%%%%%%%%%%%%%%%%%%%%%%%%%%%%%
% 
%
\ClassName[{ObitTableIDI\_WEATHER}]

\tabletitle{IDI UV data weather Table}
% table name
\tablename{IDI\_WEATHER}
\tableintro[
{This class contains tabular data and allows access.
This table is part of the IDI uv data format.
"IDI\_WEATHER" contains weather data.
An ObitTableIDI\_WEATHER is the front end to a persistent disk resident structure.
Only FITS cataloged data are supported.
This class is derived from the ObitTable class.}
]
\tableover{
In memory tables are stored in a fashion similar to how they are 
stored on disk - in large blocks in memory rather than structures.
Due to the word alignment requirements of some machines, they are 
stored by order of the decreasing element size: 
double, float long, int, short, char rather than the logical order.
The details of the storage in the buffer are kept in the 
ObitTableDesc.
}
% Table keyword description
\begin{keywords}
\tablekey[{"TABREV  "}{J}{tabrev}{2}{}
{Table format revision number}
]
\tablekey[{"RDATE   "}{A}{RefDate}{"YYYY-MM-DD"}{}
{Reference date as "YYYY-MM-DD"}
]
\tablekey[{"NO\_BAND "}{J}{no\_band}{1}{()}
{Number of frequency bands }
]
\tablekey[{"NO\_CHAN "}{J}{no\_chan}{1}{}
{Number of frequency channels}
]
\tablekey[{"NO\_STKD "}{J}{no\_stkd}{1}{}
{Number of Stokes parameters}
]
\tablekey[{"STK\_1   "}{J}{stk\_1}{-1}{}
{First Stokes parameter}
]
\tablekey[{"REF\_FREQ"}{D}{ref\_freq}{1.0}{}
{Reference frequency (Hz)}
]
\tablekey[{"CHAN\_BW "}{D}{chan\_bw}{1.0}{}
{Channel bandwidth (Hz)}
]
\tablekey[{"REF\_PIXL"}{J}{ref\_pixl}{1.0}{}
{Reference frequency bin}
]
\tablekey[{"OBSCODE "}{A}{obscode}{"    "}{}
{Observation project code}
]
\end{keywords}
%
% Table column description
\begin{columns}
\tablecol[{"TIME    "}{"DAYS   " }{D}{(1)}{Time}
{The center time}
] 
\tablecol[{"TIME\_INTERVAL"}{"DAYS    "}{E}{(1)}{TimeI}
{Time interval.}
] 
\tablecol[{"ANTENNA\_NO"}{"       " }{J}{(1)}{antNo}
{Antenna number }
] 
\tablecol[{"TEMPERATURE"}{"DEG C   "}{E}{(1)}{temperature}
{Air temperature }
] 
\tablecol[{"PRESSURE"}{"MILLIBAR   "}{E}{(1)}{pressure}
{Air pressure.}
] 
\tablecol[{"DEWPOINT"}{"DEG C  " }{E}{(1)}{dewpoint}
{Dew point}
] 
\tablecol[{"WIND\_VELOCITY"}{"M/S    " }{E}{(1)}{wind\_velocity}
{Wind velocity}
] 
\tablecol[{"WIND\_DIRECTION"}{"DEC    " }{E}{(1)}{wind\_direction}
{Wind direction, east from north}
] 
\tablecol[{"WVR\_H2O"}{"M**-2    " }{E}{(1)}{wvr\_h2o}
{Water column density}
] 
\tablecol[{"IONOS\_ELECTRON"}{"M**-2    " }{E}{(1)}{ionos\_electron}
{Electron column density}
] 
\end{columns}
%
% Table modification history
\begin{history}
\modhistory[{W. D. Cotton}{02/09/2009}{Revision 1: Initial definition}]
\end{history}
%
\clearpage
%%%%%%%%%%%%%%% ObitTableFS Class %%%%%%%%%%%%%%%%%%%%%%%%%%%%%%%%%%%%%%%%%%%%%%%%%%%
% 
\ClassName[{ObitTableFS}]
ObitTableFS Class 
\tabletitle{Spectral source catalog file}
% table name
\tablename{FS}
\tableintro[
{This class contains tabular data and allows access.
"AIPS FS" contains a catalog of sources allowing spectral information.
The table contains an index in the INDEX?? keywords that give the
first table row number (1-rel) for a given hour of RA.
An ObitTableFS is the front end to a persistent disk resident structure.
Both FITS and AIPS cataloged data are supported.
This class is derived from the ObitTable class. }
]
\tableover{
In memory tables are stored in a fashion similar to how they are 
stored on disk - in large blocks in memory rather than structures.
Due to the word alignment requirements of some machines, they are 
stored by order of the decreasing element size: 
double, float long, int, short, char rather than the logical order.
The details of the storage in the buffer are kept in the 
ObitTableDesc.
}
% Table keyword description
\begin{keywords}
\tablekey[{"REVISION"}{J}{revision}{1}{}
{Revision number of the table definition}
]
\tablekey[{"NO\_CH"}{J}{numCh}{}{()}
{The number of spectral channels}
]
\tablekey[{"BM\_MAJOR"}{E}{BeamMajor}{}{}
{Restoring beam major axis in deg.}
]
\tablekey[{"BM\_MINOR"}{E}{BeamMinor}{}{}
{Restoring beam minor axis in deg.}
]
\tablekey[{"BM\_PA"}{E}{BeamPA}{}{}
{Restoring beam position angle of major axis in deg.}
]
\tablekey[{"VEL\_REF"}{E}{VelRef}{}{}
{Velocity at reference channel (m/s).}
]
\tablekey[{"VEL\_RPIX"}{E}{VelRPix}{}{}
{Velocity reference channel (1=rel).}
]
\tablekey[{"VEL\_DEL"}{E}{VelDelt}{}{}
{Velocity channel width (m/s).}
]
\tablekey[{"SORTORT"}{J}{SortOrder}{}{}
{Column number for sort (neg -> descending)}
]
\tablekey[{"NUM\_INDE"}{J}{numIndexed}{1}{}
{Number of rows in table when indexed}
]
\tablekey[{"INDEX00"}{J}{index00}{1}{}
{First entry for RA=00 h}
]
\tablekey[{"INDEX01"}{J}{index01}{1}{}
{First entry for RA=01 h}
]
\tablekey[{"INDEX02"}{J}{index02}1{}{}
{First entry for RA=02 h}
]
\tablekey[{"INDEX03"}{J}{index03}{1}{}
{First entry for RA=03 h}
]
\tablekey[{"INDEX04"}{J}{index04}{1}{}
{First entry for RA=04 h}
]
\tablekey[{"INDEX05"}{J}{index05}{1}{}
{First entry for RA=05 h}
]
\tablekey[{"INDEX06"}{J}{index06}{1}{}
{First entry for RA=06 h}
]
\tablekey[{"INDEX07"}{J}{index07}{1}{}
{First entry for RA=07 h}
]
\tablekey[{"INDEX08"}{J}{index08}{1}{}
{First entry for RA=08 h}
]
\tablekey[{"INDEX09"}{J}{index09}{1}{}
{First entry for RA=09 h}
]
\tablekey[{"INDEX10"}{J}{index10}{1}{}
{First entry for RA=10 h}
]
\tablekey[{"INDEX11"}{J}{index11}{1}{}
{First entry for RA=11 h}
]
\tablekey[{"INDEX12"}{J}{index12}{1}{}
{First entry for RA=12 h}
]
\tablekey[{"INDEX13"}{J}{index13}{1}{}
{First entry for RA=13 h}
]
\tablekey[{"INDEX14"}{J}{index14}{1}{}
{First entry for RA=14 h}
]
\tablekey[{"INDEX15"}{J}{index15}{1}{}
{First entry for RA=15 h}
]
\tablekey[{"INDEX16"}{J}{index16}{1}{}
{First entry for RA=16 h}
]
\tablekey[{"INDEX17"}{J}{index17}{1}{}
{First entry for RA=17 h}
]
\tablekey[{"INDEX18"}{J}{index18}{1}{}
{First entry for RA=18 h}
]
\tablekey[{"INDEX19"}{J}{index19}{1}{}
{First entry for RA=19 h}
]
\tablekey[{"INDEX20"}{J}{index20}{1}{}
{First entry for RA=20 h}
]
\tablekey[{"INDEX21"}{J}{index21}{1}{}
{First entry for RA=21 h}
]
\tablekey[{"INDEX22"}{J}{index22}{1}{}
{First entry for RA=22 h}
]
\tablekey[{"INDEX23"}{J}{index23}{1}{}
{First entry for RA=23 h}
]
\end{keywords}
%
% Table column description
\begin{columns}
%] 
\tablecol[{"RA(2000)"}{"DEGREE  " }{D}{(1)}{Ra2000}
{RA (J2000)}
] 
\tablecol[{"DEC(2000)"}{"DEGREE  " }{D}{(1)}{Dec2000}
{Dec (J2000)}
] 
\tablecol[{"VELOCITY"}{"M/S     " }{D}{(1)}{Velocity}
{Overall LSR velocity}
] 
\tablecol[{"VEL WID "}{"M/S     " }{E}{(1)}{VelWidth}
{Velocity width}
] 
\tablecol[{"PEAK INT  "}{"JY/BEAM " }{E}{(1)}{PeakInt}
{Peak Ipol }
] 
\tablecol[{"MAJOR AX"}{"DEGREE  " }{E}{(1)}{MajorAxis}
{Fitted major axis size}
] 
\tablecol[{"MINOR AX"}{"DEGREE  " }{E}{(1)}{MinorAxis}
{Fitted minor axis siz}
] 
\tablecol[{"POSANGLE"}{"DEGREE  " }{E}{(1)}{PosAngle}
{Fitted PA}
] 
\tablecol[{"Q CENTER"}{"JY/BEAM " }{E}{(1)}{QCenter}
{Center Q flux density}
] 
\tablecol[{"U CENTER"}{"JY/BEAM " }{E}{(1)}{UCenter}
{enter U flux density}
] 
\tablecol[{"P FLUX"}{"JY      " }{E}{(1)}{PFlux}
{Integrated polarized flux density}
] 
\tablecol[{"I RMS "}{"JY/BEAM " }{E}{(1)}{IRMS}
{Ipol RMS uncertainty }
] 
\tablecol[{"POL RMS "}{"JY/BEAM " }{E}{(1)}{PolRMS}
{RMS (sigma) in Qpol and Upol}
] 
\tablecol[{"RES RMS"}{"JY/BEAM " }{E}{(1)}{ResRMS}
{RMS of Ipol residual}
] 
\tablecol[{"RES PEAK"}{"JY/BEAM " }{E}{(1)}{ResPeak}
{Peak in Ipol residual}
] 
\tablecol[{"RES FLUX"}{"JY      " }{E}{(1)}{ResFlux}
{Integrated Ipol residual}
] 
\tablecol[{"CENTER X"}{"PIXEL   " }{E}{(1)}{CenterX}
{Center x position in pixels in FIELD}
] 
\tablecol[{"CENTER Y"}{"PIXEL   " }{E}{(1)}{CenterY}
{Center y position in pixels in FIELD}
] 
\tablecol[{"CENTER Z"}{"PIXEL   " }{E}{(1)}{CenterZ}
{Center velocity position in pixels in FIELD}
] 
\tablecol[{"SPECTRUM"}{"JY/BEAM " }{E}{(numCh)}{Spectrum}
{Spectrum at Center X, Center Y}
] 
\tablecol[{"RMSCH"}{"JY/BEAM " }{E}{(numCh)}{RMSCh}
{RMS in each channel}
] 
\tablecol[{"PEAKCH"}{"JY/BEAM " }{E}{(numCh)}{PEAKCh}
{Peak in each channel}
] 
\tablecol[{"FIELD  "}{"       " }{A}{(8)}{Field}
{Name of survey field}
] 
\tablecol[{"JD PROCESSED"}{"DAYS    " }{J}{(1)}{JDProcess}
{Julian date on which entry was derived from image.}
] 
\end{columns}
%
% Table modification history
\begin{history}
\modhistory[{W. D. Cotton}{04/02/2012}{Revision 1: Make up}]
\modhistory[{W. D. Cotton}{10/02/2014}{Revision 2: Add velocity axis labeling}]
\end{history}
%
\clearpage

\begin{references}
\reference{Cotton, W.~D., Tody, D., and Pence, W.~D.\ 1995, \aaps, 113,
159--166.}
\reference{Flatters, C., 1998, AIPS Memo No, 102, NRAO.}
\reference{Wells, D.~C., Greisen, E.~W., and Harten, R.~H.\ 1981, \aaps, 44, 363.}
\end{references} 

\clearpage
\end{document}
\clearpage
%%%%%%%%%%%%%%% ObitTableXX Class %%%%%%%%%%%%%%%%%%%%%%%%%%%%%%%%%%%%%%%%%%%%%%%%%%%
% 
%
\ClassName[{ObitTableXX}]
ObitTableXX Class
\tabletitle{Template ObitTable document}
% table name
\tablename{XX}
\tableintro[
{This class contains tabular data and allows access.
"AIPS XX" contains highly secret information.
An ObitTableXX is the front end to a persistent disk resident structure.
Both FITS and AIPS cataloged data are supported.
This class is derived from the ObitTable class. }
]
\tableover{
In memory tables are stored in a fashion similar to how they are 
stored on disk - in large blocks in memory rather than structures.
Due to the word alignment requirements of some machines, they are 
stored by order of the decreasing element size: 
double, float long, int, short, char rather than the logical order.
The details of the storage in the buffer are kept in the 
ObitTableDesc.
}
% Table keyword description
\begin{keywords}
\tablekey[{"REVISION"}{J}{revision}{1}{}
{Revision number of the table definition.}
]
\tablekey[{"NO\_SECRET"}{J}{numSecret}{}{(1,2)}
{The number of secrets.}
]
\end{keywords}
%
% Table column description
\begin{columns}
\tablecol[{"TIME    "}{"DAYS   " }{D}{(1)}{Time}
{The center time of the secret.}
] 
\end{columns}
%
% Table modification history
\begin{history}
\modhistory[{A. N. Author}{99/99/9999}{Revision 1: Copied from AIPS}]
\end{history}
%
