% $Id$ 
%
%  OBIT class definitions
%
%\def\section #1.#2.{\medskip\leftline{\bf #1. #2.}\smallskip}
%\def\bfitem #1#2{{{\bf (#1)}}{\it #2}}
\def\bfi #1#2{{\bf #1:}{ #2}\par}
\def\extname #1#2{ {\bf 1} Extension Name {\it #2}\smallskip}
%\def\tabname #1#2{\bfitem 1{#1 Table Name #2}}
\def\keyword {\leftline{\bf Keywords:}}
%\def\table {\leftline{\bf Table Definition:}}
\documentclass[11pt]{article}
\usepackage{OBITdoc}
\begin{document}
\setcounter{section}{0}
%  Title page

%\centerline{\ttlfont PRELIMINARY}
\vskip 5cm
\centerline{\ttlfont OBIT }
\vskip 1cm
\centerline{\ttlfont Software for the Recently Deceased}
\vskip 3cm
\centerline{\secfont Draft version: 1.0 \today}
\vskip 1cm
\centerline{\secfont W. D. Cotton}
\clearpage

% Table Of Contents
\tableofcontents
\cleardoublepage

\section {Introduction}
OBIT is a software library package intended for astronomical,
especially radio--astronomical applications. 
This document describes the high level view of the software system and
is the fundamental definition of the tables  used in the Obit software
package.  

OBIT is an object--oriented set of class and utility libraries
allowing access to multiple disk--resident data formats.
In particular, access to either AIPS disk data or FITS files.
Documentation of the Obit classes uses the doxygen documentation
system and can be accessed starting at doc/doxygen/html/index.html.

\section{Obit Software system}
The Obit system maintains its basic information in global structures.
These are initialized by a call to ObitSystemStartup and released by a
call to ObitSystemShutdown.

\subsection{General Software Issues}
Obit is written in ANSI c of the glib dialect to assist in
portability.  
Thus most data types are declared with the ``g'' prefix (e.g. gfloat =
float). 
Also, g\_malloc and g\_free are used to allocate and free memory.

Each class and structure in Obit has an ``IsA'' function to test that
the pointer passed is to a valid instance of that structure or class.

The glib library provides assertion macroes which are used to test for
programming errors but the tests can be turned off in a production
compilation.
The assertion  g\_assert should be used with the class IsA test
functions to assure that the inputs to each routine are valid.
If the assertion fails, the program aborts and dumps core.

\subsection{Obit Classes}
Many, but not all, of the structures in Obit are inherited from the
basal Obit virtual class.
The Obit class system is described below.

\subsubsection{Organization}
The class name of the class being defined in a file begins with the
name of the class and include the following files for class xxx.
\begin{enumerate}
\item include/xxx.h\\
Class interface definition.  Defines structures, macroes and public
functions.
\item include/xxxDef.h\\
File included in include/xxx.h which defines the object structure.
It first includes the corresponding file from its parent class.
\item include//xxxClassDef.h\\
File included in include/xxx.h which defines the class structure.
It first includes the corresponding file from its parent class.
\item src/xxx.c\\
File containing the source code for the class; including class initialization.
\end{enumerate}
Derived classes begin with the name of the parent class such that the
inheritance is given in the class name.

Obit objects are connected to their respective classes by a pointer to
a global class structure which contains a pointer to the parent class
structure and function pointers for the class.
Class member functions may be invoked either explicitly or via the
class function pointers.
Class objects, and all parent classes are initialized upon the
instantialtion of the first object of that class.

Obit objects have explicit creators but no explicit destructors.
Instead, a referencing/defererencing system is used to destroy objects
when the last reference to them is removed.
A reference (pointer) is passed from the ObitRef (or class specific)
function which increments the objects reference count.
A reference is deleted using the ObitUnref (or class specific)
function which decrements the object reference count, destroying it if
it goes to zero and returns a null pointer.
{\bf The integrity of Obit depends on strict adherence to this convention.}

The Obit virtual class object contains the following members:
\begin{enumerate}
\item gint32 ObitId;\\
 Recognition bit pattern to identify the start of an Obit object.
\item gpointer ClassInfo;\\
ClassInfo pointer for class with name, base and function pointers.
\item gint ReferenceCount;\\
 Reference count for object (numbers of pointers attaching).
\item gchar *name;\\
 Name of object [Optional], this is used in error and informative messages.
\end{enumerate}

The Obit virtual class structure contains the following members:
\begin{enumerate}
\item gboolean initialized;\\
Have I been initialized?
\item gboolean hasScratch;\\
 Are disk resident "scratch" objects of this class possible?
\item gchar* ClassName;\\
Name of class ("Obit")
\item gconstpointer ParentClass;\\
Pointer to parent class ClassInfo, Null if none.
\item ObitClassInitFP ObitClassInit;\\
 Function pointer to Class initializer
\item newObitFP newObit;\\
Function pointer to newObit (default constructor).
Initializes class and parent if needed.
\item ObitCopyFP ObitCopy;\\
 Function pointer to shallow copy constructor.
\item ObitCloneFP ObitClone;\\
Function pointer to deep copy constructor
\item ObitRefFP ObitRef;\\
Function pointer to Object Reference
\item ObitUnrefFP ObitUnref;\\
Function pointer to Object Unreference
\item ObitIsAFP ObitIsA;\\
Function pointer to test if a class member.
\item ObitClearFP ObitClear;\\
Private Function pointer to deallocation function.
\item ObitInitFP ObitInit;\\
Private Function pointer to object initializer.
\end{enumerate}

\subsubsection{Inheritance}
The object and class strucures are defined in the *Def.h and
*ClassDef.h files as described above.
Inheritance is by means of including the class object and class
structure definition files (*Def.h and *ClassDef.h) at the beginning
of the corresponding class files.
This recursive include assures the full inherited structure is present
in each object and class.

\subsubsection{Constructors}
Obit objects are explicitly created by a constructor.
The default constructor has the name new[ObitClassname] and accepts an
optional name for the object.
This name is used in error and informative messages and should reflect
its usage.
Derived classes may have other constructors.

\subsubsection{Reference/destructors}
Obit classes have no explicit destructors.
Instead, this is handled by the Obit reference scheme.
Pointers to an object are assigned by means of the ObitRef (or class
specific) method which increments the object reference count.
The ObitUnref (or class specific) method removes a reference,
decrementing the reference count and destroying the object when the
count goes to zero.

\subsubsection{ObitInfoList}
Most Obit classes have an ObitInfoList member which is a container for
labeled arrays of information.
Most control information is passed to and stored by the object in this
member.

\subsubsection{Messages and error handling}
Messages and error handling is by means of the ObitErr class.
An ObitErr has an error condition flag and a stack of messages each
with a category, i.e. error, informative...
Low level routines enter messages on the stack which is then
explicitly logged and cleared using the ObitErrLog function.
If an error is encountered, this condition is entered into the ObitErr
structure and can be dealt with as appropriate by the calling
routine.
On entry, routines should check if an error condition exists and if so
return.

Message logging can be customized by an application specific logging
handler.
Currently all messages are written to stderr.

The most useful ways of entering information into an ObitErr are
through macroes defined in the class:
\begin{itemize}
\item Obit\_log\_error(err,errCode,format...)\\
log error message.
\item Obit\_return\_if\_fail(expr,err,format...) \\
Macro to evaluate expression, log in err and return on failure.
\item Obit\_retval\_if\_fail(expr,err,out,format...) \\
Macro to evaluate expression, log in err and return a value 
on failure.
\item Obit\_traceback\_msg(err,me,name)\\
Macro for traceback when an error in a called routine is encountered.
Writes traceback info and returns (no return value).
Gives location (file, routine, line) where message generated.
\item Obit\_traceback\_vall(err,me,name,out)\\
Macro for traceback logging with a return value
Writes traceback info and returns, passing a value.
Gives location (file, routine, line) where message generated.
\item  Obit\_cfitsio(err)\_error\\
Macro to dump cfitsio error stack to an ObitErr.
\end{itemize}
The traceback macroes should be used to check for errors in routines
called which cannot be handled in the current level and need to be
passed to higher levels.

\subsection{Obit Interface to Persistent Data}
Obit supports multiple disk-resident forms; currently AIPS and FITS
images and UV data as binary tables.
Interfaces to external data are via derivations from the ObitIO
class.
Explicit derived classes are needed for each data type (UV Image,
table...) and for each form (AIPS, FITS).
Except for associating an object with an explicit external source (data
file), most of Obit should not be aware of the source of the data
outside of the explicit IO classes.
The cfitsio package is used for basic FITS I/O.
\footnote{The fitsio library is a package of Fortran and c language
callable routines and is available at 
http://heasarc.gsfc.nasa.gov/docs/software/fitsio/fitsio.html.
This package takes care of most of the details of reading and writing
FITS files, including binary tables, and is well documented.}

Directories containing AIPS or FITS data files are registered with
Obit in the ObitSystemStartup call.
These directories are then referenced by a (1--rel) index.

\subsubsection{Images}
Images are accessed a plane (or more) at a time and are internally
kept as a 1-D array of float.
An ObitImage may have associated tables.
A descriptor object (class ObitImageDesc) is used to describe the data.
An ObitIO object is used to access the external data.

\subsubsection{UV data}
UV data is accessed as blocks of ``rows'' of visibility data which are
internally kept as 1-D arrays of float for efficient access.
The structure of UV data assumed is that of AIPS; ``Random'',
descriptive data followed by a rectangular data array.
``Compressed'' data is supported.
UV data in FITS format use binary tables for the visibility data but
these tables are handled differently from normal tables for increased
performance.
An ObitUV may have associated tables.
A descriptor object (class ObitUVDesc) is used to describe the data.
An ObitIO object is used to access the external data.

\subsubsection{Tables}
ObitTables are always associated with Image, UV data or other data.
A ``row'' in a table corresponds to a row in a FITS binary table and
consists of an array of arrays of data of the same native type.
Access is a row at a time and row data are converted to/from explicit
class structures.
A descriptor object (class ObitTableDesc) is used to describe the data.
An ObitIO object is used to access the external data.

\subsection{Threading}
The hooks for multithreaded operation (class ObitThread), e.g. mutexes
and locking/unlocking are in place but not yet implemented.
Note: cfitio is not thread--safe.

\subsection{Documentation}
Class documentation uses the doxygen system which uses specially coded
comments in the code to generate documentation.
The class documentation in html format starts at doc/doxygen/html/index.html.

\subsection{History}
Storing processing history is not yet implemented.

\section{External software}
Obit uses the following external packages:
\begin{itemize}
\item  glib \\
The gnu library for large scale software in anscii c.
Available at http://www.gtk.org/.
\item  doxygen \\
System for generating human readable software documentation from
comments in the software.
Available at http://www.stack.nl/\~\ dimitri/doxygen/ .
\item  cfitsio \\
Basic FITS I/O package.
Available at http://heasarc.gsfc.nasa.gov/docs/software/fitsio/fitsio.html.
\item  FFTW \\
Fast Fourier Transform.
Available at http://www.fftw.org/.
\item  GSL \\
GNU Scientific Library.
Available at http://www.gnu.org/software/gsl/.
\end{itemize}

\section{User interfaces}
Currently only AIPS programs and user written programs.

\section{AIPS interface}
A Fortran callable interface for Obit is defined in utility classes
ObitAIPSFortran and ObitAIPSObject.


\section {Example}
The following example copies an AIPS image to a FITS image and
illustrates the main features of an Obit program.
\begin{verbatim}
/* Include defining Obit structures and prototypes */
#include "ObitAll.h"

/* Example program to copy an AIPS image to a FITS file  */
int main ( int argc, char **argv )
{
  ObitSystem *mySystem;
  ObitErr *err;
  ObitImage *obitImage, *outImage;
  gint disk, cno, user;

  /* relative paths to AIPS data directories */
  gchar *AIPSdir[] = {"AIPSdata/"}; 

  /* relative path to FITS data directories */
  gchar *FITSdir[] = {"FITSdata/"};

  /* bottom left corner of image specified */ 
  gint blc[IM_MAXDIM] = {1,1,1,1,1,1,1};

  /* top right corner of image specified */ 
  gint trc[IM_MAXDIM] = {0,0,0,0,0,0,0};

  /*  Define input AIPS file */
  gchar Aname[13] = {"3C138       "};
  gchar Aclass[7] = {"PCube "};
  gchar Atype[3] = {"MA"};
  gint  Aseq = 2;

  /* Define output FITS file name - allow overwrite */
  gchar *Filename="!3C138.PCube.fits";

  /* Initialize Obit, define where disk directories are */
  err = newObitErr();  /* Create error/message object */
  user = 100;     /* User ID for AIPS */
  mySystem = ObitSystemStartup ("AIPS2FITS", 1, user, 1, AIPSdir, 1, FITSdir,  
                                (oint)TRUE, (oint)FALSE, err);
  ObitErrLog(err); /* show any error messages on err */

  /* Define image objects */
  obitImage = newObitImage("AIPS");
  outImage =  newObitImage("FITS");

  /* Associate input object with AIPS file */
  disk = 1;      /* AIPS and or FITS disk number */
  cno = ObitAIPSDirFindCNO(disk, user, Aname, Aclass, Atype, Aseq, err);
  if (cno<0) Obit_log_error(err, OBIT_Error, "Failure looking up input file");
  ObitErrLog(err);  /* display any errors */

  /* setup input */
  ObitImageSetAIPS(obitImage,OBIT_IO_byPlane,disk,cno,user,blc,trc,err);
 
   /* Associate output object with FITS file */
  ObitImageSetFITS(outImage,OBIT_IO_byPlane,disk,Filename,blc,trc,err);
    
  /* copy */
  outImage =  ObitImageCopy(obitImage, outImage, err);

  /* Unreference image objects - this will destroy the object but not
     the disk files. */
  obitImage = ObitImageUnref(obitImage);
  outImage =  ObitImageUnref(outImage);

  /* show any errors */
  ObitErrLog(err);

  /* Shutdown Obit */
  mySystem = ObitSystemShutdown (mySystem);
  err = ObitErrUnref(err);  /* delete error/message object */
  
  return 0;
} /* end of main */
\end{verbatim}

\section{Obit Tables}
This document uses latex macroes which are translated by a perl
script (ObitTables.pl) into the c source code.

\subsection{LaTeX Macros}
Tables used in Obit are defined in this document by using LaTeX
macroes to formally define the table.
These macroes are:
\begin{itemize}
\item tabletitle\{Title of table, e.g. ``Antenna table for uv data''\}
\item tablename\{Name of table, e.g. ``AN''\}
\item tableintro\{Short description of class\}
\item tableover\{Overview of usage of class\}
\item tablekey[\{name\}\{type code\}\{ software name\} \{default value\}
\{(range of indices)\} \{description\}]\\
Defines Table keyword.
\item tablecol[\{name\}\{units\}\{type code\} \{(dimensionality)\} \{software
name\} \{description\}]\\
Defines Table column..
\end{itemize}


%%%%%%%%%%%%%%% Obit Tables definition %%%%%%%%%%%%%%%%%%%%%%%%%%%%%%%%%%%%%%%%%%%%%%%%%%%
\clearpage
%%%%%%%%%%%%%%% ObitTableAN Class %%%%%%%%%%%%%%%%%%%%%%%%%%%%%%%%%%%%%%%%%%%%%%%%%%%
% 
%
\ClassName[{ObitTableAN}]
ObitTableAN Class
\tabletitle{Antenna table for uv data}
% table name
\tablename{AN}
\tableintro[
{This class contains tabular data and allows access.
"AIPS AN" contains information about the locations and characteristics
of antennas in a UV data set.
Also time information and the state of the Earth's orientation.
Each subarray in a uv data set will have it's own AN table in version
numbers the same order as the subarray.
Polarization calibration information is included. }
]
\tableover{
In memory tables are stored in a fashion similar to how they are 
stored on disk - in large blocks in memory rather than structures.
Due to the word alignment requirements of some machines, they are 
stored by order of the decreasing element size: 
double, float long, int, short, char rather than the logical order.
The details of the storage in the buffer are kept in the 
ObitTableDesc.
}
% Table keyword description
\begin{keywords}
\tablekey[{"ARRAYX  "}{D}{ArrayX}{0.0}{}
{Array center X coord. (meters, earth center)}
]
\tablekey[{"ARRAYY  "}{D}{ArrayY}{0.0}{}
{Array center Y coord. (meters, earth center)}
]
\tablekey[{"ARRAYZ  "}{D}{ArrayZ}{0.0}{}
{Array center Z coord. (meters, earth center)}
]
\tablekey[{"GSTIA0 "}{D}{GSTiat0}{0.0}{}
{GST at time=0 (degrees) on the reference date}
]
\tablekey[{"DEGPDY  "}{D}{DegDay}{360.0}{}
{Earth rotation rate (deg/IAT day)}
]
\tablekey[{"FREQ    "}{D}{Freq}{1.0}{}
{Obs. Reference Frequency for subarray(Hz)}
]
\tablekey[{"RDATE   "}{A}{RefDate}{"YYYYMMDD"}{}
{Reference date as "YYYYMMDD"}
]
\tablekey[{"POLARX  "}{E}{PolarX}{0.0}{}
{Polar position X (meters) on ref. date}
]
\tablekey[{"POLARY  "}{E}{PolarY}{0.0}{}
{Polar position Y (meters) on ref. date}
]
\tablekey[{"UT1UTC  "}{E}{ut1Utc}{}{}
{UT1-UTC  (time sec.)  }
]
\tablekey[{"DATUTC  "}{E}{dataUtc}{}{}
{data time-UTC  (time sec.)}
]
\tablekey[{"TIMSYS"}{A}{TimeSys}{}{}
{Time system, 'IAT' or 'UTC'}
]
\tablekey[{"ARRNAM  "}{A}{ArrName}{}{}
{Array name}
]
\tablekey[{"NUMORB  "}{J}{numOrb}{0}{()}
{Number of orbital parameters}
]
\tablekey[{"NOPCAL "}{J}{numPCal}{4}{()}
{Number of polarization calibration constants}
]
\tablekey[{"FREQID  "}{J}{FreqID}{0}{}
{Denotes the FQ ID for which the AN poln. parms have been modified.}
]
\tablekey[{"IATUTC  "}{E}{iatUtc}{}{}
{IAT - UTC (sec).}
]
\end{keywords}
%
% Table column description
\begin{columns}
\tablecol[{"ANNAME  "}{"        " }{A}{(8)}{AntName}
{Station name}
] 
\tablecol[{"STABXYZ "}{"METERS " }{D}{(3)}{StaXYZ}
{X,Y,Z offset from array center}
] 
\tablecol[{"ORBPARM "}{"       " }{D}{(numOrb)}{OrbParm}
{Orbital parameters.}
] 
\tablecol[{"NOSTA   "}{"       " }{J}{(1)}{noSta}
{Station number, used as an index in other tables, uv data}
] 
\tablecol[{"MNTSTA  "}{"       " }{J}{(1)}{mntSta}
{Mount type, 0=altaz, 1=equatorial, 2=orbiting}
] 
\tablecol[{"STAXOF  "}{"METERS " }{E}{(1)}{staXof}
{Axis offset}
] 
\tablecol[{"POLTYA  "}{"       " }{A}{(4)}{polTypeA}
{Feed A feed poln. type 'R','L','X','Y', actually only one valid character.}
] 
\tablecol[{"POLAA   "}{"DEGREES " }{E}{(1)}{PolAngA}
{Feed A feed position angle}
] 
\tablecol[{"POLCALA"}{"       " }{E}{(numPCal)}{PolCalA}
{Feed A poln. cal parameter. }
] 
\tablecol[{"POLTYB "}{"       " }{A}{(4)}{polTypeB}
{Feed B feed poln. type 'R','L','X','Y'}
] 
\tablecol[{"POLAB  "}{"DEGREES " }{E}{(1)}{PolAngB}
{Feed B feed position angle}
] 
\tablecol[{"POLCALB"}{"       " }{E}{(numPCal)}{PolCalB}
{Feed B poln. cal parameter}
] 
\end{columns}
%
% Table modification history
\begin{history}
\modhistory[{W. D. Cotton}{02/03/2003}{Revision 1: Copied from AIPS}]
\end{history}
\clearpage
%%%%%%%%%%%%%%% ObitTableBL Class %%%%%%%%%%%%%%%%%%%%%%%%%%%%%%%%%%%%%%%%%%%%%%%%%%%
% 
%
\ClassName[{ObitTableBL}]
ObitTableBL Class
\tabletitle{Template ObitTable document}
% table name
\tablename{BL}
\tableintro[
{This class contains tabular data and allows access.
"AIPS BL" contains baseline dependent additive and multiplicative
terms for the correction of UV data.
Each table row contains is for a single baseline and has a correction
for each IF. 
An ObitTableBL is the front end to a persistent disk resident structure.
Both FITS and AIPS cataloged data are supported.
This class is derived from the ObitTable class. }
]
\tableover{
In memory tables are stored in a fashion similar to how they are 
stored on disk - in large blocks in memory rather than structures.
Due to the word alignment requirements of some machines, they are 
stored by order of the decreasing element size: 
double, float long, int, short, char rather than the logical order.
The details of the storage in the buffer are kept in the 
ObitTableDesc.
}
% Table keyword description
\begin{keywords}
\tablekey[{"NO\_ANT "}{J}{numAnt}{}{}
{The number of antennas.
}
]
\tablekey[{"NO\_POL "}{J}{numPol}{}{(1,2)}
{The number of antennas.
}
]
\tablekey[{"NO\_IF  "}{J}{numIF}{}{()}
{The number of IFs.
}
]
\end{keywords}
%
% Table column description
\begin{columns}
\tablecol[{"TIME    "}{"DAYS   " }{E}{(1)}{Time}
{The center time.}
] 
\tablecol[{"SOURCE ID  "}{"        " }{J}{(1)}{SourID}
{Source ID number}
] 
\tablecol[{"SUBARRAY   "}{"        " }{J}{(1)}{SubA}
{Subarray number}
] 
\tablecol[{"ANTENNA1   "}{"        " }{J}{(1)}{ant1}
{First antenna number of baseline}
] 
\tablecol[{"ANTENNA2   "}{"        " }{J}{(1)}{ant2}
{Second antenna number of baseline}
] 
\tablecol[{"FREQ ID    "}{"        " }{J}{(1)}{FreqID}
{Freqid number}
] 
\tablecol[{"REAL M\#NO\_POL"}{"        " }{E}{(numIF)}{RealM}
{Real (Multiplicative correction Poln \# NO\_POL )}
] 
\tablecol[{"IMAG M\#NO\_POL"}{"        " }{E}{(numIF)}{ImagM}
{Imaginary (Multiplicative correction Poln \# NO\_POL )}
] 
\tablecol[{"REAL A\#NO\_POL"}{"        " }{E}{(numIF)}{RealA}
{Real (Additive correction Poln \# NO\_POL )}
] 
\tablecol[{"IMAG A\#NO\_POL"}{"        " }{E}{(numIF)}{ImagA}
{Imaginary (Additive correction Poln \# NO\_POL )}
] 
\end{columns}
%
% Table modification history
\begin{history}
\modhistory[{W. D. Cotton}{03/03/2003}{Revision 1: Copied from AIPS}]
\end{history}
%

\clearpage
%%%%%%%%%%%%%%% ObitTableBP Class %%%%%%%%%%%%%%%%%%%%%%%%%%%%%%%%%%%%%%%%%%%%%%%%%%%
% 
%
\ClassName[{ObitTableBP}]
ObitTableBP Class
\tabletitle{UV data BandPass calibration table}
% table name
\tablename{BP}
\tableintro[
{This class contains tabular data and allows access.
"AIPS BP" contains bandpass calibration informatioin for UV data.
An ObitTableBP is the front end to a persistent disk resident structure.
Both FITS and AIPS cataloged data are supported.
This class is derived from the ObitTable class. }
]
\tableover{
In memory tables are stored in a fashion similar to how they are 
stored on disk - in large blocks in memory rather than structures.
Due to the word alignment requirements of some machines, they are 
stored by order of the decreasing element size: 
double, float long, int, short, char rather than the logical order.
The details of the storage in the buffer are kept in the 
ObitTableDesc.
}
% Table keyword description
\begin{keywords}
\tablekey[{"NO\_ANT "}{J}{numAnt}{}{}
{The number of antennas}
]
\tablekey[{"NO\_POL "}{J}{numPol}{}{(1,2)}
{The number of antennas}
]
\tablekey[{"NO\_IF  "}{J}{numIF}{}{()}
{The number of IFs}
]
\tablekey[{"NO\_CHAN"}{J}{numChan}{}{()}
{Number of frequency channels}
]
\tablekey[{"STRT\_CHN"}{J}{startChan}{1}{}
{Start channel number}
]
\tablekey[{"NO\_SHFTS"}{J}{numShifts}{1}{}
{If numShifts = 1 BP entries are from cross-power data, if 2 are from total power, if 3 are a mixture, anything else then type is unknown and will assume cross-power}
]
\tablekey[{"LOW\_SHFT"}{J}{lowShift}{1}{}
{Most negative shift}
]
\tablekey[{"SHFT\_INC"}{J}{shiftInc}{1}{}
{Shift increment}
]
\tablekey[{"BP\_TYPE"}{A}{BPType}{}{}
{BP type: ' ' => standard BP table, CHEBSHEV' => Chebyshev polynomial coeff.}
]
\end{keywords}
%
% Table column description
\begin{columns}
\tablecol[{"TIME    "}{"DAYS   " }{D}{(1)}{Time}
{The center time.}
] 
\tablecol[{"INTERVAL "}{"DAYS   " }{E}{(1)}{TimeI}
{Time interval of record }
] 
\tablecol[{"SOURCE ID "}{"        " }{J}{(1)}{SourID}
{Source ID number}
] 
\tablecol[{"SUBARRAY "}{"       " }{J}{(1)}{SubA}
{Subarray number}
] 
\tablecol[{"ANTENNA "}{"       " }{J}{(1)}{antNo}
{Antenna number}
] 
\tablecol[{"BANDWIDTH "}{"HZ     " }{E}{(1)}{BW}
{andwidth of an individual channel}
] 
\tablecol[{"CHN\_SHIFT "}{"       " }{D}{(numIF)}{ChanShift}
{Frequency shift for each IF}
] 
\tablecol[{"FREQ ID "}{"       " }{J}{(1)}{FreqID}
{Freq. id number}
] 
\tablecol[{"REFANT \#NO\_POL"}{"       " }{J}{(1)}{RefAnt}
{Reference Antenna}
] 
\tablecol[{"WEIGHT \#NO\_POL"}{"       " }{E}{(numIF)}{Weight}
{Weights for complex bandpass}
] 
\tablecol[{"REAL \#NO\_POL"}{"       " }{E}{(numChan,numIF)}{Real}
{Real (channel gain Poln \# NO\_POL )}
] 
\tablecol[{"IMAG \#NO\_POL"}{"       " }{E}{(numChan,numIF)}{Imag}
{Imaginary (channel gain Poln \# NO\_POL)}
] 
\end{columns}
%
% Table modification history
\begin{history}
\modhistory[{W. D. Cotton}{03/03/2003}{Revision 1: Copied from AIPS}]
\end{history}
%
%
\clearpage
%%%%%%%%%%%%%%% ObitTableCL Class %%%%%%%%%%%%%%%%%%%%%%%%%%%%%%%%%%%%%%%%%%%%%%%%%%%
% Obit base class
%
%\title fooey
\ClassName[{ObitTableCL}]
{CaLibration table for UV data.}

\tabletitle{Solution Table}
% table name
\tablename{CL}
\tableintro[
{This class contains tabular data and allows access.
"AIPS CL" contains amplitude/phase/delay and rate calibration
information to be applied to a multi source UV data set.
An ObitTableCL is the front end to a persistent disk resident structure.
Both FITS and AIPS cataloged data are supported.
This class is derived from the ObitTable class. }
]
\tableover{
In memory tables are stored in a fashion similar to how they are 
stored on disk - in large blocks in memory rather than structures.
Due to the word alignment requirements of some machines, they are 
stored by order of the decreasing element size: 
double, float long, int, short, char rather than the logical order.
The details of the storage in the buffer are kept in the 
ObitTableDesc.
}
% Table keyword description
\begin{keywords}
\tablekey[{"REVISION"}{J}{revision}{1}{}
{Revision number of the table definition}
]
\tablekey[{"NO\_POL  "}{J}{numPol}{}{(1,2)}
{The number of polarizations}
]
\tablekey[{"NO\_IF  "}{J}{numIF}{}{()}
{The number of IFs}
]
\tablekey[{"NO\_ANT  "}{J}{numAnt}{}{}
{The number of antennas in table}
]
\tablekey[{"NO\_TERM"}{J}{numTerm}{0}{()}
{The number of terms in model polynomial}
]
\tablekey[{"MGMOD   "}{D}{mGMod}{1.0}{}
{The Mean Gain modulus 
}
]
\end{keywords}
%
% Table column description
\begin{columns}
\tablecol[{"TIME    "}{"DAYS   " }{D}{(1)}{Time}
{The center time of the solution}
] 
\tablecol[{"TIME INTERVAL"}{"DAYS    "}{E}{(1)}{TimeI}
{Solution interval.}
] 
\tablecol[{"SOURCE ID"}{"        "}{J}{(1)}{SourID}
{Source Identifier number }
] 
\tablecol[{"ANTENNA NO."}{"       " }{J}{(1)}{antNo}
{Antenna number }
] 
\tablecol[{"SUBARRAY"}{"        "}{J}{(1)}{SubA}
{Subarray number.}
] 
\tablecol[{"FREQ ID"}{"        "}{J}{(1)}{FreqID}
{Frequency ID}
] 
\tablecol[{"I.FAR.ROT"}{"RAD/M**2"}{E}{(1)}{IFR}
{Ionospheric Faraday Rotation }
] 
\tablecol[{"GEODELAY "}{"SECONDS "}{D}{(numTerm)}{GeoDelay}
{Geometric delay polynomial series at TIME}
] 
\tablecol[{"DOPPOFF  "}{"SEC/SEC "}{E}{(numIF)}{DopplerOff}
{Doppler offset for each IF}
] 
\tablecol[{"ATMOS  "}{"SECONDS "}{E}{(1)}{atmos}
{Atmospheric delay }
] 
\tablecol[{"DATMOS  "}{"SECONDS "}{E}{(1)}{Datmos}
{Time derivative of ATMOS}
] 
\tablecol[{"MBDELAY\#NO\_POL"}{"SECONDS "}{E}{(1)}{MBDelay}
{Multiband delay poln \# NO\_POL  }
] 
\tablecol[{"CLOCK \#NO\_POL"}{"SECONDS "}{E}{(1)}{clock}
{"Clock" epoch error }
] 
\tablecol[{"DCLOCK \#NO\_POL"}{"SEC/SEC"}{E}{(1)}{Dclock}
{Time derivative of CLOCK }
] 
\tablecol[{"DISP \#NO\_POL"}{"SECONDS "}{E}{(1)}{dispers}
{Dispersive delay (sec at wavelength = 1m)for Poln \# NO\_POL}
] 
\tablecol[{"DDISP \#NO\_POL"}{"SEC/SEC"}{E}{(1)}{Ddispers}
{Time derivative of DISPfor Poln \# NO\_POL}
] 
\tablecol[{"REAL\#NO\_POL"}{"       " }{E}{(numIF)}{Real}
{Real (gain Poln \# NO\_POL )}
] 
\tablecol[{"IMAG\#NO\_POL"}{"        "}{E}{(numIF)}{Imag}
{Imaginary (gain Poln \# NO\_POL)}
] 
\tablecol[{"DELAY \#NO\_POL"}{"SECONDS "}{E}{(numIF)}{Delay}
{ Residual group delay Poln \# NO\_POL}
] 
\tablecol[{"RATE \#NO\_POL"}{"SEC/SEC "}{E}{(numIF)}{Rate}
{Residual fringe rate  Poln \# NO\_POL}
] 
\tablecol[{"WEIGHT \#NO\_POL"}{"        "}{E}{(numIF)}{Weight}
{Weight of soln. Poln \# NO\_POL}
] 
\tablecol[{"REFANT \#NO\_POL"}{"        "}{J}{(numIF)}{RefAnt}
{Reference antenna Poln \# NO\_POL }
] 
\end{columns}
%
% Table modification history
\begin{history}
\modhistory[{W. D. Cotton}{28/02/2003}{Revision 1: Copied from AIPS}]
\end{history}

\clearpage
%%%%%%%%%%%%%%% ObitTableCQ Class %%%%%%%%%%%%%%%%%%%%%%%%%%%%%%%%%%%%%%%%%%%%%%%%%%%
% 
%
\ClassName[{ObitTableCQ}]
ObitTableCQ Class
\tabletitle{VLBA Correlator parameter frequency table}
% table name
\tablename{CQ}
\tableintro[
{This class contains tabular data and allows access.
"AIPS CQ" table contains VLBA-like correlation parameters which are
used for making a number of instrumental corrections.
An ObitTableCQ is the front end to a persistent disk resident structure.
Both FITS and AIPS cataloged data are supported.
This class is derived from the ObitTable class. }
]
\tableover{
In memory tables are stored in a fashion similar to how they are 
stored on disk - in large blocks in memory rather than structures.
Due to the word alignment requirements of some machines, they are 
stored by order of the decreasing element size: 
double, float long, int, short, char rather than the logical order.
The details of the storage in the buffer are kept in the 
ObitTableDesc.
}
% Table keyword description
\begin{keywords}
\tablekey[{"TABREV"}{J}{revision}{1}{}
{Revision number of the table definition}
]
\tablekey[{"NO\_IF"}{J}{numIF}{}{()}
{The number of IFs}
]
\end{keywords}
%
% Table column description
\begin{columns}
\tablecol[{"FRQSEL "}{"        " }{J}{(1)}{FrqSel}
{Frequency ID \{IFQDCQ in AIPSish\}}
] 
\tablecol[{"SUBARRAY "}{"        " }{J}{(1)}{SubA}
{Subarray number \{ISUBCQ\}}
] 
\tablecol[{"FFT\_SIZE "}{"        " }{J}{(numIF)}{FFTSize}
{Size of FFT in correlator \{NFFTCQ\}}
] 
\tablecol[{"NO\_CHAN "}{"        " }{J}{(numIF)}{numChan}
{No. of channels in correlator\{NCHCQ\}}
] 
\tablecol[{"SPEC\_AVG "}{"        " }{J}{(numIF)}{SpecAvg}
{Spectral averaging factor\{NSAVCQ\}}
] 
\tablecol[{"EDGE\_FRQ "}{"HZ      " }{D}{(numIF)}{EdgeFreq}
{Edge frequency \{DFRQCQ\}}
] 
\tablecol[{"CHAN\_BW "}{"HZ      " }{D}{(numIF)}{ChanBW}
{Channel bandwidth \{DCBWCQ\}}
] 
\tablecol[{"TAPER\_FN "}{"        " }{A}{(8,numIF)}{TaperFn}
{Taper function \{LTAPCQ\}}
] 
\tablecol[{"OVR\_SAMP "}{"        " }{J}{(numIF)}{OverSamp}
{Oversampling factor \{NOVSCQ\}}
] 
\tablecol[{"ZERO\_PAD "}{"        " }{J}{(numIF)}{ZeroPad}
{Zero-padding factor \{NZPDCQ\}}
] 
\tablecol[{"FILTER "}{"        " }{J}{(numIF)}{Filter}
{Filter type \{IFLTCQ\}}
] 
\tablecol[{"TIME\_AVG "}{"SECONDS " }{E}{(numIF)}{TimeAvg}
{Time averaging interval \{TAVGCQ\}}
] 
\tablecol[{"NO\_BITS "}{"        " }{J}{(numIF)}{numBits}
{Quantization (no. of bits per recorded sample)\{NBITCQ\}}
] 
\tablecol[{"FFT\_OVLP "}{"        " }{J}{(numIF)}{FFTOverlap}
{FFT overlap factor \{IOVLCQ\}}
] 
\end{columns}
%
% Table modification history
\begin{history}
\modhistory[{W. D. Cotton}{04/03/2003}{Revision 1: Copied from AIPS}]
\end{history}
%
%
\clearpage
%%%%%%%%%%%%%%% ObitTableFG Class %%%%%%%%%%%%%%%%%%%%%%%%%%%%%%%%%%%%%%%%%%%%%%%%%%%
% 
%
\ClassName[{ObitTableFG}]
ObitTableFG Class
\tabletitle{Flag table for uv data documentation}
% table name
\tablename{FG}
\tableintro[
{This class contains tabular data and allows access.
"AIPS FG" contains descriptions of data to be ignored
An ObitTableFG is the front end to a persistent disk resident structure.
Both FITS and AIPS cataloged data are supported.
This class is derived from the ObitTable class. }
]
\tableover{
In memory tables are stored in a fashion similar to how they are 
stored on disk - in large blocks in memory rather than structures.
Due to the word alignment requirements of some machines, they are 
stored by order of the decreasing element size: 
double, float long, int, short, char rather than the logical order.
The details of the storage in the buffer are kept in the 
ObitTableDesc.
}
% Table keyword description
No Keywords in table.
%\begin{keywords}
%\end{keywords}
%
% Table column description
\begin{columns}
\tablecol[{"SOURCE  "}{"       " }{J}{(1)}{SourID}
{Source ID as defined in the SOURCE table}
] 
\tablecol[{"SUBARRAY "}{"       " }{J}{(1)}{SubA}
{Subarray number}
] 
\tablecol[{"FREQ ID "}{"       " }{J}{(1)}{freqID}
{Frequency ID number}
] 
\tablecol[{"ANTS   "}{"       " }{J}{(2)}{ants}
{first and secong antenna  numbers for a baseline, 0=$>$all}
] 
\tablecol[{"TIME RANGE "}{"DAYS    " }{E}{(2)}{TimeRange}
{Start and end time of data to be flagged }
] 
\tablecol[{"IFS     "}{"       " }{J}{(2)}{ifs}
{First and last IF numbers to flag}
] 
\tablecol[{"CHANS   "}{"       " }{J}{(2)}{chans}
{First and last channel numbers to flag}
] 
\tablecol[{"PFLAGS  "}{"       " }{X}{(4)}{pFlags}
{Polarization flags, same order as in data, T=$>$flagged}
] 
\tablecol[{"REASON  "}{"       " }{A}{(24)}{reason}
{Reason for flagging}
] 
\end{columns}
%
% Table modification history
\begin{history}
\modhistory[{W. D. Cotton}{02/03/2003}{Revision 1: Copied from AIPS}]
\end{history}
%
\clearpage
%%%%%%%%%%%%%%% ObitTableFQ Class %%%%%%%%%%%%%%%%%%%%%%%%%%%%%%%%%%%%%%%%%%%%%%%%%%%
% Obit base class
%
%\title fooey
\ClassName[{ObitTableFQ}]
UV data Frequency information.
%\ClassEnum[{obitObjType}
%{Type of underlying data structures}]
%\begin{ObitEnum}
%\ObitEnumItem[{OBITTYPE\_FITS}{FITS file}]
%\ObitEnumItem[{OBITTYPE\_AIPS}{AIPS catalog data}]
%\end{ObitEnum}

\tabletitle{UV data FreQuency Table}
% table name
\tablename{FQ}
\tableintro[
{This class contains tabular data and allows access.
"AIPS FQ" contains frequency related information for ``IF'' in uv
data.
An ``IF'' is a construct that allows sets of arbitrarily spaced frequencies.
An ObitTableFQ is the front end to a persistent disk resident structure.
Both FITS and AIPS cataloged data are supported.
This class is derived from the ObitTable class.}
]
\tableover{
In memory tables are stored in a fashion similar to how they are 
stored on disk - in large blocks in memory rather than structures.
Due to the word alignment requirements of some machines, they are 
stored by order of the decreasing element size: 
double, float long, int, short, char rather than the logical order.
The details of the storage in the buffer are kept in the 
ObitTableDesc.
}
% Table keyword description
\begin{keywords}
\tablekey[{"NO\_IF  "}{J}{numIF}{}{()}
{The number of IFs, used to dimension table column entries}
]
\end{keywords}
%
% Table column description
\begin{columns}
\tablecol[{"FRQSEL  "}{"       " }{J}{(1)}{fqid}
{Frequency ID number for row, this is a random parameter in the uv data}
] 
\tablecol[{"IF FREQ "}{"HZ     " }{D}{(numIF)}{freqOff}
{Offset from reference frequency for each IF}
] 
\tablecol[{"CH WIDTH"}{"HZ     " }{E}{(numIF)}{chWidth}
{Bandwidth of an individual channel, now always written and read as a signed value}
] 
\tablecol[{"TOTAL BANDWIDTH "}{"HZ     " }{E}{(numIF)}{totBW}
{Total bandwidth of the IF, now written and read as an unsigned value}
] 
\tablecol[{"SIDEBAND "}{"       " }{J}{(numIF)}{sideBand}
{Sideband of the IF (-1 =$>$ lower, +1 =$>$ upper), now always written and read as +1}
] 
\end{columns}
%
% Table modification history
\begin{history}
\modhistory[{W. D. Cotton}{02/03/2002}{Revision 1: Copied from AIPS}]
\end{history}
%
\clearpage
%%%%%%%%%%%%%%% ObitTableNI Class %%%%%%%%%%%%%%%%%%%%%%%%%%%%%%%%%%%%%%%%%%%%%%%%%%%
% 
%
\ClassName[{ObitTableNI}]
ObitTableIN Class
\tabletitle{Ionospheric calibration table class}
% table name
\tablename{NI}
\tableintro[
{This class contains tabular data and allows access.
"AIPS NI" contains the results of ionospheric modeling fits.
The ionospheric phase screen over the array is modeled in terms of a
Zernike polynomial.
An ObitTableNI is the front end to a persistent disk resident structure.
Both FITS and AIPS cataloged data are supported.
This was formerly the IN table but an AIPS bug does nasty things to
``IN'' tables.
This class is derived from the ObitTable class. }
]
\tableover{
In memory tables are stored in a fashion similar to how they are 
stored on disk - in large blocks in memory rather than structures.
Due to the word alignment requirements of some machines, they are 
stored by order of the decreasing element size: 
double, float long, int, short, char rather than the logical order.
The details of the storage in the buffer are kept in the 
ObitTableDesc.
}
% Table keyword description
\begin{keywords}
\tablekey[{"REVISION"}{J}{revision}{1}{}
{Revision number of the table definition.}
]
\tablekey[{"NUM\_COEF"}{J}{numCoef}{5}{()}
{Number of Zernike coefficients}
]
\tablekey[{"H\_ION   "}{E}{heightIon}{1.0e10}{}
{Height of the ionospheric phase screen (km?)}
]
\tablekey[{"REF\_FREQ"}{D}{refFreq}{}{}
{Reference frequency for the phase screen model}
]
\end{keywords}
%
% Table column description
\begin{columns}
\tablecol[{"TIME    "}{"DAYS   " }{D}{(1)}{Time}
{The center time}
] 
\tablecol[{"TIME INTERVAL"}{"DAYS   " }{E}{(1)}{TimeI}
{ime interval of the solution}
] 
\tablecol[{"ANTENNA NO."}{"        " }{J}{(1)}{antNo}
{Antenna number, 0=$>$ all}
] 
\tablecol[{"SOURCE ID"}{"        " }{J}{(1)}{SourId}
{Source number, 0=$>$ all}
] 
\tablecol[{"SUBARRAY"}{"        " }{J}{(1)}{SubA}
{Subarray number, 0=$>$ all}
] 
\tablecol[{"WEIGHT "}{"        " }{E}{(1)}{weight}
{Weight}
] 
\tablecol[{"COEF   "}{"        " }{E}{(numCoef)}{coef}
{Zernike model coeffients}
] 
\end{columns}
%
% Table modification history
\begin{history}
\modhistory[{W. D. Cotton}{28/02/2003}{Revision 1: Copied from AIPS}]
\end{history}
%
\clearpage
%%%%%%%%%%%%%%% ObitTableMF Class %%%%%%%%%%%%%%%%%%%%%%%%%%%%%%%%%%%%%%%%%%%%%%%%%%%
% 
%
\ClassName[{ObitTableMF}]
ObitTableMF Class
\tabletitle{AIPS Model Fit Table}
% table name
\tablename{MF}
\tableintro[
{This class contains tabular data and allows access.
"AIPS MF" the results of a model fit to an image or uv data of the
type produced by SAD/VSAD.
An ObitTableMF is the front end to a persistent disk resident structure.
Both FITS and AIPS cataloged data are supported.
This class is derived from the ObitTable class. }
]
\tableover{
In memory tables are stored in a fashion similar to how they are 
stored on disk - in large blocks in memory rather than structures.
Due to the word alignment requirements of some machines, they are 
stored by order of the decreasing element size: 
double, float long, int, short, char rather than the logical order.
The details of the storage in the buffer are kept in the 
ObitTableDesc.
}
% Table keyword description
\begin{keywords}
\tablekey[{"DEPTH1"}{E}{depth1}{1}{}
{Dimensions 3 in image}
]
\tablekey[{"DEPTH2"}{E}{depth2}{1}{}
{Dimensions 4 in image}
]
\tablekey[{"DEPTH3"}{E}{depth3}{1}{}
{Dimensions 5 in image}1
]
\tablekey[{"DEPTH4"}{E}{depth4}{1}{}
{Dimensions 6 in image}
]
\tablekey[{"DEPTH5"}{E}{depth5}{1}{}
{Dimensions 7 in image}
]
\end{keywords}
%
% Table column description
\begin{columns}
\tablecol[{"PLANE "}{"        "}{E}{(1)}{plane}
{Plane number}
] 
\tablecol[{"PEAK INT"}{"JY/BEAM "}{E}{(1)}{Peak}
{Peak Ipol}
] 
\tablecol[{"I FLUX "}{"JY      "}{E}{(1)}{IFlux}
{Integrated Ipol flux}
] 
\tablecol[{"DELTAX "}{"DEGREE  "}{E}{(1)}{DeltaX}
{X offset of center}
] 
\tablecol[{"DELTAY "}{"DEGREE  "}{E}{(1)}{DeltaY}
{Y offset of center}
] 
\tablecol[{"MAJOR AX"}{"DEGREE  "}{E}{(1)}{MajorAx}
{Fitted major axis size}
] 
\tablecol[{"MINOR AX"}{"DEGREE  "}{E}{(1)}{MinorAx}
{Fitted minor axis size}
] 
\tablecol[{"POSANGLE"}{"DEGREE  "}{E}{(1)}{PosAngle}
{Fitted PA}
] 
\tablecol[{"Q FLUX  "}{"JY      "}{E}{(1)}{QFlux}
{Integrated Q flux density}
] 
\tablecol[{"U FLUX "}{"JY      "}{E}{(1)}{UFlux}
{Integrated U flux density}
] 
\tablecol[{"V FLUX  "}{"JY      "}{E}{(1)}{VFlux}
{Integrated Y flux density}
] 
\tablecol[{"ERR PEAK"}{"JY/BEAM "}{E}{(1)}{errPeak}
{rror in Peak Ipol}
] 
\tablecol[{"ERR FLUX"}{"JY      "}{E}{(1)}{errIFlux}
{rror in Integrated Ipol flux}
] 
\tablecol[{"ERR DLTX"}{"DEGREE  "}{E}{(1)}{errDeltaX}
{Error in X offset of center}
] 
\tablecol[{"ERR DLTY"}{"DEGREE  "}{E}{(1)}{errDeltaY}
{Error in Y offset of center}
] 
\tablecol[{"ERR MAJA"}{"DEGREE  "}{E}{(1)}{errMajorAx}
{Error in Fitted major axis size}
] 
\tablecol[{"ERR MINA"}{"DEGREE  "}{E}{(1)}{errMinorAx}
{Error in Fitted minor axis size}
] 
\tablecol[{"ERR PA "}{"DEGREE  " }{E}{(1)}{errPosAngle}
{Error in Fitted PA}
] 
\tablecol[{"ERR QFLX"}{"JY      " }{E}{(1)}{errQFlux}
{Error in Integrated Q flux density}
] 
\tablecol[{"ERR UFLX"}{"JY      " }{E}{(1)}{errUFlux}
{Error in Integrated U flux density}
] 
\tablecol[{"ERR VFLX"}{"JY      " }{E}{(1)}{errVFlux}
{Error in Integrated Y flux density}
] 
\tablecol[{"TYPE MOD"}{"        "}{E}{(1)}{TypeMod}
{Model type 1 = Gaussian}
] 
\tablecol[{"D0 MAJOR"}{"DEGREE  "}{E}{(1)}{D0Major}
{Deconvolved best major axis}
] 
\tablecol[{"D0 MINOR"}{"DEGREE  "}{E}{(1)}{D0Minor}
{Deconvolved best minor axis}
] 
\tablecol[{"D0 POSANG"}{"DEGREE  "}{E}{(1)}{D0PosAngle}
{Deconvolved best PA}
] 
\tablecol[{"D- MAJOR"}{"DEGREE  "}{E}{(1)}{DmMajor}
{Deconvolved least major axis}
] 
\tablecol[{"D- MINOR"}{"DEGREE  "}{E}{(1)}{DmMinor}
{econvolved least minor axis}
] 
\tablecol[{"D- POSAN"}{"DEGREE  "}{E}{(1)}{DmPosAngle}
{Deconvolved least PA}
] 
\tablecol[{"D+ MAJOR"}{"DEGREE  "}{E}{(1)}{DpMajor}
{Deconvolved most major axis}
] 
\tablecol[{"D+ MINOR"}{"DEGREE  "}{E}{(1)}{DpMinor}
{Deconvolved most minor axis}
] 
\tablecol[{"D+ POSAN"}{"DEGREE  "}{E}{(1)}{DpPosAngle}
{Deconvolved most PA}
] 
\tablecol[{"RES RMS "}{"JY/BEAM "}{E}{(1)}{ResRMS}
{RMS of Ipol residual}
] 
\tablecol[{"RES PEAK"}{"JY/BEAM "}{E}{(1)}{ResPeak}
{Peak in Ipol residual}
] 
\tablecol[{"RES FLUX"}{"JY      "}{E}{(1)}{ResFlux}
{Integrated Ipol in residual}
] 
\tablecol[{"CENTER X"}{"PIXEL   "}{E}{(1)}{PixelCenterX}
{Center x position in pixels}
] 
\tablecol[{"CENTER Y"}{"PIXEL  "}{E}{(1)}{PixelCenterY}
{Center y position in pixels}
] 
\tablecol[{"MAJ AXIS"}{"PIXEL  "}{E}{(1)}{PixelMajorAxis}
{Fitted major axis in pixels}
] 
\tablecol[{"MIN AXIS"}{"PIXEL  "}{E}{(1)}{PixelMinorAxis}
{Fitted minor axis in pixels}
] 
\tablecol[{"PIXEL PA"}{"DEGREE  "}{E}{(1)}{PixelPosAngle}
{Fitted PA(?)}
] 
\end{columns}
%
% Table modification history
\begin{history}
\modhistory[{W. D. Cotton}{03/03/2003}{Revision 1: Copied from AIPS}]
\end{history}
%
%
\clearpage
%%%%%%%%%%%%%%% ObitTableNX Class %%%%%%%%%%%%%%%%%%%%%%%%%%%%%%%%%%%%%%%%%%%%%%%%%%%
% 
%
\ClassName[{ObitTableNX}]
ObitTableNX Class
\tabletitle{iNdeX table for uv data}
% table name
\tablename{NX}
\tableintro[
{This class contains tabular data and allows access.
"AIPS NX" contains an index for a uv data file giving the times,
source and visibility range for a sequence of scans.
A scan is a set of observations in the same mode and on the same source.
An ObitTableNX is the front end to a persistent disk resident structure.
This class is derived from the ObitTable class. }
]
\tableover{
In memory tables are stored in a fashion similar to how they are 
stored on disk - in large blocks in memory rather than structures.
Due to the word alignment requirements of some machines, they are 
stored by order of the decreasing element size: 
double, float long, int, short, char rather than the logical order.
The details of the storage in the buffer are kept in the 
ObitTableDesc.
}
% Table keyword description
No Keywords in table.
%\begin{keywords}
%\end{keywords}
%
% Table column description
\begin{columns}
\tablecol[{"TIME    "}{"DAYS   " }{E}{(1)}{Time}
{The center time of the sscan.}
] 
\tablecol[{"TIME INTERVAL "}{"DAYS   " }{E}{(1)}{TimeI}
{Duration of scan}
] 
\tablecol[{"SOURCE ID "}{"       " }{J}{(1)}{SourID}
{Source ID as defined in then SOURCE table}
] 
\tablecol[{"SUBARRAY "}{"        " }{J}{(1)}{SubA}
{Subarray number}
] 
\tablecol[{"START VIS "}{"        "}{J}{(1)}{StartVis}
{First visibility number (1-rel) in scan}
] 
\tablecol[{"END VIS  "}{"        " }{J}{(1)}{EndVis}
{Last visibility number (1-rel) in scan}
] 
\tablecol[{"FREQ ID  "}{"        " }{J}{(1)}{FreqID}
{Frequency id of scan}
] 
\end{columns}
%
% Table modification history
\begin{history}
\modhistory[{W. D. Cotton}{02/03/2003}{Revision 1: Copied from AIPS}]
\end{history}
%
\clearpage
%%%%%%%%%%%%%%% ObitTableSN Class %%%%%%%%%%%%%%%%%%%%%%%%%%%%%%%%%%%%%%%%%%%%%%%%%%%
% Obit base class
%
%\title fooey
\ClassName[{ObitTableSN}]
{SolutioN table for UV data.}
%\ClassEnum[{obitObjType}
%{Type of underlying data structures}]
%\begin{ObitEnum}
%\ObitEnumItem[{OBITTYPE\_FITS}{FITS file}]
%\ObitEnumItem[{OBITTYPE\_AIPS}{AIPS catalog data}]
%\end{ObitEnum}

\tabletitle{Solution Table}
% table name
\tablename{SN}
\tableintro[
{This class contains tabular data and allows access.
"AIPS SN" amplitude/phase/delay and rate calibration information derived
from a calibration solution and can be used either for "self calibration"
or correction a multisource CaLibration table.
An ObitTableSN is the front end to a persistent disk resident structure.
Both FITS and AIPS cataloged data are supported.
This class is derived from the ObitTable class. }
]
\tableover{
In memory tables are stored in a fashion similar to how they are 
stored on disk - in large blocks in memory rather than structures.
Due to the word alignment requirements of some machines, they are 
stored by order of the decreasing element size: 
double, float long, int, short, char rather than the logical order.
The details of the storage in the buffer are kept in the 
ObitTableDesc.
}
% Table keyword description
\begin{keywords}
\tablekey[{"REVISION"}{J}{revision}{1}{}
{Revision number of the table definition.
}
]
\tablekey[{"NO\_POL  "}{J}{numPol}{}{(1,2)}
{The number of polarizations.
}
]
\tablekey[{"NO\_IF  "}{J}{numIF}{}{()}
{The number of IFs
}
]
\tablekey[{"NO\_ANT  "}{J}{numAnt}{}{}
{The number of antennas in table.
}
]
\tablekey[{"NO\_NODES"}{J}{numNodes}{0}{}
{The number of interpolation nodes.
}
]
\tablekey[{"MGMOD   "}{D}{mGMod}{1.0}{}
{The Mean Gain modulus 
}
]
\tablekey[{"APPLIED"}{L}{isApplied}{FALSE}{}
{True if table has been applied to a CL table.}
]
\end{keywords}
%
% Table column description
\begin{columns}
\tablecol[{"TIME    "}{"DAYS   " }{D}{(1)}{Time}
{The center time of the solution}
] 
\tablecol[{"TIME INTERVAL"}{"DAYS    "}{E}{(1)}{TimeI}
{Solution interval.}
] 
\tablecol[{"SOURCE ID"}{"        "}{J}{(1)}{SourID}
{Source Identifier number }
] 
\tablecol[{"ANTENNA NO."}{"       " }{J}{(1)}{antNo}
{Antenna number }
] 
\tablecol[{"SUBARRAY"}{"        "}{J}{(1)}{SubA}
{Subarray number.}
] 
\tablecol[{"FREQ ID"}{"        "}{J}{(1)}{FreqID}
{Frequency ID}
] 
\tablecol[{"I.FAR.ROT"}{"RAD/M**2"}{E}{(1)}{IFR}
{Ionospheric Faraday Rotation }
] 
\tablecol[{"NODE NO."}{"        "}{J}{(1)}{NodeNo}
{Node number }
] 
\tablecol[{"MBDELAY\#NO\_POL"}{"SECONDS "}{E}{(1)}{MBDelay}
{Multiband delay poln \# NO\_POL  }
] 
\tablecol[{"REAL\#NO\_POL"}{"       " }{E}{(numIF)}{Real}
{Real (gain Poln \# NO\_POL )}
] 
\tablecol[{"IMAG\#NO\_POL"}{"        "}{E}{(numIF)}{Imag}
{Imaginary (gain Poln \# NO\_POL)}
] 
\tablecol[{"DELAY \#NO\_POL"}{"SECONDS "}{E}{(numIF)}{Delay}
{ Residual group delay Poln \# NO\_POL}
] 
\tablecol[{"RATE \#NO\_POL"}{"SEC/SEC "}{E}{(numIF)}{Rate}
{Residual fringe rate  Poln \# NO\_POL}
] 
\tablecol[{"WEIGHT \#NO\_POL"}{"        "}{E}{(numIF)}{Weight}
{Weight of soln. Poln \# NO\_POL}
] 
\tablecol[{"REFANT \#NO\_POL"}{"        "}{J}{(numIF)}{RefAnt}
{Reference antenna Poln \# NO\_POL }
] 
\end{columns}
%
% Table modification history
\begin{history}
\modhistory[{W. D. Cotton}{28/02/2002}{Revision 1: Copied from AIPS}]
\end{history}
%
\clearpage
%%%%%%%%%%%%%%% ObitTableSU Class %%%%%%%%%%%%%%%%%%%%%%%%%%%%%%%%%%%%%%%%%%%%%%%%%%%
% 
%
\ClassName[{ObitTableSU}]
ObitTableSU Class
\tabletitle{Flag table for uv data documentation}
% table name
\tablename{SU}
\tableintro[
{This class contains tabular data and allows access.
"AIPS SU" contains information about astronomical sources.
An ObitTableSU is the front end to a persistent disk resident structure.
Both FITS and AIPS cataloged data are supported.
This class is derived from the ObitTable class. }
]
\tableover{
In memory tables are stored in a fashion similar to how they are 
stored on disk - in large blocks in memory rather than structures.
Due to the word alignment requirements of some machines, they are 
stored by order of the decreasing element size: 
double, float long, int, short, char rather than the logical order.
The details of the storage in the buffer are kept in the 
ObitTableDesc.
}
% Table keyword description
\begin{keywords}
\tablekey[{"NO\_IF  "}{J}{numIF}{}{()}
{The number of IFs}
]
\tablekey[{"VELTYP  "}{A}{velType}{}{}
{Velocity type,}
]
\tablekey[{"VELDEF  "}{A}{velDef}{}{}
{Velocity definition 'RADIO' or 'OPTICAL'}
]
\tablekey[{"FREQID  "}{J}{FreqID}{0}{}
{The Frequency ID for which the source parameters are relevant.}
\end{keywords}
%
% Table column description
\begin{columns}
\tablecol[{"ID. NO. "}{"       " }{J}{(1)}{SourID}
{Source ID}
] 
\tablecol[{"SOURCE  "}{"       " }{A}{(16)}{Source}
{Source name }
] 
\tablecol[{"QUAL    "}{"       " }{J}{(1)}{Qual}
{Source qualifier}
] 
\tablecol[{"CALCODE "}{"       " }{A}{(4)}{CalCode}
{Calibrator code}
] 
\tablecol[{"IFLUX   "}{"JY     " }{E}{(numIF)}{IFlux}
{Total Stokes I flux density per IF}
] 
\tablecol[{"QFLUX   "}{"JY     " }{E}{(numIF)}{QFlux}
{Total Stokes Q flux density per IF}
] 
\tablecol[{"UFLUX   "}{"JY     " }{E}{(numIF)}{UFlux}
{Total Stokes U flux density per IF}
] 
\tablecol[{"VFLUX   "}{"JY     " }{E}{(numIF)}{VFlux}
{Total Stokes V flux densityper IF }
] 
\tablecol[{"FREQOFF "}{"HZ     " }{D}{(numIF)}{FreqOff}
{Frequency offset (Hz) from IF nominal per IF}
] 
\tablecol[{"BANDWIDTH"}{"HZ     " }{D}{(numIF)}{Bandwidth}
{Bandwidth per IF}
] 
\tablecol[{"RAEPO   "}{"DEGREES " }{D}{(1)}{RAMean}
{Right ascension at mean EPOCH (actually equinox) }
] 
\tablecol[{"DECEPO  "}{"DEGREES " }{D}{(1)}{DecMean}
{Declination at mean EPOCH (actually equinox) }
] 
\tablecol[{"EPOCH   "}{"YEARS   " }{D}{(1)}{Epoch}
{Mean Epoch (really equinox) for position in yr. since year 0.0}
] 
\tablecol[{"RAAPP   "}{"DEGREES " }{D}{(1)}{RAApp}
{Apparent Right ascension }
] 
\tablecol[{"DECAPP  "}{"DEGREES " }{D}{(1)}{DecApp}
{Apparent Declination}
] 
\tablecol[{"LSRVEL "}{"M/SEC    " }{D}{(numIF)}{LSRVel}
{LSR velocity per IF }
] 
\tablecol[{"RESTFREQ"}{"HZ     " }{D}{(numIF)}{RestFreq}
{Line rest frequency per IF }
] 
\tablecol[{"PMRA   "}{"DEG/DAY " }{D}{(1)}{PMRa}
{Proper motion (deg/day) in RA}
] 
\tablecol[{"PMDEC  "}{"DEG/DAY " }{D}{(1)}{PMDec}
{Proper motion (deg/day) in declination}
] 
\end{columns}
%
% Table modification history
\begin{history}
\modhistory[{W. D. Cotton}{09/03/2003}{Revision 1: Copied from AIPS}]
\end{history}
%
\clearpage
%%%%%%%%%%%%%%% ObitTableVL Class %%%%%%%%%%%%%%%%%%%%%%%%%%%%%%%%%%%%%%%%%%%%%%%%%%%
% 
%
\ClassName[{ObitTableVL}]
ObitTableVL Class
\tabletitle{NVSS full format catalog file}
% table name
\tablename{VL}
\tableintro[
{This class contains tabular data and allows access.
"AIPS VL" contains a catalog of sources in the format produced for the
NVSS survey.
The tabel contains an index in the INDEX?? keywords that give the
first table row number (1-rel) for a given hour of RA.
An ObitTableVL is the front end to a persistent disk resident structure.
Both FITS and AIPS cataloged data are supported.
This class is derived from the ObitTable class. }
]
\tableover{
In memory tables are stored in a fashion similar to how they are 
stored on disk - in large blocks in memory rather than structures.
Due to the word alignment requirements of some machines, they are 
stored by order of the decreasing element size: 
double, float long, int, short, char rather than the logical order.
The details of the storage in the buffer are kept in the 
ObitTableDesc.
}
% Table keyword description
\begin{keywords}
\tablekey[{"REVISION"}{J}{revision}{1}{}
{Revision number of the table definition}
]
\tablekey[{"BM\_MAJOR"}{E}{BeamMajor}{}{}
{Restoring beam major axis in deg.}
]
\tablekey[{"BM\_MINOR"}{E}{BeamMinor}{}{}
{Restoring beam minor axis in deg.}
]
\tablekey[{"BM\_PA"}{E}{BeamPA}{}{}
{Restoring beam position angle of major axis in deg.}
]
\tablekey[{"SORTORT"}{J}{SortOrder}{}{}
{Column number for sort (neg -> descending)}
]
\tablekey[{"NUM\_INDEXED"}{J}{numImdexed}{}{}
{Number of rows in table when indexed}
]
\tablekey[{"INDEX00"}{J}{index00}{}{}
{First entry for RA=00 h}
]
\tablekey[{"INDEX01"}{J}{index01}{}{}
{First entry for RA=01 h}
]
\tablekey[{"INDEX02"}{J}{index02}{}{}
{First entry for RA=02 h}
]
\tablekey[{"INDEX03"}{J}{index03}{}{}
{First entry for RA=03 h}
]
\tablekey[{"INDEX04"}{J}{index04}{}{}
{First entry for RA=04 h}
]
\tablekey[{"INDEX05"}{J}{index05}{}{}
{First entry for RA=05 h}
]
\tablekey[{"INDEX06"}{J}{index06}{}{}
{First entry for RA=06 h}
]
\tablekey[{"INDEX07"}{J}{index07}{}{}
{First entry for RA=07 h}
]
\tablekey[{"INDEX08"}{J}{index08}{}{}
{First entry for RA=08 h}
]
\tablekey[{"INDEX09"}{J}{index09}{}{}
{First entry for RA=09 h}
]
\tablekey[{"INDEX10"}{J}{index10}{}{}
{First entry for RA=10 h}
]
\tablekey[{"INDEX11"}{J}{index11}{}{}
{First entry for RA=11 h}
]
\tablekey[{"INDEX12"}{J}{index12}{}{}
{First entry for RA=12 h}
]
\tablekey[{"INDEX13"}{J}{index13}{}{}
{First entry for RA=13 h}
]
\tablekey[{"INDEX14"}{J}{index14}{}{}
{First entry for RA=14 h}
]
\tablekey[{"INDEX15"}{J}{index15}{}{}
{First entry for RA=15 h}
]
\tablekey[{"INDEX16"}{J}{index16}{}{}
{First entry for RA=16 h}
]
\tablekey[{"INDEX17"}{J}{index17}{}{}
{First entry for RA=17 h}
]
\tablekey[{"INDEX18"}{J}{index18}{}{}
{First entry for RA=18 h}
]
\tablekey[{"INDEX19"}{J}{index19}{}{}
{First entry for RA=19 h}
]
\tablekey[{"INDEX20"}{J}{index20}{}{}
{First entry for RA=20 h}
]
\tablekey[{"INDEX21"}{J}{index21}{}{}
{First entry for RA=21 h}
]
\tablekey[{"INDEX22"}{J}{index22}{}{}
{First entry for RA=22 h}
]
\tablekey[{"INDEX23"}{J}{index23}{}{}
{First entry for RA=23 h}
]
\end{keywords}
%
% Table column description
\begin{columns}
%] 
\tablecol[{"RA(2000)"}{"DEGREE  " }{D}{(1)}{Ra2000}
{RA (J2000)}
] 
\tablecol[{"DEC(2000)"}{"DEGREE  " }{D}{(1)}{Dec2000}
{Dec (J2000)}
] 
\tablecol[{"PEAK INT  "}{"JY/BEAM " }{E}{(1)}{PeakInt}
{Peak Ipol }
] 
\tablecol[{"MAJOR AX"}{"DEGREE  " }{E}{(1)}{MajorAxis}
{Fitted major axis size}
] 
\tablecol[{"MINOR AX"}{"DEGREE  " }{E}{(1)}{MinorAxis}
{Fitted minor axis siz}
] 
\tablecol[{"POSANGLE"}{"DEGREE  " }{E}{(1)}{PosAngle}
{Fitted PA}
] 
\tablecol[{"Q CENTER"}{"JY/BEAM " }{E}{(1)}{QCenter}
{Center Q flux density}
] 
\tablecol[{"U CENTER"}{"JY/BEAM " }{E}{(1)}{UCenter}
{enter U flux density}
] 
\tablecol[{"P FLUX"}{"JY      " }{E}{(1)}{PFlux}
{Integrated polarized flux density}
] 
\tablecol[{"I RMS "}{"JY/BEAM " }{E}{(1)}{IRMS}
{Ipol RMS uncertainty }
] 
\tablecol[{"POL RMS "}{"JY/BEAM " }{E}{(1)}{PolRMS}
{RMS (sigma) in Qpol and Upol}
] 
\tablecol[{"RES RMS"}{"JY/BEAM " }{E}{(1)}{ResRMS}
{RMS of Ipol residual}
] 
\tablecol[{"RES PEAK"}{"JY/BEAM " }{E}{(1)}{ResPeak}
{Peak in Ipol residual}
] 
\tablecol[{"RES FLUX"}{"JY      " }{E}{(1)}{ResFlux}
{Integrated Ipol residual}
] 
\tablecol[{"CENTER X"}{"PIXEL   " }{E}{(1)}{CenterX}
{Center x position in pixels in FIELD}
] 
\tablecol[{"CENTER Y"}{"PIXEL   " }{E}{(1)}{CenterY}
{Center y position in pixels in FIELD}
] 
\tablecol[{"FIELD  "}{"       " }{A}{(8)}{Field}
{Name of survey field}
] 
\tablecol[{"JD PROCESSED"}{"DAYS    " }{J}{(1)}{JDProcess}
{Julian date on which entry was derived from image.}
] 
\end{columns}
%
% Table modification history
\begin{history}
\modhistory[{W. D. Cotton}{02/03/2003}{Revision 1: Copied from AIPS}]
\end{history}
%
%
\clearpage
%%%%%%%%%%%%%%% ObitTableVZ Class %%%%%%%%%%%%%%%%%%%%%%%%%%%%%%%%%%%%%%%%%%%%%%%%%%%
% 
%
\ClassName[{ObitTableVZ}]
ObitTableVZ Class
\tabletitle{NVSS short format catalog file}
% table name
\tablename{VZ}
\tableintro[
{This class contains tabular data and allows access.
"AIPS VZ" contains a catalog of sources.
An ObitTableVZ is the front end to a persistent disk resident structure.
Both FITS and AIPS cataloged data are supported.
This class is derived from the ObitTable class. }
]
\tableover{
In memory tables are stored in a fashion similar to how they are 
stored on disk - in large blocks in memory rather than structures.
Due to the word alignment requirements of some machines, they are 
stored by order of the decreasing element size: 
double, float long, int, short, char rather than the logical order.
The details of the storage in the buffer are kept in the 
ObitTableDesc.
}
% Table keyword description
\begin{keywords}
\tablekey[{"REVISION"}{J}{revision}{1}{}
{Revision number of the table definition}
]
\tablekey[{"REF\_FREQ"}{D}{refFreq}{1.4e9}{}
{The Reference frequency}
]
\end{keywords}
%
% Table column description
\begin{columns}
\tablecol[{"RA(2000)"}{"DEGREE  " }{D}{(1)}{Ra2000}
{RA (J2000)}
] 
\tablecol[{"DEC(2000)"}{"DEGREE  " }{D}{(1)}{Dec2000}
{Dec (J2000)}
] 
\tablecol[{"PEAK INT  "}{"JY/BEAM " }{E}{(1)}{PeakInt}
{Peak Ipol }
] 
\tablecol[{"RMS   "}{"JY/BEAM " }{E}{(1)}{RMS}
{RMS uncertainty }
] 
\tablecol[{"QUAL  "}{"        " }{E}{(1)}{Quality}
{Quality (crowding) measure, 0=> best, higher worse}
] 
\end{columns}
%
% Table modification history
\begin{history}
\modhistory[{W. D. Cotton}{02/03/2003}{Revision 1: Copied from AIPS}]
\end{history}
%
%
\clearpage
\begin{references}
\reference{Cotton, W.~D., Tody, D., and Pence, W.~D.\ 1995, \aaps, 113,
159--166.}
\reference{Flatters, C., 1998, AIPS Memo No, 102, NRAO.}
\reference{Wells, D.~C., Greisen, E.~W., and Harten, R.~H.\ 1981, \aaps, 44, 363.}
\end{references} 

\clearpage
\end{document}

\clearpage
%%%%%%%%%%%%%%% ObitTableXX Class %%%%%%%%%%%%%%%%%%%%%%%%%%%%%%%%%%%%%%%%%%%%%%%%%%%
% 
%
\ClassName[{ObitTableXX}]
ObitTableXX Class
\tabletitle{Template ObitTable document}
% table name
\tablename{XX}
\tableintro[
{This class contains tabular data and allows access.
"AIPS XX" contains highly secret information.
An ObitTableXX is the front end to a persistent disk resident structure.
Both FITS and AIPS cataloged data are supported.
This class is derived from the ObitTable class. }
]
\tableover{
In memory tables are stored in a fashion similar to how they are 
stored on disk - in large blocks in memory rather than structures.
Due to the word alignment requirements of some machines, they are 
stored by order of the decreasing element size: 
double, float long, int, short, char rather than the logical order.
The details of the storage in the buffer are kept in the 
ObitTableDesc.
}
% Table keyword description
\begin{keywords}
\tablekey[{"REVISION"}{J}{revision}{1}{}
{Revision number of the table definition.
}
]
\tablekey[{"NO\_SECRET"}{J}{numPol}{}{(1,2)}
{The number of secrets.
}
]
\end{keywords}
%
% Table column description
\begin{columns}
\tablecol[{"TIME    "}{"DAYS   " }{D}{(1)}{Time}
{The center time of the secret.}
] 
\tablecol[{""}{"" }{}{()}{}
{}
] 
\end{columns}
%
% Table modification history
\begin{history}
\modhistory[{A. N. Author}{99/99/9999}{Revision 1: Copied from AIPS}]
\end{history}
%
