% $Id$ 
%
%  ObitSingle dish and OTF/GBT Tables class definitions
%
%\def\section #1.#2.{\medskip\leftline{\bf #1. #2.}\smallskip}
%\def\bfitem #1#2{{{\bf (#1)}}{\it #2}}
\def\bfi #1#2{{\bf #1:}{ #2}\par}
\def\extname #1#2{ {\bf 1} Extension Name {\it #2}\smallskip}
%\def\tabname #1#2{\bfitem 1{#1 Table Name #2}}
\def\keyword {\leftline{\bf Keywords:}}
%\def\table {\leftline{\bf Table Definition:}}
\documentclass[11pt]{article}
\usepackage{PennArrayDoc}
\begin{document}
\setcounter{section}{0}
%  Title page

\vskip 5cm
\centerline{\ttlfont Single Dish Tables}
\vskip 1cm
\centerline{\ttlfont Obit: }
\centerline{\ttlfont Merx mollis mortibus nuper}
\vskip 3cm
\centerline{\secfont version: 1.1.1 \today}
\vskip 1cm
\centerline{\secfont W. D. Cotton}
\clearpage

% Table Of Contents
\tableofcontents
\cleardoublepage

\section {Introduction}
   This document is intended to define the contents and meaning of the
various tables developed for single dish radio astronomical data,
especially ``On--the--fly'' (OTF) data from the Penn Array bolometer
camera on the GBT.
This software works inside the Obit software package.

Usage documentation uses the doxygen system and is in separate
documents.

\section{Obit Tables}
This document uses latex macroes which are translated by a perl
script (bin/PennArrayTables.pl) into the c source code defining the
classes used to access these tables.
The tables defined in this document are both those needed for the OTF
data structure and for reading the GBT archive data files.

\subsection{LaTeX Macros}
Tables used in Obit are defined in this document by using LaTeX
macroes to formally define the table.
These macroes are:
\begin{itemize}
\item tabletitle\{Title of table, e.g. ``OTFArrayGeom''\}
\item tablename\{Name of table, e.g. ``OTFArrayGeom''\}
\item tableintro\{Short description of class\}
\item tableover\{Overview of usage of class\}
\item tablekey[\{name\}\{type code\}\{ software name\} \{default value\}
\{(range of indices)\} \{description\}]\\
Defines Table keyword.
\item tablecol[\{name\}\{units\}\{type code\} \{(dimensionality)\} \{software
name\} \{description\}]\\
Defines Table column..
\end{itemize}

\section{ObitOTF data}
The ObitOTF data structure class and related classes are for storing
``On--The--Fly'' single dish radio astronomy data.
The data structure is patterned after the AIPS single dish data format
but, due to AIPS table naming restrictions, only a FITS binary table
implementation is currently supported.

An ObitOTF data file can be thought of as a relational database.
The measured sky brightness measurements are kept in the OTFScanData
table together with some auxillary information such as the nominal sky
pointing position, time etc.
Other auxillary information and calibration and editing information is
kept in other tables (see below).


\subsection{ObitOTF tables}
The basic calibration strategy is to manipulate tables which tell how
to transform the data from ``raw'' data as tabulated to ``calibrated''
data.
There are two basic calibration tables, with the same internal
structure.
The OTFSoln table is a differential calibration relative to a
potential prior calibration.
A particular calibration operation determines a OTFSoln table.
A OTFCal table is a cumulative table which is obtained from a
(possible) prior calibration corrected by an OTFSoln table.

The tables in an ObitOTF data file include:
\begin{itemize}
\item  OTFScanData\\
Raw sky brightness data and auxillary information.
\item  OTFArrayGeom\\
Table giving the geometric offsets of a feed/detector array from the
pointing axis of the telescope.
\item  OTFTarget\\
Table of sources or targets. 
These are referred to in the OTFScanData table as an index into this table.
\item  OTFIndex\\
Scan table [optional] giving start and stop times and row numbers in
the OTFScanData table as well as targets etc.
This index is used to improve data access times.
\item  OTFFlag\\
Table describing ``flagged'' data - data to be ignored.
\item  OTFCal\\
Cumulative calibration table.
This table gives multiplicative and additive corrections to the raw
sky brightness measurements in the OTFScanData table as well as
corrections to the nominal telescope pointing direction.
\item  OTFSoln\\
Differential calibration (``Solution'') table.
\end{itemize}

\subsection{ObitOTF Data Structure}
Data in the OTFScanData table are stored in table records with a row
corresponding to the data obtained in a given integration.
The row contains a number of descriptive columns giving time,
celestial pointing etc. followed by a column containing a regular data
array.
The data in the OTFScanData table are all stored as floats to increase
access performance.

In general, the data in the data array is a multidimensional array with
different quanties along different axes (feed/detector, frequency,
polarization). 
The dimensionality, types and axis values are given in the header of
the OTFScanData table as the TDIMn (dimensionality), mCTYPn (axis
type), mCDLTn (increment in axis values between pixels), mCRPXn (axis
reference pixel), mCROTn (axis rotation angle) and mCRVLn (coordinate
on axis at reference pixel) keywords where m is the axis number and n
is the data column in the table.
(This is the standard FITS convention for conveying this information.)
The following illustration is for the Penn Array with one Stokes
(total power), 64 detectors (FEED) and 1 frequency; the data column is
number 9.
\begin{verbatim}
TDIM9   = '(1,64,1)'           / size of the multidimensional array
TZERO1  =   2.452814500000E+06 / Offset of Date from JD
1CTYP9  = 'STOKES  '           / Stokes axis
2CTYP9  = 'FEED    '           / Feed/detector axis
3CTYP9  = 'FREQ    '           / Frequency axis
1CDLT9  =         1.000000E+00 / Stokes ``increment''
2CDLT9  =         1.000000E+00 / Feed ``increment''
3CDLT9  =         1.000000E+00 / Frequency increment (Hz)
1CRPX9  =         1.000000E+00 / Stokes reference pixel
2CRPX9  =         1.000000E+00 / Feed reference pixel
3CRPX9  =         1.000000E+00 / Frequency reference pixel
1CROT9  =         0.000000E+00 / Stokes ``rotation'' (no meaning)
2CROT9  =         0.000000E+00 / Feed ``rotation'' (no meaning)
3CROT9  =         0.000000E+00 / Frequency ``rotation'' (no meaning)
1CRVL9  =   1.000000000000E+00 / Stokes 'I'
2CRVL9  =   1.000000000000E+00 / ``Feed'' 1
3CRVL9  =   9.000000000000E+10 / Frequency in Hz
\end{verbatim}


\section{ObitSD software}
The high level view of the Obit system is included in file
OBITdoc.ps.
The class documentation for the software for processing ObitOTF data
is derived from the source code using doxygen and is available
in html format starting at doc/doxygen/html/index.html.

\section{Building ObitSD}
The ObitSD package is an addon to the basic Obit package which should
be installed first.
See OBITdoc.ps for details.

ObitSD comes with a configure script to construct the Makefiles to build
ObitSD.
Note: there are a number of third party packages as well as basic Obit
that should be installed first.
The basic installation is thus:
\begin{verbatim}
 % setenv OBIT /where/ever/you/installed/Obit 
 % gtar xzvf ObitSD1.0.tgz
 % cd ObitSD
 % ./configure
 % make
\end{verbatim}
For ``/where/ever/you/installed/Obit'' substitute the actual path of
the Obit base directory.
Alternatively, use the ``--with-obit=DIR'' option with configure.
In addition to the packages used by basic Obit, ObitSD uses the GSL
(GNU Scientific Library) package.
The location of GSL can be specified to configure with the ''--with-gsl=DIR''
configure option if configure cannot find it.

When using ObitSD from python, specify the PYTHONPATH environment
variable as ``ObitSD/python : Obit/python'' where for ObitSD and Obit
substitute the base directories of the ObitSD and Obit packages.

\clearpage
%%%%%%%%%%%%%%% ObitTableOTFArrayGeom Class %%%%%%%%%%%%%%%%%%%%%%%%%%%%%%%%%%%%%%%%%%%%%%%%%%%
% 
%
\ClassName[{ObitTableOTFArrayGeom}]
ObitTableOTFArrayGeom Class
\tabletitle{OTFArrayGeom}
% table name
\tablename{OTFArrayGeom}
\tableintro[
{This class contains tabular data and allows access.
"OTFArrayGeom" contains information about the locations and characteristics
of detectors in the camera, the location of the telescope and time
related information.}
]
\tableover{
In memory tables are stored in a fashion similar to how they are 
stored on disk - in large blocks in memory rather than structures.
Due to the word alignment requirements of some machines, they are 
stored by order of the decreasing element size: 
double, float long, int, short, char rather than the logical order.
The details of the storage in the buffer are kept in the 
ObitTableDesc.
}
% Table keyword description
\begin{keywords}
\tablekey[{"TELEX  "}{D}{TeleX}{0.0}{}
{Telescope X coordinate. (meters, earth center)}
]
\tablekey[{"TELEY  "}{D}{TeleY}{0.0}{}
{Telescope Y coordinate. (meters, earth center)}
]
\tablekey[{"TELEZ  "}{D}{TeleZ}{0.0}{}
{Telescope Z coordinate. (meters, earth center)}
]
\tablekey[{"RDATE   "}{A}{RefDate}{"YYYYMMDD"}{}
{Reference date as "YYYYMMDD"}
]
\tablekey[{"DEGPDY  "}{D}{DegDay}{360.0}{}
{Earth rotation rate (deg/IAT day)}
]
\tablekey[{"POLARX  "}{E}{PolarX}{0.0}{}
{Polar position X (meters) on ref. date}
]
\tablekey[{"POLARY  "}{E}{PolarY}{0.0}{}
{Polar position Y (meters) on ref. date}
]/** 
\tablekey[{"GSTIA0 "}{D}{GSTiat0}{0.0}{}
{GST at time=0 (degrees) on the reference date}
]
\tablekey[{"UT1UTC  "}{E}{ut1Utc}{}{}
{UT1-UTC  (time sec.)  }
]
\tablekey[{"DATUTC  "}{E}{dataUtc}{}{}
{data time-UTC  (time sec.)}
]
\tablekey[{"IATUTC  "}{E}{iatUtc}{}{}
{IAT - UTC (sec).}
]
\tablekey[{"TIMSYS"}{A}{TimeSys}{"UTC"}{}
{Time system, 'IAT' or 'UTC'}
]
\end{keywords}
%
% Table column description
\begin{columns}
\tablecol[{"DETECTOR"}{"        " }{J}{(1)}{detector}
{Detector number}
] 
\tablecol[{"AZ\_OFF"}{"DEGREE  " }{E}{(1)}{azOff}
{``Azimuth'' offset from nominal pointing, this is formally ``cross
elevation'' in GBT-speak and is the offset on the sky in the direction
of azimuth.
}
] 
\tablecol[{"EL\_OFF"}{"DEGREE  " }{E}{(1)}{elOff}
{Elevation offset from nominal pointing}
] 
\end{columns}
%
% Table modification history
\begin{history}
\modhistory[{W. D. Cotton}{14/03/2003}{Revision 1: Initial definition}]
\end{history}
%
\clearpage
%%%%%%%%%%%%%%% ObitTableOTFCal Class %%%%%%%%%%%%%%%%%%%%%%%%%%%%%%%%%%%%%%%%%%%%%%%%%%%
% 
%
\ClassName[{ObitTableOTFCal}]
ObitTableOTFCal Class
\tabletitle{OTFCal}
% table name
\tablename{OTFCal}
\tableintro[
{This class contains tabular data and allows access.
"OTFCal" contains calibration information for OTF data.
Calibrated data for each detector are:
$${\rm cal\_data}\ = \ {\rm mult} ({\rm raw\_data}\ -\ {\rm cal}\ -\ 
{\rm add}\ -\ {\rm poly})$$
where cal is the calibration noise value for ``Cal on'' data and 0 for
``Cal off'' date and poly is evaluated in the direction of the detector.
]
\tableover{
In memory tables are stored in a fashion similar to how they are 
stored on disk - in large blocks in memory rather than structures.
Due to the word alignment requirements of some machines, they are 
stored by order of the decreasing element size: 
double, float long, int, short, char rather than the logical order.
The details of the storage in the buffer are kept in the ObitTableDesc.
}
% Table keyword description
\begin{keywords}
\tablekey[{"NO\_DETEC"}{J}{numDet}{1}{()}
{Number of detectors)}
]
\tablekey[{"NO\_POLY"}{J}{numPoly}{1}{()}
{Number of polynomial coefficients describing atmospheric emission.
0 => no polynomial model.}
]
\end{keywords}
%
% Table column description
\begin{columns}
\tablecol[{"TIME    "}{"DAYS   " }{E}{(1)}{Time}
{The center time.}
] 
\tablecol[{"TIME\_INT"}{"DAYS   " }{E}{(1)}{TimeI}
{The integration time.}
] 
\tablecol[{"TARGET"}{"       " }{J}{(1)}{Target}
{Celestial target, as index in target table.}
] 
\tablecol[{"DELTA\_AZ"}{"DEGREE  " }{E}{(1)}{dAz}
{Correction to the ``Azimuth'' for all detectors.
This is formally ``cross elevation'' in GBT-speak and is the offset on
the sky in the direction of azimuth.}
] 
\tablecol[{"DELTA\_El"}{"DEGREE  " }{E}{(1)}{dEl}
{Correction to the Elevation for all detectors.}
] 
\tablecol[{"CAL     "}{"        " }{E}{(numDet)}{cal}
{Cal value in units of raw data per detector, to be subtracted from
``Cal on'' data.}
] 
\tablecol[{"ADD     "}{"        " }{E}{(numDet)}{add}
{Additive (subtractive actually) term per detector}
] 
\tablecol[{"MULT    "}{"        " }{E}{(numDet)}{mult}
{Additive (subtractive actually) term per detector}
] 
\tablecol[{"WEIGHT  "}{"        " }{E}{(numDet)}{wt}
{ Weight value per detector}
] 
\tablecol[{"POLY    "}{"        " }{E}{(numPoly)}{poly}
{Polynomial atmosphere model, expansion in RA, dec about pointing position}
] 
\end{columns}
%
% Table modification history
\begin{history}
\modhistory[{W. D. Cotton}{03/04/2003}{Revision 1: Initial definition}]
\end{history}
%
\clearpage
%%%%%%%%%%%%%%% ObitTableOTFModel Class %%%%%%%%%%%%%%%%%%%%%%%%%%%%%%%%%%%%%%%%%%%%%%%%%%%
% 
%
\ClassName[{ObitTableOTFModel}]
ObitTableCC Class
\tabletitle{CLEAN Components Table}
% table name
\tablename{OTFModel}
\tableintro[
{This class contains tabular data and allows access.
"OTFModel" table contains image model components which may be derived
via a CLEAN or other fitting process.
An OTFModel is the front end to a persistent disk resident structure.
Only FITS data are supported.
This class is derived from the ObitTable class. }
]
\tableover{
In memory tables are stored in a fashion similar to how they are 
stored on disk - in large blocks in memory rather than structures.
Due to the word alignment requirements of some machines, they are 
stored by order of the decreasing element size: 
double, float long, int, short, char rather than the logical order.
The details of the storage in the buffer are kept in the 
ObitTableDesc.
A number of model types are supported as described with their
parameters in the following:
\begin{itemize}
\item Point\\
A Point model has a position and a flux but no extent on the sky.
This model is indicated by the absence of the Type column or a value
of 0.  
No additional parameters are needed.
\item Gaussian on Sky\\
This is a Gaussian shaped model.
This model is indicated by a value in the Type column of 1.
The extra model parameters are:.
\begin{enumerate}
\item Major axis size in asec.
\item Minor axis size in asec.
\item position angle on sky in deg.
\end{enumerate}
\item Convolved Gaussian\\
This is a Gaussian shaped model.
This model is indicated by a value in the Type column of 2.
The extra model parameters are:.
\begin{enumerate}
\item Major axis size in asec.
\item Minor axis size in asec.
\item position angle on sky in deg.
\end{enumerate}
\item Uniform optically thin sphere\\
This corresponds to a uniformly filled sphere model which is optically
thin.
This model is indicated by a value in the Type column of 3.
The extra model parameters are:.
\begin{enumerate}
\item Radius in aseconds.\\
\end{enumerate}
\end{itemize}
}
% Table keyword description
\begin{keywords}
\tablekey[{"NO\_PARM"}{J}{numParm}{0}{()}
{The number of IFs}
]
\end{keywords}
%
% Table column description
\begin{columns}
\tablecol[{"X       "}{"Degree  " }{E}{(1)}{X}
{``X'' position of component centroid as offset from reference position}
] 
\tablecol[{"Y       "}{"Degree  " }{E}{(1)}{Y}
{``Y'' position of component centroid as offset from reference position}
] 
\tablecol[{"FLUX    "}{"Jansky  " }{E}{(1)}{Flux}
{Flux density of component}
] 
\tablecol[{"TYPE    "}{"  " }{J}{(1)}{Type}
{Component type: 0: (or not present) point, 1=Gaussian on sky, 2=
convolved Gaussian, 3=Uniform optically thin sphere}
] 
\tablecol[{"PARMS   "}{"        " }{E}{(numParm)}{Parms}
{Model components as needed by model}
] 
\end{columns}
%
% Table modification history
\begin{history}
\modhistory[{W. D. Cotton}{17/12/2003}{Revision 1: Initial definition}]
\end{history}
%
%
\clearpage
%%%%%%%%%%%%%%% ObitTableSkyModel Class %%%%%%%%%%%%%%%%%%%%%%%%%%%%%%%%%%%%%%%%%%%%%%%%%%%
% 
%
\ClassName[{ObitTableSkyModel}]
ObitTableSkyModel Class
\tabletitle{SkyModel}
% table name
\tablename{SkyModel}
\tableintro[
{This class contains tabular data and allows access.
"SkyModel" contains a sky brightness model in terms of discrete components.}
]
\tableover{
In memory tables are stored in a fashion similar to how they are 
stored on disk - in large blocks in memory rather than structures.
Due to the word alignment requirements of some machines, they are 
stored by order of the decreasing element size: 
double, float long, int, short, char rather than the logical order.
The details of the storage in the buffer are kept in the 
ObitTableDesc.
}
% Table keyword description
\begin{keywords}
\tablekey[{"RA  "}{E}{RA}{0.0}{}
{Tangent point RA (deg)}
]
\tablekey[{"DEC  "}{E}{Dec}{0.0}{}
{Tangent point Dec (deg)}
]
\tablekey[{"PROJ  "}{A}{Proj}{"-SIN"}{}
{Projection code '-SIN', '-ARC', '-TAN'}
]
\end{keywords}
%
% Table column description
\begin{columns}
\tablecol[{"RA\_OFF  "}{"DEGREE  " }{E}{(1)}{RAOff}
{Right ascension offset from tangent point}
] 
\tablecol[{"DEC\_OFF "}{"DEGREE  " }{E}{(1)}{DecOff}
{Declination offset from tangent point}
] 
\tablecol[{"FLUX    "}{"JY      " }{E}{(1)}{Flux}
{Flux density}
] 
\end{columns}
%
% Table modification history
\begin{history}
\modhistory[{W. D. Cotton}{14/03/2003}{Revision 1: Initial definition}]
\end{history}
%
\clearpage
%%%%%%%%%%%%%%% ObitTableOTFSoln Class %%%%%%%%%%%%%%%%%%%%%%%%%%%%%%%%%%%%%%%%%%%%%%%%%%%
% 
%
\ClassName[{ObitTableOTFSoln}]
ObitTableOTFSoln Class
\tabletitle{OTFSoln}
% table name
\tablename{OTFSoln}
\tableintro[
{This class contains tabular data and allows access.
"OTFSoln" contains calibration solution information for OTF data.
Calibrated data for each detector are:
$${\rm cal\_data}\ = \ {\rm mult} ({\rm raw\_data}\ -\ {\rm cal}\ -\ 
{\rm add}\ -\ {\rm poly})$$
where cal is the calibration noise value for ``Cal on'' data and 0 for
``Cal off'' date and poly is evaluated in the direction of the detector.
OTFSoln tables may be applied to either an OTFCal table or directly
the data in a self-cal mode.
]
\tableover{
In memory tables are stored in a fashion similar to how they are 
stored on disk - in large blocks in memory rather than structures.
Due to the word alignment requirements of some machines, they are 
stored by order of the decreasing element size: 
double, float long, int, short, char rather than the logical order.
The details of the storage in the buffer are kept in the ObitTableDesc.
}
% Table keyword description
\begin{keywords}
\tablekey[{"NO\_DETEC"}{J}{numDet}{1}{()}
{Number of detectors)}
]
\tablekey[{"NO\_POLY"}{J}{numPoly}{1}{()}
{Number of polynomial coefficients describing atmospheric emission.
0 => no polynomial model.}
]
\end{keywords}
%
% Table column description
\begin{columns}
\tablecol[{"TIME    "}{"DAYS   " }{E}{(1)}{Time}
{The center time.}
] 
\tablecol[{"TIME\_INT"}{"DAYS   " }{E}{(1)}{TimeI}
{The integration time.}
] 
\tablecol[{"TARGET"}{"       " }{J}{(1)}{Target}
{Celestial target, as index in target table.}
] 
\tablecol[{"DELTA\_AZ"}{"DEGREE  " }{E}{(1)}{dAz}
{Correction to the ``Azimuth'' for all detectors.
This is formally ``cross elevation'' in GBT-speak and is the offset on
the sky in the direction of azimuth.}
] 
\tablecol[{"DELTA\_EL"}{"DEGREE  " }{E}{(1)}{dEl}
{Correction to the El for all detectors.}
] 
\tablecol[{"CAL     "}{"        " }{E}{(numDet)}{cal}
{Cal value in units of raw data per detector, to be subtracted from
``Cal on'' data.}
] 
\tablecol[{"ADD     "}{"        " }{E}{(numDet)}{add}
{Additive (subtractive actually) term per detector}
] 
\tablecol[{"MULT    "}{"        " }{E}{(numDet)}{mult}
{Additive (subtractive actually) term per detector}
] 
\tablecol[{"WEIGHT  "}{"        " }{E}{(numDet)}{wt}
{ Weight value per detector}
] 
\tablecol[{"POLY    "}{"        " }{E}{(numPoly)}{poly}
{Polynomial atmosphere model, expansion in RA, dec about pointing position}
] 
\end{columns}
%
% Table modification history
\begin{history}
\modhistory[{W. D. Cotton}{05/04/2003}{Revision 1: Initial definition}]
\end{history}
%
%
\clearpage
%%%%%%%%%%%%%%% ScanData Class %%%%%%%%%%%%%%%%%%%%%%%%%%%%%%%%%%%%%%%%%%%%%%%%%%%
% 
%
\ClassName[{ObitTableOTFScanData}]
ObitTableOTFScanData Class
\tabletitle{OTFScan data}
% table name
\tablename{OTFScanData}
\tableintro[
{This class contains tabular data and allows access.
An OTFScanData table has ``on the fly'' mode observational data from the
bolometer array.}
]
\tableover{
In memory tables are stored in a fashion similar to how they are 
stored on disk - in large blocks in memory rather than structures.
Due to the word alignment requirements of some machines, they are 
stored by order of the decreasing element size: 
double, float long, int, short, char rather than the logical order.
The details of the storage in the buffer are kept in the 
ObitTableDesc.
}
% Table keyword description
\begin{keywords}
\tablekey[{"NO\_DETEC"}{J}{numDet}{}{()}
{The number of detectors.}
]
\tablekey[{"ORIGIN"}{A}{origin}{}{}
{Originator of file}
]
\tablekey[{"OBJECT"}{A}{object}{}{}
{Name of object}
]
\tablekey[{"TELESCOP"}{A}{teles}{}{}
{Telescope used}
]
\tablekey[{"DATE-OBS"}{A}{obsdat}{}{}
{Date (yyyy-mm-dd) of observation}
]
\tablekey[{"EPOCH"}{E}{epoch}{}{}
{Celestial coordiate equinox}
]
\tablekey[{"BUNIT"}{A}{bunit}{}{}
{Data units}
]
\tablekey[{"OBSRA"}{D}{obsra}{}{}
{Observed Right Ascension in deg.}
]
\tablekey[{"OBSDEC"}{D}{obsdec}{}{}
{Observed declination in deg.}
]
\tablekey[{"BEAMSIZE"}{E}{beamSize}{0.00111}{}
{Gaussian FWHM of telescope beam size.}
]
\tablekey[{"DIAMETER"}{E}{diameter}{100.0}{}
{Diameter of telescope in meters.}
]
\tablekey[{"OTFTYPE"}{A}{OTFType}{"Unknown"}{}
{Type of data: ``DCR'': GBT DCR, ``SP'': GBT Spectral processor,
``CCB'':CalTech Continuum Backend, ``PAR'':Penn Array Receiver}
]
\end{keywords}
%
% Table column description
\begin{columns}
\tablecol[{"TIME    "}{"DAYS   " }{E}{(1)}{Time}
{The center time.}
] 
\tablecol[{"TIME\_INT"}{"DAYS   " }{E}{(1)}{TimeI}
{The integration time.}
] 
\tablecol[{"TARGET"}{"       " }{E}{(1)}{Target}
{Celestial target, as index in target table.}
] 
\tablecol[{"Scan"}{"       " }{E}{(1)}{Scan}
{Observing scan index.}
] 
\tablecol[{"RA  "}{"DEGREE  " }{E}{(1)}{RA}
{Nominal RA of array center}
] 
\tablecol[{"DEC    "}{"DEGREE" }{E}{(1)}{Dec}
{Nominal Dec of array center}
] 
\tablecol[{"ROTATE  "}{"        " }{E}{(1)}{rotate}
{Rotation of array on sky (parallactic angle)}
] 
\tablecol[{"CAL    "}{"        " }{E}{(1)}{cal}
{if > 0 then the cal source is on.}
] 
\tablecol[{"DATA    "}{"        " }{E}{(numDet)}{data}
{Detector sample data per detector )}
] 
\end{columns}
%
% Table modification history
\begin{history}
\modhistory[{W. D. Cotton}{14/03/2003}{Revision 1: Initial definition}]
\modhistory[{W. D. Cotton}{14/09/2003}{Added diameter, change name to
OTFScanData}]
\end{history}
%

\clearpage
%%%%%%%%%%%%%%% ObitTableOTFTarget Class %%%%%%%%%%%%%%%%%%%%%%%%%%%%%%%%%%%%%%%%%%%%%%%%%%%
% 
%
\ClassName[{ObitTableOTFTarget}]
ObitTableOTFTarget Class
\tabletitle{Target table for OTF data documentation}
% table name
\tablename{OTFTarget}
\tableintro[
{This class contains tabular data and allows access.
OTFTarget contains information about astronomical sources.
An ObitTableOTFTarget is the front end to a persistent disk resident structure.
Only FITS cataloged data are supported.
This class is derived from the ObitTable class. }
]
\tableover{
In memory tables are stored in a fashion similar to how they are 
stored on disk - in large blocks in memory rather than structures.
Due to the word alignment requirements of some machines, they are 
stored by order of the decreasing element size: 
double, float long, int, short, char rather than the logical order.
The details of the storage in the buffer are kept in the 
ObitTableDesc.
}
% Table keyword description
\begin{keywords}
\tablekey[{"VELTYP  "}{A}{velType}{}{}
{Velocity type,}
]
\tablekey[{"VELDEF  "}{A}{velDef}{}{}
{Velocity definition 'RADIO' or 'OPTICAL'}
]
\tablekey[{"FREQID  "}{J}{FreqID}{0}{}
{The Frequency ID for which the source parameters are relevant.}
]
\end{keywords}
%
% Table column description
\begin{columns}
\tablecol[{"ID. NO. "}{"       " }{J}{(1)}{TargID}
{Target ID}
] 
\tablecol[{"TARGET  "}{"       " }{A}{(16)}{Target}
{Target name }
] 
\tablecol[{"QUAL    "}{"       " }{J}{(1)}{Qual}
{Target qualifier}
] 
\tablecol[{"CALCODE "}{"       " }{A}{(4)}{CalCode}
{Calibrator code}
] 
\tablecol[{"IFLUX   "}{"JY     " }{E}{(1)}{IFlux}
{Total Stokes I flux density}
] 
\tablecol[{"QFLUX   "}{"JY     " }{E}{(1)}{QFlux}
{Total Stokes Q flux density}
] 
\tablecol[{"UFLUX   "}{"JY     " }{E}{(1)}{UFlux}
{Total Stokes U flux density}
] 
\tablecol[{"VFLUX   "}{"JY     " }{E}{(1)}{VFlux}
{Total Stokes V flux density}
] 
\tablecol[{"FREQOFF "}{"HZ     " }{D}{(1)}{FreqOff}
{Frequency offset (Hz) from nominal}
] 
\tablecol[{"BANDWIDTH"}{"HZ     " }{D}{(1)}{Bandwidth}
{Bandwidth}
] 
\tablecol[{"RAEPO   "}{"DEGREES " }{D}{(1)}{RAMean}
{Right ascension at mean EPOCH (actually equinox) }
] 
\tablecol[{"DECEPO  "}{"DEGREES " }{D}{(1)}{DecMean}
{Declination at mean EPOCH (actually equinox) }
] 
\tablecol[{"EPOCH   "}{"YEARS   " }{D}{(1)}{Epoch}
{Mean Epoch (really equinox) for position in yr. since year 0.0}
] 
\tablecol[{"RAAPP   "}{"DEGREES " }{D}{(1)}{RAApp}
{Apparent Right ascension }
] 
\tablecol[{"DECAPP  "}{"DEGREES " }{D}{(1)}{DecApp}
{Apparent Declination}
] 
\tablecol[{"LSRVEL "}{"M/SEC    " }{D}{(1)}{LSRVel}
{LSR velocity per IF }
] 
\tablecol[{"RESTFREQ"}{"HZ     " }{D}{(1)}{RestFreq}
{Line rest frequency per IF }
] 
\tablecol[{"PMRA   "}{"DEG/DAY " }{D}{(1)}{PMRa}
{Proper motion (deg/day) in RA}
] 
\tablecol[{"PMDEC  "}{"DEG/DAY " }{D}{(1)}{PMDec}
{Proper motion (deg/day) in declination}
] 
\end{columns}
%
% Table modification history
\begin{history}
\modhistory[{W. D. Cotton}{10/07/2003}{Revision 1: Initial version}]
\end{history}
\clearpage

%%%%%%%%%%%%%%% ObitTableOTFIndex Class %%%%%%%%%%%%%%%%%%%%%%%%%%%%%%%%%%%%%%%%%%%%%%%%%%%
% 
%
\ClassName[{ObitTableOTFIndex}]
ObitTableOTFIndex Class
\tabletitle{Index table for OTF data}
% table name
\tablename{OTFIndex}
\tableintro[
{This class contains tabular data and allows access.
ObitTableOTFIndex contains an index for a OTF data file giving the times,
target and datum range for a sequence of scans.
A scan is a set of observations in the same mode and on the same target.
An ObitTableOTFIndex is the front end to a persistent disk resident structure.
This class is derived from the ObitTable class. }
]
\tableover{
In memory tables are stored in a fashion similar to how they are 
stored on disk - in large blocks in memory rather than structures.
Due to the word alignment requirements of some machines, they are 
stored by order of the decreasing element size: 
double, float long, int, short, char rather than the logical order.
The details of the storage in the buffer are kept in the 
ObitTableDesc.
}
% Table keyword description
No Keywords in table.
%\begin{keywords}
%\end{keywords}
%
% Table column description
\begin{columns}
\tablecol[{"SCAN\_ID "}{"       " }{J}{(1)}{ScanID}
{Scan ID number}
] 
\tablecol[{"TIME    "}{"DAYS   " }{E}{(1)}{Time}
{The center time of the scan.}
] 
\tablecol[{"TIME\_INTERVAL "}{"DAYS   " }{E}{(1)}{TimeI}
{Duration of scan}
] 
\tablecol[{"TARGET\_ID "}{"       " }{J}{(1)}{TargetID}
{Target ID as defined in the OTFTarget table}
] 
\tablecol[{"START\_REC "}{"        "}{J}{(1)}{StartRec}
{First record number (1-rel) in scan}
] 
\tablecol[{"END\_REC  "}{"        " }{J}{(1)}{EndRec}
{Last record number (1-rel) in scan}
] 
\end{columns}
%
% Table modification history
\begin{history}
\modhistory[{W. D. Cotton}{08/06/2003}{Revision 1: Initial version}]
\end{history}
%

\clearpage
%%%%%%%%%%%%%%% ObitTableOTFFlag Class %%%%%%%%%%%%%%%%%%%%%%%%%%%%%%%%%%%%%%%%%%%%%%%%%%%
% 
%
\ClassName[{ObitTableOTFFlag}]
ObitTableOTFFlag Class
\tabletitle{Flag table for OTF data documentation}
% table name
\tablename{OTFFlag}
\tableintro[
{This class contains tabular data and allows access.
{ObitTableOTFFlag contains descriptions of data to be ignored
An {ObitTableOTFFlag is the front end to a persistent disk resident structure.
This class is derived from the ObitTable class. }
]
\tableover{
In memory tables are stored in a fashion similar to how they are 
stored on disk - in large blocks in memory rather than structures.
Due to the word alignment requirements of some machines, they are 
stored by order of the decreasing element size: 
double, float long, int, short, char rather than the logical order.
The details of the storage in the buffer are kept in the 
ObitTableDesc.
}
% Table keyword description
No Keywords in table.
%\begin{keywords}
%\end{keywords}
%
% Table column description
\begin{columns}
\tablecol[{"TARGET  "}{"       " }{J}{(1)}{TargetID}
{Target ID as defined in the OTFTarget table}
] 
\tablecol[{"FEED   "}{"       " }{J}{(1)}{Feed}
{Feed number to flag, 0=$>$all}
] 
\tablecol[{"TIME RANGE "}{"DAYS    " }{E}{(2)}{TimeRange}
{Start and end time of data to be flagged }
] 
\tablecol[{"FREQ   "}{"       " }{J}{(2)}{chans}
{First and last frequency channel numbers to flag}
] 
\tablecol[{"PFLAGS  "}{"       " }{X}{(4)}{pFlags}
{Polarization flags, same order as in data, T=$>$flagged}
] 
\tablecol[{"REASON  "}{"       " }{A}{(24)}{reason}
{Reason for flagging}
] 
\end{columns}
%
% Table modification history
\begin{history}
\modhistory[{W. D. Cotton}{08/06/2003}{Revision 1: Initial}]
\end{history}
%

% ++++++++++++++++++++++++++++++ GBT tables +++++++++++++++++++++++++++++++++++++=\
\clearpage
%%%%%%%%%%%%%%% GBT Antenna file %%%%%%%%%%%%%%%%%%%%%%%%%%%%%%%%%%%%%%%%%%%%%%%%%%%%%%%%%%%
%%%%%%%%%%%%%%% ObitTableGBTBEAM\_OFFSETS Class %%%%%%%%%%%%%%%%%%%%%%%%%%%%%%%%%%%%%%%%%%%%%%%%%%%
% 
%
\ClassName[{ObitTableGBTBEAM\_OFFSETS}]
ObitTableGBTBEAM\_OFFSETS Class
\tabletitle{Template ObitTableGBT document}
% table name
\tablename{BEAM\_OFFSETS}
\tableintro[
{Table in GBT archive/Antenna file.
This class contains tabular data and allows access.

This class is derived from the ObitTable class. }
]
\tableover{
}
% Table keyword description
%
% Table column description
\begin{columns}
\tablecol[{"NAME    "}{"        " }{A}{(32)}{Name}
{}
] 
\tablecol[{"BEAMXELOFFSET"}{"DEGREE  " }{D}{(1)}{xeloff}
{}
] 
\tablecol[{"BEAMELOFFSET"}{"DEGREE " }{D}{(1)}{eloff}
{}
] 
\tablecol[{"SRFEED1  "}{"       " }{J}{(1)}{srfeed1}
{}
] 
\tablecol[{"SRFEED2  "}{"       " }{J}{(1)}{srfeed2}
{}
] 
\end{columns}
%
% Table modification history
\begin{history}
\modhistory[{A. N. Author}{99/99/9999}{Revision 1: Copied from GBT}]
\end{history}

\clearpage
%%%%%%%%%%%%%%% ObitTableGBTANTPOSGR Class %%%%%%%%%%%%%%%%%%%%%%%%%%%%%%%%%%%%%%%%%%%%%%%%%%%
% 
%
\ClassName[{ObitTableGBTANTPOSGR}]
ObitTableGBTANTPOSGR Class
\tabletitle{Template ObitTableGBT document}
% table name
\tablename{ANTPOSGR}
\tableintro[
{Table in GBT archive/Antenna file.
This class contains tabular data and allows access.
This class is derived from the ObitTable class.  
Secondary Focus receivers.}
]
\tableover{
The details of the storage in the buffer are kept in the 
ObitTableDesc.
}
% Table keyword description
%
% Table column description
\begin{columns}
\tablecol[{"DMJD    "}{"DAY    " }{D}{(1)}{dmjd}
{}
] 
\tablecol[{"RAJ2000 "}{"DEGREE  " }{D}{(1)}{raj2000}
{}
] 
\tablecol[{"DECJ2000"}{"DEGREE  " }{D}{(1)}{decj2000}
{}
] 
\tablecol[{"MNT\_AZ "}{"DEGREE  " }{D}{(1)}{mntAaz}
{}
] 
\tablecol[{"MNT\_EL  "}{"DEGREE  " }{D}{(1)}{mntEl}
{}
] 
\tablecol[{"REFRACT "}{"DEGREE  " }{D}{(1)}{refract}
{}
] 
\tablecol[{"MAJOR "}{"DEGREE  " }{D}{(1)}{major}
{}
] 
\tablecol[{"MINOR "}{"DEGREE  " }{D}{(1)}{minor}
{}
] 
\tablecol[{"SR\_XP"}{"MM      " }{D}{(1)}{srXp}
{}
] 
\tablecol[{"SR\_YP"}{"MM      " }{D}{(1)}{srYp}
{}
] 
\tablecol[{"SR\_ZP "}{"MM      " }{D}{(1)}{srZp}
{}
] 
\tablecol[{"SR\_XT "}{"DEGREE  " }{D}{(1)}{srXt}
{}
] 
\tablecol[{"SR\_YT "}{"DEGREE  " }{D}{(1)}{srYt}
{}
] 
\tablecol[{"SR\_ZT "}{"DEGREE  " }{D}{(1)}{srZt}
{}
] 
\end{columns}
%
% Table modification history
\begin{history}
\modhistory[{A. N. Author}{99/99/9999}{Revision 1: Copied from GBT}]
\end{history}

\clearpage
%%%%%%%%%%%%%%% ObitTableGBTANTPOSF Class %%%%%%%%%%%%%%%%%%%%%%%%%%%%%%%%%%%%%%%%%%%%%%%%%%%
% 
%
\ClassName[{ObitTableGBTANTPOSPF}]
ObitTableGBTANTPOSPF Class
\tabletitle{Template ObitTableGBT document}
% table name
\tablename{ANTPOSPF}
\tableintro[
{Table in GBT archive/Antenna file.
This class contains tabular data and allows access.
This class is derived from the ObitTable class. 
Prime Focus receivers.}
]
\tableover{
The details of the storage in the buffer are kept in the 
ObitTableDesc.
}
% Table keyword description
%
% Table column description
\begin{columns}
\tablecol[{"DMJD    "}{"DAY    " }{D}{(1)}{dmjd}
{}
] 
\tablecol[{"RAJ2000 "}{"DEGREE  " }{D}{(1)}{raj2000}
{}
] 
\tablecol[{"DECJ2000"}{"DEGREE  " }{D}{(1)}{decj2000}
{}
] 
\tablecol[{"MNT\_AZ "}{"DEGREE  " }{D}{(1)}{mntAaz}
{}
] 
\tablecol[{"MNT\_EL  "}{"DEGREE  " }{D}{(1)}{mntEl}
{}
] 
\tablecol[{"REFRACT "}{"DEGREE  " }{D}{(1)}{refract}
{}
] 
\tablecol[{"MAJOR "}{"DEGREE  " }{D}{(1)}{major}
{}
] 
\tablecol[{"MINOR "}{"DEGREE  " }{D}{(1)}{minor}
{}
] 
\tablecol[{"PF\_FOCUS"}{"MM     " }{D}{(1)}{pfFocus}
{Prime focus focus.}
] 
\tablecol[{"PF\_ROTATION"}{"DEGREE  " }{D}{(1)}{pfRotation}
{Prime focus rotation}
] 
\tablecol[{"PF\_X  "}{"MM      " }{D}{(1)}{pfX}
{}
] 
\end{columns}
%
% Table modification history
\begin{history}
\modhistory[{A. N. Author}{99/99/9999}{Revision 1: Copied from GBT}]
\end{history}

\clearpage
%%%%%%%%%%%%%%% GBT Quadrant detector file %%%%%%%%%%%%%%%%%%%%%%%%%%%%%%%%%%%%%%%%%%%%%%%%%%%%%%%%%%%
%%%%%%%%%%%%%%% ObitTableGBTQUADDETECTOR Class %%%%%%%%%%%%%%%%%%%%%%%%%%%%%%%%%%%%%%%%%%%%%%%%%%%
% 
%
\ClassName[{ObitTableGBTQUADDETECTOR}]
ObitTableGBTQUADDETECTOR Class
\tabletitle{ObitTableGBTQUADDETECTOR document}
% table name
\tablename{QuadrantDetectorData}
\tableintro[
{Table in GBT archive/Quadrent detector file.
File in directory QuadrantDetector-QuadrantDetector-QuadrantDetectorData.
This class contains tabular data and allows access.

This class is derived from the ObitTable class. }
]
\tableover{
The quadrant detector measures motions of the GBT Feed arm resulting 
in pointing errors.
}
% Table keyword description
%
% Table column description
\begin{columns}
\tablecol[{"DMJD    "}{"DAY    " }{D}{(1)}{dmjd}
{Modified Julian Date of time sample taken;}
] 
\tablecol[{"ch1Voltage"}{"VOLT    " }{E}{(1)}{ch1Voltage}
{Channel 1 raw voltage}
] 
\tablecol[{"ch3Voltage"}{"VOLT    " }{E}{(1)}{ch3Voltage}
{Channel 3 raw voltage}
] 
\tablecol[{"ch4Voltage"}{"VOLT    " }{E}{(1)}{ch4Voltage}
{Channel 4 raw voltage}
] 
\tablecol[{"ch5Voltage"}{"VOLT    " }{E}{(1)}{ch5Voltage}
{Channel 5 raw voltage}
] 
\tablecol[{"X_Axis"}{"        " }{E}{(1)}{X_Axis}
{Calculated feed-arm motion in the X Axis, in arcseconds}
] 
\tablecol[{"Z_Axis"}{"        " }{E}{(1)}{Z_Axis}
{Calculated feed-arm motion in the Z Axis, in arcseconds}
] 
\tablecol[{"T1      "}{"DAY    " }{D}{(1)}{T1}
{Sample time-stamp for channel 1 data, as MJD}
] 
\tablecol[{"T3      "}{"DAY    " }{D}{(1)}{T3}
{Sample time-stamp for channel 3 data, as MJD}
] 
\tablecol[{"T4      "}{"DAY    " }{D}{(1)}{T4}
{Sample time-stamp for channel 4 data, as MJD}
] 
\tablecol[{"T5      "}{"DAY    " }{D}{(1)}{T5}
{Sample time-stamp for channel 5 data, as MJD}
] 
\tablecol[{"MedianClockOffset"}{"SECOND  " }{D}{(1)}{MedianClockOffset}
{Median estimate of GDAQ clock offset}
] 
\end{columns}
%
% Table modification history
\begin{history}
\modhistory[{A. N. Author}{99/99/9999}{Revision 1: Copied from GBT}]
\end{history}

\clearpage

%%%%%%%%%%%%%%% GBT DCR file %%%%%%%%%%%%%%%%%%%%%%%%%%%%%%%%%%%%%%%%%%%%%%%%%%%%%%%%%%%
%%%%%%%%%%%%%%% ObitTableGBTDCRSTATE Class %%%%%%%%%%%%%%%%%%%%%%%%%%%%%%%%%%%%%%%%%%%%%%%%%%%
% 
%
\ClassName[{ObitTableGBTDCRSTATE}]
ObitTableGBTSTATE Class
\tabletitle{ObitTableGBTDCRState document}
% table name
\tablename{STATE}
\tableintro[
{Table in GBT archive/DCR file.
This class contains tabular data and allows access.
This class is derived from the ObitTable class. }
]
\tableover{
The details of the storage in the buffer are kept in the 
ObitTableDesc.
}
% Table keyword description
\begin{keywords}
\tablekey[{"MASTER "}{A}{master}{"DCR"}{}
{Switching signals master
}
]
\tablekey[{"SCAN "}{J}{scan}{}{}
{Scan number
}
]
\tablekey[{"UTDATE "}{J}{utdate}{}{}
{MJD of start time
}
]
\tablekey[{"UTCSTART "}{D}{utcstart}{}{}
{Start time
}
]
\end{keywords}
%
% Table column description
\begin{columns}
\tablecol[{"BLANKTIM"}{"SECOND  " }{D}{(1)}{blanktim}
{}
] 
\tablecol[{"PHASETIM"}{"SECOND  " }{D}{(1)}{phasetim}
{}
] 
\tablecol[{"SIGREF"}{"" }{B}{(1)}{sigref}
{}
] 
\tablecol[{"CAL"}{"" }{B}{(1)}{cal}
{}
] 
\tablecol[{"SWSIG1 "}{"" }{B}{(1)}{swsig1}
{}
] 
\tablecol[{"SWSIG2 "}{"" }{B}{(1)}{swsig2}
{}
] 
\tablecol[{"SWSIG3 "}{"" }{B}{(1)}{swsig3}
{}
] 
\tablecol[{"SWSIG4"}{"" }{B}{(1)}{swsig4}
{}
] 
\tablecol[{"SWSIG5"}{"" }{B}{(1)}{swsig5}
{}
] 
\end{columns}
%
% Table modification history
\begin{history}
\modhistory[{A. N. Author}{99/99/9999}{Revision 1: Copied from GBT}]
\end{history}

\clearpage
%%%%%%%%%%%%%%% ObitTableGBTDCRRECEIVER  Class %%%%%%%%%%%%%%%%%%%%%%%%%%%%%%%%%%%%%%%%%%%%%%%%%%%
% 
%
\ClassName[{ObitTableGBTDCRRECEIVER}]
ObitTableGBTDCRRECEIVER Class
\tabletitle{ObitTableGBTDCRRECEIVER document}
% table name
\tablename{RECEIVER}
\tableintro[
{Table in GBT archive/DCR file.
This class contains tabular data and allows access.
This class is derived from the ObitTable class. }
]
\tableover{
The details of the storage in the buffer are kept in the 
ObitTableDesc.
}
%
% Table keyword description
\begin{keywords}
\tablekey[{"SCAN "}{J}{scan}{}{}
{Scan number
}
]
\tablekey[{"UTDATE "}{J}{utdate}{}{}
{MJD of start time
}
]
\tablekey[{"UTCSTART "}{D}{utcstart}{}{}
{Start time
}
]
\end{keywords}
%
% Table column description
\begin{columns}
\tablecol[{"CHANNELID"}{"" }{I}{(1)}{channelid}
{}
] 
\tablecol[{"TESTDATA"}{"" }{B}{(1)}{testdata}
{}
] 
\end{columns}
%
% Table modification history
\begin{history}
\modhistory[{A. N. Author}{99/99/9999}{Revision 1: Copied from GBT}]
\end{history}

\clearpage
%%%%%%%%%%%%%%% ObitTableGBTDCRDATA Class %%%%%%%%%%%%%%%%%%%%%%%%%%%%%%%%%%%%%%%%%%%%%%%%%%%
% 
%
\ClassName[{ObitTableGBTDCRDATA}]
ObitTableGBTDCRDATA  Class
\tabletitle{ObitTableGBTDCRDATA document}
% table name
\tablename{DATA }
\tableintro[
{Table in GBT archive/DCR file.
This class contains tabular data and allows access.
This class is derived from the ObitTable class. }
]
\tableover{
The details of the storage in the buffer are kept in the 
ObitTableDesc.
}
% Table keyword description
\begin{keywords}
\tablekey[{"SCAN "}{J}{scan}{}{}
{Scan number
}
]
\tablekey[{"UTDATE "}{J}{utdate}{}{}
{MJD of start time
}
]
\tablekey[{"UTCSTART "}{D}{utcstart}{}{}
{Start time
}
]
\tablekey[{"BACKEND"}{A}{backend}{"DCR"}{}
{Which backend
}
]
\tablekey[{"CTYPE1"}{A}{ctype1}{"STATE"}{}
{First data axis is State
}
]
\tablekey[{"CTYPE2"}{A}{ctype2}{"RECEIVER"}{}
{Second data axis is Receiver
}
]
\end{keywords}
%
% Table column description
\begin{columns}
\tablecol[{"IFFLAG  "}{"CODE " }{I}{(1)}{ifflag}
{}
] 
\tablecol[{"SUBSCAN "}{"CODE " }{J}{(1)}{subscan}
{}
] 
\tablecol[{"TIMETAG "}{"DMJD" }{D}{(1)}{timetag}
{}
] 
\tablecol[{"DATA "}{"COUNTS  " }{J}{(2,2))}{data}
{}
] 
\end{columns}
%
% Table modification history
\begin{history}
\modhistory[{A. N. Author}{99/99/9999}{Revision 1: Copied from GBT}]
\end{history}
\clearpage


%%%%%%%%%%%%%%% GBT CCB file %%%%%%%%%%%%%%%%%%%%%%%%%%%%%%%%%%%%%%%%%%%%%%%%%%%%%%%%%%%
%%%%%%%%%%%%%%% ObitTableGBTCCBSTATE Class %%%%%%%%%%%%%%%%%%%%%%%%%%%%%%%%%%%%%%%%%%%%%%%%%%%
% 
%
\ClassName[{ObitTableGBTCCBSTATE}]
ObitTableGBTSTATE Class
\tabletitle{ObitTableGBTCCBState document}
% table name
\tablename{STATE}
\tableintro[
{Table in GBT archive/CCB file.
This class contains tabular data and allows access.
This class is derived from the ObitTable class. }
]
\tableover{
A binary table extension called CCBSTATE records the physical
definitions of the phases of the data. There are a number of rows
equal to the number of phases NPHASES returned for each CCB input port
for each integration.  Columns PHIA and PHIB records the values of the
phase switches A \& B at each phase state. Valid values are 0 or 1. 
Data type of each entry in the column is an unsigned byte.  The
ordering of rows in the CCBSTATE table corresponds to ordering of the
phase columns in the DATA table.  The number of rows in the CCBSTATE
table is equal to the number of phases in the phase switch cycle
(which is in turn equal to 2 to the power of the number of active
phase switches hence 1,2, or 4). The number of phases in the
phase switch cycle is referred to as NPHASES elsewhere in this
document.\\
Comments:
\begin{itemize}
\item NPHASES is equal to 2NACTPSW where NACTPSW is the number of active phase switches.
Since NACTPSW has valid values of 0, 1, and 2, NPHASES can be 1, 2, or 4.
%
\item This table is analagous to GBT Backends' STATE tables but differs due to 
the different implementations of CALs (individual integrations are Cal On or Cal off, 
rather than having sub-integrations or ``phases'' be Cal On or Cal Off as for other 
GBT backends) and SIGREF (next bullet point) for this backend.
%
\item  The REF state corresponds to ``PHIA XOR PHIB''. A SIG state is ``NOT(PHIA XOR PHIB)''. 
With two phase switches active there will be two physically distinct rows of the 
CCBSTATE table that correspond to SIG and two that correspond to REF.
\end{itemize}
The details of the storage in the buffer are kept in the 
ObitTableDesc.
}
% Table keyword description
\begin{keywords}
\tablekey[{"NPHASES"}{J}{nphases}{}{}
{Scan number
}
]
\end{keywords}
%
% Table column description
\begin{columns}
\tablecol[{"PHIA"}{"state " }{J}{(1)}{phia}
{Value of phase switch A}
]
\tablecol[{"PHIB"}{"state " }{J}{(1)}{phib}
{Value of phase switch B}
]
\end{columns}
%
% Table modification history
\begin{history}
\modhistory[{A. N. Author}{99/99/9999}{Revision 1: Copied from GBT}]
\end{history}

\clearpage
%%%%%%%%%%%%%%% ObitTableGBTCCBRBPORT  Class %%%%%%%%%%%%%%%%%%%%%%%%%%%%%%%%%%%%%%%%%%%%%%%%%%%
% 
%
\ClassName[{ObitTableGBTCCBPORT}]
ObitTableGBTCCBPORT Class
\tabletitle{ObitTableGBTCCBPORT document}
% table name
\tablename{RECEIVER}
\tableintro[
{Table in GBT archive/CCB file.
This class contains tabular data and allows access.
This class is derived from the ObitTable class. }
]
\tableover{
A standard (SPN/004) PORT binary table extension is recorded in order
to allow the CCB inputs to be crossindexed with the physical
descriptions provided in the IF manager IF table. 
There are two columns: BANK (a character), and PORT (a non-zero integer). 
A given value of PORT uniquely identifies a physical input to the CCB, 
and may be used to index physical descriptions (frequency, feed, polarization 
,etc) in the IF manager IF table. 
It also uniquely defines a row in the PORT table. 
The BANK column is retained for compliance with the GBT FITS standard 
and are set to a fiducial value of 'A'. 
Additionally there is a SLAVE column indicating which daughter card a 
given input port is associated with. 
Data are unsigned 8-bit integers with valid values 0,1,2,3. 
The order of the rows of the PORT table correspond to the ordering of 
PORT columns in the DATA table. The number of rows NPORTS of the PORT 
table is equal to the number of ports input ports selected as active 
in the manager for the given scan.  
The details of the storage in the buffer are kept in the 
ObitTableDesc.
}
%
% Table keyword description
%
% Table column description
\begin{columns} 
\tablecol[{"BANK"}{" " }{A}{(1)}{bank}
{Always has value of ``A'' but included for compliance with GBT standards.}
] 
\tablecol[{"PORT"}{" " }{I}{(1)}{port}
{Identifier for a physical input to the CCB}
] 
\tablecol[{"SLAVE"}{" " }{I}{(1)}{slave}
{Which daughter card a given input port is associated with.}
] 
\end{columns}
%
% Table modification history
\begin{history}
\modhistory[{A. N. Author}{99/99/9999}{Revision 1: Copied from GBT}]
\end{history}

\clearpage
%%%%%%%%%%%%%%% ObitTableGBTCCBDATA Class %%%%%%%%%%%%%%%%%%%%%%%%%%%%%%%%%%%%%%%%%%%%%%%%%%%
% 
%
\ClassName[{ObitTableGBTCCBDATA}]
ObitTableGBTCCBDATA  Class
\tabletitle{ObitTableGBTCCBDATA document}
% table name
\tablename{DATA }
\tableintro[
{Table in GBT archive/CCB file.
This class contains tabular data and allows access.
This class is derived from the ObitTable class. }
]
\tableover{
The DATA binary table extension contains raw accumulated total power
integrations for each phase of each CCB input port that was used for a
given scan.  The first (DMJD) column of the data array contains the
MJD of the integration start.  The DATA column is a multidimensional
column with dimensions (NPORTS,NPHASES). Each datum is recorded as a
32 bit two's complement integer; subsequent transformation to unsigned
values is facilitated by recording a TZERO keyword with a value of
$2^31$. The order of the PORT and PHASE sub-columns should correspond to
the order of the rows in the PORT and CCBSTATE tables
respectively. The number of phases NPHASES is determined by the number
of active switches and is 1, 2, or 4. The number of ports NPORTS is
equal to the number of ports selected in the manager as active for the
given scan.\\
A second multi-dimensional OVRFLOW column, of the same dimensions as
the DATA column, comprises LOGICAL data with 'T' indicating
integrations that overflowed and 'F' indicating integrations that did
not. The value at subcolumn M row N in the OVRFLOW column denotes the
overflow status of the integration datum at subcolumn M row N of the
DATA column.  One multi-dimensional LOGICAL column contains the four
SLAVEOK flags. Two LOGICAL columns contain CAL A and CAL B ON flags
indicating whether, for a given integration, a given call was on or
not; the ``integration usable'' flag is a separate SHORT-INT column,
indicating whether each integration is usable based on the cal diode
rise and fall time flags applied by the CCB.\\
Comments\\
\begin{itemize}
\item Integration data are returned by the CCB as unsigned 32 bit integers. Conversion from signed 
32 bit two's complements values, to unsigned 32 bit values, may require use of double precision 
on the data processing end.
\item  The ``integration usable columns'' is short int not logical in order to more closely 
line up the columns with machine byte boundaries, for better performance.
\item  The SLAVEOK flags can be associated with individual columns of data 
(ie input ports) using the information in the SLAVE column of the PORT table.
\item  The cabling-dependent mapping of the CCB's ``Cal A'' and ``Cal B'' to a physical Cal diode 
(nominally Left and Right, or perhaps, tags in the calibration FITS file database) is 
presently unspecified.
\end{itemize}
The details of the storage in the buffer are kept in the 
ObitTableDesc.
The ordering of ``PORTS'' is given in the IF table (Freq, feed, poln).
CCBSTATE given in CCBSTATE table; guessing PHIA = sig/ref (ref state
swap the two feeds), PHIB = cal, 1=on.
}

% Table keyword description
% None
%
% Table column description
\begin{columns}
\tablecol[{"DMJD  "}{"d " }{D}{(1)}{dmjd}
{MJD of the integration start.}
] 
\tablecol[{"SLAVEOK"}{"bool " }{L}{(4)}{slaveok}
{Is each daughter card in an OK state?}
] 
\tablecol[{"USABLE "}{"status" }{I}{(1)}{usable}
{}
] 
\tablecol[{"CALA "}{"status" }{L}{(1)}{cala}
{Is Cal A ``on''?}
] 
\tablecol[{"CALB "}{"status" }{L}{(1)}{calb}
{Is Cal B ``on''?}
] 
\tablecol[{"OVRFLOW "}{"bool  " }{L}{(16,4))}{ovrflow}
{ DATA overflowed? - same order as DATA}
] 
\tablecol[{"DATA "}{"ulong " }{J}{(16,4))}{data}
{unsigned (PORT,CCBSTATE)}
] 
\end{columns}
%
% Table modification history
\begin{history}
\modhistory[{A. N. Author}{99/99/9999}{Revision 1: Copied from GBT}]
\end{history}
\clearpage

%%%%%%%%%%%%%%% GBT Penn Array Camera file %%%%%%%%%%%%%%%%%%%%%%%%%%%%%%%%%%%%%%%%%%%%%%%%%
%%%%%%%%%%%%%% ObitTableGBTPARDATA Class %%%%%%%%%%%%%%%%%%%%%%%%%%%%%%%%%%%%%%%%%%%%%%%%%%%
% 
%
\ClassName[{ObitTableGBTPARDATA}]
ObitTableGBTPARDATA  Class
\tabletitle{ObitTableGBTPARDATA document}
% table name
\tablename{DATA }
\tableintro[
{Table in GBT archive/Penn Array Camera file.
This class contains tabular data and allows access.
This class is derived from the ObitTable class. }
]
\tableover{
The DATA binary table extension contains raw bolometer data.
The ``TimeStamp'' column contains the modified Julian day number
in spite of any indications from the name and Units .\\
The main file HDU contains the following keywords:
\begin{itemize}
\item ``DATE-OBS''\\
String in form yyyy-mm-dd giving UTC start date
\item ``INSTRUME''\\
String 'PennArrayReceiver' 
\item ``UTCSTART''\\
String: UTC start time in seconds since midnight
\item ``UTDSTART''\\
String: UTC starttime in MJD, ``unknown'' = unspecified
\item ``UTCEND''\\
String: UTC of exposure end, ``unknown'' = unspecified
\item ``SCANNUM''\\
Integer: scan number(?).
\item ``PROJID''\\
String: Project ID
\end{itemize}
}
% Table keyword description
\begin{keywords}
\tablekey[{"CFGVALID"}{L}{cfgvalid}{TRUE}{}
{If true, configuration has not changed during scan}
]
\end{keywords}

%
% Table column description
\begin{columns}
\tablecol[{"TimeStamp"}{"Second" }{D}{(1)}{TimeStamp}
{Unix time in seconds.}
] 
\tablecol[{"daccounts"}{"Count" }{D}{(72)}{daccounts}
{daccounts}
] 
\tablecol[{"saecounts"}{"Count" }{D}{(72)}{saecounts}
{saecounts}
] 
\tablecol[{"DigInput"}{"bool" }{B}{(6)}{DigInput}
{DigInput}
] 
\end{columns}
%
% Table modification history
\begin{history}
\modhistory[{A. N. Author}{99/99/9999}{Revision 1: Copied from GBT}]
\end{history}
\clearpage

%%%%%%%%%%%%%%% GBT Penn Array Camera file %%%%%%%%%%%%%%%%%%%%%%%%%%%%%%%%%%%%%%%%%%%%%%%%%
%%%%%%%%%%%%%% ObitTableGBTPARDATA2 Class %%%%%%%%%%%%%%%%%%%%%%%%%%%%%%%%%%%%%%%%%%%%%%%%%%%
% 
%
\ClassName[{ObitTableGBTPARDATA2}]
ObitTableGBTPARDATA  Class
\tabletitle{ObitTableGBTPARDATA2 document}
% table name
\tablename{DATA }
\tableintro[
{Table in GBT archive/Penn Array Camera file.
This class contains tabular data and allows access.
This class is derived from the ObitTable class. }
]
\tableover{
The DATA binary table extension contains raw bolometer data.
The ``TimeStamp'' column contains the number of seconds since the
beginning of Unix Time (1 Jan 1976?).\\
The main file HDU contains the following keywords:
\begin{itemize}
\item ``DATE-OBS''\\
String in form yyyy-mm-dd giving UTC start date
\item ``INSTRUME''\\
String 'PennArrayReceiver' 
\item ``UTCSTART''\\
String: UTC start time in seconds since midnight
\item ``UTDSTART''\\
String: UTC starttime in MJD, ``unknown'' = unspecified
\item ``UTCEND''\\
String: UTC of exposure end, ``unknown'' = unspecified
\item ``SCANNUM''\\
Integer: scan number(?).
\item ``PROJID''\\
String: Project ID
\end{itemize}
}
% Table keyword description
\begin{keywords}
\tablekey[{"CFGVALID"}{L}{cfgvalid}{TRUE}{}
{If true, configuration has not changed during scan}
]
\end{keywords}

%
% Table column description
\begin{columns}
\tablecol[{"DMJD"}{"Day" }{D}{(1)}{DMJD}
{MJD.}
] 
\tablecol[{"daccounts"}{"Count" }{E}{(72)}{daccounts}
{daccounts}
] 
\tablecol[{"saecounts"}{"Count" }{E}{(72)}{saecounts}
{saecounts}
] 
\tablecol[{"DigInput"}{"bool" }{B}{(6)}{DigInput}
{DigInput}
] 
\end{columns}
%
% Table modification history
\begin{history}
\modhistory[{A. N. Author}{99/99/9999}{Revision 1: Copied from GBT}]
\modhistory[{A. N. Author}{99/99/9999}{Revision 2: GBT changed format}]
\end{history}
\clearpage

%%%%%%%%%%%%%%% ObitTableGBTPARSENSOR Class %%%%%%%%%%%%%%%%%%%%%%%%%%%%%%%%%%%%%%%%%%%%%%%%%%%
% 
%
\ClassName[{ObitTableGBTPARSENSOR}]
ObitTableGBTPARSENSOR Class
\tabletitle{ObitTableGBTPARSensor document}
% table name
\tablename{Sensor}
\tableintro[
{Table in GBT archive/PAR file.
This class contains tabular data and allows access.
This class is derived from the ObitTable class. }
]
\tableover{
A binary table extension called PARSENSOR Gives the row and column
numbers of each sensor.
NB this does not appear to correspond to sky position.
Eight of the sensors are ``dark'', i.e. don't see the sky.
}
% Table keyword description
% none
%
% Table column description
\begin{columns}
\tablecol[{"Row"}{" " }{J}{(1)}{row}
{Row number (0-rel) of corresponding sensor, row number=index in data arrays}
]
\tablecol[{"Col"}{" " }{J}{(1)}{col}
{Column number (0-rel) of corresponding sensor, row number=index in data arrays}
]
\end{columns}
%
% Table modification history
\begin{history}
\modhistory[{A. N. Author}{99/99/9999}{Revision 1: Copied from GBT}]
\end{history}

\clearpage

%%%%%%%%%%%%%%% GBT SpectralProcessor file %%%%%%%%%%%%%%%%%%%%%%%%%%%%%%%%%%%%%%%%%%%%%%%%%%%%%%%%%%%
%%%%%%%%%%%%%%%1 ObitTableGBTSTATE Class %%%%%%%%%%%%%%%%%%%%%%%%%%%%%%%%%%%%%%%%%%%%%%%%%%%
% 
%
\ClassName[{ObitTableGBTSPSTATE}]
ObitTableGBTSTATE Class
\tabletitle{ObitTableGBTSPSTATE document}
% table name
\tablename{STATE}
\tableintro[
{Table in GBT archive/SpectralProcessor file.
This class contains tabular data and allows access.
This class is derived from the ObitTable class. }
]
\tableover{
The details of the storage in the buffer are kept in the 
ObitTableDesc.
}
% Table keyword description
\begin{keywords}
\tablekey[{"FORMATID"}{A}{formatid}{"GBSDD007"}{}
{SDD\_FORMAT\_ID
}
]
\tablekey[{"SCAN"}{J}{scan}{}{}
{Scan number
}
]
\tablekey[{"SUBSCAN"}{J}{subscan}{}{}
{Scan record number
}
]
\tablekey[{"UTDATE"}{J}{utdate}{}{}
{MJD of start time
}
]
\tablekey[{"UTCSTART"}{D}{utcstart}{}{}
{UTC start time seconds.
}
]
\tablekey[{"UTCSTOP"}{D}{utcstop}{}{}
{Stop time seconds.
}
]
\tablekey[{"RCVRS"}{J}{rcvrs}{}{}
{Each item is index by RCVRS.
}
]
\end{keywords}
%
% Table column description
\begin{columns}
\tablecol[{"BLANKTIM"}{"SECOND  " }{E}{(2)}{blanktim}
{}
] 
\tablecol[{"PHASETIM"}{"SECOND  " }{E}{(2)}{phasetim}
{}
] 
\tablecol[{"SIGREF"}{"" }{B}{(2)}{sigref}
{}
] 
\tablecol[{"CAL"}{"" }{B}{(2)}{cal}
{}
] 
\tablecol[{"FFTS "}{"" }{J}{(2)}{ffts}
{}
] 
\tablecol[{"DELETED "}{"" }{J}{(2)}{deleted}
{}
] 
\end{columns}
%
% Table modification history
\begin{history}
\modhistory[{A. N. Author}{99/99/9999}{Revision 1: Copied from GBT}]
\end{history}

\clearpage
%%%%%%%%%%%%%%% ObitTableGBTSPRECEIVER  Class %%%%%%%%%%%%%%%%%%%%%%%%%%%%%%%%%%%%%%%%%%%%%%%%%%%
% 
%
\ClassName[{ObitTableGBTSPRECEIVER}]
ObitTableGBTSPRECEIVER Class
\tabletitle{ObitTableGBTSPRECEIVER document}
% table name
\tablename{RECEIVER}
\tableintro[
{Table in GBT archive/SpectralProcessor file.
This class contains tabular data and allows access.
This class is derived from the ObitTable class. }
]
\tableover{
The details of the storage in the buffer are kept in the 
ObitTableDesc.
}
%
% Table keyword description
\begin{keywords}
\tablekey[{"SCAN "}{J}{scan}{}{}
{Scan number
}
]
\tablekey[{"UTDATE "}{J}{utdate}{}{}
{MJD of start time
}
]
\tablekey[{"UTCSTART "}{D}{utcstart}{}{}
{Start time
}
]
\end{keywords}
%
% Table column description
\begin{columns}
\tablecol[{"RCVRID"}{" "}{J}{(1)}{rcvrid}
{}
] 
\tablecol[{"TAPER"}{" "}{A}{(8)}{taper}
{}
] 
\tablecol[{"OBSFREQ"}{"HZ"}{D}{(1)}{obsfreq}
{}
] 
\tablecol[{"IFF"}{"HZ"}{D}{(1)}{iff}
{}
] 
\tablecol[{"FREQRES"}{"HZ"}{D}{(1)}{freqres}
{}
] 
\tablecol[{"BANDWD"}{"HZ"}{E}{(1)}{bandwd}
{}
] 
\tablecol[{"TCAL"}{"DEGREE"}{E}{(1)}{tcal}
{}
] 
\tablecol[{"TPLEVEL"}{" "}{E}{(1)}{tplevel}
{}
] 
\tablecol[{"FASTTIM"}{"SECOND"}{E}{(1)}{fasttim}
{}
] 
\tablecol[{"SLOWTIM"}{"SECOND"}{E}{(1)}{slowtim}
{}
] 
\tablecol[{"CLIP"}{"SECOND"}{E}{(1)}{clip}
{}
] 
\tablecol[{"THRESH"}{"SECOND"}{E}{(1)}{thresh}
{}
] 
\tablecol[{"SYNTHL"}{"CODE"}{B}{(1)}{synthl}
{}
] 
\tablecol[{"OVERL"}{"CODE"}{B}{(1)}{overl}
{}
] 
\tablecol[{"IMODF"}{"CODE"}{B}{(1)}{imodf}
{}
] 
\tablecol[{"IFSYNTH"}{"CODE"}{B}{(1)}{ifsynth}
{}
] 
\tablecol[{"TAPEROFF"}{"CODE"}{B}{(1)}{taperoff}
{}
] 
\tablecol[{"RFIEXC"}{"CODE"}{B}{(1)}{rfiexc}
{}
] 
\tablecol[{"CLKSRC"}{"CODE"}{B}{(1)}{clksrc}
{}
] 
\tablecol[{"IFLO"}{"CODE"}{B}{(1)}{iflo}
{}
] 
\tablecol[{"IFSIDE"}{"CODE"}{B}{(1)}{ifside}
{}
] 
\tablecol[{"RFSIDE"}{"CODE"}{B}{(1)}{rfside}
{}
] 
\end{columns}
%
% Table modification history
\begin{history}
\modhistory[{A. N. Author}{99/99/9999}{Revision 1: Copied from GBT}]
\end{history}

\clearpage
%%%%%%%%%%%%%%% ObitTableGBTSPDATA Class %%%%%%%%%%%%%%%%%%%%%%%%%%%%%%%%%%%%%%%%%%%%%%%%%%%
% 
%
\ClassName[{ObitTableGBTSPDATA}]
ObitTableGBTSPDATA  Class
\tabletitle{Template ObitTableGBTSPDATA document}
% table name
\tablename{DATA}
\tableintro[
{Table in GBT archive/SpectralProcessor file.
This class contains tabular data and allows access.
This class is derived from the ObitTable class. }
]
\tableover{
The details of the storage in the buffer are kept in the 
ObitTableDesc.
}
% Table keyword description
\begin{keywords}
\tablekey[{"OBJECT "}{A}{object}{}{}
{Source name 
}
]
\tablekey[{"SCAN "}{J}{scan}{}{}
{Scan number 
}
]
\tablekey[{"UTDATE "}{J}{utdate}{}{}
{MJD of start time 
}
]
\tablekey[{"UTCSTART "}{D}{utcstart}{}{}
{Start time in seconds. 
}
]
\tablekey[{"UTCSTOP "}{D}{utcstop}{}{}
{Stop time in seconds. 
}
]
\\tablekey[{"INTTIME "}{D}{inttime}{}{}
{Integration time in seconds. 
}
]
tablekey[{"BACKEND"}{A}{backend}{"DCR"}{}
{Which backend 
}
]
\tablekey[{"CTYPE1"}{A}{ctype1}{"STATE"}{}
{First data axis is State 
}
]
\tablekey[{"CTYPE2"}{A}{ctype2}{"RECEIVER"}{}
{Second data axis is Receiver 
}
]
\end{keywords}
%
% Table column description
\begin{columns}
\tablecol[{"SUBSCAN "}{"CODE " }{J}{(1)}{subscan}
{}
] 
\tablecol[{"UTDATE "}{"DAY" }{J}{(1)}{utdate}
{}
] 
\tablecol[{"UTCSTART "}{"SECOND" }{D}{(1)}{utcstart}
{}
] 
\tablecol[{"PSRPER "}{" " }{D}{(1)}{psrper}
{Pulsar period.}
] 
\tablecol[{"DATA "}{"COUNT  " }{E}{(1024,2,2))}{data}
{Data}
] 
\end{columns}
%
% Table modification history
\begin{history}
\modhistory[{A. N. Author}{99/99/9999}{Revision 1: Copied from GBT}]
\end{history}
\clearpage

%%%%%%%%%%%%%%% GBT IF file %%%%%%%%%%%%%%%%%%%%%%%%%%%%%%%%%%%%%%%%%%%%%%%%%%%%%%%%%
%%%%%%%%%%%%%%% ObitTableGBTIF Class %%%%%%%%%%%%%%%%%%%%%%%%%%%%%%%%%%%%%%%%%%%%%%%%%%%
% 
%
\ClassName[{ObitTableGBTIF}]
ObitTableGBTIF Class
\tabletitle{ObitTableGBTIF document}
% table name
\tablename{IF }
\tableintro[
{Table in GBT archive/IF file.
This class contains tabular data and allows access.
This class is derived from the ObitTable class. }
]
\tableover{
The details of the storage in the buffer are kept in the 
ObitTableDesc.
IF controler information about frequency and polarization setup.
Sky Frequency Formula:
$$ sky = SFF\_SIDEBAND*IF + SFF\_MULTIPLIER*LO1 + SFF\_OFFSET $$
Signed Sum of the LOs:
$$  sum =  -(SFF\_MULTIPLIER*LO1 + SFF\_OFFSET)/SFF\_SIDEBAND $$
}
% Table keyword description
%
% Table column description
\begin{columns}
\tablecol[{"BACKEND"}{" " }{A}{(32)}{backend}
{Name of the terminating backend.}
] 
\tablecol[{"BANK "}{" " }{A}{(2)}{bank}
{Name of the backend's set of inputs.}
] 
\tablecol[{"PORT "}{" " }{J}{(1)}{port}
{Index of the backend's input.}
] 
\tablecol[{"RECEIVER"}{" " }{A}{(32)}{receiver}
{Name of the receiver of origin.}
] 
\tablecol[{"FEED"}{" " }{J}{(1)}{feed}
{Index of receiver RF entry point (0 indicates none.}
] 
\tablecol[{"SRFEED1"}{" " }{J}{(1)}{srfeed1}
{Index of first FEED of a sig/ref pair.}
] 
\tablecol[{"SRFEED2"}{" " }{J}{(1)}{srfeed2}
{Index of second FEED of a sig/ref pair.}
] 
\tablecol[{"RECEPTOR"}{" " }{A}{(8)}{receptor}
{Name of the receiver's detector.}
] 
\tablecol[{"LO\_CIRCUIT"}{" " }{A}{(32)}{loCircuit}
{Circuit producing the tracking frequency.}
] 
\tablecol[{"LO\_COMPONENT"}{" " }{A}{(32)}{loComponent}
{component producing the tracking frequency.}
] 
\tablecol[{"SIDEBAND"}{" " }{A}{(2)}{sideband}
{Resulting sideband: upper or lower.}
] 
\tablecol[{"POLARIZE"}{" " }{A}{(2)}{polarize}
{Resulting polarization ('X', 'Y', 'R', 'L').}
] 
\tablecol[{"CENTER\_IF"}{"HZ" }{E}{(1)}{CenterIF}
{Approximate physical center frequency.}
] 
\tablecol[{"CENTER\_SKY"}{"HZ" }{E}{(1)}{CenterSky}
{Approximate center frequency on the sky.}
] 
\tablecol[{"BANDWDTH"}{"HZ" }{E}{(1)}{bandwdth}
{Approximate resulting bandwidth.
BANDWDTH of 0 denotes the bandpass is outside the optimal range.
}
] 
\tablecol[{"HIGH\_CAL"}{" " }{J}{(1)}{highCal}
I{ndicates a high powered calibrator was used.}
] 
\tablecol[{"TEST\_TONE\_IF"}{"HZ" }{E}{(1)}{testToneIF}
{Approximate physical test tone frequency, if any.}
] 
\tablecol[{"TEST\_TONE\_SKY"}{"HZ" }{E}{(1)}{YestToneSky}
{Approximate test tone frequency on the sky, if any.}
] 
\tablecol[{"TEST\_TONE\_CIRCUIT"}{" " }{A}{(32)}{TestToneCircuit}
{Circuit producing the test tone, if any.}
] 
\tablecol[{"TEST\_TONE\_COMPONENT"}{" " }{A}{(32)}{testToneComponent}
{Component producing the test tone, if any.}
] 
\tablecol[{"SFF\_MULTIPLIER"}{" " }{D}{(1)}{sffMultiplier}
{Sky Frequency Formula multiplier coefficient.}
] 
\tablecol[{"SFF\_SIDEBAND"}{" " }{D}{(1)}{sffSideband}
{Sky Frequency Formula sideband coefficient.}
] 
\tablecol[{"SFF\_OFFSET"}{" " }{D}{(1)}{sffOffset}
{Sky Frequency Formula offset coefficient.}
] 
\tablecol[{"TRANSFORM\_COUNT"}{" " }{J}{(1)}{transformCount}
{Number of transform.}
] 
\tablecol[{"TRANSFORMS"}{" " }{A}{(4096)}{transforms}
{Matrix of transform descriptions (frequencies in MHz).}
] 
\end{columns}
%
% Table modification history
\begin{history}
\modhistory[{A. N. Author}{99/99/9999}{Revision 1: Copied from GBT}]
\end{history}
\end{document}

\clearpage
%%%%%%%%%%%%%%% ObitTableXX Class %%%%%%%%%%%%%%%%%%%%%%%%%%%%%%%%%%%%%%%%%%%%%%%%%%%
% 
%
\ClassName[{ObitTableXX}]
ObitTableXX Class
\tabletitle{Template ObitTable document}
% table name
\tablename{XX}
\tableintro[
{This class contains tabular data and allows access.
"AIPS XX" contains highly secret information.
An ObitTableXX is the front end to a persistent disk resident structure.
Both FITS and AIPS cataloged data are supported.
This class is derived from the ObitTable class. }
]
\tableover{
In memory tables are stored in a fashion similar to how they are 
stored on disk - in large blocks in memory rather than structures.
Due to the word alignment requirements of some machines, they are 
stored by order of the decreasing element size: 
double, float long, int, short, char rather than the logical order.
The details of the storage in the buffer are kept in the 
ObitTableDesc.
}
% Table keyword description
\begin{keywords}
\tablekey[{"REVISION"}{J}{revision}{1}{}
{Revision number of the table definition.
}
]
\tablekey[{"NO\_SECRET"}{J}{numPol}{}{(1,2)}
{The number of secrets.
}
]
\end{keywords}
%
% Table column description
\begin{columns}
\tablecol[{"TIME    "}{"DAYS   " }{D}{(1)}{Time}
{The center time of the secret.}
] 
\tablecol[{""}{"" }{}{()}{}
{}
] 
\end{columns}
%
% Table modification history
\begin{history}
\modhistory[{A. N. Author}{99/99/9999}{Revision 1: Copied from AIPS}]
\end{history}
\end{document}

\clearpage
%%%%%%%%%%%%%%% ObitTableXX Class %%%%%%%%%%%%%%%%%%%%%%%%%%%%%%%%%%%%%%%%%%%%%%%%%%%
% 
%
\ClassName[{ObitTableXX}]
ObitTableXX Class
\tabletitle{Template ObitTable document}
% table name
\tablename{XX}
\tableintro[
{This class contains tabular data and allows access.
"AIPS XX" contains highly secret information.
An ObitTableXX is the front end to a persistent disk resident structure.
Both FITS and AIPS cataloged data are supported.
This class is derived from the ObitTable class. }
]
\tableover{
In memory tables are stored in a fashion similar to how they are 
stored on disk - in large blocks in memory rather than structures.
Due to the word alignment requirements of some machines, they are 
stored by order of the decreasing element size: 
double, float long, int, short, char rather than the logical order.
The details of the storage in the buffer are kept in the 
ObitTableDesc.
}
% Table keyword description
\begin{keywords}
\tablekey[{"REVISION"}{J}{revision}{1}{}
{Revision number of the table definition.
}
]
\tablekey[{"NO\_SECRET"}{J}{numPol}{}{(1,2)}
{The number of secrets.
}
]
\end{keywords}
%
% Table column description
\begin{columns}
\tablecol[{"TIME    "}{"DAYS   " }{D}{(1)}{Time}
{The center time of the secret.}
] 
\tablecol[{""}{"" }{}{()}{}
{}
] 
\end{columns}
%
% Table modification history
\begin{history}
\modhistory[{A. N. Author}{99/99/9999}{Revision 1: Copied from AIPS}]
\end{history}
\end{document}

\clearpage
%%%%%%%%%%%%%%% ObitTableXX Class %%%%%%%%%%%%%%%%%%%%%%%%%%%%%%%%%%%%%%%%%%%%%%%%%%%
% 
%
\ClassName[{ObitTableXX}]
ObitTableXX Class
\tabletitle{Template ObitTable document}
% table name
\tablename{XX}
\tableintro[
{This class contains tabular data and allows access.
"AIPS XX" contains highly secret information.
An ObitTableXX is the front end to a persistent disk resident structure.
Both FITS and AIPS cataloged data are supported.
This class is derived from the ObitTable class. }
]
\tableover{
In memory tables are stored in a fashion similar to how they are 
stored on disk - in large blocks in memory rather than structures.
Due to the word alignment requirements of some machines, they are 
stored by order of the decreasing element size: 
double, float long, int, short, char rather than the logical order.
The details of the storage in the buffer are kept in the 
ObitTableDesc.
}
% Table keyword description
\begin{keywords}
\tablekey[{"REVISION"}{J}{revision}{1}{}
{Revision number of the table definition.
}
]
\tablekey[{"NO\_SECRET"}{J}{numPol}{}{(1,2)}
{The number of secrets.
}
]
\end{keywords}
%
% Table column description
\begin{columns}
\tablecol[{"TIME    "}{"DAYS   " }{D}{(1)}{Time}
{The center time of the secret.}
] 
\tablecol[{""}{"" }{}{()}{}
{}
] 
\end{columns}
%
% Table modification history
\begin{history}
\modhistory[{A. N. Author}{99/99/9999}{Revision 1: Copied from AIPS}]
\end{history}
%
